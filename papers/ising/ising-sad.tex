\documentclass[letterpaper,twocolumn,amsmath,amssymb,pre,aps,10pt]{revtex4-1}
\usepackage{graphicx} % Include figure files
\usepackage{color}
\usepackage{nicefrac} % Include for in-line fractions
\usepackage{harpoon}% <---

\newcommand{\red}[1]{{\bf \color{red} #1}}
\newcommand{\green}[1]{{\bf \color{green} #1}}
\newcommand{\blue}[1]{{\bf \color{blue} #1}}
\newcommand{\cyan}[1]{{\bf \color{cyan} #1}}

\newcommand{\davidsays}[1]{{\color{red} [\green{David:} \emph{#1}]}}
\newcommand{\jpsays}[1]{{\color{red} [\blue{Jordan:} \emph{#1}]}}
\newcommand{\tssays}[1]{{\color{red} [\cyan{Tanner:} \emph{#1}]}}

\newcommand{\Int}{\int}
\newcommand\Emin{E_{\min}}
\newcommand\Nmax{N_{\max}}
\newcommand\Tmin{T_{\min}}
\newcommand\mumax{\mu_{\max}}

\begin{document}
\title{SAD Monte Carlo}

\author{Jordan K. Pommerenck}
\author{David Roundy}
\affiliation{Department of Physics, Oregon State University,
  Corvallis, OR 97331}

\begin{abstract}
In this work, we finish writing everything before the abstract~\jpsays{FIX AT END}
\end{abstract}

\maketitle

\section{Introduction}
In the early 2000s, Wang and Landau introduced a novel flat-histogram
Monte-Carlo algorithm that accurately determined the density of states
(DOS) for a statistical
system~\cite{wang2001determining,wang2001efficient}. For all of its
power, the method unfortunately requires a priori knowledge of several
user-defined parameters. Thus, for any given system under study, the
user needs to determine the ideal parameters in order to apply the
method. The Wang-Landau algorithm is also known to violate detailed
balance (although only for brief time intervals)~\cite{yan2003fast,
shell2002generalization}. With the violation of detailed balance,
convergence of the algorithm is not guaranteed. The convergence rate
for WL is well known~\cite{zhou2005understanding,lee2006convergence,
belardinelli2007wang}. Modification factors that decrease faster than
$1/t$ lead to non-convergence~\cite{belardinelli2007fast}.  These
findings lead to the formation of the $1/t$-WL algorithm.  The ideal
choice of decreasing the modification factor prevents CPU time from
being wasted by continuing to perform calculations after the error in
the density of states is saturated~\cite{belardinelli2008analysis}.
Liang independently considered whether WL could be treated as a special
case of Stochastic Approximation whose convergence could be
mathematically proven~\cite{liang2006theory, liang2007stochastic}. In
2007, Liang et al.~\cite{liang2007stochastic} argued that WL can be
considered a form of Stochastic Approximation Monte Carlo (SAMC).
Unlike WL, SAMC can guarantee convergence (if certain conditions are
met). Despite the added benefit of guaranteed convergence, the method
still has a system specific user-defined variable. Such variables often
create difficulty when applying Monte-Carlo methods across arbitrary
systems.  The SAD (Dynamic Stochastic Approximation)
method~\jpsays{outlined by Pommerenck et.al include David and my paper
on SAD square well in citations!} is special version of the SAMC
Algorithm that dynamically chooses the modification factor rather than
relying on system dependent parameters. SAD shares the same convergence
properties with SAMC while replacing unphysical user-defined parameters
with the algorithms dynamic choice.  In this work, we apply the family
of Monte-Carlo algorithms (WL, SAMC, SAD) to the 2D Potts Model. We
explore the convergence properties of the three algorithms in an effort
to demonstrate the system scalability of SAD.

\section{Ising Model}
The 2D ten state Potts Model was used as a test system for the Wang 
Landau method during its inception~\cite{wang2001determining,wang2001efficient}.

Apparently Wang and Landau cite~\cite{wu1982potts} because of exact solutions 
and extensive simulation data are available. Since they use a Q = 10 2D Potts
Model I assume they are comparing 1 of 2 things.
(1) The latent heat of the system which can be obtained by plotting U vs T
and looking at how much U drops at Tc.
(2) Looking at the value for the critical temperature Tc.
Or it is possible that they clarify in a later work~\cite{landau2004new} how
to extract the correct density of states. 
\begin{align}
\sum_E g_N\left( E\right)=Q^N
\end{align}
Also the number of ground states (where $E = -2N$) is $Q$. He then rescales
the density of states $g_N\left( E\right)$ and writes
\begin{align}
\ln \left[ g_N\left(E\right)\right]=\ln \left[ g\left( E\right)\right]
-\ln \left[ g\left(E=-2N\right)\right]+\ln\left[ Q \right]
\end{align}
He calls this the normallized density of states (for his work Q = 10). The
transition temperature for a 2D lattice $k_B T_C\left(L\right)$ can be directly
calculated~\cite{landau2004new} and compared with the known value,~\cite{wu1982potts}
\begin{align}
k_B T_C = \frac{1}{\ln\left(1+(Q)^0.5\right)}
\end{align}

\section{Methods}
\subsection{Wang-Landau: WL}

\subsection{Stochastic Approximation Monte Carlo: SAMC}

\subsection{Dynamic Stochastic Approximation: SAD}

\section{Results}
\section{Conclusion}
\section{Acknowledgements}

\maketitle


\bibliography{paper}% Produces the bibliography via BibTeX.

\end{document}
