\documentclass[letterpaper,twocolumn,amsmath,amssymb,pre,aps,10pt]{revtex4-1}
\usepackage{graphicx}% Include figure files
\usepackage{dcolumn}% Align table columns on decimal point
\usepackage{bm}% bold math
\usepackage{color}

\newcommand{\red}[1]{{\bf \color{red} #1}}
\newcommand{\rr}{\textbf{r}}
\newcommand{\refnote}{\red{[ref]}}

\newcommand{\fixme}[1]{\text{\red{[#1]}}}

%\newcommand{\derivation}[1]{#1} % Use this to show all derivations in detail
\newcommand{\derivation}[1]{} % Use this for nice pegagogical paper...

\begin{document}
\title{Using Fundamental Measure Theory to treat the Pair
  Distribution Function of the Inhomogeneous Hard-Sphere Fluid}

\author{Paho Lurie-Gregg}
\author{Jeff B. Schulte}
\author{David Roundy}
\affiliation{Department of Physics, Oregon State University, Corvallis, OR 97331}

%\pacs{61.20.Ne, 61.20.Gy, 61.20.Ja}
%%%%%%%%%%%%%%%%%%%%%%%%%%%%%%%%%%%%%%%%%%%%%%%%%%%%%%%%%%%%
\begin{abstract}
  We have developed an approximation for the pair distribution
  function of hard spheres, which is based on the average value of the
  pair distribution function at contact described in
  Section~\ref{sec:contact}.  Our pair distribution is constructed so
  as to be computable with only single-site convolutions and to
  accurately reproduce the averaged distribution function at contact.
  This results in a functional that is efficient to compute, and
  predicts the pair distribution function as well as the most accurate
  existing analytical approximation.
\end{abstract}

\maketitle

\newcommand\saftlocaldft{}
% The following are papers that use a SAFT-based classical DFT with
% all the terms that should be non-local being non-local.
\newcommand\saftnonlocaldft{}

The standard approach in liquid state theory is to treat a liquid as a
hard-sphere reference fluid with attractive interactions that are
treated perturbatively~\cite{hansen2006theory}.  Recently there have
been considerable advances in the development of classical density
functionals have been based on the
application of these theoretical approaches to inhomogeneous
densities~\cite{ jain2007modified, gloor2007prediction, gross2009density,
  %felipe2001examination, gloor2002saft,
  %gloor2004accurate, clark2006developing,
  kahl2008modified,
  % yu2002fmt-dft-inhomogeneous-associating,
  %fu2005vapor-liquid-dft, bryk2006density,
  hughes2013classical,
  %segura1998comparison, felipe2001examination,
  %gloor2002saft, gloor2004accurate, fu2005vapor-liquid-dft,
  bryk2006density, clark2006developing,
  kahl2008modified, gross2009density,
  %yu2002fmt-dft-inhomogeneous-associating,
  sundararaman2012computationally, marshall2013density}.  The
perturbation theory treatment of intermolecular interactions relies on
the pair distribution function of the reference fluid:
$g^{(2)}(\rr_1,\rr_2)$.  Unlike the radial distribution function of a
homogeneous fluid, there does not currently exist a tractable form for
the pair distribution function of an inhomogeneous hard-sphere fluid,
suitable for use in constructing a density
functional~\cite{gloor2007prediction, jain2007modified}.

At its core, thermodynamic perturbation theory---sometimes referred to
as the cluster, or high-temperature expansion---is an expansion of the
free energy in powers of a small parameter, which typically is a
pairwise attractive interaction:
\begin{align}
  F &= F_0 + F_1 + \beta F_2 + \mathcal{O}(\beta^2)
\end{align}
where the terms $F_n$ are corrections to the free energy at order $n$
in the small interaction.  The first and largest term in this
expansion is
\begin{align}
  F_1 &= \tfrac12 \int g^{(2)}_{HS}(\rr_1,\rr_2)n(\rr_1)n(\rr_2)\Phi(|\rr_1-\rr_2|) d\rr_1d\rr_2
\end{align}
where $g^{(2)}_{HS}(\rr_1,\rr_2)$ is the pair distribution function of
the hard-sphere reference fluid, and $\Phi(r)$ is the pair potential.
In this paper, we will introduce an approximation for the hard-sphere
pair distribution function which is suitable for use in the creation
of classical density functionals based on a cluster expansion.

\section{Theoretical approaches}

The classic (and earliest) approach for computing the pair
distribution function is to use Percus' trick of treating one sphere
as an external field, and using the resultant equilibrium density to
find the pair distribution function~\cite{hansen2006theory}.  This
elegant approach lends itself to DFT, and can be used to compute and
plot the pair distribution function, but requires a full free-energy
minimization \emph{for each position} $\rr_1$ in
$g^{(2)}(\rr_1,\rr_2)$, and hence would be prohibitively expensive as
a tool in constructing a free energy functional.

The first canonical inhomogeneous configuration for the hard-sphere
fluid is the system consisting of a hard spheres at a hard wall.  In
1986, Plischke and Henderson solved the pair distribution function of
this system using integral equation theory under the Percus-Yevick
approximation~\cite{plischke1986pair}.  Ten years later,
G{\"o}tzelmann \emph{et al.} found the pair distribution function near
a hard wall using classical DFT by finding the direct correlation
function by taking a functional derivative of the free energy
functional and then solving the Ornstein-Zernicke
equation~\cite{gotzelmann1996structure}.  The approach of solving the
Ornstein-Zernicke equation for an inhomogeneous system is
computationally challenging~\cite{paul2003variational}.

A second inhomogeneous configuration that is of interest is the
test-particle configuration, in which one hard sphere is fixed in the
otherwise homogeneous fluid.  Where the hard-wall is a surface with no
curvature, the test-particle configuration has a surface with
curvature at the molecular length scale.  In this case, the density
gives the pair distribution function---this is just Percus'
trick---and the pair distribution function gives the triplet
distribution function.  The triplet distribution function was computed
by Gonz\'alez \emph{et al.} using the test-particle approach with
\emph{two} spheres fixed~\cite{gonzalez1999test}.

Lado recently introduced a new and improved efficient algorithm for
implementing integral equation theory for inhomogeneous fluids, which
computes $g^{(2)}(\rr_1,\rr_2)$~\cite{lado2009efficient}.  While this
approach is two orders of magnitude more efficient than previous
implementations, it remains a computationally expensive approach, and
unsuitable for repeated evaluation within a free-energy minimization.

\section{Analytic approximations}
There are few analytic approximations for the inhomogeneous pair
distribution function, which extend the radial distribution function
to inhomogeneous scenarios.  These works differ in what density they
use to evaluate $g(r)$.

Several recent works have cited a
mean-density approximation:
\begin{align}
  g^{(2)}(\rr_1,\rr_2) \approx g\left(r_{12}, \frac{\pi\sigma^3}{6}\tfrac12
  (n(\rr_1)+n(\rr_2))\right)
\end{align}
where $g(r,\eta)$ is the radial distribution function as a function of
radius and packing fraction $\eta$~\cite{gloor2007prediction,
  gross2009density}.  These works did not however use this approach,
which fails dramatically at higher densities, as the packing fraction
found from the mean density exceeds unity.

Approaches that do not diverge use an average of the density over some
volume. Fischer and Methfessel~\cite{fischer1980born} introduce the
approximation
\begin{align}
  g^{(2)}(\rr_1,\rr_2) \approx g\left(r_{12}, n_3\left(\tfrac12
  (\rr_1+\rr_2)\right)\right)
  \label{eq:fischer}
\end{align}
where $n_3$ is one of the fundamental measures defined in Fundamental
Measure Theory~\cite{rosenfeld1989free}, which is an integral of the
density over a spherical volume:
\begin{align}
  n_3(\rr) = \int n(\rr')\Theta(\tfrac12 \sigma - |\rr-\rr'|) d\rr'
\end{align}
where $\sigma$ is the hard-sphere diameter.  Equation~\ref{eq:fischer}
is computationally awkward, because it relies on the midpoint
$\tfrac12(\rr_1+\rr_2)$.  For this reason Sokolowski and Fischer
modified this approach to use averages centered on the two points
$\rr_1$ and $\rr_2$:
\begin{align}
  g^{(2)}(\rr_1,\rr_2) \approx g\left(r_{12},
  \tfrac12(\bar{n}(\rr_1)+\bar{n}(\rr_2))\right)
  \label{eq:sokolowski}
\end{align}
where
\begin{align}
  \bar{n}(\rr) = \frac{3}{4\pi (0.8\sigma)^3}\int n(\rr')\Theta(0.8\sigma - |\rr-\rr'|) d\rr'
\end{align}
is the density averaged over a sphere with diameter
$0.8\sigma$~\cite{sokolowski1992role}.  The value 0.8 in this formula
was arrived at by fitting to Monte Carlo simulation.  Although
Eq.~\ref{eq:sokolowski} has the advantages of only involving density
averages at the points at which the pair distribution function is
desired, cannot be written as a single-site convolution, since the
convolution kernel depends on both points.



\section{What is our new idea?}
We approximate the pair correlation with the average of two radial
distribution functions, evaluated at the distance between the two
points, that are themselves functions of our published pair
correlation at contact $g_{\sigma}$ evaluated at the two points.

\begin{align}
  g^{(2)}(\rr_1,\rr_2) = \frac{g(|r_{12}|, g_\sigma(\rr_1)) + g(|r_{12}|, g_\sigma(\rr_2))}{2}
\end{align}

In the previous approaches mentioned above, function forms of radial
distributions (used to approximate pair correlation) were dependent
upon independent variables $\eta$, the density, wether at single
points or averaged over sourrounding volumes.  We can achieve more
accuracy by using instead a function dependent upon our $g_{\sigma}$,
which holds more information about an inhomogeneous system
then an average $\eta$ does.  We therefore need a function form for
our radial distribution that is dependent upon $g_{\sigma}$ and that
will approach the correct solution in the homogeneous limit and also
in the limit of contact distance between the two points.

The demand that our solution is correct in the homogeneous limit forms
the basis of the derivation of our radial distribtuion's function
form. We use the well known and very successful Carnahan Starling
equation to calculated $g_{\sigma}$ and then simple monte-carlo
simulations to calculate values of the homogeneous radial distribution
function at varying distance perameter $r$ (Our previously published
$g_{\sigma}$ approaches the Carnahan Starling version in the
homogeneous limit and it's simplest to use an analytical expression
for this calculation).  We then construct our radial distribution
function that has the form of an expansion of powers of $h_{\sigma} =
g_{\sigma}-1$ and perform a least squares fit to the monte-carlo
solution to find the needed constant perameters.  The function is
constructed in such a way as to appoach $g_{\sigma}$ in the limit of
contact distance between the particles.  After this process we have a
radial distribution function that is dependent only on $g_{\sigma}$
and that gives the correct solution in the homogeneous and contact
distance limit.

The function has the form:
\begin{align}
  g(|\rr|) &= 1 + g_{\sigma}e^{-x_0(r-\sigma)} + x_1g_{\sigma}\sin(x_2(r-\sigma))e^{-x_3(r-\sigma)}\\
  &~~-x_4g_{\sigma}^2\sin(x_5(r-\sigma))e^(-x_6(r-\sigma))
\end{align}
The finalalized form of our pair correlation function is:
\begin{align}
  g^{(2)}(\rr_1,\rr_2) = \frac{g(|r_{12}|, g_\sigma(\rr_1)) + g(|r_{12}|, g_\sigma(\rr_2))}{2}
\end{align}



%% gsigma = gsigconcatenated[i] - 1
%%         h0 = gsigma # was x[0]*gsig
%%         f0 = numpy.exp(-x[0]*rconcatenated[i])
%%         h1 = x[1]*gsigma
%%         f1 = numpy.sin(x[2]*rconcatenated[i]) * numpy.exp(-x[3]*rconcatenated[i])
%%         h2 = -x[4]*gsigma**(2)
%%         f2 = numpy.sin(x[5]*rconcatenated[i]) * numpy.exp(-x[6]*rconcatenated[i])
%%         g[i] = 1 + h0*f0 + h1*f1 + h2*f2
%%     return g

%% The homogeneoues limit $g_{sigm}$ can be calculated using the well
%% known and very successful Carnahan Starling equation while the
%% homogeneous limit radial distribution function can be calculated using
%% Monte-Carlo methods.  We used these calculations in a least squares
%% fit along the data points at increasing radial distances in order to
%% fit the perameters of our more general pair correlation function.  Our
%% function will therefore approach the most accurate solution possible
%% in the homogeneous limit.  Created a function that has the form of an
%% expansion in powers of $g_{\sigma}$.  $g_{sigm}$ in the homogeneous
%% limit can be calculated using the well known and very successful
%% Carnahan Starling equation.  has a well known and Perameters in the
%% expansion where fit
\begin{figure}
  \centering
  \includegraphics[width=\columnwidth]{figs/ghs-g}% \\
  %\includegraphics[width=\columnwidth]{figs/ghs-g-ghs}
  \caption{Plot of the radial distribution function, with our fit.}\label{fig:radial-distribution}
\end{figure}
\section{What does it look like (graphs that show the function itself)?}
\subsection{Separable fit to the radial distribution function}
Figure~\ref{fig:radial-distribution} shows...

We here construct a separable fit to the radial distribution function
$g(r, g_\sigma)$, which is a function of the radius $r$ and its value
at contact $g_\sigma$.  We could convert this to a more conventional
function of the packing fraction $\eta$ by inverting the
Carnahan-Starling formula for $g_\sigma$
\begin{equation}
  g_\sigma = \frac{1-\tfrac{\eta}{2}}{(1-\eta)^3}
\end{equation}
but that would add an unnecessary step in the use of $g(r)$, since the
averaged value of $g_\sigma$ is what we can directly compute for the
inhomogeneous system.

We construct $g(r, g_\sigma)$ using several constraints.  Naturally,
\begin{equation}
  g(\sigma, g_\sigma) = g_\sigma.
\end{equation}
In addition, we know that
\begin{align}
  1 + n\int\left[g(r)-1\right]d\rr &= nkT\chi_T
\end{align}
\begin{equation}
  \begin{split}
    \chi_T  &= -\frac{1}{V}\left(\frac{\partial V}{\partial
      p}\right)_T \\
    &= - \left(V\left(\frac{\partial p}{\partial
      V}\right)_T\right)^{-1}
  \end{split}
\end{equation}
\begin{equation}
    p = nkT\frac{1+\eta+\eta^2-\eta^3}{(1-\eta)^3}
\end{equation}
\begin{equation}
    \left(\frac{\partial p}{\partial V}\right)_T = \frac{\partial p}{\partial\eta}\frac{\partial\eta}{\partial V} + \frac{\partial p}{\partial n}\frac{\partial n}{\partial V}
\end{equation}
\begin{equation}
    n = \frac{N}{V}\Longrightarrow \frac{\partial n}{\partial V} = -\frac{N}{V^2} = -\frac{n}{V}
\end{equation}
\begin{equation}
    \eta = \frac{4\pi R^3N}{3V}\Longrightarrow \frac{d\eta}{dV} = -\frac{\eta}{V}
\end{equation}
\begin{equation}
  \begin{split}
    \left(\frac{\partial p}{\partial V}\right)_T &= -nkT\frac{\eta}{V}\left[\frac{1 + 2\eta - 3\eta^2}{(1-\eta)^3} + 3\frac{1 + \eta + \eta^2 - \eta^3}{(1-\eta)^4}\right]\\
    &\quad  - \frac{nkT}{V}\left[\frac{1+\eta+\eta^2-\eta^3}{(1-\eta)^3}\right]\\
    &= -nkT\frac{\eta}{V}\left[\frac{(1+2\eta-3\eta^2)(1-\eta) + 3 + 3\eta + 3\eta^2 - 3\eta^3}{(1-\eta)^4}\right]\\
    &\quad  - \frac{nkT}{V}\left[\frac{(1+\eta+\eta^2-\eta^3)(1-\eta)}{(1-\eta)^4}\right]\\
    &= -\frac{nkT}{V}\left[\frac{4\eta + 4\eta^2 - 2\eta^3}{(1-\eta)^4} + \frac{1 - 2\eta^3 + \eta^4}{(1-\eta)^4}\right]\\
  \end{split}
\end{equation}
\begin{equation}
    nkT\chi_T = \frac{(1-\eta)^4}{1 + 4\eta + 4\eta^2 - 4\eta^3 + \eta^4}
  %\int (g(r)-1)d\rr &= \frac{\chi_T}{kT} \\
  %&= \eta \eta \cdots \fixme{fill in, double-check}
\end{equation}
where $\chi_T$ is the isothermal compressibility, for which we use the
Carnahan-Starling equation of state.


An additional constraint that we use sets the slope of $g(r,
g_\sigma)$ at $r=\sigma$.  \fixme{explain this}

Given these constraints, we use a functional form of
\begin{equation}
  g(r,g_\sigma) = 1 + A e^{-a(r-\sigma)}
  + B (r-\sigma) e^{-b(r-\sigma)} + C (r-\sigma)^2 e^{-c(r-\sigma)}
\end{equation}
where $A$, $B$, and $C$ are amplitudes determined by the above
constraints, and $a$, $b$, and $c$ are parameters fit to Monte Carlo
simulations of the hard-sphere radial distribution function.

\fixme{add paragraphs and equations with $A$, $B$ and $C$}

\fixme{add table with parameters}

\subsection{Pair distribution function}
%% Figure~\ref{fig:pair-distribution} shows...
%% \begin{figure}
%%   %%\includegraphics[width=6cm]{figs/pair-correlation-2-395-0.pdf}
%%   \includegraphics[width=6cm]{figs/pair-correlation-2-395-5.pdf}
%%   %%\includegraphics[width=6cm]{figs/pair-correlation-2-395-10.pdf}
%%   \caption{Plot of the pair distribution function near a hard wall.
%%     In the interest of space and memeory, we should look at the function and decided which data to add?}\label{fig:pair-distribution}
%% \end{figure}

\begin{figure}
  \includegraphics[width=\columnwidth]{figs/pair-correlation-pretty-1.pdf}
  \caption{Pretty plot of the pair distribution function near a hard
    wall.}\label{fig:pair-distribution-1}
\end{figure}

\begin{figure}
  \includegraphics[width=\columnwidth]{figs/pair-correlation-pretty-2.pdf}
  \caption{Pretty plot of the pair distribution function near a hard
    wall.}\label{fig:pair-distribution-2}
\end{figure}
\begin{figure}
  \includegraphics[width=\columnwidth]{figs/pair-correlation-pretty-3.pdf}
  \caption{Pretty plot of the pair distribution function near a hard
    wall.}\label{fig:pair-distribution-3}
\end{figure}
\begin{figure}
  \includegraphics[width=\columnwidth]{figs/pair-correlation-pretty-4.pdf}
  \caption{Pretty plot of the pair distribution function near a hard
    wall.}\label{fig:pair-distribution-4}
\end{figure}

Discuss the pair distribution function here, explaining plots
  and discussing results and comparison.

\subsection{Triplet distribution function}

\begin{figure}
  \includegraphics[width=\columnwidth]{figs/triplet-correlation-pretty-contact-3.pdf}
  \caption{The triplet distribution function
    $g^{(3)}(\rr_1,\rr_2,\rr_3)$ at packing fraction 0.3, plotted when
    $\rr_1$ and $\rr_2$ are in contact.  The color plot on the left
    shows the triplet distribution function, with white corresponding
    to the contact value of the \emph{pair} distribution function.
    The plot on the right shows the value of $g^{(3)}$ along the path
    shown as a dotted lin on the plot to the
    left.}\label{fig:triplet-contact-distribution-3}
\end{figure}
\begin{figure}
  \includegraphics[width=\columnwidth]{figs/triplet-correlation-pretty-contact-4.pdf}
  \caption{The triplet distribution function
    $g^{(3)}(\rr_1,\rr_2,\rr_3)$ at packing fraction 0.3, plotted when
    $\rr_1$ and $\rr_2$ are in contact.  The color plot on the left
    shows the triplet distribution function, with white corresponding
    to the contact value of the \emph{pair} distribution function.
    The plot on the right shows the value of $g^{(3)}$ along the path
    shown as a dotted lin on the plot to the
    left.}\label{fig:triplet-contact-distribution-4}
\end{figure}

Just as the radial distribution function of a homogeneous fluid may be
computed from the density of an inhomogeneous one using Percus'
test-particle trick, the triplet distribution function of a
homogeneous system can be computed using an approximation of the pair
distribution for an inhomogeneous fluid, such as we have developed.
\begin{align}
  g^{(3)}(\rr_1,\rr_2,\rr_3) &= \frac{n_{\textrm{TP}(\rr_1)}(\rr_2)
  n_{\textrm{TP}(\rr_1)}(\rr_3)}{n^2}
  g^{(2)}_{\textrm{TP}(\rr_1)}(\rr_2,\rr_3)
\end{align}
where $g^{(3)}(\rr_1,\rr_2,\rr_3)$ is the triplet distribution
function of a homogeneous fluid with density $n$, and the
$\textrm{TP}(\rr_1)$ subscript indicates quantities computed for the
inhomogeneous density configuration in which one sphere is fixed at
position $\rr_1$.  This method treats one of the three positions---the
location of the test particle---differently from the others, which
means that a poor approximation to the pair distribution function may
beak the symmetry which is present in the true triplet distribution
function.

Figures~\ref{fig:triplet-contact-distribution-3}-\ref{fig:triplet-inbetween-distribution-4}
compare the triplet distribution function computed using our
approximation to $g^{(2)}$ with results from Monte Carlo simulation.
In each figure $\rr_1$ and $\rr_2$ are held fixed, while the third
position $\rr_3$ is varied.  The test-particle position in each case
is $\rr_1$, which is on the left-hand side of the figure.  In the
color plot on the left, the first and second spheres are displayed in
grey.  On the bottom half of each left-hand plot is the triplet
correlation function as computed using the approximation presented in
this work.  The top half displays the triplet correlation function
from Monte Carlo.  On the right, we display 

\begin{figure}
  \includegraphics[width=\columnwidth]{figs/triplet-correlation-pretty-inbetween-3.pdf}
  \caption{The triplet distribution function
    $g^{(3)}(\rr_1,\rr_2,\rr_3)$ plotted when $\rr_1$ and $\rr_2$ are
    separated by the diameter of one sphere.}\label{fig:triplet-inbetween-distribution-3}
\end{figure}
\begin{figure}
  \includegraphics[width=\columnwidth]{figs/triplet-correlation-pretty-inbetween-4.pdf}
  \caption{The triplet distribution function
    $g^{(3)}(\rr_1,\rr_2,\rr_3)$ plotted when $\rr_1$ and $\rr_2$ are
    separated by the diameter of one sphere.}\label{fig:triplet-inbetween-distribution-4}
\end{figure}


\section{How well does it work?}
Figure~\ref{fig:dadz} shows...

\appendix

\section{Properties of $g_\sigma$}
We begin with some definitions of basic quantities, looking at the
homogeneous hard-sphere fluid.  I am ignoring here the distinction
between $N$ and $N-1$, which is unimportant in the thermodynamic limit
$N\gg 1$.
\begin{align}
  Z(\sigma) &= \iiiint d\rr_1\cdots d\rr_N \prod_{i<j} \Theta(r_{ij}-\sigma)
  \\
  n^{(2)}(\rr_1,\rr_2) &= \frac{N^2}{Z}\iiiint d\rr_3\cdots d\rr_N \prod_{i<j} \Theta(r_{ij}-\sigma)
\end{align}
\begin{align}
  g^{(2)}(\rr_1,\rr_2) &=
  \frac{n^{(2)}(\rr_1,\rr_2)}{n(\rr_1)n(\rr_2)}
  \\
  &= \frac{V^2}{Z}\iiiint d\rr_3\cdots d\rr_N \prod_{i<j}
  \Theta(r_{ij}-\sigma)
  \label{eq:g2-integral-defn}
\end{align}


\begin{figure}
  \includegraphics[width=3cm]{figs/dadz-3-2.pdf}
  \includegraphics[width=3cm]{figs/dadz-3-3.pdf}
  \includegraphics[width=3cm]{figs/dadz-2-2.pdf}
  \includegraphics[width=3cm]{figs/dadz-2-3.pdf}
  %%\includegraphics[width=3cm]{figs/dadz-4-3}
  \caption{Plot of $\frac{da_1}{dz}$ near a hard
    wall. For what $\Phi$?}\label{fig:dadz}
\end{figure}

At this point we consider the effect of increasing the diameter
$\sigma$ of all our spheres.  This will reduce the number of available
states, thus reducing the partition function $Z$, which we will
ultimately relate to a change in free energy.
\begin{align}
  Z(\sigma+\delta) &= \iiiint d\rr_1\cdots d\rr_N \prod_{i<j} \Theta(r_{ij}-\sigma-\delta)
  \\
  &= \iiiint d\rr_1\cdots d\rr_N \prod_{i<j} \Theta(r_{ij}-\sigma)(1-\Theta(\sigma+\delta-r_{ij}))
\end{align}
I have written $Z(\sigma+\delta)$ in such a way that it is clear that
it is equal to $Z(\sigma)$ plus a small quantity, provided $\delta$ is
small.  We will now investigate that small quantity, which we will
write as a power series in $\delta$.  We begin by expanding the
product of $(1-\Theta(\sigma+\delta-r_{ij}))$:
\begin{multline}
    \prod_{i<j}(1-\Theta(\sigma+\delta-r_{ij})) =
    1 - \sum_{i<j} \Theta(\sigma+\delta-r_{ij}) \\
    + \frac{1}{2}\mathop{\mathop{\sum_{i<j}}_{k<l}}_{kl \ne ij}
    \Theta(\sigma+\delta-r_{ij}) \Theta(\sigma+\delta-r_{kl})
    \\ + \cdots.
\end{multline}
Given this expansion, we can readily expand the partition function:
\begin{multline}
  Z(\sigma+\delta) = Z(\sigma) \\ -
  \sum_{k<l} \iiiint \Theta(\sigma+\delta-r_{kl}) \prod_{i<j}
  \Theta(r_{ij}-\sigma) \\
  + \frac{1}{2}\mathop{\mathop{\sum_{k<l}}_{m<n}}_{kl\ne mn}
  \iiiint \Theta(\sigma+\delta-r_{kl}) \Theta(\sigma+\delta-r_{mn}) \prod_{i<j}
  \Theta(r_{ij}-\sigma) \\
  + \cdots.\label{eq:z-power-series}
\end{multline}
Our challenge now is to manipulate this into forms involving $g(r)$,
which we can then simplify using a Taylor expansion around
$r=\sigma$.  We will begin by taking advantage of the fact that the
summations in Eq.~\ref{eq:z-power-series} are mostly of identical
terms, since the integrals are over all coordinates.
\begin{multline}
  Z(\sigma)-Z(\sigma+\delta) =
  \frac{N^2}{2}\iiiint \Theta(\sigma+\delta-r_{12})
  \prod_{i<j} \Theta(r_{ij}-\sigma) \\
  - \frac{N^4}{8}\iiiint
  \Theta(\sigma+\delta-r_{12})
  \Theta(\sigma+\delta-r_{34})
  \prod_{i<j} \Theta(r_{ij}-\sigma) \\
  + \cdots \label{eq:z-expansion}
\end{multline}
At this point we are ready to connect with $g(r)$, which we can do by
considering an integral of the pair distribution function, from
Eq.~\ref{eq:g2-integral-defn}.
\begin{multline}
  \int_{\sigma}^{\sigma+\delta} g(r) d\rr\\
  \shoveleft{\quad= \int g^{(2)}(\rr_1,\rr_2)
  \Theta(\sigma+\delta-r_{12})d\rr_2} \\
  \shoveleft{\quad= \frac{1}{V}\int g^{(2)}(\rr_1,\rr_2)
  \Theta(\sigma+\delta-r_{12})d\rr_1d\rr_2} \\
  \shoveleft{\quad= \frac{V}{Z}\iiiint d\rr_1\cdots d\rr_N
  \Theta(\sigma+\delta-r_{12})}\\
  \prod_{i<j} \Theta(r_{ij}-\sigma)
  \label{eq:g-shell-integral}
\end{multline}
Looking back at Eq.~\ref{eq:z-expansion}, we can see that the first
term is easily related to Eq.~\ref{eq:g-shell-integral}.  The second
term has two shell $\delta$ functions, and is thus a bit trickier.
Fortunately, in a very large system, $\rr_3$ and $\rr_4$ are
vanishingly seldom near $\rr_1$ or $\rr_2$, so we can write the second
term as a product.
\begin{multline}
  \frac{Z(\sigma)-Z(\sigma+\delta)}{Z(\sigma)}\\
  \shoveleft{\quad= \frac{nN}{2} \int_{\sigma}^{\sigma+\delta} g(r) d\rr }
  - \frac{n^2N^2}{8} \left(\int_{\sigma}^{\sigma+\delta} g(r)
  d\rr\right)^2
  + \cdots
  \label{eq:zdiff-g}
\end{multline}
Now let us look at how this integral relates to $g(r)$ evaluated at
contact.  I'll skip a few steps doing this integral...
\begin{equation}
   \int_{\sigma}^{\sigma+\delta} g(r)
  d\rr = 4\pi\sigma^2\left(g_\sigma\delta + \left(\tfrac12g_\sigma' +
  \frac{g_\sigma}{\sigma}\right)\delta^2 + \cdots \right)
\end{equation}
Putting this together with the above we find
\begin{multline}
  \frac{Z(\sigma)-Z(\sigma+\delta)}{Z(\sigma)} =
  2\pi\sigma^2 n N g_\sigma \delta +
  \pi\sigma n N  \left(g_\sigma'\sigma + 2g\right)
  \delta^2 \\
  - 2\pi^2\sigma^4n^2N^2\sigma^4g_\sigma^2\delta^2 + \mathcal{O}(\delta^3)
\end{multline}
But by considering a Taylor expansion, we can also see that
\begin{multline}
  Z(\sigma)-Z(\sigma+\delta) =
  -\frac{\partial Z}{\partial \sigma} \delta
  -\frac12 \frac{\partial^2 Z}{\partial \sigma^2} \delta^2  + \mathcal{O}(\delta^3)
\end{multline}
So that
\begin{multline}
  \frac{\partial Z}{\partial \sigma} =
  -2ZnN\pi\sigma^2g_\sigma\\
  \frac{\partial^2 Z}{\partial \sigma^2} =
  -2ZnN\pi\sigma\left(g_\sigma'\sigma + 2g\sigma \right)
  +4Zn^2N^2\sigma^4g^2
\end{multline}
from which we can extract relationships for the derivatives of $Z$
with respect to $\sigma$.  Which if we knew what those derivatives
were, would tell us about $g_\sigma'$ and $g_\sigma$, which is what we
are after here.  We find this out by considering the Helmholtz free
energy.
\begin{equation}
  F = -kT \ln Z
\end{equation}
The derivatives are then given by
\begin{equation}
  \frac{\partial F}{\partial \sigma} = -\frac{kT}{Z} \frac{\partial
    Z}{\partial \sigma}
\end{equation}
and
\begin{equation}
  \frac{\partial^2 F}{\partial \sigma^2} =
  \frac{1}{kT}\left(\frac{\partial F}{\partial \sigma}\right)^2
  -\frac{kT}{Z} \frac{\partial^2
    Z}{\partial \sigma^2}.
\end{equation}
With a bunch more work, we find out that
\begin{equation}
  g_\sigma' = \frac{\frac{\partial^2 F}{\partial
      \sigma^2}}{2\pi\sigma^2nkT}
  - \frac{2g_\sigma}{\sigma}
\end{equation}
except that there is a mistake in here.  I can tell because $F$ is
extensive and $g_\sigma'$ is not.  This tracks back to
Eq.~\ref{eq:zdiff-g}, which is correct in this \LaTeX, but is wrong on
my scratch paper.

\bibliography{paper}% Produces the bibliography via BibTeX.

\section{THE END OF THE PAPER}

%%%%%%%%%%%%%%%%%%%%%%%%%%%%%%%%%%%%%%%%%%%%%%%%%%%%%%%%%%%%
\section{Introduction}

Attard worked out the triplet correlation function, which is something
that we can access from the pair distribution function, if we consider
the pair correlation near a single hard sphere
solute\cite{attard1989spherically}.  Gonz\'alez \emph{et al} wrote an
interesting paper using DFT and Monte Carlo to compute the
three-particle distribution function (same as the triplet correlation
function)\cite{gonzalez1999test}.  Their theory compares favorably
with the ``superposition approximation'', and we could probably do
something similar.

Plischke and Henderson worked out the pair correlation function near a
hard wall using Percus-Yevick, and compare with Monte Carlo
results\cite{plischke1986pair}.  They plot the pair correlation
function along interesting paths.  \fixme{Jeff read this paper for a
  while and does not understand what they do in oder to find the pair
  correlation function as compared to the radial distribution function}

Lado recently introduced a new and improved efficient algorithm for
implementing integral equation theory for inhomogeneous fluids, which
computes $g^{(2)}(\rr_1,\rr_2)$ among other
things\cite{lado2009efficient}.  Lado's work looks pretty awkward to
use, and their example is limited to 1D systems.  And the scaling is
still (so far as I can tell) $O(N^2)$, since their improvement is in
the prefactor, by developing a nice basis set.

\section{Previous approaches}

Gloor \emph{et al} explain that little is known of the pair
distribution function of the hard-sphere fluid, and therefore use the
formula
\begin{align}
  g^{(2)}(\rr_1,\rr_2) \approx g\left(|\rr_1-\rr_2|, \frac{\rho(\rr_1)+\rho(\rr_2)}2\right)
\end{align}
They go on to further approximate the radial distribution function,
but we can just stop here.

Gloor et al \cite{gloor2007prediction} uses average density of two points:
\begin{align}
\overline{\rho} &\equiv \frac{\rho(\rr)+\rho(\rr')}{2}
\end{align}
and the uses bulk radial distribution for the reference system (hard
spheres) at this density, in order to approximate pair correlation.

Evans \cite{evans1992density}perscribes to use the same approximation but with an unspecified 'mean denisty':

\begin{align}
g^{(2)}(\rr_1,\rr_2) &\equiv g_r(\rho^m,\rr_{12})
\end{align}

Gross \cite{gross2009density} uses a hard chain reference fluid.  In
his functional he needs a pair correlation beween the different
segments of chains.  He approximates this function with the radial
distribution given by the Carnahan Starling at a density that is the
average density of the segment positions $\rr$ and $\rr'$:

\begin{align}
\overline{\rho} &\equiv \frac{\rho(\rr)+\rho(\rr')}{2}
\end{align}

Fu and Wu \cite{fu2005vapor} do I think the same as Gloor et al but
maybe you (Dr. Roundy) should take a look at it.  They reference Tang
\cite{tang2008accurate} which I've mentioned below in the Appendix.
\fixme{DJR: Not same as Gloor, they do a WDA approximation.}


Sokolowski and Fischer \cite{sokolowski1992role} do the same as Gloor
et al except that instead of taking the exact densities
$\rho(\mathbf{r_1})$ and $\rho(\mathbf{r_2})$ to average he takes an
integral $\int_{V_{sphere}} \rho(\mathbf{r})d\mathbf{r}$ over a sphere
of some radius centered at each point.  It's these two averaged
densities that he takes the average of.

Fischer and Methfessel \cite{fischer1980born}, in a paper that is
cited by Sokolowski and Fischer (above) do the same thing (the
originate the idea of the spherical average density.  Their
Introduction has a very good discussion of the different averaging
methods to get to $g(\mathbf{r})$ tried up until then (paper is
written in 1980).


McCabe and Kiselev \cite{mccabe2004application} model a square well
potential.  One of their terms uses a radial distribution function
that once again is taken to be the bulk function evaluated at an
effective density.  They reference a source that calculates this
effective density (or filling fraction) to be $\eta_{effective} =
c_1\eta + c_2\eta^2 + c_3\eta^3$ where the coeffecients are gotten
from three coupled polynomial equations with $1,\lambda$, and
$\lambda^2$, where $\lambda$ is a measure of the model's potential
range.  They reference another source who calculates the effective
density in a similar (but more complicated) way.  I can look more at
these if you'd like.  They also use a different first order expansion
formula for the ditribution at contact in the chain term.


Toxvaerd \cite{toxvaerd1976hydrostatic}approximates the radial
distribution as the radial distribution of the bulk at the same
density.  Uses the first two terms in a tempurature expansion.  First
term is just Carnahan Starling derived, second is empirical.


\section{Pair distribution function with inhomogeneity}

The pair distribution function can be defined using the two particle
density as
\begin{align}
  g^{(2)}(\rr_1,\rr_2) &\equiv \frac{n^{(2)}(\rr_1,\rr_2)}{n(\rr_1)n(\rr_2)}
\end{align}

We may also want to consider the \emph{triplet} distribution function
\begin{align}
  g^{(3)}(\rr_1,\rr_2,\rr_3) &\equiv \frac{n^{(3)}(\rr_1,\rr_2,\rr_3)}{n(\rr_1)n(\rr_2)n(\rr_3)}
\end{align}
The triplet distribution is relevant, because we can compute the
triplet distribution of the homogeneous system by considering the
\emph{pair} distribution of the inhomogeneous ``test-particle
scenario.''  The process is that we solve the inhomogeneous situation
in which we have one hard sphere located at the origin.  We solve for
the density around that hard sphere, and this gives us the pair
distribution function of the homogeneous system $g^{(2)}(0, \rr)$.
When we solve for the pair distribution $g^{(2)}(\rr_1,\rr_2)$ of the
inhomogeneous system, this will give us the \emph{triplet}
distribution function of the homogeneous system
$g^{(3)}(0,\rr_1,\rr_2)$.

\section{Goals of this paper (to be integrated into rest of paper)}


In this paper, we seek an efficient approximation for the pair
correlation function of an inhomogeneous hard-sphere fluid, where we
define efficient as an algorithm that scales as $O(N\log N)$ where $N$
scales as the volume of the system under consideration.
Integral equation theory (look this up) scales as $O(N^2)$, which can
be very expensive for large three-dimensional systems.

Fundamental-Measure Theory (FMT) is constructed using only single-point
convolutions (check terminology) such as
\begin{align}
  n_\alpha(\rr) &= \int n(\rr')w_\alpha(|\rr-\rr'|) d\rr'
\end{align}
These convolutions can be computed in Fourier space in $O(N\log N)$
time, with the result that the FMT free energy---and its
gradient---can be computed in $O(N\log N)$ time.

The pair distribution function is used in computing the first term in
the high-temperature perturbation expansion, which is used in SAFT
(for instance) to treat dispersion interactions.  This term is written
as
\begin{align}\label{eq:a1}
  a_1 &\propto \iint n(\rr)n(\rr') \Phi(|\rr-\rr'|)
  g^{(2)}(\rr,\rr')d\rr d\rr'
\end{align}
where $\Phi(r)$ is the pair potential.  If $g^{(2)}(\rr,\rr')$ could
be written as $g(|\rr-\rr'|)$, this term could be computed in $O(N\log
N)$ time.  However, this is not possible when the density is
inhomogeneous, because the pair distribution function depends on the
density.

In this paper, we consider a superposition approximation, in which
construct an approximate representation for the pair distribution
function, which allows the first term in the high temperature
expansion given in Equation~\ref{eq:a1} to be computed in $O(N\log N)$
time.  One input to this approach is our recently published expression
for an average of the value of the pair distribution function at
contact $g_\sigma(\rr)$, which itself can be computed in $O(N\log N)$
time.

\begin{align}
  g^{(2)}(\rr_1,\rr_2) = \frac{g(|r_{12}|, \eta(g_\sigma(\rr_1))) + g(|r_{12}|, \eta(g_\sigma(\rr_2)))}{2}
\end{align}
Bad:
\begin{align}
  g^{(2)}(\rr_1,\rr_2) = g\left(|r_{12}|, \frac{\eta(g_\sigma(\rr_1)) + \eta(g_\sigma(\rr_2))}{2}\right)
\end{align}

The general idea is to write
\begin{align}
  g^{(2)}(\rr_1,\rr_2) = \Theta(r_{12}-\sigma)
  \left(1 + \sum_{\alpha} \frac{f_{\alpha}(g_\sigma(\rr_1)) + f_{\alpha}(g_\sigma(\rr_2))}{2}g_{\alpha}(r_{12})\right)\label{eq:gpair-approx}
\end{align}
where $f_{\alpha}$ and $g_{\alpha}$ are a set of functions that
satisfy the following constraint:
\begin{align}\label{eq:definealpha}
  \sum_{\alpha} f_{\alpha}(g_\sigma(\eta)) g_{\alpha}(r) \approx g(r; \eta)
\end{align}
which ensures that our pair distribution function reduces
(approximately) to the radial distribution function in the homogeneous
limit.  Equation~\ref{eq:gpair-approx} is constructed such that
Equation~\ref{eq:a1} can be evaluated in $O(N\log N)$ time, since all
the required convolutions may be performed with a fixed kernel.

Our proposed pair distribution function has several good properties.
When evaluated at contact ($r_{12}=\sigma$), it gives the same good
result as our previously-published functional for the averaged pair
distribution function at contact, which is constructed to give an
accurate prediction for the contribution to Equation~\ref{eq:a1} from
spheres in contact.  In the homogeneous limit,
Equation~\ref{eq:gpqair-approx} should give a prediction that is as
accurate as the approximate equality in
Equation~\ref{eq:definealpha}.  In inhomogeneous distributions and at
intermediate distances, we expect our pair distribution function to be
least accurate, but one hopes that it will be reasonable, simply
because it is an average of two reasonable results.


\subsection*{Fundamental-Measure Theory}

We use the White Bear version of the Fundamental-Measure Theory~(FMT)
functional~\cite{roth2002whitebear}, which describes the excess free
energy of a hard-sphere fluid.  The White Bear functional reduces to
the Carnahan-Starling equation of state for homogeneous systems.  It
is written as an integral over all space of a local function of a set
of ``fundamental measures'' $n_\alpha(\rr)$, each of which is written
as a one-center convolution of the density.  The White Bear free
energy is thus
\begin{equation}
A_{\textit{HS}}[n] = k_B T \int \left(\Phi_1(\rr) + \Phi_2(\rr) + \Phi_3(\rr)\right) d\rr \; ,
\end{equation}
with integrands
\begin{align}
\Phi_1 &= -n_0 \ln\left( 1 - n_3\right) \label{eq:Phi1}\\
\Phi_2 &= \frac{n_1 n_2 - \mathbf{n}_{V1} \cdot\mathbf{n}_{V2}}{1-n_3} \\
\Phi_3 &= (n_2^3 - 3 n_2 \mathbf{n}_{V2} \cdot \mathbf{n}_{V2}) \frac{
  n_3 + (1-n_3)^2 \ln(1-n_3)
}{
  36\pi n_3^2\left( 1 - n_3 \right)^2
} , \label{eq:Phi3}
\end{align}
using the fundamental measures
\begin{align}
  n_3(\rr) &= \int n(\rr') \Theta(\sigma/2 -\left|\rr - \rr'\right|)
  d\rr' \label{eq:FMn3} \\
  n_2(\rr) &= \int n(\rr') \delta(\sigma/2 -\left|\rr - \rr'\right|) d\rr' \\
  \mathbf{n}_{2V}(\rr) &= \int n(\rr') \delta(\sigma/2 -\left|\rr - \rr'\right|) \frac{\rr-\rr'}{|\rr-\rr'|}d\rr'
\end{align}
\begin{align}
  \mathbf{n}_{V1} = \frac{\mathbf{n}_{V2}}{2\pi \sigma}, \quad
  n_1 &= \frac{n_2}{2\pi \sigma} , \quad
  n_0 = \frac{n_2}{\pi \sigma^2} \label{eq:FMrest}
\end{align}


\section{Theoretical Approaches}

The traditional approach for solving for the pair distribution
function of an inhomogeneous density distribution is to use an
integral equation theory...

\section{Comparison with simulation}\label{sec:comparison}

\section{Conclusion}

\appendix

\section*{Appendix}
\cite{jin2011perturbative} Combines hard sphere (white bear) with
attraction that is not mean field but rather uses two other
correltation expressions given in -\cite{tang2008accurate} and
\cite{hlushak2009direct}

Tang \cite{tang2008accurate}Derives the direct correltation for a SW
and Yukawa and LJ potentials by first getting the radial distribution
function through a method I don't really understand.  Composes fourier
transforms $c(k)$ and $h(k)$ of these complicated functions that
depend on filling fraction and the potential in an integral. Uses
first order mean spherical approximation.  This is just square
potential well around particle but the radial distribution function
takes only the first order in an expansion in... (high temp?).

\cite{hlushak2009direct} Cites \cite{tang2008accurate} (above) and
does largely the same thing but then adds a complication for $2 <
\lambda \leq 3$

\cite{levesque2012scalar} Uses direct correlation funtion that's
gotten from a radial distribution function of a homogeneous density.
The direct correlation itself is only dependent on distance.  The
$\rho_0$ that they use is the same $\rho_0$ that the used in the
reference HS term in their functional.  The use the Carnahan Starling
or the Perkis Yevick for these hard sphere free energies.  The other
terms involve $\delta\rho$ so the change is a variation from the
reference density.  They further modify it by looking at molecular
dynamic simulations of a certain model of water molecules and looking
at the oxygen-oxygen correlation.





\end{document}


%% G{\"o}tzelmann \emph{et al.} worked out the pair distribution function near a hard
%% wall by finding the direct correlation function by taking a functional
%% derivative of the free energy functional and then solving the
%% Ornstein-Zernicke equation~\cite{gotzelmann1996structure}.  They have plots like our pretty plots,
%% but with the fixed sphere one radius away from touching.
%% Interestingly, they plot the second bit as $z$, which enables them to
%% have just two axis values.  I think I prefer our ``total distance''
%% method, but there is a certain elegance to just plotting $z$ for the
%% curvy part as well as the straight-away part.  They found poor
%% quantitative agreement, particularly at contact.  They also studied
%% the pair correlation function in a hard slit~\cite{gotzelmann1997density}.

%% Integrating the Ornstein-Zernicke equation is challenging, requiring a
%% variational minimization itself~\cite{paul2003variational}.

%% Attard worked out the triplet correlation function, which is something
%% that we can access from the pair distribution function, if we consider
%% the pair correlation near a single hard sphere
%% solute\cite{attard1989spherically}.  Gonz\'alez \emph{et al.} wrote an
%% interesting paper using DFT (test-particle approach of Percus) and
%% Monte Carlo to compute the three-particle distribution function (same
%% as the triplet correlation function)\cite{gonzalez1999test}.  Their
%% theory compares favorably with the ``superposition approximation'',
%% and we could probably do something similar.

%% Plischke and Henderson worked out the pair correlation function near a
%% hard wall using Percus-Yevick, and compare with Monte Carlo
%% results\cite{plischke1986pair}.  They plot the pair correlation
%% function along interesting paths.  \fixme{Jeff read this paper for a
%%   while and does not understand what they do in oder to find the pair
%%   correlation function as compared to the radial distribution function}

%% Lado recently introduced a new and improved efficient algorithm for
%% implementing integral equation theory for inhomogeneous fluids, which
%% computes $g^{(2)}(\rr_1,\rr_2)$ among other
%% things\cite{lado2009efficient}.  Lado's work looks pretty awkward to
%% use, and their example is limited to 1D systems.  And the scaling is
%% still (so far as I can tell) $O(N^2)$, since their improvement is in
%% the prefactor, by developing a nice basis set.

%% Gloor \emph{et al} explain that little is known of the pair
%% distribution function of the hard-sphere fluid, and therefore use the
%% formula
%% \begin{align}
%%   g^{(2)}(\rr_1,\rr_2) \approx g\left(|\rr_1-\rr_2|, \frac{\rho(\rr_1)+\rho(\rr_2)}2\right)
%% \end{align}
%% They go on to further approximate the radial distribution function,
%% but we can just stop here.

%% Gloor et al \cite{gloor2007prediction} uses average density of two points:
%% \begin{align}
%% \overline{\rho} &\equiv \frac{\rho(\rr)+\rho(\rr')}{2}
%% \end{align}
%% and the uses bulk radial distribution for the reference system (hard
%% spheres) at this density, in order to approximate pair correlation.



%% \section{What has been done before?}
%% \subsection{Gloor simple method, problematic}
%% Gil-Villegas\cite{gil1997statistical} approximates the first order
%% inverse tenpurature Helmholtz Free energy term, which involves an
%% integral of a potential multiplied by a radial distribution function
%% $g^{HS}(r,\eta)$, by first taking the distribution function out of the
%% integral and evaluating it at the correct mean distance $\zeta$,
%% according to the mean value theorem.  They then approximate the pair
%% correlation function $g^{HS}(\zeta,\eta)$ with the correlation at
%% contact and at a effective $\eta_{eff}$.  The $\eta_{eff}$ is found by
%% fitting a general function form to correct energy solutions $a_1$ with
%% depenedent variable $\eta$ and


%% \subsection{Fischer accurate but slow}
