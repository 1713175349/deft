\documentclass[letterpaper,twocolumn,amsmath,amssymb,pre]{revtex4-1}
\usepackage{graphicx}% Include figure files
\usepackage{dcolumn}% Align table columns on decimal point
\usepackage{bm}% bold math
\usepackage{color}
\usepackage{breqn}

\newcommand{\red}[1]{{\bf \color{red} #1}}
\newcommand{\blue}[1]{{\bf \color{blue} #1}}
\newcommand{\green}[1]{{\bf \color{green} #1}}
\newcommand{\rr}{\textbf{r}}
\newcommand{\refnote}{\red{[ref]}}

\newcommand{\fixme}[1]{\red{[#1]}}

%\newcommand{\derivation}[1]{#1} % Use this to show all derivations in detail
\newcommand{\derivation}[1]{} % Use this for nice pegagogical paper...

% needsworklater is used to annotate bits that need work, but that we
% can postpone for a while.
\newcommand{\needsworklater}[1]{\emph{[#1]}}
% needsworknow is intended to prioritize stuff that needs fixing.
\newcommand{\needsworknow}[1]{\textcolor{red}{[\emph{#1}]}}

\begin{document}
\title{Using Fundamental Measure Theory to treat the two-point pair
  distribution function of the Inhomogeneous Hard-Sphere Fluid}

\author{Paho}
\author{???}
\author{David Roundy}
\affiliation{Department of Physics, Oregon State University, Corvallis, OR 97331}

%\pacs{61.20.Ne, 61.20.Gy, 61.20.Ja}
%%%%%%%%%%%%%%%%%%%%%%%%%%%%%%%%%%%%%%%%%%%%%%%%%%%%%%%%%%%%
\begin{abstract}
  We develop and test an efficient approximation for the two-point
  pair distribution function of an inhomogeneous hard-sphere fluid.
\end{abstract}

%\maketitle

%%%%%%%%%%%%%%%%%%%%%%%%%%%%%%%%%%%%%%%%%%%%%%%%%%%%%%%%%%%%
\section{Introduction}

Attard worked out the triplet correlation function, which is something
that we can access from the pair distribution function, if we consider
the pair correlation near a single hard sphere
solute\cite{attard1989spherically}.  Gonz\'alez \emph{et al} wrote an
interesting paper using DFT and Monte Carlo to compute the
three-particle distribution function (same as the triplet correlation
function)\cite{gonzalez1999test}.  Their theory compares favorably
with the ``superposition approximation'', and we could probably do
something similar.

Plischke and Henderson worked out the pair correlation function near a
hard wall using Percus-Yevick, and compare with Monte Carlo
results\cite{plischke1986pair}.  They plot the pair correlation
function along interesting paths.

Lado recently introduced a new and improved efficient algorithm for
implementing integral equation theory for inhomogeneous fluids, which
computes $g^{(2)}(\rr_1,\rr_2)$ among other
things\cite{lado2009efficient}.  Lado's work looks pretty awkward to
use, and their example is limited to 1D systems.  And the scaling is
still (so far as I can tell) $O(N^2)$, since their improvement is in
the prefactor, by developing a nice basis set.

\section{Pair distribution function with inhomogeneity}

The pair distribution function can be defined using the two particle
density as
\begin{align}
  g^{(2)}(\rr_1,\rr_2) &\equiv \frac{n^{(2)}(\rr_1,\rr_2)}{n(\rr_1)n(\rr_2)}
\end{align}

We may also want to consider the \emph{triplet} distribution function
\begin{align}
  g^{(3)}(\rr_1,\rr_2,\rr_3) &\equiv \frac{n^{(3)}(\rr_1,\rr_2,\rr_3)}{n(\rr_1)n(\rr_2)n(\rr_3)}
\end{align}
The triplet distribution is relevant, because we can compute the
triplet distribution of the homogeneous system by considering the
\emph{pair} distribution of the inhomogeneous ``test-particle
scenario.''  The process is that we solve the inhomogeneous situation
in which we have one hard sphere located at the origin.  We solve for
the density around that hard sphere, and this gives us the pair
distribution function of the homogeneous system $g^{(2)}(0, \rr)$.
When we solve for the pair distribution $g^{(2)}(\rr_1,\rr_2)$ of the
inhomogeneous system, this will give us the \emph{triplet}
distribution function of the homogeneous system
$g^{(3)}(0,\rr_1,\rr_2)$.

\section{Goals of this paper (to be integrated into rest of paper)}


In this paper, we seek an efficient approximation for the pair
correlation function of an inhomogeneous hard-sphere fluid, where we
define efficient as an algorithm that scales as $O(N\log N)$ where $N$
scales as the volume of the system under consideration.
Integral equation theory (look this up) scales as $O(N^2)$, which can
be very expensive for large three-dimensional systems.

Fundamental-Measure Theory (FMT) is constructed using only single-point
convolutions (check terminology) such as
\begin{align}
  n_\alpha(\rr) &= \int n(\rr')w_\alpha(|\rr-\rr'|) d\rr'
\end{align}
These convolutions can be computed in Fourier space in $O(N\log N)$
time, with the result that the FMT free energy---and its
gradient---can be computed in $O(N\log N)$ time.

The pair distribution function is used in computing the first term in
the high-temperature perturbation expansion, which is used in SAFT
(for instance) to treat dispersion interactions.  This term is written
as
\begin{align}\label{eq:a1}
  a_1 &\propto \iint n(\rr)n(\rr') \Phi(|\rr-\rr'|)
  g^{(2)}(\rr,\rr')d\rr d\rr'
\end{align}
where $\Phi(r)$ is the pair potential.  If $g^{(2)}(\rr,\rr')$ could
be written as $g(|\rr-\rr'|)$, this term could be computed in $O(N\log
N)$ time.  However, this is not possible when the density is
inhomogeneous, because the pair distribution function depends on the
density.

In this paper, we consider a superposition approximation, in which
construct an approximate representation for the pair distribution
function, which allows the first term in the high temperature
expansion given in Equation~\ref{eq:a1} to be computed in $O(N\log N)$
time.  One input to this approach is our recently published expression
for an average of the value of the pair distribution function at
contact $g_\sigma(\rr)$, which itself can be computed in $O(N\log N)$
time.

The general idea is to write
\begin{dmath}
  g^{(2)}(\rr_1,\rr_2) = \Theta(r_{12}-\sigma)
  \left(1 + \sum_{\alpha} \frac{f_{\alpha}(g_\sigma(\rr_1)) + f_{\alpha}(g_\sigma(\rr_2))}{2}g_{\alpha}(r_{12})\right)\label{eq:gpair-approx}
\end{dmath}
where $f_{\alpha}$ and $g_{\alpha}$ are a set of functions that
satisfy the following constraint:
\begin{dmath}\label{eq:definealpha}
  \sum_{\alpha} f_{\alpha}(g_\sigma(\eta)) g_{\alpha}(r) \approx g(r; \eta)
\end{dmath}
which ensures that our pair distribution function reduces
(approximately) to the radial distribution function in the homogeneous
limit.  Equation~\ref{eq:gpair-approx} is constructed such that
Equation~\ref{eq:a1} can be evaluated in $O(N\log N)$ time, since all
the required convolutions may be performed with a fixed kernel.

Our proposed pair distribution function has several good properties.
When evaluated at contact ($r_{12}=\sigma$), it gives the same good
result as our previously-published functional for the averaged pair
distribution function at contact, which is constructed to give an
accurate prediction for the contribution to Equation~\ref{eq:a1} from
spheres in contact.  In the homogeneous limit,
Equation~\ref{eq:gpqair-approx} should give a prediction that is as
accurate as the approximate equality in
Equation~\ref{eq:definealpha}.  In inhomogeneous distributions and at
intermediate distances, we expect our pair distribution function to be
least accurate, but one hopes that it will be reasonable, simply
because it is an average of two reasonable results.


\subsection*{Fundamental-Measure Theory}

We use the White Bear version of the Fundamental-Measure Theory~(FMT)
functional~\cite{roth2002whitebear}, which describes the excess free
energy of a hard-sphere fluid.  The White Bear functional reduces to
the Carnahan-Starling equation of state for homogeneous systems.  It
is written as an integral over all space of a local function of a set
of ``fundamental measures'' $n_\alpha(\rr)$, each of which is written
as a one-center convolution of the density.  The White Bear free
energy is thus
\begin{equation}
A_{\textit{HS}}[n] = k_B T \int \left(\Phi_1(\rr) + \Phi_2(\rr) + \Phi_3(\rr)\right) d\rr \; ,
\end{equation}
with integrands
\begin{align}
\Phi_1 &= -n_0 \ln\left( 1 - n_3\right) \label{eq:Phi1}\\
\Phi_2 &= \frac{n_1 n_2 - \mathbf{n}_{V1} \cdot\mathbf{n}_{V2}}{1-n_3} \\
\Phi_3 &= (n_2^3 - 3 n_2 \mathbf{n}_{V2} \cdot \mathbf{n}_{V2}) \frac{
  n_3 + (1-n_3)^2 \ln(1-n_3)
}{
  36\pi n_3^2\left( 1 - n_3 \right)^2
} , \label{eq:Phi3}
\end{align}
using the fundamental measures
\begin{align}
  n_3(\rr) &= \int n(\rr') \Theta(\sigma/2 -\left|\rr - \rr'\right|)
  d\rr' \label{eq:FMn3} \\
  n_2(\rr) &= \int n(\rr') \delta(\sigma/2 -\left|\rr - \rr'\right|) d\rr' \\
  \mathbf{n}_{2V}(\rr) &= \int n(\rr') \delta(\sigma/2 -\left|\rr - \rr'\right|) \frac{\rr-\rr'}{|\rr-\rr'|}d\rr'
\end{align}
\begin{align}
  \mathbf{n}_{V1} = \frac{\mathbf{n}_{V2}}{2\pi \sigma}, \quad
  n_1 &= \frac{n_2}{2\pi \sigma} , \quad
  n_0 = \frac{n_2}{\pi \sigma^2} \label{eq:FMrest}
\end{align}


\section{Theoretical Approaches}

The traditional approach for solving for the pair distribution
function of an inhomogeneous density distribution is to use an
integral equation theory...

\section{Comparison with simulation}\label{sec:comparison}

\section{Conclusion}

\appendix

\section*{Appendix}

\bibliography{paper}% Produces the bibliography via BibTeX.

\end{document}
