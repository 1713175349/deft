\documentclass[letterpaper,twocolumn,amsmath,amssymb,pre]{revtex4-1}
\usepackage{graphicx}% Include figure files
\usepackage{dcolumn}% Align table columns on decimal point
\usepackage{bm}% bold math
\usepackage{color}

\newcommand{\red}[1]{{\bf \color{red} #1}}
\newcommand{\blue}[1]{{\bf \color{blue} #1}}
\newcommand{\green}[1]{{\bf \color{green} #1}}
\newcommand{\rr}{\textbf{r}}
\newcommand{\refnote}{\red{[ref]}}

\newcommand{\fixme}[1]{\red{[#1]}}

%\newcommand{\derivation}[1]{#1} % Use this to show all derivations in detail
\newcommand{\derivation}[1]{} % Use this for nice pegagogical paper...

% needsworklater is used to annotate bits that need work, but that we
% can postpone for a while.
\newcommand{\needsworklater}[1]{\emph{[#1]}}
% needsworknow is intended to prioritize stuff that needs fixing.
\newcommand{\needsworknow}[1]{\textcolor{red}{[\emph{#1}]}}

\begin{document}
\title{Using Fundamental Measure Theory to treat the two-point pair
  distribution function of the Inhomogeneous Hard-Sphere Fluid}

\author{Paho}
\author{???}
\author{David Roundy}
\affiliation{Department of Physics, Oregon State University, Corvallis, OR 97331}

%\pacs{61.20.Ne, 61.20.Gy, 61.20.Ja}
%%%%%%%%%%%%%%%%%%%%%%%%%%%%%%%%%%%%%%%%%%%%%%%%%%%%%%%%%%%%
\begin{abstract}
  We develop and test an efficient approximation for the two-point
  pair distribution function of an inhomogeneous hard-sphere fluid.
\end{abstract}

\maketitle

%%%%%%%%%%%%%%%%%%%%%%%%%%%%%%%%%%%%%%%%%%%%%%%%%%%%%%%%%%%%
\section{Introduction}

Attard worked out the triplet correlation function, which is something
that we can access from the pair distribution function, if we consider
the pair correlation near a single hard sphere
solute\cite{attard1989spherically}.  Gonz\'alez \emph{et al} wrote an
interesting paper using DFT and Monte Carlo to compute the
three-particle distribution function (same as the triplet correlation
function)\cite{gonzalez1999test}.  Their theory compares favorably
with the ``superposition approximation'', and we could probably do
something similar.

Plischke and Henderson worked out the pair correlation function near a
hard wall using Percus-Yevick, and compare with Monte Carlo
results\cite{plischke1986pair}.  They plot the pair correlation
function along interesting paths.

\section{Pair distribution function with inhomogeneity}

\subsection*{Fundamental-Measure Theory}

We use the White Bear version of the Fundamental-Measure Theory~(FMT)
functional~\cite{roth2002whitebear}, which describes the excess free
energy of a hard-sphere fluid.  The White Bear functional reduces to
the Carnahan-Starling equation of state for homogeneous systems.  It
is written as an integral over all space of a local function of a set
of ``fundamental measures'' $n_\alpha(\rr)$, each of which is written
as a one-center convolution of the density.  The White Bear free
energy is thus
\begin{equation}
A_\textit{HS}[n] = k_B T \int \left(\Phi_1(\rr) + \Phi_2(\rr) + \Phi_3(\rr)\right) d\rr \; ,
\end{equation}
with integrands
\begin{align}
\Phi_1 &= -n_0 \ln\left( 1 - n_3\right) \label{eq:Phi1}\\
\Phi_2 &= \frac{n_1 n_2 - \mathbf{n}_{V1} \cdot\mathbf{n}_{V2}}{1-n_3} \\
\Phi_3 &= (n_2^3 - 3 n_2 \mathbf{n}_{V2} \cdot \mathbf{n}_{V2}) \frac{
  n_3 + (1-n_3)^2 \ln(1-n_3)
}{
  36\pi n_3^2\left( 1 - n_3 \right)^2
} , \label{eq:Phi3}
\end{align}
using the fundamental measures
\begin{align}
  n_3(\rr) &= \int n(\rr') \Theta(\sigma/2 -\left|\rr - \rr'\right|)
  d\rr' \label{eq:FMn3} \\
  n_2(\rr) &= \int n(\rr') \delta(\sigma/2 -\left|\rr - \rr'\right|) d\rr' \\
  \mathbf{n}_{2V}(\rr) &= \int n(\rr') \delta(\sigma/2 -\left|\rr - \rr'\right|) \frac{\rr-\rr'}{|\rr-\rr'|}d\rr'
\end{align}
\begin{align}
  \mathbf{n}_{V1} = \frac{\mathbf{n}_{V2}}{2\pi \sigma}, \quad
  n_1 &= \frac{n_2}{2\pi \sigma} , \quad
  n_0 = \frac{n_2}{\pi \sigma^2} \label{eq:FMrest}
\end{align}


\section{Theoretical Approaches}

\section{Comparison with simulation}\label{sec:comparison}

\section{Conclusion}

\appendix

\section*{Appendix}

\bibliography{paper}% Produces the bibliography via BibTeX.

\end{document}
