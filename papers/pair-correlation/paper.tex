\documentclass[letterpaper,twocolumn,amsmath,amssymb,pre,aps,10pt]{revtex4-1}
\usepackage{graphicx}% Include figure files
\usepackage{dcolumn}% Align table columns on decimal point
\usepackage{bm}% bold math
\usepackage{color}
\usepackage{subcaption} % provides subfigure

\newcommand{\red}[1]{{\bf \color{red} #1}}
\newcommand{\green}[1]{{\bf \color{green} #1}}
\newcommand{\rr}{\textbf{r}}
\newcommand{\refnote}{\red{[ref]}}

\newcommand{\fixme}[1]{\red{[#1]}}

%\newcommand{\derivation}[1]{#1} % Use this to show all derivations in detail
\newcommand{\derivation}[1]{} % Use this for nice pegagogical paper...
\newcommand{\davidsays}[1]{{\color{red} [\green{David:} \emph{#1}]}}
\newcommand{\jeffsays}[1]{{\color{red} [\green{Jeff:} \emph{#1}]}}
\newcommand{\pahosays}[1]{{\color{red} [\green{Paho:} \emph{#1}]}}

\begin{document}
\title{The Pair Distribution Function of the Inhomogeneous Hard-Sphere
  Fluid}

\author{Paho Lurie-Gregg}
\author{Jeff B. Schulte}
\author{David Roundy}
\affiliation{Department of Physics, Oregon State University, Corvallis, OR 97331}

%\pacs{61.20.Ne, 61.20.Gy, 61.20.Ja}
%%%%%%%%%%%%%%%%%%%%%%%%%%%%%%%%%%%%%%%%%%%%%%%%%%%%%%%%%%%%
\begin{abstract}
  We have developed an approximation for the pair distribution
  function of hard spheres, which is based on the average value of the
  pair distribution function at contact we recently developed.  Our
  pair distribution function is constructed so as to be computable
  with only single-site convolutions and to accurately reproduce the
  averaged distribution function at contact.  This results in a
  functional that is efficient to compute, and predicts the pair
  distribution function as well as the most accurate existing
  analytical approximation.  \fixme{needs special attention.}
\end{abstract}

\maketitle

\newcommand\saftlocaldft{}
% The following are papers that use a SAFT-based classical DFT with
% all the terms that should be non-local being non-local.
\newcommand\saftnonlocaldft{}

The standard approach in liquid state theory is to treat a liquid as a
hard-sphere reference fluid with attractive interactions that are
treated perturbatively~\cite{hansen2006theory}.  Recent advances have
extended these perturbative approaches to inhomogeneous density
distributions---that is, liquid interfaces---through the use of
classical density functional theory (DFT), in which the grand free
energy is found by minimizing a free energy functional of the
density~\cite{ jain2007modified, gloor2007prediction,
  gross2009density,
  %felipe2001examination, gloor2002saft,
  %gloor2004accurate, clark2006developing,
  kahl2008modified,
  % yu2002fmt-dft-inhomogeneous-associating,
  %fu2005vapor-liquid-dft, bryk2006density,
  hughes2013classical,
  %segura1998comparison, felipe2001examination,
  %gloor2002saft, gloor2004accurate, fu2005vapor-liquid-dft,
  bryk2006density, clark2006developing,
  kahl2008modified, gross2009density,
  %yu2002fmt-dft-inhomogeneous-associating,
  sundararaman2012computationally, marshall2013density}.  The
perturbation theory treatment of intermolecular interactions relies on
the pair distribution function of the reference fluid:
$g^{(2)}(\rr_1,\rr_2)$.  Unlike the radial distribution function of a
homogeneous fluid, there does not currently exist a tractable form for
the pair distribution function of an inhomogeneous hard-sphere fluid,
suitable for use in constructing a density
functional~\cite{gloor2007prediction, jain2007modified}.

At its core, thermodynamic perturbation theory---sometimes referred to
as the cluster, or high-temperature expansion---is an expansion of the
free energy in powers of a small parameter, which typically is a
pairwise attractive interaction:
\begin{align}
  F &= F_0 + F_1 + \beta F_2 + \mathcal{O}(\beta^2)
\end{align}
where the terms $F_n$ are corrections to the free energy at order $n$
in the small interaction.  The first and largest term in this
expansion is
\begin{align}
  F_1 &= \tfrac12 \iint
  g^{(2)}_{HS}(\rr_1,\rr_2)n(\rr_1)n(\rr_2)\Phi(|\rr_1-\rr_2|)
  d\rr_1d\rr_2
  \label{eq:mean-field}
\end{align}
where $g^{(2)}_{HS}(\rr_1,\rr_2)$ is the pair distribution function of
the hard-sphere reference fluid, and $\Phi(r)$ is the pair potential.
In this paper, we will introduce an approximation for the hard-sphere
pair distribution function which is suitable for use in the creation
of classical density functionals based on a cluster expansion.

\section{Theoretical approaches}

The classic (and earliest) approach for computing the pair
distribution function is to use Percus' trick of treating one sphere
as an external field, and using the resultant equilibrium density to
find the pair distribution function~\cite{hansen2006theory}.  This
elegant approach lends itself to DFT, and can be used to compute and
plot the pair distribution function, but requires a full free-energy
minimization \emph{for each position} $\rr_1$ in
$g^{(2)}(\rr_1,\rr_2)$, and hence would be prohibitively expensive as
a tool in constructing a free energy functional.

The canonical inhomogeneous configuration for the hard-sphere fluid is
the system consisting of a hard sphere at a hard wall.  In 1986,
Plischke and Henderson solved the pair distribution function of this
system using integral equation theory under the Percus-Yevick
approximation~\cite{plischke1986pair}.  Ten years later,
G{\"o}tzelmann \emph{et al.} found the pair distribution function near
a hard wall by solving the Ornstein-Zernicke equation using the direct
correlation function found from a functional derivative of a classical
DFT free energy functional~\cite{gotzelmann1996structure}.
Unfortunately, solving the inhomogeneous Ornstein-Zernicke equation is
computationally challenging, although more efficient approximate
algorithms have been developed~\cite{paul2003variational}.

Another inhomogeneous configuration that is of interest is the
test-particle configuration, in which one hard sphere is fixed in the
otherwise homogeneous fluid.  Where the hard-wall is a surface with no
curvature, the test-particle configuration has a surface with
curvature at the molecular length scale.  In this case, the density
gives the pair distribution function---this is just Percus'
trick---and the pair distribution function gives the triplet
distribution function.  The triplet distribution function has been computed
by Gonz\'alez \emph{et al.} using the test-particle approach with
\emph{two} spheres fixed~\cite{gonzalez1999test}.

Lado recently introduced a new and improved efficient algorithm for
implementing integral equation theory for inhomogeneous fluids, which
computes $g^{(2)}(\rr_1,\rr_2)$~\cite{lado2009efficient}.  While this
approach is two orders of magnitude more efficient than previous
implementations, it remains a computationally expensive approach, and
unsuitable for repeated evaluation within a free-energy minimization.

\section{Approximations to the pair distribution function}
There are several analytic approximations for the inhomogeneous pair
distribution function, which extend the radial distribution function
to inhomogeneous scenarios.  These works differ in what density is
used when evaluating the radial distribution function $g(r,\eta)$.

Several recent works have assumed a mean-density approximation for
the pair distribution function:
\begin{align}
  g^{(2)}(\rr_1,\rr_2) \approx g\left(r_{12}, \tfrac{\pi\sigma^3}{6}\tfrac12
  (n(\rr_1)+n(\rr_2))\right)
\end{align}
where $g(r,\eta)$ is the radial distribution function as a function of
radius and packing fraction $\eta$, and $\sigma$ is the hard-sphere
diameter~\cite{gloor2007prediction, gross2009density}.  This approach,
however, cannot be applied to highly inhomogeneous systems such as a
dense fluid at a solid surface.  These systems exhibit large density
oscillations that lead to mean packing fractions greater than unity,
which cannot occur in the bulk reference system that defines $g(r,
\eta)$.  The above papers restricted themselves to the liquid-vapor
interface, which does not exhibit this pathology.

Non-pathological approaches use an average of the density over some
volume. Fischer and Methfessel~\cite{fischer1980born} introduce the
approximation
\begin{align}
  g^{(2)}(\rr_1,\rr_2) \approx g\left(r_{12}, n_3\left(\tfrac12
  (\rr_1+\rr_2)\right)\right)
  \label{eq:fischer}
\end{align}
where $n_3$ is one of the fundamental measures defined in Fundamental
Measure Theory~\cite{rosenfeld1989free}, which is an integral of the
density over a spherical volume:
\begin{align}
  n_3(\rr) = \int n(\rr')\Theta(\tfrac12 \sigma - |\rr-\rr'|) d\rr'
\end{align}
where $\sigma$ is the hard-sphere diameter.  Equation~\ref{eq:fischer}
is computationally awkward, because it treats as special the midpoint
$\tfrac12(\rr_1+\rr_2)$.  Moreover, the approach of Fischer and
Methfessel is intended to approximate the pair distribution function only
at contact, when the distance between $\rr_1$ and $\rr_2$ is the hard-sphere
diameter.

Sokolowski and Fischer addressed the shortcomings of the theory of
Fischer and Methfessel by modifying this approach to use
averages, centered on the two points $\rr_1$ and $\rr_2$:
\begin{align}
  g^{(2)}(\rr_1,\rr_2) \approx g\left(r_{12},
  \tfrac12(\bar{n}(\rr_1)+\bar{n}(\rr_2))\right)
  \label{eq:sokolowski}
\end{align}
where
\begin{align}
  \bar{n}(\rr) = \frac{3}{4\pi (0.8\sigma)^3}\int n(\rr')\Theta(0.8\sigma - |\rr-\rr'|) d\rr'
\end{align}
is the density averaged over a sphere with diameter
$0.8\sigma$~\cite{sokolowski1992role}.  The value 0.8 in this formula
was arrived at by fitting to Monte Carlo simulation.  Although
Eq.~\ref{eq:sokolowski} has the advantages of only involving density
averages at the points at which the pair distribution function is
desired, it cannot be written as a single-site convolution, since the
convolution kernel depends on both points.

In a previous paper~\cite{schulte2012using}, we have introduced a
functional that gives a good approximation for the pair distribution
function averaged over positions $\rr_2$ that are in contact with
$\rr_1$, defined as:
\begin{align}
  g_\sigma(\rr_1) &\equiv \frac{ \int g^{(2)}(\rr_1,\rr_2) \delta(\sigma -|\rr_1-\rr_2|)n(\rr_2)
    d\rr_2 }{ \tilde{n}(\rr_1)  }
\end{align}
where the averaged density $\tilde{n}(\rr_1)$ is defined by
\begin{align}
  \tilde{n}(\rr_1) &\equiv \int \delta(\sigma -|\rr_1 - \rr_2|)n(\rr_2) d\rr_2.
\end{align}
We use the contact-value theorem to derive the exact formula:
\begin{align}
  g_\sigma(\rr)% &= \frac{\ncontact(\rr)}{n_A(\rr)} \\
  &= \frac12 \frac{1}{k_BT n(\rr) \tilde{n}(\rr)} \frac{\delta
    A}{\delta \sigma(\mathbf{r})} \label{eq:gsigma-exact}
\end{align}
where $\sigma(\rr)$ is the diameter of hard spheres located at
position $\rr$, and $A$ is the Helmholtz free energy of the
hard-sphere fluid.  Our approximation is obtained by using the White
Bear functional for Helmholtz free energy of the hard-sphere
fluid~\cite{roth2002whitebear} in Eq.~\ref{eq:gsigma-exact}.  This
provides an excellent approximation for this averaged value of the
pair distribution function for a variety of interfaces, and over a
wide range of densities.

\section{Our approach}
%% We approximate the pair correlation with the average of two radial
%% distribution functions, evaluated at the distance between the two
%% points, that are themselves functions of the averaged pair correlation
%% at contact $g_{\sigma}(\rr)$ discussed above~\cite{schulte2012using},
%% evaluated at the two points.
In the approaches for the pair distribution function mentioned above,
the radial distribution function used to approximate pair correlation
was dependent upon the density averaged over some volume.  We seek to
achieve greater accuracy by using instead a function dependent upon
our averaged $g_{\sigma}(\rr)$ discussed above, which holds more
information about an inhomogeneous system than does a simple
convolution of the density.
%
We approximate the pair correlation with the average of two radial
distribution functions, evaluated at the distance between the two
points, that are themselves functions of the averaged pair correlation
at contact $g_{\sigma}(\rr)$ evaluated at the two points:
%
%% We therefore seek a function form
%% for our radial distribution that is dependent upon $g_{\sigma}(\rr)$
%% and that will approach the correct solution in the homogeneous limit
%% and also in the limit of contact distance between the two points.
\begin{align}
  g^{(2)}(\rr_1,\rr_2) = \frac{g(|r_{12}|, g_\sigma(\rr_1)) +
    g(|r_{12}|, g_\sigma(\rr_2))}{2}. \label{eq:g2-our-mean}
\end{align}
$g^{(2)}(\rr_1,\rr_2)$ above is constructed to reproduce the exact
value for the integral$ \int n(\rr_1)n(\rr_2)
g^{(2)}(\rr_1,\rr_2)\delta(|\rr_1-\rr_2|-\sigma)d\rr_1d\rr_2$.  \par
The approximation described in Eq.~\ref{eq:g2-our-mean} requires the
radial distribution function as a function of $r$ and $g_\sigma$.  One
approach to constructing $g(r,g_\sigma)$ is to compute $\eta$ from
$g_\sigma$ and then find $g(r,\eta)$ either using some approximate
functional form (e.g. Percus-Yevick) or using Monte Carlo simulation
results.  In this paper we first use the Monte Carlo results (which
are also used in evaluating the approximations of earlier works), and
refer to this approximation (which has an exact $g(r,g_\sigma)$) as
``MCRD.''  \jeffsays{I thought maybe for Monte Carlo Radial
  Distribution} We will then construct our own, more computationally
efficient radial distribution approximation, which is addressed in the
following section.

\newcommand\kappaO{\kappa_0}
\newcommand\kappaI{\kappa_1}
\newcommand\kappaZ{\kappa_2}

\input{figs/fit-parameters}

\subsection{Separable fit to the radial distribution function}

The approximation described in Eq.~\ref{eq:g2-our-mean} is reasonably
accurate, but cannot be computed efficiently with single-center
convolutions as written.  In order to ease computational application
of this pair distribution function in classical density functionals,
we have developed a separable appoximation for the function
$g(r,g_\sigma)$.
\begin{align}
  g(|r_{12}|, g_\sigma) &= \sum_{i} a_i(r_{12}) b_i(g_\sigma)
\end{align}
This separable approximation allows the mean field contribution to
the free energy (Eq.~\ref{eq:mean-field}) to be computed with only
single-center convolutions, which can be efficiently computed using
fast Fourier transforms.

%% We construct our function's form in terms of an expansion of powers of
%% $h_{\sigma} = g_{\sigma}-1$, ensuring that it approaches $g_{\sigma}$
%% in the contact distance limit, leaving our function with tunable
%% constant perameters.  We run a simple monte-carlo simulation to
%% calculate values of the homogeneous radial distribution function at
%% varying distance perameter $r$.  We then tune our function's constant
%% parameters by matching our function's results against the monte-carlo
%% data, using a least squares fit method.  During this process it is
%% simplest to use an analytic expression for $g_{\sigma}$ in our
%% function and so we use the well known and very successful Carnahan
%% Starling equation to calculate $g_{\sigma}$.  This is appropriate
%% since our published $g_{\sigma}$ approaches the Carnahan Starling
%% version in the homogeneous limit and it's this limit, along with the
%% contact limit, that we would like our constructed radial distribution
%% function to get right.  After this process we have an appropriate
%% radial distribution function that is dependent only on $r$ and
%% $g_{\sigma}$.

%% The function has the form:
%% \begin{multline}
%%   g(r,g_\sigma) = 1 + h_\sigma e^{-\kappaO \left(\frac{r}{\sigma}-1\right)} \\
%%   + h_\sigma(\kappaO - 2g_\sigma)  \left(\frac{r}{\sigma}-1\right)e^{-\kappaI  \left(\frac{r}{\sigma}-1\right)} \\
%%   + I(g_\sigma)  \left(\frac{r}{\sigma}-1\right)^2e^{-\kappaZ  \left(\frac{r}{\sigma}-1\right)}
%% \end{multline}
%% where $\kappaO$,$\kappaI$, and $\kappaZ$ are fitted constants and
%% $I(g_\sigma)$ is a function of $g_\sigma$.

%% The finalized form of our pair correlation function is:
%% \begin{align}
%%   g^{(2)}(\rr_1,\rr_2) = \frac{g(|r_{12}|, g_\sigma(\rr_1)) + g(|r_{12}|, g_\sigma(\rr_2))}{2}
%% \end{align}
\begin{figure}
  \centering
  \includegraphics[width=\columnwidth]{figs/ghs-g2}% \\
  %\includegraphics[width=\columnwidth]{figs/ghs-g-ghs}
  \caption{Plot of the hard-sphere radial distribution function, with
    our fit.}\label{fig:radial-distribution}
\end{figure}

We construct $g(r, g_\sigma)$ to satisfy several constraints.  Naturally,
\begin{equation}
  g(\sigma, g_\sigma) = g_\sigma.
\end{equation}
A second constraint applies to the integral of the total correlation
function
\begin{align}
  1 + n\int h(r)d\rr &= nkT\chi_T \label{eq:total-constraint}
\end{align}
where $h(r) = g(r) - 1$ is the total correlation function.
A final constraint is that the slope of the radial distribution
function at contact is given to a very good approximation by
\begin{align}
  g'(\sigma) \approx - g_\sigma (g_\sigma - 1).
\end{align}
Under these three constraints, we applied a least-squares fit to
simulation data for the radial distribution function at packing
fractions from 0.05 to 0.4 by steps of 0.05.  Our fitted radial
distribution function is given by
\begin{multline}
  g(r,g_\sigma) = 1 + h_\sigma e^{-\kappaO \left(\frac{r}{\sigma}-1\right)} \\
  + h_\sigma(\kappaO - 2g_\sigma)  \left(\frac{r}{\sigma}-1\right)e^{-\kappaI  \left(\frac{r}{\sigma}-1\right)} \\
  + I(g_\sigma)  \left(\frac{r}{\sigma}-1\right)^2e^{-\kappaZ  \left(\frac{r}{\sigma}-1\right)}
\end{multline}
\begin{equation}
  I(g_\sigma) = \kappaZ^5 \frac{
    \frac{\chi-1}{24\eta} - h_\sigma \frac{(\kappaO-2
      g_\sigma)(\kappaI^2 + 4 \kappaI + 6)}{\kappaI^4}
    - h_\sigma\frac{\kappaO^2 + 2\kappaO + 2}{\kappaO}
  }{2 \kappaZ^2 + 12 \kappaZ + 24}
\end{equation}
\begin{equation}
  \chi = \frac{(1-\eta)^4}{\eta^4 - 4\eta^3 + 4\eta^2 + 4\eta + 1}
\end{equation}
Finally, we can convert between the packing fraction $\eta$ and the
radial distribution function at contact $g_\sigma$ using the
Carnahan-Starling approximation for $g_\sigma$:
\newcommand\nastyetacuberoot{\sqrt[3]{54 g_\sigma^2 -
    6\sqrt{81g_\sigma^4 - 6g_\sigma^3}}}
\begin{align}
  g_\sigma &= \frac{1-\tfrac{\eta}{2}}{(1-\eta)^3}\text{, and its inverse} \\
  \eta &= 1 - \frac{1}{\nastyetacuberoot} - \frac{\nastyetacuberoot}{6g_\sigma}.
\end{align}
This function includes three fitted parameters, $\kappaO =
\kappazero$, $\kappaI = \kappaone$,
and $\kappaZ = \kappatwo$.
The fit is displayed in Fig.~\ref{fig:radial-distribution}.  In that
figure, it is apparent that the integral of $h(r)$ from
Eq.~\ref{eq:total-constraint} results in cancelling errors at large
distances.  The maximum error in $g(r)$ at packing fractions up to
0.4 is \maxerr, which occurs at $\eta = \etamaxerr$ and $r =
\rmaxerr$.

\newcommand\paircaption{The pair distribution function near a hard wall, with
  packing fractions of 0.1 and 0.3 and $\rr_1$ in contact with the hard wall.
  On the left is a 2D plot of $g^{(2)}(\rr_1,\rr_2)$ as $\rr_2$ varies.  The top
  half of this figure shows the result of the Monte Carlo simulation, while the
  bottom half shows the result of the appproximation developed in this work.  On
  the right is a plot of $g^{(2)}(\rr_1,\rr_2)$ on the path illustrated in the
  figure to the left.  This plot compares the Monte Carlo results (black) with
  those of this work (blue), Sokolowski and Fischer
  (red)~\cite{sokolowski1992role}, and Fischer and Methfessel
  (green)~\cite{fischer1980born}.  The latter is only plotted at contact, where
  it is defined.  }

\section{Results}

\begin{figure*}
  \begin{subfigure}{\textwidth}
    \includegraphics[width=\linewidth]{figs/pair-correlation-pretty-1.pdf}
    \vspace{-0.6cm}
  \end{subfigure}
  \begin{subfigure}{\textwidth}
    \includegraphics[width=\linewidth]{figs/pair-correlation-pretty-3.pdf}
    \vspace{-0.6cm}
  \end{subfigure}
  \caption{\paircaption}
  \label{fig:pair-distribution}
\end{figure*}

\subsection{Pair distribution function}

%% \begin{figure}
%%   \includegraphics[width=\columnwidth]{figs/pair-correlation-pretty-2.pdf}
%%   \caption{\paircaption{0.2}}\label{fig:pair-distribution-2}
%% \end{figure}
% \begin{figure}
%   \includegraphics[width=\columnwidth]{figs/pair-correlation-pretty-3.pdf}
%   \caption{\paircaption{0.3}}\label{fig:pair-distribution-3}
% \end{figure}
%% \begin{figure}
%%   \includegraphics[width=\columnwidth]{figs/pair-correlation-pretty-4.pdf}
%%   \caption{\paircaption{0.4}}\label{fig:pair-distribution-4}
%% \end{figure}

We begin by examining the pair distribution function near a hard wall,
with a focus on the case where one of the two spheres is in contact
with the hard wall.  Figures~\ref{fig:pair-distribution}a,\ref{fig:pair-distribution}b
and Figures~\ref{fig:pair-distribution}c,\ref{fig:pair-distribution}d compare the results of our
approximate functional with Monte Carlo simulation at packing
fractions of 0.1 and 0.3 respectively.  We see reasonable agreement at
the lower density, with some reduced oscillations at larger distances,
and a flatter angular dependence when the two spheres are in contact.
At the higher density, we see significant structure developing in the
simulation results that is not reflected in our approximation.

Figures~\ref{fig:pair-distribution}b and~\ref{fig:pair-distribution}d
show the pair distribution function as plotted along paths
illustrated in Figures~\ref{fig:pair-distribution}a and
\ref{fig:pair-distribution}c.  These plots compare this work with
Monte Carlo results, as well as the approximations of Sokolowski and
Fischer~\cite{sokolowski1992role} and that of Fischer and
Methfessel~\cite{fischer1980born} at packing fractions of 0.1 and 0.3.
The approach of Fischer and Methfessel is only defined when the two
spheres are in contact, and is therefore only plotted on that segment
of the path.  As an input to both previous approximations we use the
hard sphere radial distribution function found with Monte Carlo
simulation, interpolated as necessary.  We find that both previous
approximations to the pair distribution function predict the
qualitative angular dependence of the pair distribution function at
contact better than this work.  However, in each case the pair
distribution function has a systematic error at contact---either too
high or too low.  Thus while our approximation is smoother than either
of the existing approaches, its errors will have a tendancy to cancel
when used in a perturbation expansion.  At higher densities, the
approximation of Fischer and Methfessel requires evaluating the radial
distribution function at densities significantly higher than the
freezing density, which poses numerical difficulties when using the
radial distribution function from simulation.  When the two points are
more than a radius away from contact, we find that any of these
approaches gives a reasonable prediction, with our approach
underestimating the oscillations in $g^{(2)}$, as expected based on
the fit in Fig.~\ref{fig:radial-distribution}.

\subsection{Triplet distribution function}

\begin{figure*}
  \includegraphics[width=\textwidth]{figs/triplet-correlation-pretty-contact-3.pdf}\\
  \includegraphics[width=\textwidth]{figs/triplet-correlation-pretty-inbetween-3.pdf}
  \caption{\fixme{Make note of the double curves}\fixme{Add backward curve for
      this work mc?}The triplet distribution function
    $g^{(3)}(\rr_1,\rr_2,\rr_3)$ at packing fraction 0.3, plotted when $\rr_1$
    and $\rr_2$ are in contact.  The color plot on the left shows the triplet
    distribution function, with white corresponding to the contact value of the
    \emph{pair} distribution function.  The plot on the right shows the value of
    $g^{(3)}$ along the path shown as a dotted line on the plot to the
    left.}\label{fig:triplet-contact-distribution}
\end{figure*}

Just as the radial distribution function of a homogeneous fluid may be
computed from the density of an inhomogeneous one using Percus'
test-particle trick, the triplet distribution function of a
homogeneous system can be computed using an approximation of the pair
distribution for an inhomogeneous fluid, such as we have developed.
\begin{multline}
    g^{(3)}(\rr_1,\rr_2,\rr_3) =\\
    \frac{n_{\textrm{TP}(\rr_1)}(\rr_2)
      n_{\textrm{TP}(\rr_1)}(\rr_3)}{n^2}
    g^{(2)}_{\textrm{TP}(\rr_1)}(\rr_2,\rr_3)
\end{multline}
where $g^{(3)}(\rr_1,\rr_2,\rr_3)$ is the triplet distribution
function of a homogeneous fluid with density $n$, and the
$\textrm{TP}(\rr_1)$ subscript indicates quantities computed for the
inhomogeneous density configuration in which one sphere is fixed at
position $\rr_1$.  This method treats one of the three positions---the
location of the test particle---differently from the others, which
means that a poor approximation to the pair distribution function may
beak the symmetry which is present in the true triplet distribution
function.

Figure~\ref{fig:triplet-contact-distribution} compares the triplet
distribution function at a packing fraction of 0.3 computed using our
approximation to $g^{(2)}$ with results from Monte Carlo simulation.
In each subfigure, $\rr_1$ and $\rr_2$ are held fixed, while the third
position $\rr_3$ is varied.  The test-particle position in each case
is $\rr_1$, which is on the left-hand side of the figure.  In the
color plot on the left, the first and second spheres are displayed in
grey.  On the bottom half of
figures~\ref{fig:triplet-contact-distribution}a and
\ref{fig:triplet-contact-distribution}c is the triplet distribution
function as computed using the approximation presented in this work.
The top half of those figures displays the triplet distribution
function from Monte Carlo.
Figures~\ref{fig:triplet-contact-distribution}b
and~\ref{fig:triplet-contact-distribution}d show the triplet
distribution function along the paths illustrated in the corresponding
2D plots.

Figures~\ref{fig:triplet-contact-distribution}a
and~\ref{fig:triplet-contact-distribution}b show the triplet
distribution
function when two of the positions $\rr_1$ and $\rr_2$ are in contact.
The triplet distribution function predicted by this work is slightly
asymmetric, indicating a failure in thermodynamic consistency.  Once
again, we observe the lack of oscillations at large distances.
Figures~\ref{fig:triplet-contact-distribution}c
and~\ref{fig:triplet-contact-distribution}d show the triplet distribution
function when two of the positions $\rr_1$ and $\rr_2$ are spaced so
that the third could just fit between them.  Again, the triplet
distribution function predicted by this work is slightly asymmetric,
and lacks the oscillations seen in the Monte Carlo simulations at
large distances.  In both cases, our agreement with simulation is
comparable to that of previous approximations.

\begin{figure}
  \begin{subfigure}{1.0\columnwidth}
    \includegraphics[width=\columnwidth]{figs/dadz-3-2.pdf}
    \vspace{-0.8cm}
    \caption{Sticky hard-sphere fluid}\label{fig:dadz-delta}
  \end{subfigure}
  \begin{subfigure}{1.0\columnwidth}
    \includegraphics[width=\columnwidth]{figs/dadz-square-well-3.pdf}
    \vspace{-0.8cm}
    \caption{Hard-core square well fluid}\label{fig:dadz-square-well}
  \end{subfigure}
  \begin{subfigure}{1.0\columnwidth}
    \includegraphics[width=\columnwidth]{figs/dadz-inverse-sixth-3.pdf}
    \vspace{-0.8cm}
    \caption{Hard-core inverse-sixth fluid}\label{fig:dadz-inverse-sixth}
  \end{subfigure}
  \caption{Plot of $\frac{da_1}{dz}$ near a hard wall.  (a) shows a
    sticky hard-sphere fluid defined by a pair potential
    $\delta(\sigma-r+\delta)$ where $\sigma$ is the hard-sphere
    diameter, and $\delta$ is an infinitesimal distance; (b) shows a
    square well fluid defined by a pair potential $\Theta(1.79
    \sigma-r)$; and (c) shows a hard-core inverse-sixth potential
    fluid with an attractive pair potential proportional to $r^{-6}$.
  }
  \label{fig:dadz}
\end{figure}

\section{Performance in thermodynamic perturbation theory}

A particularly relevant quantitative test of a pair distribution
function is how well it predicts the interaction energy due to a pair
potential.  To this end, we have computed the error in the first term
in a high-temperature perturbation expansion $a_1$ for several pair
potentials.  In order to focus on effects at the interface, we have
defined a position-dependent pair interaction energy as
\begin{align}
  \frac{dF_1}{dz} &=
  \tfrac12 \int g^{(2)}_{HS}(\rr,\rr')n(\rr)n(\rr')\Phi(|\rr-\rr'|)
  d\rr'\, dxdy\label{eq:da1}
\end{align}
which gives the contribution to the mean-field free energy due to
molecules located at position $z$.  \jeffsays{About this last
  sentence: this is molecules in the plane $z$, right?  Wanted to make
  sure that's the case before changing to something like 'in the
  plane'.  Saying the position z makes it seem that there's only one.}

We plot in Fig.~\ref{fig:dadz} this quantity for hard-core fluids with
three representative pair potentials near a hard wall.  We have chosen
to illustrate a delta-function ineraction at contact (i.e. ``sticky
hard spheres''), a hard-core square-well fluid, with the length-scale
of interaction taken from the optimal SAFT model for water found by
Clark \emph{et al.}~\cite{clark2006developing}, and a $1/r^6$
power-law attraction.  These pair potentials represent a short-range
interaction, a medium-range discontinuous interaction, and a
long-range smooth interaction.

Figure~\ref{fig:dadz-delta} shows the results for the sticky
hard-sphere fluid.  Our pair distribution function is constructed to
produce this result exactly, provided the averaged pair distribution
function at contact from Ref.~\citenum{schulte2012using} is exact.  As
expected, we see excellent agreement with the Monte Carlo simulation
results, while the approximations of Fischer and Sokolowski each show
deviations near the interface.  Figure~\ref{fig:dadz-square-well}
shows the same curve from Eq.~\ref{eq:da1} for the square-well fluid.
This potential stresses our approximation to the radial distribution
function, since the edge of the attraction occurs at the dip in the
radial distribution function at this density, with poor results for
the bulk fluid.  We see that over these length scales, Sokolowski's
approximation is very accurate.  Finally, in the case of the power-law
attraction both approximations work very well.

\section{Conclusion}

We have introduced and tested an approximation for the pair
distribution function of the inhomogeneous hard-sphere fluid.  Our
approximation is suitable for use in the development of classical
density functionals based constructed using perturbation theory, and
may be computed using exclusively single-center convolutions.  We have
tested this function at a hard wall and near a single fixed hard
sphere, and find that it gives qualitatively reasonable results.
Tests of the pair distribution function in integrals that arise in
thermodynamic perturbation theory suggest that our approximation gives
good results for both short-range and long-range attractions.

\bibliography{paper}% Produces the bibliography via BibTeX.

\end{document}
