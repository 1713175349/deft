\documentclass{beamer}
\usepackage{amsmath}
\usepackage{graphicx}
\usepackage{color}
\usepackage{cancel}
\graphicspath{ {/home/krebse/deft/papers/water-SAFT/figs/} }
\usetheme{Warsaw}
\newcommand{\rr}{\textbf{r}}
\title[Testing Soft FMT]{Testing a ``soft'' Fundamental Measure Theory}
\author{Eric Krebs, Patrick Kreitzberg, David Roundy}
\institute{Oregon State University}
\date{March, 20 2013}
\begin{document}

\begin{frame}
 \titlepage
\end{frame}

\begin{frame}{Motivation}
  \begin{itemize}
    \item Want theory that works for inhomogeneous liquids
    \item The hard sphere fluid is well studied
    \item Hard spheres used as reference for real liquids
      \begin{itemize}
        \item Hard spheres are not very physical
        \item Not computationally convenient (delta functions)
      \end{itemize}
    \item We would like to ``soften'' spheres
  \end{itemize}
  \begin{block}{Classical density functional theory}
    \begin{align}
    \Omega(T) = \underset{n(\rr)}{min} \left\{F[n(\rr),T] + \int
                \left(V_{ext}(\rr) - \mu \right)n(\rr)d\rr \right\}
    \end{align}
    \begin{itemize}
      \item $n(\rr)$ can be an inhomogeneous particle density
    \end{itemize}
  \end{block}
\end{frame}

\begin{frame}{Fundamental measure theory}
  \begin{itemize}
    \item Classical DFT for hard sphere fluid
    \item Introduced by Rosenfeld in 1989
    \item Hard sphere fluid represented by fundamental measures
    \begin{itemize}
      \item Volume, surface area, mean curvature
    \end{itemize}
    \item $n_{\alpha}(\rr) = \int n(\rr')w_{\alpha}(\rr - \rr')d\rr'$
  \end{itemize}
  \begin{block}{Some weight functions for sphere of radius R}
    \begin{align}
      w_3(\rr) = \theta (R - |\rr|) \\
      w_2(\rr) = \delta (R - |\rr|) \\
      \notag
    \end{align}
  \end{block}
\end{frame}

\begin{frame}{Soft fundamental measure theory}
  \begin{itemize}
    \item Modification of FMT to treat not-so-hard spheres
    \item Introduced by Schmidt in 1999
    \item Weight functions are modified
  \end{itemize}
  Schmidt showed that:
  \begin{align}
    \int_{-\infty}^{\infty}w_2(r')w_2(r - r') dr' = \frac{\partial f(r)}{\partial r}
  \end{align}
  \begin{block}{Mayer Function}
    \begin{center}
      $f(r) = e^{-\beta V(r)} - 1$ \\
    \end{center}
\end{block}
  \begin{itemize}
    \item If $w_2$ is known, all weight functions, $w_{\alpha}$'s, can
      be derived
  \end{itemize}
\end{frame}

\begin{frame}{Choosing a $w_2(r)$}
  \begin{itemize}
    \item Want spheres to overlap a little at ambient temperatures
    \begin{itemize}
      \item We use a harmonic potential in Mayer function\\
      \includegraphics[width = 4cm]{figs/harmonic}
      \includegraphics[width = 4cm]{figs/fprime}
      \item Could not find simple analytic solution for $w_2$
    \end{itemize}
  \end{itemize}
 %% \begin{align}
 %%    w_2 &= \frac{2 \gamma r}{(\sqrt{\pi \gamma}-1)R^2} e^{-\gamma
 %%       \left(1 - \frac{r}{2r_0} \right)^2} \Theta(r) \Theta(2r_0-r)\\
 %%    \gamma &= \frac{1}{32} \left( \sqrt{\pi \beta V_0} + \sqrt{\pi \beta
 %%        V_0 - 16\sqrt{\beta V_0}} \right)^2
 %%  \end{align}
\end{frame}

\begin{frame}{Comparing $w_2$ with $f'(r)$}
  \begin{itemize}
    \item Recall: $\int_{-\infty}^{\infty}w_2(r')w_2(r - r') dr' = \frac{\partial f(r)}{\partial r}$
    \item Chose a simple form of $w_2(r)$ with fitting parameters
  \end{itemize}
  \begin{center}
    \includegraphics[trim = 0 13 0 37, clip, scale = 0.40]{figs/w2convolves}  \end{center}
\end{frame}

\begin{frame}{Weight functions comparison with FMT}
  \begin{center}
  \begin{tabular}{ccc}
  %\includegraphics[trim = 0 13 0 37, clip, scale = 0.20]{figs/w_3} &
  \includegraphics[width=4cm]{figs/w_3} \hspace{-1.5em} &
  \includegraphics[width=4cm]{figs/w_2} \hspace{-1.5em} &
  \includegraphics[width=4cm]{figs/w_1} \\
  \includegraphics[width=4cm]{figs/step} \hspace{-1.5em} &
  \includegraphics[width=4cm]{figs/delta} \hspace{-1.5em} &
  \includegraphics[width=4cm]{figs/delta}  
  \end{tabular}
  \end{center}
  %% \begin{tabular}{cc}
  %% %\includegraphics[trim = 0 13 0 37, clip, scale = 0.20]{figs/w_3} &
  %% \includegraphics[width=4cm]{figs/w_3} &
  %% \includegraphics[width=4cm]{figs/step} \\
  %% \includegraphics[width=4cm]{figs/w_2}
  %% \vspace{-.7em} & \\
  %% \includegraphics[width=4cm]{figs/w_1} &
  %% \end{tabular}
\end{frame}

\begin{frame}{Homogeneous fluid}
  \begin{center}
    \includegraphics[trim = 0 13 0 35, clip, scale = 0.35]{figs/p-vs-packing}
  \end{center}
  \begin{itemize}
    \item At low temps, should reduce to hard spheres
    \item At low packing fraction, agrees with ideal gas
  \end{itemize}
\end{frame}

\begin{frame}{Conclusion}
  \begin{itemize}
    \item Soft FMT might be a possible alternative to hard spheres
    \item Our approximate $w_2$ works well for homogeneous case
    \item Reduces to hard spheres at low temps
  \end{itemize}
  \begin{block}{Future work}
    \begin{itemize}
      \item Compare with Monte Carlo simulation
      \item Simulate soft spheres near a hard wall
    \end{itemize}
  \end{block}
\end{frame}

\end{document}
