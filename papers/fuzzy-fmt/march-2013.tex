\documentclass{beamer}
\usepackage{amsmath}
\usepackage{graphicx}
\usepackage{color}
\usepackage{cancel}
\graphicspath{ {/home/krebse/deft/papers/water-SAFT/figs/} }
\usetheme{Warsaw}
\newcommand{\rr}{\textbf{r}}
\title[Soft FMT]{Testing a ``soft'' Fundamental Measure Theory}
\author{Eric Krebs, Patrick Kreitzberg, David Roundy}
\institute{Oregon State University}
\date{}
\begin{document}

\begin{frame}
 \titlepage
\end{frame}

\begin{frame}{Motivation}
  \begin{itemize}
    \item We currently model water as a hard sphere fluid
    \item Hard spheres not physical for water
    \item We would like to ``soften'' spheres
  \end{itemize}
\end{frame}

\begin{frame}{Fundamental measure theory}
  \begin{itemize}
    \item Introduced by Rosenfeld in 1989
    \item Hard sphere fluid can be represented by fundamental measures
  \end{itemize}
  \begin{block}{weight functions for sphere of radius R}
    \begin{align}
      w_3(\rr) = \theta (|\rr| - R) \\
      w_2(\rr) = \delta (|\rr| - R)
    \end{align}
  \end{block}
\end{frame}

\begin{frame}{Schmidt paper (2001)}
  \begin{block}{Mayer Function}
    $f(r) = e^{-\beta V(r)} - 1$
    \begin{itemize}
      \item $\beta = (k_BT)^{-1}$
      \item $V(r)$ is a pair potential
    \end{itemize}
  \end{block}
  Schmidt shows that:
  \begin{align}
    \frac{\partial f(r)}{\partial r} = \int_{-\infty}^{\infty}w_2(r')
    \ast w_2(r - r') dr'
  \end{align}
\end{frame}

\begin{frame}{Choosing a $w_2(r)$}
  \begin{itemize}
    \item We considered linear and quadratic pair potentials, $V(r)$
    \item
  \end{itemize}
\end{frame}

\begin{frame}{theory results}
\end{frame}

\begin{frame}{Monte Carlo simulation results}
\end{frame}

\begin{frame}{Comparison}
\end{frame}

\begin{frame}{Finish}
  Thank you!
\end{frame}

\end{document}
