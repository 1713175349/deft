% TODO:

% Write ``Liquid-vapor interface'' section A.

% Look up question-mark references (citations).

\documentclass[letterpaper,twocolumn,amsmath,amssymb,prb]{revtex4-1}
\usepackage{graphicx}% Include figure files
\usepackage{dcolumn}% Align table columns on decimal point
\usepackage{bm}% bold math
\usepackage{color}

\newcommand{\red}[1]{{\bf \color{red} #1}}
\newcommand{\blue}[1]{{\bf \color{blue} #1}}
\newcommand{\green}[1]{{\bf \color{green} #1}}
\newcommand{\rr}{\textbf{r}}
\newcommand{\refnote}{\red{[ref]}}

\newcommand{\fixme}[1]{\red{[#1]}}

\begin{document}
\title{Smoothing the Fundamental Measure Theory Functional for Hard Spheres}

\author{David Roundy}
\affiliation{Department of Physics, Oregon State University, Corvallis, OR 97331}

%%%%%%%%%%%%%%%%%%%%%%%%%%%%%%%%%%%%%%%%%%%%%%%%%%%%%%%%%%%%
\begin{abstract}
\fixme{ We examine a functional based on SFMT~\cite{schmidt2000fluid}.
  We pick a slightly soft potential, and demonstrate that it gives
  good results.}
\end{abstract}

\maketitle

%%%%%%%%%%%%%%%%%%%%%%%%%%%%%%%%%%%%%%%%%%%%%%%%%%%%%%%%%%%%
\section{Introduction}



\section{Fundamental-Measure Theory}

We use the White Bear version of the Fundamental-Measure Theory~(FMT)
functional published in reference~\cite{roth2002whitebear}.  The FMT
functional describes the excess free energy of a hard-sphere fluid.
This particular FMT reduces to the Carnahan-Starling equation of state
for homogeneous systems.
\begin{equation}
A_\textit{HS}[n] = k_B T \int \left(\Phi_1(\rr) + \Phi_2(\rr) + \Phi_3(\rr)\right) d\rr \; ,
\end{equation}
with integrands
\begin{align}
\Phi_1 &= -n_0 \ln\left( 1 - n_3\right)\\
\Phi_2 &= \frac{n_1 n_2 - \mathbf{n}_{V1} \cdot\mathbf{n}_{V2}}{1-n_3} \\
\Phi_3 &= (n_2^3 - 3 n_2 \mathbf{n}_{V2} \cdot \mathbf{n}_{V2}) \frac{
  n_3 + (1-n_3)^2 \ln(1-n_3)
}{
  36\pi n_3^2\left( 1 - n_3 \right)^2
} ,
\end{align}
using the weighted densities
\begin{align}
  n_3(\rr) &= \int n(\rr') \Theta(\left|\rr - \rr'\right| - R) d\rr' \\
  n_2(\rr) &= \int n(\rr') \delta(\left|\rr - \rr'\right| - R) d\rr'
\end{align}
\begin{align}
  \mathbf{n}_{V2} &= \mathbf{\nabla} n_3 , \quad
  \mathbf{n}_{V1} = \frac{\mathbf{n}_{V2}}{4\pi R} \\
  n_1 &= \frac{n_2}{4\pi R} , \quad
  n_0 = \frac{n_2}{4\pi R^2}
\end{align}

\section{Theory}

We begin by examining the Mayer function:

\begin{equation}
f(r) = \exp (-\beta V(r)) - 1.
\end{equation}

From Schmidt\cite{schmidt2000fluid},

\begin{equation}
\frac{d f(r)}{dr} = \int dr' w_2(r') w_2 (r-r')
\end{equation}

We have a choice to either start with a pair potential, $V(r)$, and
derive the appropriate $w_2(r)$, or we start with the latter and find
the former. All other weighting functions can be derived from
$w_2(r)$ (see Appendix A).  Another way to attack this problem is to
pick a pair potential and determine a $w_2(r)$ to ``fit'' to the
potential.

We choose a quadratic pair potential:

\begin{align}
  V(r) =
  \begin{cases}
    V_0 \left ( 1 - \frac{r}{r_0} \right )^2 & 0 \leq r \leq r_0\\
    0 & otherwise.
  \end{cases}
\end{align}
This gives a Mayer function with a quadratic potential: 
\begin{align}
  f(r) =
  \begin{cases}
    e^{-\beta V_0 \left( 1 - \frac{r}{r_0} \right)^2} - 1 & 0 \leq r
    \leq r_0 \\
    0 & otherwise.
  \end{cases} 
\end{align}
Because $w_2(r)$ will \emph{look} similar to \emph{-f(r)} but at
$\frac{r_0}{2}$, we guess that
\begin{equation}
w_2(r) = B r e^{-\gamma \left ( 1 - \frac{r}{R} \right )^2}\Theta_1
(r) \Theta_2 (R - r)
\end{equation}
where B is a constant to be determined by fitting the integral of
$f'(r)$ with the integral of $w_2(\rr)\ast W_2(\rr)$ and $R =
\frac{r_0}{2}$.  $gamma$ is determined by the slope at $r = r_0$.  
%% Now we'll first want to convolve $w_2(r)$ with
%% itself:
%% \begin{align*}
%%   f'(r) &= \int dr' w_2 (r')w_2(r - r') \\ 
%%      &= \int dr' \left(B r' e^{- \gamma \left (1 - \frac{r}{R} \right )^2}
%%                               \Theta_1 (r') \Theta_2 (R - r') \right) \\
%%      &\times \left( B (r-r') e^{- \gamma \left(1 - \frac{r-r'}{R}
%%                                 \right)^2}\Theta_{1'} (r-r') \Theta_{2'} (R -
%%                                 (r-r')) \right) \\
%%      &= \alpha \int dr' r'(r-r') \exp \left [-\frac{2\gamma}{R^2} \left (r' -
%%                               \frac{r}{2} \right)^2 \right ] \Theta_1
%%                               \Theta_2 \Theta_{1'} \Theta_{2'}
%% \end{align*}
%% where we have substituted $\alpha = B^2 e^{- 2 \gamma \left(1 -
%%   \frac{r}{R} + (\frac{r}{2R})^2 \right )}$. The Theta functions create two separate
%% cases to integrate.  The case that is, perhaps, more important for us
%% is is when $R \leq r \leq 2R$.  The integral is

%% \begin{align}
%%  f_{r \geq R}'(r) &= \alpha \int_{r - R}^R dr' r'(r-r') \exp \left [ -
%%    \frac {2\gamma}{R^2}\left( r' - \frac{r}{2} \right)^2 \right]
%% \end{align}

%% Now, for $ 0 \leq r \leq 2R$
%% \begin{align}
%%   f'(r) = \frac{\beta V_0}{R} \left( 1 - \frac{r}{2R} \right)
%%               e^{-\beta V_0 \left( 1 - \frac{r}{2R} \right)^2}
%% \end{align}
%% If we integrate $f'(r)$ for the quadratic potential we get
%% \begin{align}
%%   \int_0^{\infty} f'(r)dr &= f(\infty) - f(0)\\
%%                     &= 1 - e^{-\beta V_0}
%% \end{align}
%% For the $w_2(r)$ convolved with itself,
%% \begin{widetext}
%% \begin{align*}
%%   w_2\ast w_2 &= \alpha \frac{R}{4}\sqrt{\frac{\pi}{2\gamma}}
%%                \left( r^2 - \frac{R^2}{\gamma}
%%                \right)\textrm{erf} \left( \sqrt{2\gamma} \left(1 - \frac{r}{2R}
%%                \right) \right)%\\
%%                %&
%%                + \alpha \frac{R^2}{2\gamma} \left( R - \frac{r}{2}
%%                \right) e^{-2\gamma \left( 1 - \frac{r}{2R} \right)^2}.
%% \end{align*}
%% \end{widetext}
%% We substitute $u = \sqrt{2\gamma}\left( 1 - \frac{r}{2R} \right)$ and
%% plug in our expression above for $\alpha$ so that,
\begin{align*}
  w_2\ast w_2 &= \frac{B^2 R}{\sqrt{2\gamma}} \left[
               \frac{\sqrt{\pi}}{4} \left( r^2 - \frac{R^2}{\gamma}
               \right) e^{-u^2} \textrm{erf}(u)
               + \frac{R^2 u}{\sqrt{2\gamma}}
               e^{-2u^2} \right]
\end{align*}
Now we'll need to integrate this and set it to the integral of $f'(r)$
to solve for the constant $B$. Since this expression is only the part
of $w_2(r)$ from $R \leq r \leq 2R$, we need to justify only using
this portion by saying that at low temps (which is our region of
interest), $w_2(r)$ below $R$ (or $\frac{r_0}{2}$) is nearly zero.
This upper part of $w_2(r)$ which we have used above goes to very
small numbers in the region $0 \leq r \leq R$.  Both parts of $w_2(r)$
are nearly zero for $r \leq 0$, so we can carry out the integrate to
$-\infty$ so that the error function integrals are analytic.
Therefore, we can integrate from $-\infty$ to $r_0$ and it is
approximately correct.

\begin{figure}
\begin{center}
\includegraphics[width=3.5in]{figs/w2convolves}
\end{center}
\caption{Comparison of the $-f'(r)$ function with various $w_2(r)\ast
  w_2(r)$.}
\label{fig:w2convolves}
\end{figure}

If we do this integration on over $r$, then the limits for $u$ are
$\infty$ to zero.  I'll set \fixme{for now to save space} $w_2\ast
w_2 = w_{2c}(u)$. Then,
\begin{widetext}
\begin{align}
\int_{\infty}^0 w_{2c}(u) du &= -\frac{B^2 R^4}{2 \gamma^2}
           \int_\infty^0 \left[ \frac{\sqrt{\pi}}{2} \left( 4\gamma - 1 -
           4\sqrt{2\gamma}u + 2 u^2 \right) e^{-u^2} \textrm{erf}(u) +
           u e^{-2u^2} \right]du \\
           &= \left[ \frac{B R^2}{2\gamma} (\sqrt{\pi \gamma} -1) \right]^2
\end{align}
\end{widetext}
Now, to find B we set
\begin{align}
  \int_0^{\infty} f'(r)dr &= \int_{-\infty}^0 w_2(r)\ast w_2(r) dr\\
   1 - e^{-\beta V_0} &= \left[ \frac{B R^2}{2\gamma} (\sqrt{\pi
       \gamma} -1)  \right]^2
\end{align}
Which gives
\begin{align}
  B = \frac{2\gamma}{(\sqrt{\pi \gamma}-1)R^2} \left(
    1-e^{-\beta V_0} \right)^{1/2}
\end{align}

we could also change the limits so that
\begin{align}
  \int_{-\infty}^{r_0} f'(r)dr &= 1 \\
  \implies B &= \frac{2\gamma}{(\sqrt{\pi \gamma}-1)R^2}
\end{align}

\subsection{Weighting functions in real space}

\begin{align}
  w_2(r) &=\frac{2\gamma \left(e^{-\beta V_0} - 1\right)^{1/2} r}
  {(\sqrt{\pi \gamma}-1)R^2}e^{-\gamma \left ( 1 - \frac{r}{R} \right)^2}
           \Theta(r) \Theta(R - r )\\
  w_3(r) &= \int_r^\infty w_2(r') dr'\\
         &= -B \int_r^R r' e^{-\gamma \left(1- \frac{r'}{R} \right)^2}dr'\\
         &= \frac{B R^2}{2\gamma}\left[ 1 - e^{-\gamma
      \left(1-\frac{r}{R} \right)^2} - \sqrt{\pi \gamma} \mathrm{erf} \left[
        \sqrt{\gamma} \left( 1- \frac{r}{R} \right) \right] \right]\\
  w_1(r) &= \frac{B}{4\pi}e^{-\gamma \left ( 1 - \frac{r}{R}
            \right)^2} \Theta_1(r) \Theta_2 (R - r)\\
  w_0(r) &= \frac{B}{4\pi r}e^{-\gamma \left ( 1 - \frac{r}{R}
            \right)^2} \Theta_1(r) \Theta_2 (R - r)\\
  \mathbf{w}_{2V}(\rr) &= B e^{-\gamma \left ( 1 - \frac{r}{R} \right)^2}
           \Theta_1(r) \Theta_2 (R - r) \rr\\
  \mathbf{w}_{1V}(\rr) &= \frac{B}{4\pi r} e^{-\gamma \left ( 1 - \frac{r}{R} \right)^2}
           \Theta_1(r) \Theta_2 (R - r )\rr
\end{align}

\subsection{Weighting functions in fourier space}\

\begin{align}
  \tilde{w}_2(k) &= \frac{4\pi B}{k} \int_{0}^{\infty} r^2 
   e^{-\gamma \left(1- \frac{r}{R} \right)^2} \sin(kr) dr \\
  \tilde{w}_3(k) &=
\end{align}
\begin{align}
  \tilde{w}_1(k) &= \int_0^\infty \int_0^{2\pi} \int_{-1}^{1} w_1(r)
           e^{-ikr\cos\theta} r^2drd\phi d\cos\theta
    \\
    &= \frac{4\pi}{k} \int_0^\infty r w_1(r) \sin(kr) dr
    \\
    &= -\frac{2\pi}{ik} \int_{-\infty}^\infty r w_1(|r|) e^{-ikr} dr
    \\
    &= \frac{2\pi}{k} \int_0^{R} r \frac{B}{4\pi} e^{-\gamma\left(1-\frac{|r|}{R}\right)^2} \sin(kr) dr
    \\
    &= -\frac{B}{ik} \int_{-R}^{R} r e^{-\gamma\left(1-\frac{|r|}{R}\right)^2} e^{-ikr} dr
    \\
    &= -\frac{B}{ik} \int_{-R}^{R} r
    e^{-\gamma\left(1-\frac{|r|}{R}\right)^2 - ikr} dr
    \\
    x &\equiv \frac{r}{R} \\
    \\
    \tilde{w}_1(k) &=
    -\frac{BR^2}{ik} \int_{-1}^{1} x
    e^{-\gamma\left(1-|x|\right)^2 - ikRx} dx
    \\
    &=
    -\frac{BR^2}{ik}
    \bigg(
    \int_{0}^{1} x e^{-\gamma\left(1-x\right)^2 - ikRx} dx
    \notag\\&\quad\quad\quad\quad\quad\quad+
    \int_{-1}^{0} x e^{-\gamma\left(1+x\right)^2 - ikRx} dx
    \bigg)
\end{align}
Eric has checked this, and it looks fine, and looks solvable by
Mathematica.  The final integral is alternatively solvable by
completing the square in the exponent.
\begin{align}
  \tilde{w}_0(k) &= \frac{B}{k} \int_0^\infty \frac{w_2(r)}{r} \sin(kr) dr
    \\
  \mathbf{\tilde{w}}_{2V}(k) &= \\
  \mathbf{\tilde{w}}_{1V}(k) &= 
\end{align}

\subsection{Homogeneous limit}

\begin{align}
  \int w_3(\rr)d\rr &= \frac{4\pi BR^2}{2\gamma}\int_0^R r^2 \bigg\{1-
                     e^{-\gamma \left(1-\frac{r}{R} \right)^2} \notag\\
                    &-\sqrt{\pi \gamma} \mathrm{erf}\left[ \gamma \left(1 - \frac{r}{R}
                      \right) \right] \bigg\} d\rr\\
                     &= \frac{\pi R
                       e^{-\gamma}}{3\gamma^{3/2}(\sqrt{\pi \gamma})}
                     \bigg\{2\sqrt{\gamma}
                     \left(8e^\gamma(1+\gamma)-2\gamma -5 \right) \notag\\
                     &- e^\gamma \sqrt{\pi} \left(4\gamma^2 +12\gamma
                     +3 \right) \mathrm{erf}(\sqrt{\gamma}) \bigg\}\\
  \int w_2(\rr)d\rr &= 4\pi B \int_0^R r^3 e^{-\gamma \left ( 1 - \frac{r}{R}
           \right)^2} dr \\
        &= \frac{-2\pi R^2}{(\sqrt{\pi
           \gamma} -1)}\big[ \sqrt{\pi \gamma}(2\gamma
           +3)\mathrm{erf}(\sqrt{\gamma})\notag \\
           & - 2(7\gamma + 1)e^{-\gamma} +6\gamma +2 \big] \\
  \int w_1(\rr)d\rr &= B\int_0^R r^2 e^{-\gamma \left ( 1 - \frac{r}{R}
            \right)^2} dr\\
         &= \frac{R\left[ \sqrt{\pi}(2\gamma + 1)\mathrm{erf}(\sqrt{\gamma})
          + \left(4-6e^{-\gamma}\right)\sqrt{\gamma}
          \right]}{2\sqrt{\gamma}( \sqrt{\pi \gamma} -1)} \\
  \int w_0(\rr)d\rr &= B\int_0^R r e^{-\gamma \left ( 1 - \frac{r}{R}
            \right)^2} dr\\
              &= \frac{\sqrt{\gamma}}{\sqrt{\pi \gamma} -1}\left[
              \sqrt{\pi} \mathrm{erf}(\sqrt{\gamma}) +
               \frac{e^{-\gamma}-1}{\sqrt{\gamma}} \right]
\end{align}
 
\section{Comparison with simulation}

\subsection{Hard spheres confined in a spherical cavity}

\fixme{TODO: Create actual Monte Carlo data, and make a comparison.}

%%%%%%%%%%%%%%%%%%%%%%%%%%%%%%%%%%%%%%%%%%%%%%%%%%%%%%%%%%%%
\section{Conclusion}

\appendix

\section{Soft Fundamental-Measure Theory}

We follow the Soft Fundamental-Measure Theory introduced by
Schmidt~\cite{schmidt2000fluid}.  Here I will derive the fundamental
measures for a hard-sphere fluid, so we can see how this works.
Schmidt defines the relationship between the weighting functions as:
\begin{align}
  w_2(r) &= -\frac{\partial w_3(r)}{\partial r} \\
  \mathbf{w}_{v2}(\rr) &= w_2(r)\frac{\rr}{r} \\
  \mathbf{w}_{m2}(\rr) &= w_2(r)\left( \frac{\rr \rr}{r^2}
                              - \frac{\mathbf{\hat{1}}}{3} \right) \\
  w_1(r) &= \frac{w_2(r)}{4\pi r} \\
  \mathbf{w}_{v1}(\rr) &= w_1(r) \frac{\rr}{r} \\
  w_0(r) &= \frac{w_1(r)}{r}
\end{align}
The weighting functions are themselves given by a deconvolution of the
Mayer $f$ function:
\begin{align}
  f(r) &= \exp\left(-\frac{V(r)}{k_BT}\right) - 1 \\
  &= w_0 * w_3 + w_1 * w_2 - \mathbf{w}_{v1} * \mathbf{w}_{v2}
\end{align}
where the $*$ operator denotes a three-dimensional convolution, and a
scalar product between vectors.  Schmidt explains that there is a
simple relationship in fourier space between the Mayer $f$ function
and the weighting function $w_2$:
\begin{align}
  \tilde{w}_2(k) &= \pm \sqrt{ik\tilde{f}(k)} \label{eq:w2fromf}
\end{align}
where the tilde denotes a \emph{one-dimensional} Fourier transform:
\begin{align}
  \tilde{f}(k) &= \int_{-\infty}^\infty f(r) e^{ikr} dr
\end{align}

I am writing this section to explore this formula, and to help me to
understand how to deal with this branch cut (i.e. the sign in
Equation~\ref{eq:w2fromf}).  One interesting challenge here is that
$\tilde{w}_2(k)$ is \emph{not} actually the function that we will use
in our convolutions, which are three-dimensional convolutions.  So we
will still have to go into real space and back again to get our
convolution kernel.  $\ddot\frown$

\section{Hard spheres}

For the hard-sphere fluid,
\begin{align}
  f(r) = - \Theta(r - 2R)
\end{align}
which means that
\begin{align}
  \tilde{f}(k) &= \int_{-\infty}^\infty f(r) e^{ikr} dr \\
  &= -\int_{-2R}^{2R} e^{ikr} dr \\
  &= -\int_{-2R}^{2R} \cos(kr) dr \\
  &= -\frac{2}{k}\sin(2kR)
\end{align}
At this point, we want to solve for $w_2$.
\begin{align}
  \tilde{w}_2(k) &= \pm \sqrt{ik\tilde{f}(k)} \\
  &= \pm \sqrt{-2i\sin(2kR)} \\
  &= \pm \sqrt{-4i\sin(kR)\cos(kR)}
\end{align}
I happen to know that
\begin{align}
  w_2(r) &= \delta(r - R) \\
  \tilde{w}_2(k) &= \int_{-\infty}^\infty w_2(r) e^{ikr} dr \\
  &= \int_{-\infty}^\infty \delta(r-R) e^{ikr} dr \\
  &= e^{ikr} + e^{-ikr + i\Phi}
\end{align}
where in the last step, I use the fact that there is a physical
ambiguity as to whether $w_2$ is even or odd or somewhere in between
when understood in one dimension, since only even values for $r$ make
physical sense.  Thus looking at the square of this, we see
\begin{align}
  \tilde{w}_2(k)^2 &= e^{i2kr} + e^{-i2kr + 2\Phi} + 2e^{i\Phi} \\
  &= -2i\sin(2kR) \\
  &= -2i\frac{e^{i2kR} - e^{-i2kR}}{2i} \\
  &= e^{-i2kR} - e^{i2kR}
\end{align}
This almost makes sense (if $\Phi = \pi/2$), but not quite, because
there is the extra term of $2e^{i\Phi}$, which would give us an extra
$2i$.

This looks like a contradiction.  However, we assumed here that there
was a $\delta$-function at $-R$, and that $f(r)$ was a symmetric
function, simply because the three-dimensional function is symmetric.
What if there is only one spike at $R$ and $f(r)$ is asymmetric?  Then
we would have
\begin{align}
  \tilde{f}(k) &= \int_{-\infty}^\infty f(r) e^{ikr} dr \\
  &= -\int_{-\infty}^{2R} e^{ik2R} dr
\end{align}
This integral isn't very nicely behaved.  But if I were to ignore the
lower bound---which is actually okay for computing the slope, which is
all we need---then I would find that
\begin{align}
  \tilde{f}(k) &= -\frac{e^{ikr}}{ik} + C
\end{align}
When I solve for $\tilde{w}_2$ I get something similar:
\begin{align}
  w_2(r) &= \delta(r - R) \\
  \tilde{w}_2(k) &= \int_{-\infty}^\infty \delta(r-R) e^{ikr} dr \\
  &= e^{ikR}
\end{align}
And now it's obvious that the solution is correct, apart from an
unexplained minus sign in $\tilde{f}$, which I imagine I could find in
time.  So it looks like we want to assume that the slope of $f(r)$ is
zero for $r<0$, and therefore that $w_2(r<0)=0$.

\section{Linear soft potential}

Now let's imagine a potential
\begin{align}
  V(r) =
  \begin{cases}
    V_0 & r < 0 \\
    V_0\left(1 - \frac{r}{r_0}\right) & r < r_0 \\
    0 & r \ge r_0
  \end{cases}
\end{align}
This gives us a Mayer $f$ function of
\begin{align}
  f(r) &= e^{-\beta V(r)} - 1 \\
  f'(r) &=
  \begin{cases}
    \frac{\beta V_0}{r_0} e^{-\beta V_0(1-r/r_0)} & 0 < r < r_0 \\
    0 & \text{otherwise}
  \end{cases} \\
  \tilde{f'}(k) &= \frac{\beta V_0}{r_0}\int_{0}^{r_0} e^{-\beta
    V_0(1-r/r_0)}e^{ikr}dr \\
  &= \frac{\beta V_0}{r_0}e^{-\beta V_0}
  \int_{0}^{r_0} e^{(\beta V_0/r_0 + ik)r} dr
  \\
  &= \frac{\beta V_0}{r_0}e^{-\beta V_0} \frac{1}{\beta V_0/r_0 + ik}\left(
  e^{\beta V_0+ikr_0} - 1 \right)
  \\
  &= \frac{e^{ikr_0} - e^{-\beta V_0}}{1 + \frac{ikr_0}{\beta V_0}}
  \\
  &= \tilde{w}_2(k)^2
\end{align}
That's just a little bit yucky, and I'm not sure I can solve for it in
real space.

Now let's imagine a potential that has some interesting (but
numerically convenient) values for $r<0$
\begin{align}
  V(r) =
  \begin{cases}
    V_0\left(1 - \frac{r}{r_0}\right) & r < r_0 \\
    0 & r \ge r_0
  \end{cases}
\end{align}
This gives us a Mayer $f$ function of
\begin{align}
  f(r) &= e^{-\beta V(r)} - 1 \label{eq:ftrytwo} \\
  f'(r) &=
  \begin{cases}
    \frac{\beta V_0}{r_0} e^{-\beta V_0(1-r/r_0)} & r < r_0 \\
    0 & \text{otherwise}
  \end{cases} \\
  \tilde{f'}(k) &= \frac{\beta V_0}{r_0}\int_{-\infty}^{r_0} e^{-\beta
    V_0(1-r/r_0)}e^{ikr}dr \\
  &= \frac{\beta V_0}{r_0}e^{-\beta V_0}
  \int_{-\infty}^{r_0} e^{(\beta V_0/r_0 + ik)r} dr
  \\
  &= \frac{\beta V_0}{r_0}e^{-\beta V_0} \frac{1}{\beta V_0/r_0 + ik}e^{\beta V_0+ikr_0}
  \\
  &= \frac{e^{ikr_0}}{1 + \frac{ikr_0}{\beta V_0}}
  \\
  &= \tilde{w}_2(k)^2
\end{align}
Now this is nicer.  The top is really easy to square root.  The bottom
is ever so slightly harder, but not really hard.
\begin{align}
  \tilde{w}_2(k) &=
  \frac1{\sqrt{2}}\frac{e^{ikr_0/2}}{\sqrt{\sqrt{1+\frac{kr_0}{\beta
          V_0}}+1} +
      i\sqrt{\sqrt{1+\frac{kr_0}{\beta V_0}}-1}}
\end{align}
Okay, that's a bit ugly, and I'm not sure that we can Fourier
transform it to get a useful result.  We could potentially approximate
it in the limit that $\beta V_0 \gg 1$, although any simple power
series expansion would require us to prove that $kr_0\gg1$ is
unimportant, possibly by arguing that phase cancellation will cause it
to be unimportant when we evaluate our actual $w_2(r)$.
\begin{align}
  w_2(r) &= \int_{-\infty}^\infty \tilde{w}_2(k) \\
 &=
  \frac1{\sqrt{2}}\frac{e^{ikr_0/2}}{\sqrt{\sqrt{1+\frac{kr_0}{\beta
          V_0}}+1} +
      i\sqrt{\sqrt{1+\frac{kr_0}{\beta V_0}}-1}}
\end{align}

It occurs to me now that Schmidt may have assumed right after his
Equation~A7 that $w_2(r<0)=0$, in which case this attempt won't work:
we'll need to make things zero for $r<0$ if we're going to use this
Fourier transform trick at all.  On the other hand, if this comes out
to be \emph{effectively} zero for reasonable temperatures, perhaps
this will be a good approach.  After all, I'm interested in $\beta V_0
\gg 1$, which means that the Mayer function is essentially flat in
Equation~\ref{eq:ftrytwo} for $r<0$.

\bibliography{paper}% Produces the bibliography via BibTeX.

\end{document}

