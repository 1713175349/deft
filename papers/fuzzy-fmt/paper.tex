% TODO:

% Write ``Liquid-vapor interface'' section A.

% Look up question-mark references (citations).

\documentclass[letterpaper,twocolumn,amsmath,amssymb,prb]{revtex4-1}
\usepackage{graphicx}% Include figure files
\usepackage{dcolumn}% Align table columns on decimal point
\usepackage{bm}% bold math
\usepackage{color}

\newcommand{\red}[1]{{\bf \color{red} #1}}
\newcommand{\blue}[1]{{\bf \color{blue} #1}}
\newcommand{\green}[1]{{\bf \color{green} #1}}
\newcommand{\rr}{\textbf{r}}
\newcommand{\refnote}{\red{[ref]}}

\newcommand{\fixme}[1]{\red{[#1]}}

\begin{document}
\title{Smoothing the Fundamental Measure Theory Functional for Hard Spheres}

\author{David Roundy}
\affiliation{Department of Physics, Oregon State University, Corvallis, OR 97331}

%%%%%%%%%%%%%%%%%%%%%%%%%%%%%%%%%%%%%%%%%%%%%%%%%%%%%%%%%%%%
\begin{abstract}
\fixme{ We examine a functional based on SFMT~\cite{schmidt2000fluid}.
  We pick a slightly soft potential, and demonstrate that it gives
  good results.}
\end{abstract}

\maketitle

%%%%%%%%%%%%%%%%%%%%%%%%%%%%%%%%%%%%%%%%%%%%%%%%%%%%%%%%%%%%
\section{Introduction}



\section{Fundamental-Measure Theory}

We use the White Bear version of the Fundamental-Measure Theory~(FMT)
functional published in reference~\cite{roth2002whitebear}.  The FMT
functional describes the excess free energy of a hard-sphere fluid.
This particular FMT reduces to the Carnahan-Starling equation of state
for homogeneous systems.
\begin{equation}
A_\textit{HS}[n] = k_B T \int \left(\Phi_1(\rr) + \Phi_2(\rr) + \Phi_3(\rr)\right) d\rr \; ,
\end{equation}
with integrands
\begin{align}
\Phi_1 &= -n_0 \ln\left( 1 - n_3\right)\\
\Phi_2 &= \frac{n_1 n_2 - \mathbf{n}_{V1} \cdot\mathbf{n}_{V2}}{1-n_3} \\
\Phi_3 &= (n_2^3 - 3 n_2 \mathbf{n}_{V2} \cdot \mathbf{n}_{V2}) \frac{
  n_3 + (1-n_3)^2 \ln(1-n_3)
}{
  36\pi n_3^2\left( 1 - n_3 \right)^2
} ,
\end{align}
using the weighted densities
\begin{align}
  n_3(\rr) &= \int n(\rr') \Theta(\left|\rr - \rr'\right| - R) d\rr' \\
  n_2(\rr) &= \int n(\rr') \delta(\left|\rr - \rr'\right| - R) d\rr'
\end{align}
\begin{align}
  \mathbf{n}_{V2} &= \mathbf{\nabla} n_3 , \quad
  \mathbf{n}_{V1} = \frac{\mathbf{n}_{V2}}{4\pi R} \\
  n_1 &= \frac{n_2}{4\pi R} , \quad
  n_0 = \frac{n_2}{4\pi R^2}
\end{align}

\section{Comparison with simulation}

\subsection{Hard spheres confined in a spherical cavity}

\fixme{TODO: Create actual Monte Carlo data, and make a comparison.}

%%%%%%%%%%%%%%%%%%%%%%%%%%%%%%%%%%%%%%%%%%%%%%%%%%%%%%%%%%%%
\section{Conclusion}

\appendix

\section{Soft Fundamental-Measure Theory}

We follow the Soft Fundamental-Measure Theory introduced by
Schmidt~\cite{schmidt2000fluid}.  Here I will derive the fundamental
measures for a hard-sphere fluid, so we can see how this works.
Schmidt defines the relationship between the weighting functions as:
\begin{align}
  w_2(r) &= -\frac{\partial w_3(r)}{\partial r} \\
  \mathbf{w}_{v2}(\rr) &= w_2(r)\frac{\rr}{r} \\
  \mathbf{w}_{m2}(\rr) &= w_2(r)\left( \frac{\rr \rr}{r^2}
                              - \frac{\mathbf{\hat{1}}}{3} \right) \\
  w_1(r) &= \frac{w_2(r)}{4\pi r} \\
  \mathbf{w}_{v1}(\rr) &= w_1(r) \frac{\rr}{r} \\
  w_0(r) &= \frac{w_1(r)}{r}
\end{align}
The weighting functions are themselves given by a deconvolution of the
Mayer $f$ function:
\begin{align}
  f(r) &= \exp\left(-\frac{V(r)}{k_BT}\right) - 1 \\
  &= w_0 * w_3 + w_1 * w_2 - \mathbf{w}_{v1} * \mathbf{w}_{v2}
\end{align}
where the $*$ operator denotes a three-dimensional convolution, and a
scalar product between vectors.  Schmidt explains that there is a
simple relationship in fourier space between the Mayer $f$ function
and the weighting function $w_2$:
\begin{align}
  \tilde{w}_2(k) &= \pm \sqrt{ik\tilde{f}(k)} \label{eq:w2fromf}
\end{align}
where the tilde denotes a \emph{one-dimensional} Fourier transform:
\begin{align}
  \tilde{f}(k) &= \int_{-\infty}^\infty f(r) e^{ikr} dr
\end{align}

I am writing this section to explore this formula, and to help me to
understand how to deal with this branch cut (i.e. the sign in
Equation~\ref{eq:w2fromf}).  One interesting challenge here is that
$\tilde{w}_2(k)$ is \emph{not} actually the function that we will use
in our convolutions, which are three-dimensional convolutions.  So we
will still have to go into real space and back again to get our
convolution kernel.  $\ddot\frown$

\section{Hard spheres}

For the hard-sphere fluid,
\begin{align}
  f(r) = - \Theta(r - 2R)
\end{align}
which means that
\begin{align}
  \tilde{f}(k) &= \int_{-\infty}^\infty f(r) e^{ikr} dr \\
  &= -\int_{-2R}^{2R} e^{ikr} dr \\
  &= -\int_{-2R}^{2R} \cos(kr) dr \\
  &= -\frac{2}{k}\sin(2kR)
\end{align}
At this point, we want to solve for $w_2$.
\begin{align}
  \tilde{w}_2(k) &= \pm \sqrt{ik\tilde{f}(k)} \\
  &= \pm \sqrt{-2i\sin(2kR)} \\
  &= \pm \sqrt{-4i\sin(kR)\cos(kR)}
\end{align}
I happen to know that
\begin{align}
  w_2(r) &= \delta(r - R) \\
  \tilde{w}_2(k) &= \int_{-\infty}^\infty w_2(r) e^{ikr} dr \\
  &= \int_{-\infty}^\infty \delta(r-R) e^{ikr} dr \\
  &= e^{ikr} + e^{-ikr + i\Phi}
\end{align}
where in the last step, I use the fact that there is a physical
ambiguity as to whether $w_2$ is even or odd or somewhere in between
when understood in one dimension, since only even values for $r$ make
physical sense.  Thus looking at the square of this, we see
\begin{align}
  \tilde{w}_2(k)^2 &= e^{i2kr} + e^{-i2kr + 2\Phi} + 2e^{i\Phi} \\
  &= -2i\sin(2kR) \\
  &= -2i\frac{e^{i2kR} - e^{-i2kR}}{2i} \\
  &= e^{-i2kR} - e^{i2kR}
\end{align}
This almost makes sense (if $\Phi = \pi/2$), but not quite, because
there is the extra term of $2e^{i\Phi}$, which would give us an extra
$2i$.

This looks like a contradiction.  However, we assumed here that there
was a $\delta$-function at $-R$, and that $f(r)$ was a symmetric
function, simply because the three-dimensional function is symmetric.
What if there is only one spike at $R$ and $f(r)$ is asymmetric?  Then
we would have
\begin{align}
  \tilde{f}(k) &= \int_{-\infty}^\infty f(r) e^{ikr} dr \\
  &= -\int_{-\infty}^{2R} e^{ik2R} dr
\end{align}
This integral isn't very nicely behaved.  But if I were to ignore the
lower bound---which is actually okay for computing the slope, which is
all we need---then I would find that
\begin{align}
  \tilde{f}(k) &= -\frac{e^{ikr}}{ik} + C
\end{align}
When I solve for $\tilde{w}_2$ I get something similar:
\begin{align}
  w_2(r) &= \delta(r - R) \\
  \tilde{w}_2(k) &= \int_{-\infty}^\infty \delta(r-R) e^{ikr} dr \\
  &= e^{ikR}
\end{align}
And now it's obvious that the solution is correct, apart from an
unexplained minus sign in $\tilde{f}$, which I imagine I could find in
time.  So it looks like we want to assume that the slope of $f(r)$ is
zero for $r<0$, and therefore that $w_2(r<0)=0$.

\section{Linear soft potential}

Now let's imagine a potential
\begin{align}
  V(r) =
  \begin{cases}
    V_0 & r < 0 \\
    V_0\left(1 - \frac{r}{r_0}\right) & r < r_0 \\
    0 & r \ge r_0
  \end{cases}
\end{align}
This gives us a Mayer $f$ function of
\begin{align}
  f(r) &= e^{-\beta V(r)} - 1 \\
  f'(r) &=
  \begin{cases}
    \frac{\beta V_0}{r_0} e^{-\beta V_0(1-r/r_0)} & 0 < r < r_0 \\
    0 & \text{otherwise}
  \end{cases} \\
  \tilde{f'}(k) &= \frac{\beta V_0}{r_0}\int_{0}^{r_0} e^{-\beta
    V_0(1-r/r_0)}e^{ikr}dr \\
  &= \frac{\beta V_0}{r_0}e^{-\beta V_0}
  \int_{0}^{r_0} e^{(\beta V_0/r_0 + ik)r} dr
  \\
  &= \frac{\beta V_0}{r_0}e^{-\beta V_0} \frac{1}{\beta V_0/r_0 + ik}\left(
  e^{\beta V_0+ikr_0} - 1 \right)
  \\
  &= \frac{e^{ikr_0} - e^{-\beta V_0}}{1 + \frac{ikr_0}{\beta V_0}}
  \\
  &= \tilde{w}_2(k)^2
\end{align}
That's just a little bit yucky, and I'm not sure I can solve for it in
real space.

Now let's imagine a potential that has some interesting (but
numerically convenient) values for $r<0$
\begin{align}
  V(r) =
  \begin{cases}
    V_0\left(1 - \frac{r}{r_0}\right) & r < r_0 \\
    0 & r \ge r_0
  \end{cases}
\end{align}
This gives us a Mayer $f$ function of
\begin{align}
  f(r) &= e^{-\beta V(r)} - 1 \label{eq:ftrytwo} \\
  f'(r) &=
  \begin{cases}
    \frac{\beta V_0}{r_0} e^{-\beta V_0(1-r/r_0)} & r < r_0 \\
    0 & \text{otherwise}
  \end{cases} \\
  \tilde{f'}(k) &= \frac{\beta V_0}{r_0}\int_{-\infty}^{r_0} e^{-\beta
    V_0(1-r/r_0)}e^{ikr}dr \\
  &= \frac{\beta V_0}{r_0}e^{-\beta V_0}
  \int_{-\infty}^{r_0} e^{(\beta V_0/r_0 + ik)r} dr
  \\
  &= \frac{\beta V_0}{r_0}e^{-\beta V_0} \frac{1}{\beta V_0/r_0 + ik}e^{\beta V_0+ikr_0}
  \\
  &= \frac{e^{ikr_0}}{1 + \frac{ikr_0}{\beta V_0}}
  \\
  &= \tilde{w}_2(k)^2
\end{align}
Now this is nicer.  The top is really easy to square root.  The bottom
is ever so slightly harder, but not really hard.
\begin{align}
  \tilde{w}_2(k) &=
  \frac1{\sqrt{2}}\frac{e^{ikr_0/2}}{\sqrt{\sqrt{1+\frac{kr_0}{\beta
          V_0}}+1} +
      i\sqrt{\sqrt{1+\frac{kr_0}{\beta V_0}}-1}}
\end{align}
Okay, that's a bit ugly, and I'm not sure that we can Fourier
transform it to get a useful result.  We could potentially approximate
it in the limit that $\beta V_0 \gg 1$, although any simple power
series expansion would require us to prove that $kr_0\gg1$ is
unimportant, possibly by arguing that phase cancellation will cause it
to be unimportant when we evaluate our actual $w_2(r)$.
\begin{align}
  w_2(r) &= \int_{-\infty}^\infty \tilde{w}_2(k) \\
 &=
  \frac1{\sqrt{2}}\frac{e^{ikr_0/2}}{\sqrt{\sqrt{1+\frac{kr_0}{\beta
          V_0}}+1} +
      i\sqrt{\sqrt{1+\frac{kr_0}{\beta V_0}}-1}}
\end{align}

It occurs to me now that Schmidt may have assumed right after his
Equation~A7 that $w_2(r<0)=0$, in which case this attempt won't work:
we'll need to make things zero for $r<0$ if we're going to use this
Fourier transform trick at all.  On the other hand, if this comes out
to be \emph{effectively} zero for reasonable temperatures, perhaps
this will be a good approach.  After all, I'm interested in $\beta V_0
\gg 1$, which means that the Mayer function is essentially flat in
Equation~\ref{eq:ftrytwo} for $r<0$.

\bibliography{paper}% Produces the bibliography via BibTeX.

\end{document}

