% TODO:

% Write ``Liquid-vapor interface'' section A.

% Look up question-mark references (citations).

\documentclass[letterpaper,twocolumn,amsmath,amssymb,prb]{revtex4-1}
\usepackage{graphicx}% Include figure files
\usepackage{dcolumn}% Align table columns on decimal point
\usepackage{bm}% bold math
\usepackage{color}

\newcommand{\red}[1]{{\color{red} #1}}
\newcommand{\blue}[1]{{\bf \color{blue} #1}}
\newcommand{\green}[1]{{\bf \color{green} #1}}
\newcommand{\rr}{\textbf{r}}
\newcommand{\kk}{\textbf{k}}
\newcommand{\refnote}{\red{[ref]}}

\newcommand{\fixme}[1]{\red{[#1]}}

\begin{document}
\title{Soft Fundamental Measure Theory Functional for the
  Weeks-Chandler-Anderson Repulsive Potential}

\author{Eric J. Krebs}
\affiliation{Department of Physics, Oregon State University, Corvallis, OR 97331}

\author{Patrick Kreitzberg}
\affiliation{Department of Physics, Oregon State University, Corvallis, OR 97331}

\author{David Roundy}
\affiliation{Department of Physics, Oregon State University, Corvallis, OR 97331}

%%%%%%%%%%%%%%%%%%%%%%%%%%%%%%%%%%%%%%%%%%%%%%%%%%%%%%%%%%%%
\begin{abstract}
\fixme{ We will demonstrate a easy-to-implement variation on the Soft
  Fundamental Measure Theory (SFMT) approach for treating fluids
  having a soft repulsive potential.  We will show that over a wide
  range of temperatures and densities our theory gives accurate
  results, as tested by comparison with Monte Carlo simulation of a
  fluid using a typical soft repulsive potential.  We will present
  results for the homogeneous fluid, the radial distribution function,
  and for inhomogeneous distributions at a fluid-solid interface. }
\end{abstract}

\maketitle

%%%%%%%%%%%%%%%%%%%%%%%%%%%%%%%%%%%%%%%%%%%%%%%%%%%%%%%%%%%%a
\section{Introduction}

The idea of liquids as composed of hard spheres dates back over two
millenia~\cite{lucretius}.  In the 20th century, we came to understand
atoms as inherently soft, but it was shown that their repulsion could
still be accurately described using a hard-sphere model, provided the
radius is chosen to be temperature
dependent~\cite{rowlinson1964statistical, barker1967perturbation,
  andersen1971relationship}.  These works cemented the hard-sphere
model as the reference system of choice for the theory of
liquids~\cite{gil-villegas-1997-SAFT-VR, clark2006developing,
  lafitte2013accurate}.  One reason for the wide use of the
hard-sphere fluid hard sphere fluid as a reference system is that it
is widely studied and well understood, but only for the homogeneous
fluid~\cite{carnahan1969equation}, but also in the more challenging
case of the inhomogeneous fluid~\cite{rosenfeld1989, rosenfeld1997,
  roth2002whitebear}.  However, the hard-sphere fluid remains a
non-physical model, which is also numerically inconvenient in its use
of delta functions.

Fundamental Measure Theory~(FMT) is a classical density functional
theory for the free energy of the hard-sphere fluid developed by
Rosenfeld~\cite{rosenfeld1989}.  Due to its combination of
computational efficiency with accuracy, FMT has since been used as the
basis for a wide variety of classical density
functionals~\cite{cuesta1997dimensional, hansen2009fundamental,
  marechal2013density}.
%
In 1999, Schmidt introduced \emph{Soft} Fundamental Measure Theory
(SFMT)~\cite{schmidt1999density}, which directly treats soft repulsive
potentials in a framework based on the highly successful FMT developed
by Rosenfeld~\cite{rosenfeld1989}.  SFMT has been used to describe the
behavior of a star polymer in solution~\cite{schmidt2000density,
  groh2001density, kim2001adsorption, sweatman2002fundamental}, as
well as repulsive potentials applicable to
atoms~\cite{schmidt2000fluid, sweatman2002fundamental}.

%% In his 2010 review of FMT, Roth states that the most important future
%% developments in equilibrium DFT will involve treating soft repulsions
%% and attractions~\cite{}.

In this paper, we will apply SFMT to study the Weeks-Chandler-Anderson
repulsive potential~\cite{weeks1971}.  This potential reproduces the repulsive force of
a Lennard-Jones interaction, which makes it an ideal model for
interatomic repulsion.

\section{Soft Fundamental Measure Theory}

Schmidt introduced Soft Fundamental Measure Theory~(SFMT) based on
FMT~\cite{schmidt1999density} where spheres can penetrate at an
energy cost determined by a pair potential. SFMT was created to
generalize the weight functions of FMT while keeping the range of
the weight functions at the soft sphere radius. Schmidt gives three
criteria for the weight functions: they will deconvolve the Mayer
function, yield the exact zero-dimensional single cavity limit, and
they give a reasonable approximation fo the multi-cavity
limit~\cite{schmidt1999density}.

Schmidt tested SFTM for a mixture of soft and hard spheres as well as
for star polymers~\cite{schmidt2000density, groh2001density}. SFMT was
also tested in a number of other papers~\cite{rosenfeld2000fluid}
\fixme{more citations}.

\section{Barker-Henderson hard sphere}

Another approach to account for the temperature dependence of
realistic fluids is the from Barker-Herson's
approach\cite{barker1967perturbation}. Their theory is a model for
taking any general repulsive pair potential and creating a hard sphere
reference fluid with a temperature dependent diameter along with
higher order repulsive terms. The had sphere diameter is
\begin{align}
  d = \int_0^\sigma\left( 1 - e^{\beta V(r)} \right)\mathrm{dr}
\end{align} 
where $\beta = \frac{1}{kT}$, $V(r)$ is the pair potential, and $\sigma$
is the radius for which $V(r)$ goes to zero. We use this result in a White
Bear hard sphere DFT to compare against our approximation of SFMT.


\section{Theory}

We use the White Bear version of the Fundamental-Measure Theory~(FMT)
functional published in reference~\cite{roth2002whitebear}.  The FMT
functional describes the excess free energy of a hard-sphere fluid.
This particular FMT reduces to the Carnahan-Starling equation of state
for homogeneous systems.
\begin{equation}
A_\textit{HS}[n] = k_B T \int \left(\Phi_1(\rr) + \Phi_2(\rr) + \Phi_3(\rr)\right) d\rr \; ,
\end{equation}
with integrands
\begin{align}
\Phi_1 &= -n_0 \ln\left( 1 - n_3\right)\\
\Phi_2 &= \frac{n_1 n_2 - \mathbf{n}_{V1} \cdot\mathbf{n}_{V2}}{1-n_3} \\
\Phi_3 &= (n_2^3 - 3 n_2 \mathbf{n}_{V2} \cdot \mathbf{n}_{V2}) \frac{
  n_3 + (1-n_3)^2 \ln(1-n_3)
}{
  36\pi n_3^2\left( 1 - n_3 \right)^2
} ,
\end{align}
using the weighted densities
\begin{align}
  n_3(\rr) &= \int n(\rr') \Theta(\left|\rr - \rr'\right| - R) d\rr' \\
  n_2(\rr) &= \int n(\rr') \delta(\left|\rr - \rr'\right| - R) d\rr'
\end{align}
\begin{align}
  \mathbf{n}_{V2} &= \mathbf{\nabla} n_3 , \quad
  \mathbf{n}_{V1} = \frac{\mathbf{n}_{V2}}{4\pi R} \\
  n_1 &= \frac{n_2}{4\pi R} , \quad
  n_0 = \frac{n_2}{4\pi R^2}
\end{align}

\section{Theory}

Soft Fundamental-Measure Theory~(SFMT) is a generalization of FMT to
soft potentials. In SFMT, the weight functions are modified but retain:
the direct correlation in the low-density limit, the exact free energy
in the zero-dimensional cavity limit, and relation to the generating
weight function $w_3(\rr)$~\cite{schmidt1999density} \fixme{this is an assumption
  as stated in Schmidt 1999}.  The new weight functions are also made
to deconvolve the Mayer function   

\begin{equation}
f(r) = \exp (-\beta V(r)) - 1
\end{equation}
and to give an approximate multi-cavity limit. Schmidt furthur shows that\cite{schmidt2000fluid}

\begin{equation}\label{eq:mayerandw2}
\frac{d f(r)}{dr} = \int dr' w_2(r') w_2 (r-r').
\end{equation}

\begin{figure}
\begin{center}
\includegraphics[width=\columnwidth]{figs/potential-plot}
\end{center}
\caption{plot of potential.}
\label{fig:potential-plot}
\end{figure}

\begin{figure}
\begin{center}
\includegraphics[width=\columnwidth]{figs/w2-comparison}
\end{center}
\caption{plot of potential.}
\label{fig:w2-comparison}
\end{figure}

We use a WCA (Weeks-Chandler-Anderson) pair
potential\cite{weeks1971}, which is the Lennard-Jones potential
with the attractive portion removed:
\begin{align}
  V_{wca}(r) =
  \begin{cases}
    4\epsilon \left[ \left(\frac{\sigma}{r}\right)^{12} -
    \left(\frac{\sigma}{r}\right)^{6} \right] + \epsilon, & 0 < r < 2R \\
    0, & \textrm{otherwise}.
  \end{cases}
\end{align}
We approximate the Mayer function as
\begin{align}
  f(r) \approx \tfrac12 \left( \mathrm{erf}\left( \frac{r - \alpha}{\Xi} \right) - 1 \right)
\end{align}
then
\begin{align}
  V_{erf}(r) \approx -kT\ln\left[\tfrac12 \left( \mathrm{erf}\left( \frac{r -
    \alpha}{\Xi} \right) + 1 \right) \right].
\end{align}
Expanding $V_{erf}(r)$ in a power series gives the approximation
\begin{align}
  V_{erf}(\alpha) &\approx k_BT \ln 2.
\end{align}
Setting $V_{wca}$ and and the approximate $V_{erf}$ equal at $r=\alpha$, we solve
for $\alpha$ and find that
\begin{align}
  \alpha = \sigma \left( \frac{2}{1 + \sqrt{\frac{k_BT}{\epsilon}
        \ln 2}} \right)^{\frac{1}{6}}.
\end{align}
Next, we set $V_{wca}$ and both potential's slopes equal at $r = \alpha$ and find that
\begin{align}
  \Xi = \frac{\alpha}{6\sqrt{\pi} \left( \sqrt{\frac{\epsilon}{k_BT} \ln
      2} + \ln 2 \right)}
\end{align}

The approximate weight functions from the erf model are
\begin{align}
  w_2(r) &= \frac{1}{\Xi \sqrt{\pi}} e^{-\frac{(r-\alpha/2)^2}{\Xi^2}} \\
  w_3(r) &= \tfrac12 ( 1 - \mathrm{erf}((r-\alpha/2)/\Xi) ).
\end{align}
with all other weight functions retaining their relation to $w_2(r)$
as in FMT.

\subsection{Weighting functions in fourier space}

We Fourier transform our weight functions because simulations are
carried out in Fourier space.  We find that
\begin{align}
  \tilde{w}_3(k) &= \frac{4\pi}{k} \int_0^\infty r w_3(r) \sin(kr) dr \\ 
  &= \frac{4\sqrt{\pi}}{k^3}\int_{-\sigma/2a}^\infty \left[
    \sin(k(au+\frac{\sigma}{2})) - k(au + \frac{\sigma}{2})
    \cos(k(au+\frac{\sigma}{2}))\right] e^{-u^2} du.
\end{align}
This is not an analytic function, but we assume that we are working at
low enough temperatures so that our function reduces to zero by $u=
-\frac{\sigma}{2a}$. We extend the lower limit to $-\infty$ then
\begin{align}
  \tilde{w}_3(k) &\approx
  \frac{4\pi}{k^3}e^{-\left(\frac{ak}{2}\right)^2}\left[ \left(1 +
    \frac{a^2k^2}{2} \right) \sin\left(\frac{k\sigma}{2}\right) -
    \frac{k}{2} \sigma\cos\left(\frac{k \sigma}{2}\right) \right]
\end{align}

We apply the same method to the other weight functions. These are
\begin{align}
  \tilde{w}_2(k) &=\frac{2\pi}{k} e^{-\left(\frac{ak}{2} \right)^2}
  \bigg(a^2 k \cos\left(\frac{k\sigma}{2}\right) \notag \\
  & \hspace{8em}+\sigma \sin\left(\frac{k\sigma}{2}\right) \bigg) \\ 
%%%%%%%%%%%%%%%%%%%%%%%%%%%%%%%%%%%%%%%%%%%%%%%%%%%%%%%%%%%%
  \tilde{w}_1(k) &=
  \frac{1}{k}e^{-\left(\frac{ak}{2} \right)^2}
  \sin\left(\frac{k\sigma}{2}\right) \\
%%%%%%%%%%%%%%%%%%%%%%%%%%%%%%%%%%%%%%%%%%%%%%%%%%%%%%%%%%%%
  \tilde{\mathbf{w}}_{2V}(\kk) &=
  \frac{i \pi}{k} e^{-\left(\frac{ak}{2} \right)^2}\bigg[ \left(\sigma^2 - a^4k^2\right)
    \cos\left(\frac{k\sigma}{2}\right) \notag\\ 
    & \hspace{5em}- 2\sigma \left(a^2k + \frac{1}{k}
    \right)\sin\left(\frac{k\sigma}{2}\right) \bigg]
  \mathbf{\hat{k}} \\ 
%%%%%%%%%%%%%%%%%%%%%%%%%%%%%%%%%%%%%%%%%%%%%%%%%%%%%%%%%%%%
  \mathbf{\tilde{w}}_{1V}(k)&= \frac{i}{k}
   e^{-\left(\frac{ak}{2} \right)^2} \bigg[ \frac{\sigma}{2}
    \cos\left(\frac{k\sigma}{2}\right) \notag\\
    & \hspace{6em}- \left( \frac{a^2k}{2} +
    \frac{1}{k} \right) \sin\left(\frac{k\sigma}{2}\right) \bigg]
  \mathbf{\hat{k}}
\end{align}
$\tilde{w}_0(k)$ contains a $\frac{1}{r}$ term that expands the
integrand  as an infinite series. The first two terms were kept and
could be expressed in terms of $\tilde{w}_1(k)$ and
$\tilde{w}_2(k)$.
\begin{align}
  \tilde{w}_0(k) &= \frac{2}{\sigma} \left[2\tilde{w}_1(k) - \frac{1}{2 \pi
      \sigma}\tilde{w}_2(k) \right]
\end{align}
 
\section{Comparison with simulation}

\subsection{Homogeneous limit}

As a simple test for the equation of state, we compare the theory for
a homogeneous soft-sphere fluid and compare to Monte-Carlo (MC)
simulation. The results show very good agreement for lower densities
across all temperatures shown in Figure \ref{fig:p-vs-packing}, and at
higher temperatures not shown. Differences between the DFT and MC results
become quite apparent at $n^*=0.6$ and above which includes fluid to solid
transitions starting at $n^*=0.8$ that are not predicted by our DFT.

\begin{figure}
\begin{center}
\includegraphics[width=3.5in]{figs/p-vs-T}
\end{center}
\caption{Pressure versus packing fraction.  The SFMT result is plotted
  as solid lines, with simulation results as solid circles.  The
  pressure is divided by the hard-sphere pressure in each case, in
  order to highlight the effect of the soft interactions.}
\label{fig:p-vs-packing}
\end{figure}

\subsection{Soft spheres radial distribution function}

\begin{figure}
\begin{center}
\includegraphics[width=3.5in]{figs/radial-distribution}
\end{center}
\caption{Radial distribution functions with 0.6(top) and 1.0(bottom) reduced densities.}
\label{fig:radial-distribution}
\end{figure}

For three dimensional comparisons, we plot radial distribution
funtions.  Results for reduced densitied below $n^* = 0.60$ have been
ommitted as they were exact over temperature ranges from $T^*=0.01$ to
$T^*=10$. The top frame of Figure \ref{fig:radial-distribution} shows
the radial distribution for a range of temperatures at a redued
density of $n^* = 0.60$. We see very good agreement of our DFT with
Monte-Carlo simulation at this reduced density. Our DFT's behaviour at
higher temperatures are in almost exact agreement, while the lower
temperatures have slight disagreement just after the first peak at
contact and in the subsequent oscillations where it
\fixme{underestimate} the peaks. Comparison with the Barker-Henderson
results shows our DFT to have a similar magnitude of error to the
exact radial distribution function.

In Figure \ref{fig:radial-distribution}(bottom), we plot the results
for a reduced density of $n^*=1.0$ for three different
tempreatures. While both Barker-Henderson hard spheres and our DFT
both overstimate the density at contact, ours is marginally worse at
the lowest temperature shown here. For the density oscillations, our
DFT's error is as good as the Barker-Henderson results.

\subsection{Soft spheres near a hard wall}

As a test for one dimensional behavior, we will look at the density
profile for the soft sphere fluid near a hard wall.


\begin{figure}
\begin{center}
\includegraphics[width=3.5in]{figs/hard-walls}
\end{center}
\caption{Density distribution near a hard wall.}
\label{fig:hard-walls}
\end{figure}

\subsection{Soft spheres near a soft wall}

If we construct a wall of WCA spheres of density $rho$, we
find that the potential at a distance z from the wall is

\begin{align}
V_{SW}(0 < z \leq R_0) &= 2\pi\rho\epsilon\Big[ \frac{z^3-R_0^3}{6} +
  \frac{2\sigma_W^{12}}{45} \left(\frac{1}{z^9}-\frac{1}{R_0^9}\right)
  \notag \\
  &+ \left( R_0 - z \right)\left(\frac{R_0^2}{2} +
  \frac{\sigma_W^6}{R_0^4} - \frac{2\sigma_W^{12}}{5R_0^{10}}\right)
  \notag \\
  &\hspace{2cm}+\frac{\sigma_W^6}{3}\left(\frac{1}{R_0^3}-\frac{1}{z^3}
  \right) \Bigg].
\label{eq:soft-wall-potential}
\end{align}
The distance $R_0$ is equal to $R_W + R_f$, where $R_W$ is the radius of a
sphere which makes up the wall and $R_f$ is the radius of a sphere
from the fluid. Both $\epsilon$ and $\sigma_W$ are Lennard-Jones
parameters between the wall and the fluid.

In Figure \ref{fig:soft-walls}, we compare our soft sphere DFT
against MC simulation and a Barker-Henerson fluid near a soft wall
with the potential given in Equation \ref{eq:soft-wall-potential}. We
plot the reduced density versus reduced distance from the surface of
the wall for $n^*=0.6$ at different temperatures. Again, the results
of our soft sphere fluid is as good as the Barker-Henderson(BH) fluid
overall.

\begin{figure}
  \includegraphics[width=\columnwidth]{figs/soft-walls}
  \caption{Soft walls plot, generated with new code generator.}
\label{fig:soft-walls}
\end{figure}

%%%%%%%%%%%%%%%%%%%%%%%%%%%%%%%%%%%%%%%%%%%%%%%%%%%%%%%%%%%%
\section{Comparison with experiment}

Lennard Jones data we will use will be from
papers~\cite{mikolaj2004structure, eggert2002quantitative, yarnell1973structure}.
For Lennard Jones Argon, $\sigma = 3.405 ~\textrm{\AA}$ and $\epsilon = 119.8$~K~\cite{verlet1967computer}.
\begin{figure*}
  \begin{center}
    \includegraphics[width=7.0in]{figs/argon-plots}
  \end{center}
  \caption{Radial distribution functions of Argon.}
  \label{fig:argon-plots}
\end{figure*}

\section{Conclusion}

In conclusion, our theory is \fixme{pretty awesome}.

\bibliography{paper}% Produces the bibliography via BibTeX.

\end{document}
