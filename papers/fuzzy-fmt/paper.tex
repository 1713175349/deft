% TODO:

% Write ``Liquid-vapor interface'' section A.

% Look up question-mark references (citations).

\documentclass[letterpaper,twocolumn,amsmath,amssymb,prb]{revtex4-1}
\usepackage{graphicx}% Include figure files
\usepackage{dcolumn}% Align table columns on decimal point
\usepackage{bm}% bold math
\usepackage{color}

\newcommand{\red}[1]{{\bf \color{red} #1}}
\newcommand{\blue}[1]{{\bf \color{blue} #1}}
\newcommand{\green}[1]{{\bf \color{green} #1}}
\newcommand{\rr}{\textbf{r}}
\newcommand{\refnote}{\red{[ref]}}

\newcommand{\fixme}[1]{\red{[#1]}}

\begin{document}
\title{Smoothing the Fundamental Measure Theory Functional for Hard Spheres}

\author{Eric J. Krebs}
\affiliation{Department of Physics, Oregon State University, Corvallis, OR 97331}

\author{David Roundy}
\affiliation{Department of Physics, Oregon State University, Corvallis, OR 97331}

%%%%%%%%%%%%%%%%%%%%%%%%%%%%%%%%%%%%%%%%%%%%%%%%%%%%%%%%%%%%
\begin{abstract}
\fixme{ We examine a functional based on SFMT~\cite{schmidt2000fluid}.
  We pick a slightly soft potential, and demonstrate that it gives
  good results.}
\end{abstract}

\maketitle

%%%%%%%%%%%%%%%%%%%%%%%%%%%%%%%%%%%%%%%%%%%%%%%%%%%%%%%%%%%%
\section{Introduction}



\section{Fundamental-Measure Theory}

We use the White Bear version of the Fundamental-Measure Theory~(FMT)
functional published in reference~\cite{roth2002whitebear}.  The FMT
functional describes the excess free energy of a hard-sphere fluid.
This particular FMT reduces to the Carnahan-Starling equation of state
for homogeneous systems.
\begin{equation}
A_\textit{HS}[n] = k_B T \int \left(\Phi_1(\rr) + \Phi_2(\rr) + \Phi_3(\rr)\right) d\rr \; ,
\end{equation}
with integrands
\begin{align}
\Phi_1 &= -n_0 \ln\left( 1 - n_3\right)\\
\Phi_2 &= \frac{n_1 n_2 - \mathbf{n}_{V1} \cdot\mathbf{n}_{V2}}{1-n_3} \\
\Phi_3 &= (n_2^3 - 3 n_2 \mathbf{n}_{V2} \cdot \mathbf{n}_{V2}) \frac{
  n_3 + (1-n_3)^2 \ln(1-n_3)
}{
  36\pi n_3^2\left( 1 - n_3 \right)^2
} ,
\end{align}
using the weighted densities
\begin{align}
  n_3(\rr) &= \int n(\rr') \Theta(\left|\rr - \rr'\right| - R) d\rr' \\
  n_2(\rr) &= \int n(\rr') \delta(\left|\rr - \rr'\right| - R) d\rr'
\end{align}
\begin{align}
  \mathbf{n}_{V2} &= \mathbf{\nabla} n_3 , \quad
  \mathbf{n}_{V1} = \frac{\mathbf{n}_{V2}}{4\pi R} \\
  n_1 &= \frac{n_2}{4\pi R} , \quad
  n_0 = \frac{n_2}{4\pi R^2}
\end{align}

\section{Theory}

We begin by examining the Mayer function:

\begin{equation}
f(r) = \exp (-\beta V(r)) - 1.
\end{equation}

From Schmidt\cite{schmidt2000fluid},

\begin{equation}\label{eq:mayerandw2}
\frac{d f(r)}{dr} = \int dr' w_2(r') w_2 (r-r')
\end{equation}

%% We have a choice to either start with a pair potential, $V(r)$, and
%% derive the appropriate $w_2(r)$, or we start with the latter and find
%% the former. All other weighting functions can be derived from
%% $w_2(r)$ (see Appendix A).  Another way to attack this problem is to
%% pick a pair potential and determine a $w_2(r)$ to ``fit'' to the
%% potential.

We use a harmonic pair potential

\begin{align}
  V(r) =
  \begin{cases}
    V_0 \left ( 1 - \frac{r}{r_0} \right )^2 & 0 \leq r \leq r_0\\
    0 & otherwise
  \end{cases}
\end{align}
which gives the Mayer function 
\begin{align}\label{eq:harmonicmayer}
  f(r) =
  \begin{cases}
    e^{-\beta V_0 \left( 1 - \frac{r}{r_0} \right)^2} - 1 & 0 \leq r
    \leq r_0 \\
    0 & otherwise.
  \end{cases} 
\end{align}
We use this Mayer function to approximate a $w_2(r)$ by choosing a
simple form similar to $f(r)$ with fitting parameters. We fit the
integral of boths sides of equation \ref{eq:mayerandw2} and the
slope at $r=r_0$ to find our approximate $w_2(r)$: 
\begin{equation}
  w_2(r) = \frac{2\gamma r}{(\sqrt{\pi \gamma} - 1)R^2} e^{-\gamma
    \left ( 1 - \frac{r}{R} \right )^2}\Theta(r) \Theta (R - r)
\end{equation}
where
\begin{equation}
  \gamma = 2 \left(\frac{\sqrt{\pi \beta V_0} + \sqrt{\pi \beta V_0 -
    16\sqrt{\beta V_0}}}{8}\right)^2
\end{equation}

Comparisons of $w_2(r)\ast w_2(r)$ and $\frac{df}{dr}$ for the harmonic
pair potential are shown in fig. \ref{fig:w2convolves}.

where B is a constant to be determined by fitting the integral of
$f'(r)$ with the integral of $w_2(\rr)\ast W_2(\rr)$ and $R =
\frac{r_0}{2}$.  $\gamma$ is determined by the slope at $r = r_0$.  
Now we'll first want to convolve $w_2(r)$ with
itself:
\begin{align*}
  w_2 \ast w_2 &= \int dr' w_2 (r')w_2(r - r') \\ 
     &= \int dr' \left(B r' e^{- \gamma \left (1 - \frac{r}{R} \right )^2}
                              \Theta_1 (r') \Theta_2 (R - r') \right) \\
     &\times \left( B (r-r') e^{- \gamma \left(1 - \frac{r-r'}{R}
                                \right)^2}\Theta_{1'} (r-r') \Theta_{2'} (R -
                                (r-r')) \right) \\
     &= \alpha \int dr' r'(r-r') \exp \left [-\frac{2\gamma}{R^2} \left (r' -
                              \frac{r}{2} \right)^2 \right ] \Theta_1
                              \Theta_2 \Theta_{1'} \Theta_{2'}
\end{align*}
where we have substituted $\alpha = B^2 e^{- 2 \gamma \left(1 -
  \frac{r}{R} + (\frac{r}{2R})^2 \right )}$. The Theta functions create two separate
cases to integrate.  The case that is, perhaps, more important for us
is is when $R \leq r \leq 2R$.  The integral is

\begin{align}
(w_2 \ast w_2)_{r \geq R} &= \alpha \int_{r - R}^R dr' r'(r-r') \exp \left [ -
   \frac {2\gamma}{R^2}\left( r' - \frac{r}{2} \right)^2 \right]
\end{align}

Now, for $ 0 \leq r \leq 2R$
\begin{align}
  w_2 \ast w_2 = \frac{\beta V_0}{R} \left( 1 - \frac{r}{2R} \right)
              e^{-\beta V_0 \left( 1 - \frac{r}{2R} \right)^2}
\end{align}
If we integrate $f'(r)$ for the quadratic potential we get
\begin{align}
  \int_0^{\infty} f'(r)dr &= f(\infty) - f(0)\\
                    &= 1 - e^{-\beta V_0}
\end{align}
For the $w_2(r)$ convolved with itself,
\begin{widetext}
\begin{align*}
  w_2\ast w_2 &= \alpha \frac{R}{4}\sqrt{\frac{\pi}{2\gamma}}
               \left( r^2 - \frac{R^2}{\gamma}
               \right)\textrm{erf} \left( \sqrt{2\gamma} \left(1 - \frac{r}{2R}
               \right) \right)%\\
               %&
               + \alpha \frac{R^2}{2\gamma} \left( R - \frac{r}{2}
               \right) e^{-2\gamma \left( 1 - \frac{r}{2R} \right)^2}.
\end{align*}
\end{widetext}
%% We substitute $u = \sqrt{2\gamma}\left( 1 - \frac{r}{2R} \right)$ and
%% plug in our expression above for $\alpha$ so that,
\begin{align*}
  w_2\ast w_2 &= \frac{B^2 R}{\sqrt{2\gamma}} \left[
               \frac{\sqrt{\pi}}{4} \left( r^2 - \frac{R^2}{\gamma}
               \right) e^{-u^2} \textrm{erf}(u)
               + \frac{R^2 u}{\sqrt{2\gamma}}
               e^{-2u^2} \right]
\end{align*}
%% Now we'll need to integrate this and set it to the integral of $f'(r)$
%% to solve for the constant $B$. Since this expression is only the part
%% of $w_2(r)$ from $R \leq r \leq 2R$, we need to justify only using
%% this portion by saying that at low temps (which is our region of
%% interest), $w_2(r)$ below $R$ (or $\frac{r_0}{2}$) is nearly zero.
%% This upper part of $w_2(r)$ which we have used above goes to very
%% small numbers in the region $0 \leq r \leq R$.  Both parts of $w_2(r)$
%% are nearly zero for $r \leq 0$, so we can carry out the integrate to
%% $-\infty$ so that the error function integrals are analytic.
%% Therefore, we can integrate from $-\infty$ to $r_0$ and it is
%% approximately correct.

\begin{figure}
\begin{center}
\includegraphics[width=3.5in]{figs/w2convolves}
\end{center}
\caption{Comparison of the $-f'(r)$ function with various $w_2(r)\ast
  w_2(r)$.}
\label{fig:w2convolves}
\end{figure}

If we do this integration on over $r$, then the limits for $u$ are
$\infty$ to zero.  I'll set \fixme{for now to save space} $w_2\ast
w_2 = w_{2c}(u)$. Then,
\begin{widetext}
\begin{align}
\int_{\infty}^0 w_{2c}(u) du &= -\frac{B^2 R^4}{2 \gamma^2}
           \int_\infty^0 \left[ \frac{\sqrt{\pi}}{2} \left( 4\gamma - 1 -
           4\sqrt{2\gamma}u + 2 u^2 \right) e^{-u^2} \textrm{erf}(u) +
           u e^{-2u^2} \right]du \\
           &= \left[ \frac{B R^2}{2\gamma} (\sqrt{\pi \gamma} -1) \right]^2
\end{align}
\end{widetext}
Now, to find B we set
\begin{align}
  \int_0^{\infty} f'(r)dr &= \int_{-\infty}^0 w_2(r)\ast w_2(r) dr\\
   1 - e^{-\beta V_0} &= \left[ \frac{B R^2}{2\gamma} (\sqrt{\pi
      \gamma} -1)  \right]^2
 \end{align}
To find B, we approximate the integral so that it is from $-\infty$ to $r_0$.
\begin{align}
  \int_{-\infty}^{r_0} f'(r)dr &= 1 \\
  \implies B &= \frac{2\gamma}{(\sqrt{\pi \gamma}-1)R^2}
\end{align}

\subsection{Weighting functions in real space}

\begin{align}
  w_2(r) &=\frac{2\gamma r}
  {(\sqrt{\pi \gamma}-1)R^2}e^{-\gamma \left ( 1 - \frac{r}{R} \right)^2}
           \Theta(r) \Theta(R - r )\\
  w_3(r) &= \int_r^\infty w_2(r') dr'\\
         &= -B \int_r^R r' e^{-\gamma \left(1- \frac{r'}{R} \right)^2}dr'\\
         &= \frac{B R^2}{2\gamma}\left[ 1 - e^{-\gamma
      \left(1-\frac{r}{R} \right)^2} - \sqrt{\pi \gamma} \mathrm{erf} \left[
        \sqrt{\gamma} \left( 1- \frac{r}{R} \right) \right] \right]\\
  w_1(r) &= \frac{B}{4\pi}e^{-\gamma \left ( 1 - \frac{r}{R}
            \right)^2} \Theta_1(r) \Theta_2 (R - r)\\
  w_0(r) &= \frac{B}{4\pi r}e^{-\gamma \left ( 1 - \frac{r}{R}
            \right)^2} \Theta_1(r) \Theta_2 (R - r)\\
  \mathbf{w}_{2V}(\rr) &= B e^{-\gamma \left ( 1 - \frac{r}{R} \right)^2}
           \Theta_1(r) \Theta_2 (R - r) \rr\\
  \mathbf{w}_{1V}(\rr) &= \frac{B}{4\pi r} e^{-\gamma \left ( 1 - \frac{r}{R} \right)^2}
           \Theta_1(r) \Theta_2 (R - r )\rr
\end{align}

\subsection{Weighting functions in fourier space}\

\begin{align}
  \tilde{w}_2(k) &= \frac{4\pi B}{k} \int_{0}^{\infty} r^2 
   e^{-\gamma \left(1- \frac{r}{R} \right)^2} \sin(kr) dr \\
  \tilde{w}_3(k) &=
\end{align}
\begin{align}
  \tilde{w}_1(k) &= \int_0^\infty \int_0^{2\pi} \int_{-1}^{1} w_1(r)
           e^{-ikr\cos\theta} r^2drd\phi d\cos\theta
    \\
    &= \frac{4\pi}{k} \int_0^\infty r w_1(r) \sin(kr) dr
    \\
    &= -\frac{2\pi}{ik} \int_{-\infty}^\infty r w_1(|r|) e^{-ikr} dr
    \\
    &= \frac{2\pi}{k} \int_0^{R} r \frac{B}{4\pi} e^{-\gamma\left(1-\frac{|r|}{R}\right)^2} \sin(kr) dr
    \\
    &= -\frac{B}{ik} \int_{-R}^{R} r e^{-\gamma\left(1-\frac{|r|}{R}\right)^2} e^{-ikr} dr
    \\
    &= -\frac{B}{ik} \int_{-R}^{R} r
    e^{-\gamma\left(1-\frac{|r|}{R}\right)^2 - ikr} dr
    \\
    x &\equiv \frac{r}{R} \\
    \\
    \tilde{w}_1(k) &=
    -\frac{BR^2}{ik} \int_{-1}^{1} x
    e^{-\gamma\left(1-|x|\right)^2 - ikRx} dx
    \\
    &=
    -\frac{BR^2}{ik}
    \bigg(
    \int_{0}^{1} x e^{-\gamma\left(1-x\right)^2 - ikRx} dx
    \notag\\&\quad\quad\quad\quad\quad\quad+
    \int_{-1}^{0} x e^{-\gamma\left(1+x\right)^2 - ikRx} dx
    \bigg)
\end{align}
Eric has checked this, and it looks fine, and looks solvable by
Mathematica.  The final integral is alternatively solvable by
completing the square in the exponent.
\begin{align}
  \tilde{w}_0(k) &= \frac{B}{k} \int_0^\infty \frac{w_2(r)}{r} \sin(kr) dr
    \\
  \mathbf{\tilde{w}}_{2V}(k) &= \\
  \mathbf{\tilde{w}}_{1V}(k) &= 
\end{align}

\subsection{Homogeneous limit}

\begin{align}
  \int w_3(\rr)d\rr &= \frac{4\pi BR^2}{2\gamma}\int_0^R r^2 \bigg\{1-
                     e^{-\gamma \left(1-\frac{r}{R} \right)^2} \notag\\
                    &-\sqrt{\pi \gamma} \mathrm{erf}\left[ \gamma \left(1 - \frac{r}{R}
                      \right) \right] \bigg\} d\rr\\
                     &= \frac{\pi R
                       e^{-\gamma}}{3\gamma^{3/2}(\sqrt{\pi \gamma})}
                     \bigg\{2\sqrt{\gamma}
                     \left(8e^\gamma(1+\gamma)-2\gamma -5 \right) \notag\\
                     &- e^\gamma \sqrt{\pi} \left(4\gamma^2 +12\gamma
                     +3 \right) \mathrm{erf}(\sqrt{\gamma}) \bigg\}\\
  \int w_2(\rr)d\rr &= 4\pi B \int_0^R r^3 e^{-\gamma \left ( 1 - \frac{r}{R}
           \right)^2} dr \\
        &= \frac{2\pi R^2}{(\sqrt{\pi
           \gamma} -1)}\big[ \sqrt{\pi \gamma}(2\gamma
           +3)\mathrm{erf}(\sqrt{\gamma})\notag \\
           & + 2(\gamma + 1)e^{-\gamma} -6\gamma -2 \big] \\
  \int w_1(\rr)d\rr &= B\int_0^R r^2 e^{-\gamma \left ( 1 - \frac{r}{R}
            \right)^2} dr\\
         &= \frac{R\left[ \sqrt{\pi}(2\gamma + 1)\mathrm{erf}(\sqrt{\gamma})
          + \sqrt{\gamma}\left(2-4e^{-\gamma}\right)
          \right]}{2\sqrt{\gamma}( \sqrt{\pi \gamma} -1)} \\
  \int w_0(\rr)d\rr &= B\int_0^R r e^{-\gamma \left ( 1 - \frac{r}{R}
            \right)^2} dr\\
              &= \frac{\sqrt{\gamma}}{\sqrt{\pi \gamma} -1}\left[
              \sqrt{\pi} \mathrm{erf}(\sqrt{\gamma}) +
               \frac{e^{-\gamma}-1}{\sqrt{\gamma}} \right]
\end{align}
 
\section{Comparison with simulation}

\begin{figure}
\begin{center}
\includegraphics[width=3.5in]{figs/p-vs-packing}
\end{center}
\caption{Pressure versus packing fraction.  The SFMT result is plotted
  as solid lines, with simulation results as solid circles.  The
  pressure is divided by the hard-sphere pressure in each case, in
  order to highlight the effect of the soft interactions.}
\label{fig:p-vs-packing}
\end{figure}

\begin{figure}
\begin{center}
\includegraphics[width=3.5in]{figs/radial-distribution-10}
\end{center}
\caption{Radial distribution function at low density, with 0.1 packing
  fraction.}
\label{fig:radial-distribution-10}
\end{figure}

\begin{figure}
\begin{center}
\includegraphics[width=3.5in]{figs/radial-distribution-20}
\end{center}
\caption{Radial distribution function at low density, with 0.2 packing
  fraction.}
\label{fig:radial-distribution-20}
\end{figure}

\begin{figure}
\begin{center}
\includegraphics[width=3.5in]{figs/radial-distribution-30}
\end{center}
\caption{Radial distribution function with 0.3 packing
  fraction.}
\label{fig:radial-distribution-30}
\end{figure}

\begin{figure}
\begin{center}
\includegraphics[width=3.5in]{figs/radial-distribution-40}
\end{center}
\caption{Radial distribution function with 0.4 packing
  fraction.}
\label{fig:radial-distribution-40}
\end{figure}

\begin{figure}
\begin{center}
\includegraphics[width=3.5in]{figs/radial-distribution-50}
\end{center}
\caption{Radial distribution function with 0.5 packing
  fraction.}
\label{fig:radial-distribution-50}
\end{figure}

\begin{figure}
\begin{center}
\includegraphics[width=3.5in]{figs/radial-distribution-60}
\end{center}
\caption{Radial distribution function with 0.6 packing
  fraction.}
\label{fig:radial-distribution-60}
\end{figure}

\begin{figure}
\begin{center}
\includegraphics[width=3.5in]{figs/radial-distribution-70}
\end{center}
\caption{Radial distribution function with 0.7 packing
  fraction.}
\label{fig:radial-distribution-70}
\end{figure}

\begin{figure}
\begin{center}
\includegraphics[width=3.5in]{figs/radial-distribution-80}
\end{center}
\caption{Radial distribution function with 0.8 packing
  fraction.}
\label{fig:radial-distribution-80}
\end{figure}

\subsection{Soft spheres near a hard wall}

\begin{figure}
\begin{center}
\includegraphics[width=3.5in]{figs/walls-10}
\end{center}
\caption{Density distribution near a hard wall.}
\label{fig:walls-10}
\end{figure}

\begin{figure}
\begin{center}
\includegraphics[width=3.5in]{figs/walls-20}
\end{center}
\caption{Density distribution near a hard wall.}
\label{fig:walls-20}
\end{figure}

\begin{figure}
\begin{center}
\includegraphics[width=3.5in]{figs/walls-30}
\end{center}
\caption{Density distribution near a hard wall.}
\label{fig:walls-30}
\end{figure}

\begin{figure}
\begin{center}
\includegraphics[width=3.5in]{figs/walls-40}
\end{center}
\caption{Density distribution near a hard wall.}
\label{fig:walls-40}
\end{figure}

\begin{figure}
\begin{center}
\includegraphics[width=3.5in]{figs/walls-50}
\end{center}
\caption{Density distribution near a hard wall.}
\label{fig:walls-50}
\end{figure}


%%%%%%%%%%%%%%%%%%%%%%%%%%%%%%%%%%%%%%%%%%%%%%%%%%%%%%%%%%%%
\section{Conclusion}

\appendix

\section{Soft Fundamental-Measure Theory}

We follow the Soft Fundamental-Measure Theory introduced by
Schmidt~\cite{schmidt2000fluid}.  Here I will derive the fundamental
measures for a hard-sphere fluid, so we can see how this works.
Schmidt defines the relationship between the weighting functions as:
\begin{align}
  w_2(r) &= -\frac{\partial w_3(r)}{\partial r} \\
  \mathbf{w}_{v2}(\rr) &= w_2(r)\frac{\rr}{r} \\
  \mathbf{w}_{m2}(\rr) &= w_2(r)\left( \frac{\rr \rr}{r^2}
                              - \frac{\mathbf{\hat{1}}}{3} \right) \\
  w_1(r) &= \frac{w_2(r)}{4\pi r} \\
  \mathbf{w}_{v1}(\rr) &= w_1(r) \frac{\rr}{r} \\
  w_0(r) &= \frac{w_1(r)}{r}
\end{align}
The weighting functions are themselves given by a deconvolution of the
Mayer $f$ function:
\begin{align}
  f(r) &= \exp\left(-\frac{V(r)}{k_BT}\right) - 1 \\
  &= w_0 * w_3 + w_1 * w_2 - \mathbf{w}_{v1} * \mathbf{w}_{v2}
\end{align}
where the $*$ operator denotes a three-dimensional convolution, and a
scalar product between vectors.  Schmidt explains that there is a
simple relationship in fourier space between the Mayer $f$ function
and the weighting function $w_2$:
\begin{align}
  \tilde{w}_2(k) &= \pm \sqrt{ik\tilde{f}(k)} \label{eq:w2fromf}
\end{align}
where the tilde denotes a \emph{one-dimensional} Fourier transform:
\begin{align}
  \tilde{f}(k) &= \int_{-\infty}^\infty f(r) e^{ikr} dr
\end{align}

I am writing this section to explore this formula, and to help me to
understand how to deal with this branch cut (i.e. the sign in
Equation~\ref{eq:w2fromf}).  One interesting challenge here is that
$\tilde{w}_2(k)$ is \emph{not} actually the function that we will use
in our convolutions, which are three-dimensional convolutions.  So we
will still have to go into real space and back again to get our
convolution kernel.  $\ddot\frown$

\section{Hard spheres}

For the hard-sphere fluid,
\begin{align}
  f(r) = - \Theta(r - 2R)
\end{align}
which means that
\begin{align}
  \tilde{f}(k) &= \int_{-\infty}^\infty f(r) e^{ikr} dr \\
  &= -\int_{-2R}^{2R} e^{ikr} dr \\
  &= -\int_{-2R}^{2R} \cos(kr) dr \\
  &= -\frac{2}{k}\sin(2kR)
\end{align}
At this point, we want to solve for $w_2$.
\begin{align}
  \tilde{w}_2(k) &= \pm \sqrt{ik\tilde{f}(k)} \\
  &= \pm \sqrt{-2i\sin(2kR)} \\
  &= \pm \sqrt{-4i\sin(kR)\cos(kR)}
\end{align}
I happen to know that
\begin{align}
  w_2(r) &= \delta(r - R) \\
  \tilde{w}_2(k) &= \int_{-\infty}^\infty w_2(r) e^{ikr} dr \\
  &= \int_{-\infty}^\infty \delta(r-R) e^{ikr} dr \\
  &= e^{ikr} + e^{-ikr + i\Phi}
\end{align}
where in the last step, I use the fact that there is a physical
ambiguity as to whether $w_2$ is even or odd or somewhere in between
when understood in one dimension, since only even values for $r$ make
physical sense.  Thus looking at the square of this, we see
\begin{align}
  \tilde{w}_2(k)^2 &= e^{i2kr} + e^{-i2kr + 2\Phi} + 2e^{i\Phi} \\
  &= -2i\sin(2kR) \\
  &= -2i\frac{e^{i2kR} - e^{-i2kR}}{2i} \\
  &= e^{-i2kR} - e^{i2kR}
\end{align}
This almost makes sense (if $\Phi = \pi/2$), but not quite, because
there is the extra term of $2e^{i\Phi}$, which would give us an extra
$2i$.

This looks like a contradiction.  However, we assumed here that there
was a $\delta$-function at $-R$, and that $f(r)$ was a symmetric
function, simply because the three-dimensional function is symmetric.
What if there is only one spike at $R$ and $f(r)$ is asymmetric?  Then
we would have
\begin{align}
  \tilde{f}(k) &= \int_{-\infty}^\infty f(r) e^{ikr} dr \\
  &= -\int_{-\infty}^{2R} e^{ik2R} dr
\end{align}
This integral isn't very nicely behaved.  But if I were to ignore the
lower bound---which is actually okay for computing the slope, which is
all we need---then I would find that
\begin{align}
  \tilde{f}(k) &= -\frac{e^{ikr}}{ik} + C
\end{align}
When I solve for $\tilde{w}_2$ I get something similar:
\begin{align}
  w_2(r) &= \delta(r - R) \\
  \tilde{w}_2(k) &= \int_{-\infty}^\infty \delta(r-R) e^{ikr} dr \\
  &= e^{ikR}
\end{align}
And now it's obvious that the solution is correct, apart from an
unexplained minus sign in $\tilde{f}$, which I imagine I could find in
time.  So it looks like we want to assume that the slope of $f(r)$ is
zero for $r<0$, and therefore that $w_2(r<0)=0$.

\section{Linear soft potential}

Now let's imagine a potential
\begin{align}
  V(r) =
  \begin{cases}
    V_0 & r < 0 \\
    V_0\left(1 - \frac{r}{r_0}\right) & r < r_0 \\
    0 & r \ge r_0
  \end{cases}
\end{align}
This gives us a Mayer $f$ function of
\begin{align}
  f(r) &= e^{-\beta V(r)} - 1 \\
  f'(r) &=
  \begin{cases}
    \frac{\beta V_0}{r_0} e^{-\beta V_0(1-r/r_0)} & 0 < r < r_0 \\
    0 & \text{otherwise}
  \end{cases} \\
  \tilde{f'}(k) &= \frac{\beta V_0}{r_0}\int_{0}^{r_0} e^{-\beta
    V_0(1-r/r_0)}e^{ikr}dr \\
  &= \frac{\beta V_0}{r_0}e^{-\beta V_0}
  \int_{0}^{r_0} e^{(\beta V_0/r_0 + ik)r} dr
  \\
  &= \frac{\beta V_0}{r_0}e^{-\beta V_0} \frac{1}{\beta V_0/r_0 + ik}\left(
  e^{\beta V_0+ikr_0} - 1 \right)
  \\
  &= \frac{e^{ikr_0} - e^{-\beta V_0}}{1 + \frac{ikr_0}{\beta V_0}}
  \\
  &= \tilde{w}_2(k)^2
\end{align}
That's just a little bit yucky, and I'm not sure I can solve for it in
real space.

Now let's imagine a potential that has some interesting (but
numerically convenient) values for $r<0$
\begin{align}
  V(r) =
  \begin{cases}
    V_0\left(1 - \frac{r}{r_0}\right) & r < r_0 \\
    0 & r \ge r_0
  \end{cases}
\end{align}
This gives us a Mayer $f$ function of
\begin{align}
  f(r) &= e^{-\beta V(r)} - 1 \label{eq:ftrytwo} \\
  f'(r) &=
  \begin{cases}
    \frac{\beta V_0}{r_0} e^{-\beta V_0(1-r/r_0)} & r < r_0 \\
    0 & \text{otherwise}
  \end{cases} \\
  \tilde{f'}(k) &= \frac{\beta V_0}{r_0}\int_{-\infty}^{r_0} e^{-\beta
    V_0(1-r/r_0)}e^{ikr}dr \\
  &= \frac{\beta V_0}{r_0}e^{-\beta V_0}
  \int_{-\infty}^{r_0} e^{(\beta V_0/r_0 + ik)r} dr
  \\
  &= \frac{\beta V_0}{r_0}e^{-\beta V_0} \frac{1}{\beta V_0/r_0 + ik}e^{\beta V_0+ikr_0}
  \\
  &= \frac{e^{ikr_0}}{1 + \frac{ikr_0}{\beta V_0}}
  \\
  &= \tilde{w}_2(k)^2
\end{align}
Now this is nicer.  The top is really easy to square root.  The bottom
is ever so slightly harder, but not really hard.
\begin{align}
  \tilde{w}_2(k) &=
  \frac1{\sqrt{2}}\frac{e^{ikr_0/2}}{\sqrt{\sqrt{1+\frac{kr_0}{\beta
          V_0}}+1} +
      i\sqrt{\sqrt{1+\frac{kr_0}{\beta V_0}}-1}}
\end{align}
Okay, that's a bit ugly, and I'm not sure that we can Fourier
transform it to get a useful result.  We could potentially approximate
it in the limit that $\beta V_0 \gg 1$, although any simple power
series expansion would require us to prove that $kr_0\gg1$ is
unimportant, possibly by arguing that phase cancellation will cause it
to be unimportant when we evaluate our actual $w_2(r)$.
\begin{align}
  w_2(r) &= \int_{-\infty}^\infty \tilde{w}_2(k) \\
 &=
  \frac1{\sqrt{2}}\frac{e^{ikr_0/2}}{\sqrt{\sqrt{1+\frac{kr_0}{\beta
          V_0}}+1} +
      i\sqrt{\sqrt{1+\frac{kr_0}{\beta V_0}}-1}}
\end{align}

It occurs to me now that Schmidt may have assumed right after his
Equation~A7 that $w_2(r<0)=0$, in which case this attempt won't work:
we'll need to make things zero for $r<0$ if we're going to use this
Fourier transform trick at all.  On the other hand, if this comes out
to be \emph{effectively} zero for reasonable temperatures, perhaps
this will be a good approach.  After all, I'm interested in $\beta V_0
\gg 1$, which means that the Mayer function is essentially flat in
Equation~\ref{eq:ftrytwo} for $r<0$.

\section{Erf model (scratch notes)}
If we consider that
\begin{align}
  f'(r) \approx \frac{1}{\sqrt{\pi}} e^{-(r-\sigma)^2/a^2}
\end{align}
then we conclude that
\begin{align}
  f(r) \approx \tfrac12 ( \mathrm{erf}((r-\sigma)/a) - 1 ).
\end{align}
\begin{figure}
  \includegraphics[width=\columnwidth]{figs/erf}
  \caption{Comparison of erf approximation with $f'(r)$ for our
    harmonic potential.}
\end{figure}
This matches all right, if we choose to match the first couple of
terms in a power series expansion around $\sigma$, in which case we
find that
\begin{align}
  V(\sigma) &= k_BT \ln 2 \\
  V'(\sigma) &= \frac{k_BT}{a}
\end{align}
This gives us an approximate model that we could apply to any
potential $V(r)$, with $\sigma$ being a function of temperature.  If
we do this, we can analytically solve for $w_2(r)$ and $w_3(r)$.
\begin{align}
  w_2(r) &= ? e^{-\frac{(r-\sigma/2)^2}{a^2}} \\
  w_3(r) &= \tfrac12 ( 1 - \mathrm{erf}((r-\sigma/2)/a) )
\end{align}
This is presented in Reference \citenum{schmidt2000fluid}.  The
Fourier transform of $w_3$ is surprisingly simple, although
Mathematica cannot handle it.
\begin{align}
  \tilde{w}_3(k) &= -\frac{?}{ik} \int_0^\infty r w_3(r) \sin(kr) dr
  \\
  &=  -\frac{?}{2ik} \int_0^\infty ( 1 -
  \mathrm{erf}((r-\sigma/2)/a) ) r \sin(kr) dr
  \\
  &=  -\frac{?}{2ik} \left(
  \int_0^\infty r \sin(kr) dr
  -
  \int_0^\infty \mathrm{erf}((r-\sigma/2)/a) r \sin(kr) dr
  \right)
\end{align}
At this point, we can integrate by parts the second bit.
\begin{align}
  u &= \mathrm{erf}((r-\sigma/2)/a) \\
  du &= \frac{2}{\sqrt{\pi}a^2} e^{-\frac{(r-\sigma/2)^2}{a^2}} du \\
  dv &= r \sin(kr) dr \\
  v &= \frac{\sin(kr) - kr \cos(kr)}{k^2}
\end{align}
Putting these in we get
\begin{align}
  \tilde{w}_3(k)
  &=  -\frac{?}{2ik} \Big(
  \int_0^\infty r \sin(kr) dr\\ &\quad\quad
  -
  \left.\mathrm{erf}((r-\sigma/2)/a)\frac{\sin(kr) - kr
    \cos(kr)}{k^2}\right|_0^\infty\\ &\quad\quad
  +
  \int_0^\infty \frac{\sin(kr) - kr \cos(kr)}{k^2}\frac{2}{\sqrt{\pi}a^2} e^{-\frac{(r-\sigma/2)^2}{a^2}} dr
  \Big)
\end{align}
At this stage things look a bit scary, since the second term is not
well defined.  Fortunately, the first and second terms exactly cancel,
as you can see by doing both integrals prior to taking the limit to
infinity.  So we get
\begin{align}
  \tilde{w}_3(k)
  &=  -\frac{?}{2ik}\int_0^\infty \frac{\sin(kr) - kr
    \cos(kr)}{k^2}\frac{2}{\sqrt{\pi}a^2}
  e^{-\frac{(r-\sigma/2)^2}{a^2}} dr
\end{align}
which is still a little bit scary, and is not yet something
Mathematica can handle.  Fortunately, the terms can be broken down
into simpler definite integrals that Mathematica \emph{can} handle
such as
\begin{align}
  \int_{-\infty}^\infty \cos(a u)e^{-u^2}du
\end{align}
and others that are zero by symmetry (where we assume the gaussian
drops to zero before reaching the origin).
\begin{widetext}
Beginning with $\tilde{w}_3(k)$ from above and letting $u = (r-\sigma/2)/a,$
  \begin{align}
    \tilde{w}_3(k)
    &=  \frac{?}{i\sqrt{\pi}ak^3}\int_{-\sigma/2a}^\infty \left[ \sin(k(au+\frac{\sigma}{2}))
      - k(au + \frac{\sigma}{2}) \cos(k(au+\frac{\sigma}{2}))\right] e^{-u^2} du
  \end{align}
We can take the lower limit to $-\infty$ since the gaussian should go
to zero well before that. Using mathematica,
  \begin{align}
    \tilde{w}_3(k) &= \frac{?}{i\sqrt{\pi}ak^3}
    \left(\sqrt{\pi}e^{-\left(\frac{ak}{2}\right)^2}
    \sin{\frac{k\sigma}{2}} \right) - \left( \frac{k}{2}
    \sqrt{\pi}e^{-\left(\frac{ak}{2}\right)^2} \left[\sigma
      \cos{\frac{k \sigma}{2}} - a^2 k \sin{\frac{k \sigma}{2}}
      \right] \right) \\ &=
    \frac{?}{iak^3}e^{-\left(\frac{ak}{2}\right)^2}\left[ \left(1 + \frac{a^2k^3}{2} \right)  
      \sin{\frac{k\sigma}{2}} - \frac{k}{2} \sigma\cos{\frac{k \sigma}{2}}  \right]
  \end{align}
\end{widetext}
\fixme{This bothers me, though, because it is now purely imaginary}

For the $w_2$,
\begin{align}
  \tilde{w}_2(k) &= ?\int e^{-\frac{(r-\sigma/2)^2}{a^2}} e^{i \mathbf{k} \cdot
    \rr} \textrm{d} \rr \\
  &= ? 2\pi \int_0^\infty \int_{-1}^1 r^2 e^{-\frac{(r-\sigma/2)^2}{a^2}}
  e^{ikr\cos{\theta}} \textrm{drd}\cos{\theta} \\
  &= \frac{?2\pi}{ik} \int_0^\infty r e^{-\frac{(r-\sigma/2)^2}{a^2}}
  \left(e^{ikr} - e^{-ikr}\right) \textrm{dr} \\
  &= \frac{?4\pi}{k} \int_0^\infty r e^{-\frac{(r-\sigma/2)^2}{a^2}}
  \sin{kr} ~\textrm{dr} \\
  &= \frac{?a4\pi}{k} \int_{-\sigma/2a}^\infty \left(au +
  \frac{\sigma}{2}\right) e^{-u^2} \sin{k\left(au +
  \frac{\sigma}{2}\right)} ~\textrm{du}
\end{align}
where, again, we take the lower limit to $-\infty$ since the gaussian
should to zero well before the lower limit. Using mathematica, I get
\begin{align}
  \tilde{w}_2(k) &=\frac{?a4\pi}{k}\frac{\sqrt{\pi}}{2}
  e^{-\left(\frac{ak}{2} \right)^2} \left(a^2 k
  \cos{\frac{k\sigma}{2}} +\sigma \sin{\frac{k\sigma}{2}} \right)
\end{align}
$w_1$ will have a similar integral, but without the stray $r$, so we
get
\begin{align}
  \tilde{w}_1{k}&= \frac{?a}{k} \int_{-\infty}^\infty e^{-u^2} \sin{k\left(au +
  \frac{\sigma}{2}\right)} ~\textrm{du} \\
  &= \frac{?a}{k}\sqrt{\pi}e^{-\left(\frac{ak}{2} \right)^2} \sin{\frac{k\sigma}{2}}
\end{align}

$w_0$ is quite a bit trickier.  I won't go through all the math here,
but essentially we have
\begin{align}
  \tilde{w}_0(k) &= \frac{?}{k}\int_0^\infty
  e^{-\frac{(r-\sigma/2)^2}{a^2}} \frac{\sin{kr}}{r}\textrm{dr}
\end{align}
Since the gaussian should, again, go to zero before r = 0, then we can
take this to negative infinity so \fixme{without all the details}:
\begin{align}
  \tilde{w}_0(k) &= \frac{?}{k}\int_{-\infty}^\infty
  e^{-\frac{(r-\sigma/2)^2}{a^2}} \frac{\sin{kr}}{r}\textrm{dr} \\
  &= \frac{?\sqrt{\pi}}{k}\int_0^{\frac{k}{\sqrt{\pi}}}
  e^{-\frac{t}{4}} \cos{\frac{\sigma t}{2a}}\textrm{dt}
\end{align}
So we have a closed form, but we will still have to do this numerically

\bibliography{paper}% Produces the bibliography via BibTeX.

\end{document}
