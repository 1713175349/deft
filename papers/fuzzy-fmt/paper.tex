% TODO:

% Write ``Liquid-vapor interface'' section A.

% Look up question-mark references (citations).

\documentclass[letterpaper,twocolumn,amsmath,amssymb,prb]{revtex4-1}
\usepackage{graphicx}% Include figure files
\usepackage{dcolumn}% Align table columns on decimal point
\usepackage{bm}% bold math
\usepackage{color}

\newcommand{\red}[1]{{\color{red} #1}}
\newcommand{\blue}[1]{{\bf \color{blue} #1}}
\newcommand{\green}[1]{{\bf \color{green} #1}}
\newcommand{\rr}{\textbf{r}}
\newcommand{\kk}{\textbf{k}}
\newcommand{\refnote}{\red{[ref]}}

\newcommand{\fixme}[1]{\red{[#1]}}

\begin{document}
\title{Smoothing the Fundamental Measure Theory Functional for Hard Spheres}

\author{Eric J. Krebs}
\affiliation{Department of Physics, Oregon State University, Corvallis, OR 97331}

\author{David Roundy}
\affiliation{Department of Physics, Oregon State University, Corvallis, OR 97331}

%%%%%%%%%%%%%%%%%%%%%%%%%%%%%%%%%%%%%%%%%%%%%%%%%%%%%%%%%%%%
\begin{abstract}
\fixme{ We examine a functional based on SFMT~\cite{schmidt2000fluid}.
  We pick a slightly soft potential, and demonstrate that it gives
  good results.}
\end{abstract}

\maketitle

%%%%%%%%%%%%%%%%%%%%%%%%%%%%%%%%%%%%%%%%%%%%%%%%%%%%%%%%%%%%a
\section{Introduction}

The hard-sphere fluid is \emph{the} reference fluid for models of real
liquids, such as water \fixme{citations?}. The hard sphere fluid is
widely studied and well understood \fixme{citations}, but has the
disadvantage of not being physical since it contains step and delta
functions which are hard to use in computation. One way to deal with
this difficulty is by ``smearing'' the functions so they are not
discontinuous; another is to use a theory where the spheres themselves
are not so hard.

Fundamental Measure Theory~(FMT) is a classic density functional
theory (cDFT) for the free energy of the hard-sphere fluid developed
by Rosenfeld~\cite{rosenfeld1989}.  FMT is often used for the
hard-sphere portion of the total free energy which \fixme{citations}.
While this has been used to some success to study the behavior of real
liquids\fixme{citations}, it is not physical as the hard sphere cores
are never allowed to penetrate.

Schmidt introduced a Soft Fundamental Measure Theory~(SFMT) based on
FMT~\cite{schmidt1999density} where the spheres can penetrate at a an
energy cost determined by a pair potential. Schmidt tested SFTM for a
mixture of soft and hard spheres as well as for star
polymers~\cite{schmidt2000density, groh2001density}. SFMT was also
tested in a number of other papers~\cite{rosenfeld2000fluid}
\fixme{more citations} but not the soft fluid with a harmonic
potential.

\section{Fundamental-Measure Theory}

We use the White Bear version of the Fundamental-Measure Theory~(FMT)
functional published in reference~\cite{roth2002whitebear}.  The FMT
functional describes the excess free energy of a hard-sphere fluid.
This particular FMT reduces to the Carnahan-Starling equation of state
for homogeneous systems.
\begin{equation}
A_\textit{HS}[n] = k_B T \int \left(\Phi_1(\rr) + \Phi_2(\rr) + \Phi_3(\rr)\right) d\rr \; ,
\end{equation}
with integrands
\begin{align}
\Phi_1 &= -n_0 \ln\left( 1 - n_3\right)\\
\Phi_2 &= \frac{n_1 n_2 - \mathbf{n}_{V1} \cdot\mathbf{n}_{V2}}{1-n_3} \\
\Phi_3 &= (n_2^3 - 3 n_2 \mathbf{n}_{V2} \cdot \mathbf{n}_{V2}) \frac{
  n_3 + (1-n_3)^2 \ln(1-n_3)
}{
  36\pi n_3^2\left( 1 - n_3 \right)^2
} ,
\end{align}
using the weighted densities
\begin{align}
  n_3(\rr) &= \int n(\rr') \Theta(\left|\rr - \rr'\right| - R) d\rr' \\
  n_2(\rr) &= \int n(\rr') \delta(\left|\rr - \rr'\right| - R) d\rr'
\end{align}
\begin{align}
  \mathbf{n}_{V2} &= \mathbf{\nabla} n_3 , \quad
  \mathbf{n}_{V1} = \frac{\mathbf{n}_{V2}}{4\pi R} \\
  n_1 &= \frac{n_2}{4\pi R} , \quad
  n_0 = \frac{n_2}{4\pi R^2}
\end{align}

\section{Theory}

Soft Fundamental-Measure Theory~(SFMT) is a generalization of FMT to
soft potentials. In SFMT, the weight functions are modified but retain:
the direct correlation in the low-density limit, the exact free energy
in the zero-dimensional cavity limit \fixme{It looks like we don't to
  this with our harmonic potential}, relation to the generating
weight function $w_3(\rr)$~\cite{schmidt1999density} \fixme{this is an assumption
  as stated in Schmidt 1999}.  The new weight functions are also made
to deconvolve the Mayer function   

\begin{equation}
f(r) = \exp (-\beta V(r)) - 1
\end{equation}
and to give an approximate multi-cavity limit. Schmidt furthur shows that\cite{schmidt2000fluid}

\begin{equation}\label{eq:mayerandw2}
\frac{d f(r)}{dr} = \int dr' w_2(r') w_2 (r-r').
\end{equation}

\begin{figure}
\begin{center}
\includegraphics[width=3.5in]{figs/w2convolves}
\end{center}
\caption{Comparison of the $-f'(r)$ function with various $w_2(r)\ast
  w_2(r)$.}
\label{fig:w2convolves}
\end{figure}

\begin{figure}
\begin{center}
\includegraphics[width=\columnwidth]{figs/potential-plot}
\end{center}
\caption{plot of potential.}
\label{fig:potential-plot}
\end{figure}

\begin{figure}
\begin{center}
\includegraphics[width=\columnwidth]{figs/w2-comparison}
\end{center}
\caption{plot of potential.}
\label{fig:w2-comparison}
\end{figure}

We use a WCA (Weeks-Chandler-Anderson) pair potential
\begin{align}
  V(r) =
  \begin{cases}
    \frac{\epsilon}{9} \left[ \left(\frac{\sigma}{r}\right)^{12} -
    \left(\frac{\sigma}{r}\right)^{6} \right] +
    \frac{\epsilon}{36}, & 0 < r < 2R \\
    0, & \textrm{otherwise}
  \end{cases}
\end{align}
and approximate the Mayer function as
\begin{align}
  f(r) \approx \tfrac12 \left( \mathrm{erf}\left( \frac{r - \alpha}{\Xi} \right) - 1 \right)
\end{align}
then
\begin{align}
  V(r) \approx -kT\ln\left[\tfrac12 \left( \mathrm{erf}\left( \frac{r -
    \alpha}{\Xi} \right) + 1 \right) \right].
\end{align}
Expanding this in a power series gives.
\begin{align}
  V(\alpha) &\approx -kT\ln\left(\tfrac12\right) \\
  &= k_BT \ln 2.
\end{align}
Setting the potentials equal at $r=\alpha$
\begin{align}
  k_BT \ln 2 = \frac{\epsilon}{9} \left[ \left(\frac{\sigma}{\alpha}\right)^{12} -
    \left(\frac{\sigma}{\alpha}\right)^{6} \right] +
    \frac{\epsilon}{36} \\
\end{align}
we then let $\xi = \left(\frac{\sigma}{\alpha}\right)^6$. Now we'll put it in a
quadratic form.
\begin{align}
  k_BT \ln 2 &= \frac{\epsilon}{9} \left[ \xi^2 - \xi  \right] +
  \frac{\epsilon}{36} \\
  0 &= \frac{\epsilon}{9} \left[ \xi^2 - \xi  \right] + \frac{\epsilon}{36} -
  k_BT \ln 2 \\
  0 &= \xi^2 - \xi + \frac{1}{4} - 9\frac{k_BT}{\epsilon} \ln 2
\end{align}
so then
\begin{align}
  \xi &= \frac{1\pm \sqrt{1-4 \left( \frac{1}{4} - 9 \frac{k_BT}{\epsilon}
    \ln 2 \right)}}{2} \\
  &= \frac{1}{2}\left(1 \pm \sqrt{36 \frac{k_BT}{\epsilon} \ln 2} \right)
\end{align}
We take the plus solution and solve for $\alpha$ so that
\begin{align}
  \left(\frac{\sigma}{\alpha}\right)^6 &= \frac{1}{2}\left(1 +
  \sqrt{36 \frac{k_BT}{\epsilon} \ln 2} \right) \\
  \alpha &= \sigma \left( \frac{2}{1 + \sqrt{6 \frac{k_BT}{\epsilon}
        \ln 2}} \right)^{\frac{1}{6}}.
\end{align}

Next we set both the two potential's slopes equal at $r = \alpha$
\begin{align}
  V_{erf}'(\alpha) &= V_{wca}'(\alpha) \\
  -\frac{2 k_B T}{\Xi \sqrt{\pi}} &= \frac{\epsilon}{36} \left(
  \frac{24}{\alpha} \xi - \frac{48}{\alpha} \xi^2 \right) \\
  -\frac{\alpha k_B T}{\epsilon \sqrt{\pi}} &= \Xi \frac{1}{3} \left(
  \xi - 2\xi^2 \right)
\end{align}
With some algebra, we can find $\Xi$:
\begin{align}
  \Xi &= \frac{3 \alpha k_B T}{\epsilon \sqrt{\pi} \left( 2 \xi^2 - \xi
    \right)}
\end{align}
Focusing on the $\xi$ terms so things don't get too messy,
\begin{align}
  2 \xi^2 - \xi &=  \frac{2}{4}\left(1 +
  6\sqrt{ \frac{k_BT}{\epsilon} \ln 2} \right)^2
  - \frac{1}{2}\left(1 +
  6\sqrt{ \frac{k_BT}{\epsilon} \ln 2} \right) \\
  &= \frac{1}{2}\Bigg[ 1 + 12\sqrt{ \frac{k_BT}{\epsilon}\ln 2} + 36
    \frac{k_BT}{\epsilon}\ln 2 \notag \\
    &\hspace{3cm}-1 - 6\sqrt{ \frac{k_BT}{\epsilon} \ln 2} \Bigg] \\
  &=\frac{1}{2}\left[ 6 \sqrt{ \frac{k_BT}{\epsilon} \ln 2} + 36
    \frac{k_BT}{\epsilon} \ln 2 \right] \\
  &=3\left[ \sqrt{ \frac{k_BT}{\epsilon} \ln 2} + 6
    \frac{k_BT}{\epsilon} \ln 2 \right].
\end{align}
Now if we plug that back into our terms in $\Xi$, we get
\begin{align}
  \Xi &= \frac{3 \alpha k_B T}{\epsilon \sqrt{\pi} 3\left(
      \sqrt{ \frac{k_BT}{\epsilon} \ln 2} + 6 \frac{k_BT}{\epsilon}
      \ln 2 \right) } \\
  &= \frac{\alpha}{\sqrt{\pi} \left( \sqrt{\frac{\epsilon}{k_BT} \ln
      2} + 6 \ln 2 \right)}
\end{align}

The approximate weight functions from the erf model are
\begin{align}
  w_2(r) &= \frac{1}{\Xi \sqrt{\pi}} e^{-\frac{(r-\alpha/2)^2}{\Xi^2}} \\
  w_3(r) &= \tfrac12 ( 1 - \mathrm{erf}((r-\alpha/2)/\Xi) ).
\end{align}
with all other weight functions retaining their relation to $w_2(r)$
as in FMT.

\subsection{Weighting functions in fourier space}

We Fourier transform our weight functions because simulations are
carried out in Fourier space.  We find that
\begin{align}
  \tilde{w}_3(k) &= \frac{4\pi}{k} \int_0^\infty r w_3(r) \sin(kr) dr \\ 
  &= \frac{4\sqrt{\pi}}{k^3}\int_{-\sigma/2a}^\infty \left[
    \sin(k(au+\frac{\sigma}{2})) - k(au + \frac{\sigma}{2})
    \cos(k(au+\frac{\sigma}{2}))\right] e^{-u^2} du.
\end{align}
This is not an analytic function, but we assume that we are working at
low enough temperatures so that our function reduces to zero by $u=
-\frac{\sigma}{2a}$. We extend the lower limit to $-\infty$ then
\begin{align}
  \tilde{w}_3(k) &\approx
  \frac{4\pi}{k^3}e^{-\left(\frac{ak}{2}\right)^2}\left[ \left(1 +
    \frac{a^2k^2}{2} \right) \sin\left(\frac{k\sigma}{2}\right) -
    \frac{k}{2} \sigma\cos\left(\frac{k \sigma}{2}\right) \right]
\end{align}

We apply the same method to the other weight functions. These are
\begin{align}
  \tilde{w}_2(k) &=\frac{2\pi}{k} e^{-\left(\frac{ak}{2} \right)^2}
  \bigg(a^2 k \cos\left(\frac{k\sigma}{2}\right) \notag \\
  & \hspace{8em}+\sigma \sin\left(\frac{k\sigma}{2}\right) \bigg) \\ 
%%%%%%%%%%%%%%%%%%%%%%%%%%%%%%%%%%%%%%%%%%%%%%%%%%%%%%%%%%%%
  \tilde{w}_1(k) &=
  \frac{1}{k}e^{-\left(\frac{ak}{2} \right)^2}
  \sin\left(\frac{k\sigma}{2}\right) \\
%%%%%%%%%%%%%%%%%%%%%%%%%%%%%%%%%%%%%%%%%%%%%%%%%%%%%%%%%%%%
  \tilde{\mathbf{w}}_{2V}(\kk) &=
  \frac{i \pi}{k} e^{-\left(\frac{ak}{2} \right)^2}\bigg[ \left(\sigma^2 - a^4k^2\right)
    \cos\left(\frac{k\sigma}{2}\right) \notag\\ 
    & \hspace{5em}- 2\sigma \left(a^2k + \frac{1}{k}
    \right)\sin\left(\frac{k\sigma}{2}\right) \bigg]
  \mathbf{\hat{k}} \\ 
%%%%%%%%%%%%%%%%%%%%%%%%%%%%%%%%%%%%%%%%%%%%%%%%%%%%%%%%%%%%
  \mathbf{\tilde{w}}_{1V}(k)&= \frac{i}{k}
   e^{-\left(\frac{ak}{2} \right)^2} \bigg[ \frac{\sigma}{2}
    \cos\left(\frac{k\sigma}{2}\right) \notag\\
    & \hspace{6em}- \left( \frac{a^2k}{2} +
    \frac{1}{k} \right) \sin\left(\frac{k\sigma}{2}\right) \bigg]
  \mathbf{\hat{k}}
\end{align}
$\tilde{w}_0(k)$ contains a $\frac{1}{r}$ term that expands the
integrand  as an infinite series. The first two terms were kept and
could be expressed in terms of $\tilde{w}_1(k)$ and
$\tilde{w}_2(k)$.
\begin{align}
  \tilde{w}_0(k) &= \frac{2}{\sigma} \left[2\tilde{w}_1(k) - \frac{1}{2 \pi
      \sigma}\tilde{w}_2(k) \right]
\end{align}

\subsection{Homogeneous limit}

As a simple test for the equation of state, we compare the theory for
a homogeneous soft-sphere fluid and compate to Monte-Carlo (MC) simulation.
 
\section{Comparison with simulation}

\begin{figure}
\begin{center}
\includegraphics[width=3.5in]{figs/p-vs-packing}
\end{center}
\caption{Pressure versus packing fraction.  The SFMT result is plotted
  as solid lines, with simulation results as solid circles.  The
  pressure is divided by the hard-sphere pressure in each case, in
  order to highlight the effect of the soft interactions.}
\label{fig:p-vs-packing}
\end{figure}

\begin{figure}
\begin{center}
\includegraphics[width=3.5in]{figs/radial-distribution-10}
\end{center}
\caption{Radial distribution function with 0.1 packing
  fraction.}
\label{fig:radial-distribution-10}
\end{figure}

\begin{figure}
\begin{center}
\includegraphics[width=3.5in]{figs/radial-distribution-30}
\end{center}
\caption{Radial distribution function with 0.3 packing
  fraction.  \fixme{Change legend to proper ``reduced'' temperature
    $T^* = kT/\epsilon$, and density perhaps to $n^* = n\sigma^3$?
    Start cutting harmonic?}}
\label{fig:radial-distribution-30}
\end{figure}

\begin{figure}
\begin{center}
\includegraphics[width=3.5in]{figs/radial-distribution-40}
\end{center}
\caption{Radial distribution function with 0.4 packing
  fraction.}
\label{fig:radial-distribution-40}
\end{figure}

\subsection{Soft spheres near a hard wall}

\begin{figure}
\begin{center}
\includegraphics[width=3.5in]{figs/walls-10}
\end{center}
\caption{Density distribution near a hard wall.}
\label{fig:walls-10}
\end{figure}

\begin{figure}
\begin{center}
\includegraphics[width=3.5in]{figs/walls-20}
\end{center}
\caption{Density distribution near a hard wall.}
\label{fig:walls-20}
\end{figure}

\begin{figure}
\begin{center}
\includegraphics[width=3.5in]{figs/walls-30}
\end{center}
\caption{Density distribution near a hard wall.}
\label{fig:walls-30}
\end{figure}

\begin{figure}
\begin{center}
\includegraphics[width=3.5in]{figs/walls-40}
\end{center}
\caption{Density distribution near a hard wall.}
\label{fig:walls-40}
\end{figure}

%%%%%%%%%%%%%%%%%%%%%%%%%%%%%%%%%%%%%%%%%%%%%%%%%%%%%%%%%%%%
\section{Comparison with experiment}

Lennard Jones data we will use will be from
papers~\cite{mikolaj2004structure, eggert2002quantitative, yarnell1973structure}.
For Lennard Jones Argon, $\sigma = 3.405 ~\textrm{\AA}$ and $\epsilon = 119.8$~K~\cite{verlet1967computer}.
\begin{figure}
\begin{center}
\includegraphics[width=3.5in]{figs/Argon-vapor_pressure-85K}
\end{center}
\caption{Radial distribution fucntion of Argon at 85 K and vapor
  pressure.  \fixme{The reduced density and temperature are ???.
    i.e. what is $kT/\epsilon$ and $n\sigma^3$?  Then run MC and DFT
    simulations.}}
\label{fig:argon-85K}
\end{figure}

\begin{figure}
\begin{center}
\includegraphics[width=3.5in]{figs/Argon-0_6GPa-RT}
\end{center}
\caption{Radial distribution fucntion of Argon at room temperature and
0.6 GPa.  \fixme{The reduced density and temperature are ???.
    Then run MC and DFT simulations.}}
\label{fig:argon-0.6GPa}
\end{figure}

\begin{figure}
\begin{center}
\includegraphics[width=3.5in]{figs/Argon-1_1GPa-RT}
\end{center}
\caption{Radial distribution fucntion of Argon at room temperature and
1.1 GPa.  \fixme{The reduced density and temperature are ???.
    Then run MC and DFT simulations.}}
\label{fig:argon-1.1GPa}
\end{figure}

\section{Conclusion}

\appendix

\section{Soft Fundamental-Measure Theory}

We follow the Soft Fundamental-Measure Theory introduced by
Schmidt~\cite{schmidt2000fluid}.  Here I will derive the fundamental
measures for a hard-sphere fluid, so we can see how this works.
Schmidt defines the relationship between the weighting functions as:
\begin{align}
  w_2(r) &= -\frac{\partial w_3(r)}{\partial r} \\
  \mathbf{w}_{v2}(\rr) &= w_2(r)\frac{\rr}{r} \\
  \mathbf{w}_{m2}(\rr) &= w_2(r)\left( \frac{\rr \rr}{r^2}
                              - \frac{\mathbf{\hat{1}}}{3} \right) \\
  w_1(r) &= \frac{w_2(r)}{4\pi r} \\
  \mathbf{w}_{v1}(\rr) &= w_1(r) \frac{\rr}{r} \\
  w_0(r) &= \frac{w_1(r)}{r}
\end{align}
The weighting functions are themselves given by a deconvolution of the
Mayer $f$ function:
\begin{align}
  f(r) &= \exp\left(-\frac{V(r)}{k_BT}\right) - 1 \\
  &= w_0 * w_3 + w_1 * w_2 - \mathbf{w}_{v1} * \mathbf{w}_{v2}
\end{align}
where the $*$ operator denotes a three-dimensional convolution, and a
scalar product between vectors.  Schmidt explains that there is a
simple relationship in fourier space between the Mayer $f$ function
and the weighting function $w_2$:
\begin{align}
  \tilde{w}_2(k) &= \pm \sqrt{ik\tilde{f}(k)}
  \label{eq:w2fromf}
\end{align}
where the tilde denotes a \emph{one-dimensional} Fourier transform:
\begin{align}
  \tilde{f}(k) &= \int_{-\infty}^\infty f(r) e^{ikr} dr
\end{align}

I am writing this section to explore this formula, and to help me to
understand how to deal with this branch cut (i.e. the sign in
Equation~\ref{eq:w2fromf}).  One interesting challenge here is that
$\tilde{w}_2(k)$ is \emph{not} actually the function that we will use
in our convolutions, which are three-dimensional convolutions.  So we
will still have to go into real space and back again to get our
convolution kernel.  $\ddot\frown$

\section{Hard spheres}

For the hard-sphere fluid,
\begin{align}
  f(r) = - \Theta(r - 2R)
\end{align}
which means that
\begin{align}
  \tilde{f}(k) &= \int_{-\infty}^\infty f(r) e^{ikr} dr \\
  &= -\int_{-2R}^{2R} e^{ikr} dr \\
  &= -\int_{-2R}^{2R} \cos(kr) dr \\
  &= -\frac{2}{k}\sin(2kR)
\end{align}
At this point, we want to solve for $w_2$.
\begin{align}
  \tilde{w}_2(k) &= \pm \sqrt{ik\tilde{f}(k)} \\
  &= \pm \sqrt{-2i\sin(2kR)} \\
  &= \pm \sqrt{-4i\sin(kR)\cos(kR)}
\end{align}
I happen to know that
\begin{align}
  w_2(r) &= \delta(r - R) \\
  \tilde{w}_2(k) &= \int_{-\infty}^\infty w_2(r) e^{ikr} dr \\
  &= \int_{-\infty}^\infty \delta(r-R) e^{ikr} dr \\
  &= e^{ikr} + e^{-ikr + i\Phi}
\end{align}
where in the last step, I use the fact that there is a physical
ambiguity as to whether $w_2$ is even or odd or somewhere in between
when understood in one dimension, since only even values for $r$ make
physical sense.  Thus looking at the square of this, we see
\begin{align}
  \tilde{w}_2(k)^2 &= e^{i2kr} + e^{-i2kr + 2\Phi} + 2e^{i\Phi} \\
  &= -2i\sin(2kR) \\
  &= -2i\frac{e^{i2kR} - e^{-i2kR}}{2i} \\
  &= e^{-i2kR} - e^{i2kR}
\end{align}
This almost makes sense (if $\Phi = \pi/2$), but not quite, because
there is the extra term of $2e^{i\Phi}$, which would give us an extra
$2i$.

This looks like a contradiction.  However, we assumed here that there
was a $\delta$-function at $-R$, and that $f(r)$ was a symmetric
function, simply because the three-dimensional function is symmetric.
What if there is only one spike at $R$ and $f(r)$ is asymmetric?  Then
we would have
\begin{align}
  \tilde{f}(k) &= \int_{-\infty}^\infty f(r) e^{ikr} dr \\
  &= -\int_{-\infty}^{2R} e^{ik2R} dr
\end{align}
This integral isn't very nicely behaved.  But if I were to ignore the
lower bound---which is actually okay for computing the slope, which is
all we need---then I would find that
\begin{align}
  \tilde{f}(k) &= -\frac{e^{ikr}}{ik} + C
\end{align}
When I solve for $\tilde{w}_2$ I get something similar:
\begin{align}
  w_2(r) &= \delta(r - R) \\
  \tilde{w}_2(k) &= \int_{-\infty}^\infty \delta(r-R) e^{ikr} dr \\
  &= e^{ikR}
\end{align}
And now it's obvious that the solution is correct, apart from an
unexplained minus sign in $\tilde{f}$, which I imagine I could find in
time.  So it looks like we want to assume that the slope of $f(r)$ is
zero for $r<0$, and therefore that $w_2(r<0)=0$.

\bibliography{paper}% Produces the bibliography via BibTeX.

\end{document}
