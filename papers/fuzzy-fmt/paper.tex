% TODO:

% Write ``Liquid-vapor interface'' section A.

% Look up question-mark references (citations).

\documentclass[letterpaper,twocolumn,amsmath,amssymb,prb]{revtex4-1}
\usepackage{graphicx}% Include figure files
\usepackage{dcolumn}% Align table columns on decimal point
\usepackage{bm}% bold math
\usepackage{color}

\newcommand{\red}[1]{{\bf \color{red} #1}}
\newcommand{\blue}[1]{{\bf \color{blue} #1}}
\newcommand{\green}[1]{{\bf \color{green} #1}}
\newcommand{\rr}{\textbf{r}}
\newcommand{\xx}{\textbf{x}}
\newcommand{\refnote}{\red{[ref]}}

\newcommand{\fixme}[1]{\red{[#1]}}

% needsworklater is used to annotate bits that need work, but that we
% can postpone for a while.
\newcommand{\needsworklater}[1]{\emph{[#1]}}
% needsworknow is intended to prioritize stuff that needs fixing.
\newcommand{\needsworknow}[1]{\textcolor{red}{[\emph{#1}]}}

\begin{document}
\title{A Fundamental Measure Theory Functional for Hard-Sphere Contact Densities}

\author{David Roundy}
\affiliation{Department of Physics, Oregon State University, Corvallis, OR 97331}

%%%%%%%%%%%%%%%%%%%%%%%%%%%%%%%%%%%%%%%%%%%%%%%%%%%%%%%%%%%%
\begin{abstract}
\needsworklater{ We examine a functional based on FMT~\cite{roth2002whitebear}
 with smoother convolution kernels.  This functional is numerically
 easier to converge, and yet (hopefully) gives comparable results.}
\end{abstract}

\maketitle

%%%%%%%%%%%%%%%%%%%%%%%%%%%%%%%%%%%%%%%%%%%%%%%%%%%%%%%%%%%%
\section{Introduction}



\section{Fundamental-Measure Theory}

We use the White Bear version of the Fundamental-Measure Theory~(FMT)
functional published in reference~\cite{roth2002whitebear}.  The FMT
functional describes the excess free energy of a hard-sphere fluid.
This particular FMT reduces to the Carnahan-Starling equation of state
for homogeneous systems.
\begin{equation}
A_\textit{HS}[n] = k_B T \int \left(\Phi_1(\xx) + \Phi_2(\xx) + \Phi_3(\xx)\right) d\xx \; ,
\end{equation}
with integrands
\begin{align}
\Phi_1 &= -n_0 \ln\left( 1 - n_3\right)\\
\Phi_2 &= \frac{n_1 n_2 - \mathbf{n}_{V1} \cdot\mathbf{n}_{V2}}{1-n_3} \\
\Phi_3 &= (n_2^3 - 3 n_2 \mathbf{n}_{V2} \cdot \mathbf{n}_{V2}) \frac{
  n_3 + (1-n_3)^2 \ln(1-n_3)
}{
  36\pi n_3^2\left( 1 - n_3 \right)^2
} ,
\end{align}
using the weighted densities
\begin{align}
  n_3(\xx) &= \int n(\xx') \Theta(\left|\xx - \xx'\right| - R) d\xx' \\
  n_2(\xx) &= \int n(\xx') \delta(\left|\xx - \xx'\right| - R) d\xx'
\end{align}
\begin{align}
  \mathbf{n}_{V2} &= \mathbf{\nabla} n_3 , \quad
  \mathbf{n}_{V1} = \frac{\mathbf{n}_{V2}}{4\pi R} \\
  n_1 &= \frac{n_2}{4\pi R} , \quad
  n_0 = \frac{n_2}{4\pi R^2}
\end{align}

\section{Comparison with simulation}

\subsection{Hard spheres confined in a spherical cavity}

\begin{figure}
  \includegraphics[height=5cm]{figs/density-08-013}
  \caption{Density and contact densities of thirteen hard spheres in a
    spherical cavity with diameter 8.}
  \label{fig:sphere-8}
\end{figure}

\begin{figure}
  \includegraphics[height=5cm]{figs/density-12-112}
  \caption{Density and contact densities of 112 hard spheres in a
    spherical cavity with diameter 12.}
  \label{fig:sphere-12}
\end{figure}

\begin{figure}
  \includegraphics[height=5cm]{figs/density-16-265}
  \caption{Density and contact densities of 265 hard spheres in a
    spherical cavity with diameter 16.}
  \label{fig:sphere-16}
\end{figure}

\subsection{Hard spheres confined in a cubical cavity}

\begin{figure*}
\includegraphics[height=5cm]{figs/box-100c--05,05,05-13}
\includegraphics[height=5cm]{figs/box-100s--05,05,05-13}
\includegraphics[height=5cm]{figs/box-110c--05,05,05-13}
\caption{Density of hard spheres in a $5\times5\times5$ box.}
\label{fig:box-density}
\end{figure*}


%%%%%%%%%%%%%%%%%%%%%%%%%%%%%%%%%%%%%%%%%%%%%%%%%%%%%%%%%%%%
\section{Conclusion}

\bibliography{paper}% Produces the bibliography via BibTeX.

\end{document}

