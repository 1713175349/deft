%Notes on MS Project

\documentclass{article}

\usepackage[superscript]{cite}
\usepackage{amsmath}

\renewcommand{\thesection}{}

\begin{document}
\begin{sffamily}

\title{Notes on MS Project}
\author{D. Roth}
\date{}
\maketitle

\section{c. 26 Aug 13}
Started with group. My project will generally be based on using
Renormalization Group (RG) theory to get a better model of liquids
near the critical point. The theory has been developed by Forte
\textit{et al.}\cite{Forte11}. My project will mostly focus on the
numerics of this method---can it be efficiently implemented in a
computational model?

As an introduction, Roundy set me working on coding the recent Hughes
paper\cite{Hughes13} in Python, for the homogeneous (i.e. number
density is independent of position) case.

\section{11 Sept 13}
Using $a_1$ and $a_2$ from Gil-Villegas\cite{Gil-Villegas97}. In
particular, I am using the square-well model: $a_1^{SW}$ and
$a_2^{SW}$.

\section{13 Sept 13 (Friday the 13$^{th}$)}
Using the constants from Clark \textit{et al.}\cite{Clark06}. In
particular, I'm using the W1 set from Wertheim's four-cite model
(Table 1 in the paper). Reasons for using W1 are described in the
Clark paper, and this is the set that is defined as ``optimal'' in
Table 6.

\section{16 Sept 13}
Dr. Roundy took a look at my Python code for the 1$^{st}$ time. He had
suggestions on code readability, and we did some debugging. We got
some `nan' (not a number) results in my function $X$, and we narrowed
it down to trouble with my $g_\sigma^{SW}$ function, which seemed to
be far too large. It was on the order of 10$^3$, which is certainly
\textit{not} a small perturbation! The likely culprit is the
derivatives, which I shall re-evaluate and check against the code that
I have written. Hopefully it was a simple mistake on my part.

I plan on having plots of pressure vs.\ density in the time frame of a few weeks.

\section{17 Sept 13}
From ref \citenum{Gil-Villegas97}, with $g^{HS}(1;\eta_{eff}) \equiv g_{eff}^{HS}$\ :
\begin{equation*}
  a_1^{SW}(\eta,\lambda) = a_1^{VDW}(\eta, \lambda)g_{eff}^{HS}(\eta_{eff}(\eta,\lambda))
\end{equation*}
\renewcommand{\arraystretch}{2}
\begin{tabular}{| c | c |}
\hline
  Correctly Coded? & Equation \\
\hline
  Y & $ \partial_\eta a_1^{SW} = \partial_\eta a_1^{VDW}g_{eff}^{HS} + a_1^{VDW}\partial_\eta g_{eff}^{HS} $ \\
\hline
  Y & $\partial_\eta a_1^{VDW} = -4\epsilon(\lambda^3 - 1)$ \\
\hline
  N & $\begin{aligned}
        \partial_\eta g_{eff}^{HS} &= \partial_{\eta_{eff}}g_{\eta_{eff}}^{HS}\partial_\eta\eta_{eff} \\
        &= \frac{\partial}{\partial\eta_{eff}}\left[\frac{1-\frac{1}{2}\eta_{eff}}{(1-\eta_{eff})^3}\right]\frac{\partial\eta_{eff}}{\partial\eta} \\
        &= \left[\frac{1}{2(1-\eta_{eff})^3} + \frac{3(1 - \frac{1}{2}\eta_{eff})}{(1 - \eta_{eff})^4} \right]\left[c_1(\lambda) + 2c_2(\lambda)\eta + 3c_3(\lambda)\eta^2\right]
      \end{aligned}$ \\
\hline
  Probably not & $\partial_\lambda a_1^{SW} = \partial_\lambda a_1^{VDW}g_{eff}^{HS} + a_1^{VDW}\partial_\lambda g_{eff}^{HS}$ \\
\hline
  Y & $\partial_\lambda a_1^{VDW} = -12\eta\epsilon\lambda^2$ \\
\hline
  N & $\partial_\lambda g_{eff}^{HS} = \partial_{\eta_{eff}}g_{eff}^{HS}\partial_\lambda\eta_{eff}$ \\
\hline
  Y & $\partial_\eta\eta_{eff} = dc_1(\lambda)\eta + dc_2(\lambda)\eta^2 + dc_3(\lambda)\eta^3$ \\
\hline
\end{tabular}

After making appropriate changes, $g_\sigma^{SW}$ is on the order of 10$^0$, and no more `nan' results.

\section*{18 Sept 13}
Roundy suggested that $g_\sigma^{SW}$ should be larger than
$g_\sigma^{HS}$, which is not the case.

\section*{19 Sept 13}
I leared a trick to debugging: just because two expressions (in your
code and a reference code, for example) are mathematically the same
\textit{and} coded the same/correctly does not mean that your code is
correct. The trick is to actually have the program print what it is
doing.

\section*{24 Sept 13}
Jess Hughes' project report has the derivation for $p(T,n)$. It's a
bit different than ``normal'' because our functional gives free energy
per volume---so it's not straightforward to take a derivative wrt
volume. The formula ends up being
\begin{equation*}
  p(T,n) = \frac{\partial f}{\partial n}\bigg|_T + f(n)
\end{equation*}
where $f(n) = \tfrac{F(n)}{V}$.

\section*{Coexistance curve algorith using Hughes}
See comments in prefac.py

\bibliographystyle{ieeetr} %ieeetr is similar to unsrt combined with abbr
\bibliography{refs}

\end{sffamily}
\end{document}
