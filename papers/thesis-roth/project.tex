\documentclass[letterpaper,twocolumn,amsmath,amssymb,prb]{revtex4-1}
\usepackage{graphicx}% Include figure files. When building at home, use the option [demo] to supress actual need of figures
\usepackage{dcolumn}% Align table columns on decimal point
\usepackage{bm}% bold math
\usepackage{color}

%%%%%%%%%%%%%%%%%%%%%%%%%%%%%%%%%%%%%%%%%%%%%%%%%%%%%%%%%%%%
%% Definitions

% 1/k_B/T
\newcommand{\kT}{\ensuremath{k_BT}}

% vector
\newcommand{\rr}{\ensuremath{\mathbf{r}}}

% special values for n
\newcommand{\npart}{\ensuremath{n_\textit{part}}}
\newcommand{\nliq}{\ensuremath{n_\textit{liquid}}}
\newcommand{\nvap}{\ensuremath{n_\textit{vapor}}}
\newcommand{\neff}{\ensuremath{n_\textit{eff}}}

% Free energy terms
\newcommand{\Fattr}{\ensuremath{F_\text{attr}(T,n)}}
\newcommand{\Fex}{\ensuremath{F_\text{ex}}}

\newcommand{\fid}{\ensuremath{f_\text{ID}(T,n)}}
\newcommand{\fhs}{\ensuremath{f_\text{HS}(T,n)}}
\newcommand{\fdisp}{\ensuremath{f_\text{disp}(T,n)}}
\newcommand{\fattr}{\ensuremath{f_\text{attr}(T,n)}}
\newcommand{\fsw}{\ensuremath{f_\text{SW}(T,n)}}

% lambda
\newcommand{\lambdaSW}{\ensuremath{\lambda_\text{sw}}}
\newcommand{\lambdaRG}{\ensuremath{\lambda_\text{rg}}}

% TPT
\newcommand{\III}{\ensuremath{\textbf{r}_{12}}} % Roman numerals for \12
\newcommand{\Vtilde}{\ensuremath{\widetilde{V}}}
\newcommand{\Wtilde}{\ensuremath{\widetilde{W}}}

% RG stuff
\newcommand{\FbarD}{\ensuremath{\bar{F}(T,n,x)}}
\newcommand{\UbarD}{\ensuremath{\bar{U}(T,n,x,\lambdaRG)}}

% fixme
\newcommand{\fixme}[1]{\textcolor{red}{\textbf{[#1]}}}
\newcommand{\davidsays}[1]{\textcolor{blue}{\textit{[#1]}}}

%%%%%%%%%%%%%%%%%%%%%%%%%%%%%%%%%%%%%%%%%%%%%%%%%%%%%%%%%%%%

\begin{document}
\title{Applying Renormalization Group Theory to the Square Well Liquid}

\author{Dan Roth}
\affiliation{Department of Physics, Oregon State University, Corvallis, OR
97331}

%%%%%%%%%%%%%%%%%%%%%%%%%%%%%%%%%%%%%%%%%%%%%%%%%%%%%%%%%%%%
\begin{abstract}

This work develops a method for calculating liquid-vapor coexistence,
and applies renormalization group theory to the square well
liquid. Near the critical point of a liquid, there are density
fluctuations at all length scales. In this regime, mean field theories
break down. As the liquid approaches the critical point, the
difference in free energy between the liquid and vapor states
diminishes; this causes difficulty in modeling liquid-vapor
coexistence.

%\tableofcontents
\end{abstract}

\maketitle

%%%%%%%%%%%%%%%%%%%%%%%%%%%%%%%%%%%%%%%%%%%%%%%%%%%%%%%%%%%%
\section{Introduction}
Gases have been accurately modeled since the 19$^{th}$
century\cite{Lederman92} and formulation of the kinetic theory of gasses. With the
advent of quantum mechanics in the early 20$^{th}$ century, we have
had a pretty good grasp on solids. A robust model of liquids continues
to be elusive. Liquids are not periodic structures like solids, but
neither are they disperse, as are gasses.

However, understanding liquids has both philosophical and economic
desirability. Liquid water is---as far as we know---essential for
developing and sustaining life. All cellular processes take place in an
aqueous environment. Understanding liquids is in many ways a step
towards understanding the development and existence of life. On the
economic side of the coin, we recognize liquids to be an important
part of a wide array of industrial applications.

One particularly troublesome regime is around the critical
point of a liquid. At the appropriate density and temperature, the
distinction between liquid and vapor disappears. Beyond this point,
you have a ``supercritical fluid.'' Supercritical fluids are used in
many industrial processes.\cite{Perrut00} For example (and of personal
interest), supercritical fluids are used in coffee decaffeination and
the extraction of hops resins for brewing beer.

In this work, I focus exclusively on what is called the square well
liquid. The simplest model of a liquid is to consider each molecule to
be a hard sphere (i.e.~spheres which cannot overlap) with a square-well
potential between any pair of spheres. This is a widely used
approximation to simple liquids, and one of the foundations of
this work.

There are several approaches to study liquids, each with advantages
and disadvantages. Some methods, such as Monte Carlo and Molecular
Dynamics, are simulations of systems with a discrete number of
particles. Other methods are more abstract in nature. Methods such as
Thermodynamic Perturbation Theory, Density Functional Theory and
Renormalization Group Theory are mathematical theories that were not
necessarily developed with computational methods in mind (though in
practice numerical solutions are required).

Monte Carlo simulations (MC) are one method. These simulations can
give exact answers, if you run them ``long enough.''  So, MC methods
are very accurate, but very computationally expensive. Molecular
dynamics simulations (MD) calculate forces on molecules over a
particular span of time. These simulations are also accurate but even
more computationally expensive than MC. Calculations simulating only a
few hundred molecules for picoseconds to nanoseconds of interactions
can be expected.\cite{Hughes13}

Thermodynamic Perturbation Theory (TPT) gives a good description of
liquid systems far from the critical point. Similar to quantum
perturbation theory, TPT utilizes the power and simplicity of a power
series expansion in a small parameter. Unfortunately, TPT does break
down near the critical point.

Density Functional Theory (DFT) implements calculations based on
ground-state electron density distributions. DFT is good for solid
state systems, but liquids are too complicated. Simple DFT MD
simulations ``predict'' a melting point of water at 120$^\circ$C. More
complicated---and therefore computationally expensive---models are
slightly more accurate; one model calculates a melting point of ice as
87$^\circ$C.\cite{Yoo11} Classical DFT (cDFT) describes a fluid by a
free energy functional of the particle density
distribution.\cite{Krebs13} cDFT efficiently calculates density
profiles for inhomogeneous systems, that is, systems where there are
liquid interfaces such as dissolved solutes or ``hard walls.''
However, existing cDFTs do not accurately model the critical point of
a fluid.

It is the aim of renormalization group theory to tackle the critical
point. Calculations are difficult near the critical point because
there are large-scale fluctuations of the density. Indeed, there are
density fluctuations at \emph{all} length scales, which causes mean
field theories to fail. Renormalization group theory is an iterative
procedure that was developed in the 1950s, and Kenneth Wilson's work
with the theory in the 1970s won him the 1982 Nobel Prize in
physics.\cite{Nobel82} The end goal of this particular work is to
investigate into whether or not implementing renormalization group
theory is worthwhile to our research group. Many other groups use
renormalization group theory, but none of them report the iteration
depth used in their
calculations.\cite{Forte11,Ramana12,Ghobadi13,Wang13,Zhao13} If only
few iterations are required, then this could be an effective tool to
improve the accuracy of cDFTs. However, if many iterations are
required, the computational cost could be unwieldy.

In section~\ref{sec:theory}, I present some of the main theories I
deal with: thermodynamic perturbation
theory, the square well liquid free energy and renormalization group theory. In
section~\ref{sec:methods} I discuss the the methods used. I cover the
calculation of liquid-vapor coexistence curves in some detail, as
calculating coexistence near the critical point is tricky. The
results---such as they are---are covered in section~\ref{sec:results}.
%%%%%%%%%%%%%%%%%%%%%%%%%%%%%%%%%%%%%%%%%%%%%%%%%%%%%%%%%%%%
\section{Theory}\label{sec:theory}

\subsection{Thermodynamic perturbation theory}\label{subsec:TPT}

Thermodynamic perturbation theory (TPT) gives a good description of
simple liquids, as long a system is far from the critical point---which
is most of the time. TPT is a rigorous power series
expansion in a small parameter; as long as the parameter is small, the
theory is valid. This discussion follows primarily from \textit{Theory
  of Simple Liquids}, by Hansen and McDonald\cite{Hansen06}.

Thermodynamic perturbation theory treats the intermolecular potential
as separable into two parts, one for short-range repulsion and another
for long-range attraction. The short-range part is considered to be a
reference potential to which we add a perturbation. The following pair
potential $v(\III)$ describing the interaction between hard spheres is
our reference:
\begin{align}
  v_\text{hs}(\III) &=
    \begin{cases}
      \infty & r < \sigma \\
      0 & \sigma \leq r
    \end{cases}
\end{align}
To this reference, we take a square well attraction as a perturbation:
\begin{align}
  w_\text{sw}(\III) &= - \varepsilon \quad \sigma < r < \lambdaSW\sigma
\end{align}
so that we have
\begin{align}
  v_\xi(\III) &= v_\text{hs}(\III) + \xi w_\text{sw}(\III) \label{eqn:small-perturbation}
\end{align}
where $\xi$ is a parameter that allows us to make the perturbation arbitrarily small.
The total potential energy $\Vtilde$ of our system and its derivatives with
respect to $\xi$ (which become important later) are
\begin{align}
  \Vtilde &= \sum_{i=1}^N\sum_{j>i}^N v_\xi(\mathbf{r}_{ij})
\end{align}
\begin{align}
  \frac{\partial \Vtilde}{\partial\xi} &= \sum_{i=1}^N\sum_{j>i}^N w_\text{sw}(\mathbf{r}_{ij}) \nonumber \\
  &\equiv \Wtilde \\
  \frac{\partial^2 \Vtilde}{\partial\xi^2} &= 0
\end{align}

The \emph{configuration integral} (the potential energy component of
the partition function) and its derivative with respect to $\xi$
are
\begin{align}
  Z_\xi &= \int d\rr^N\, e^{-\beta \Vtilde}
\end{align}
\begin{align}
  \frac{\partial Z_\xi}{\partial\xi} &=  \int d\rr^N\, \frac{\partial}{\partial\xi}\left( e^{-\beta \Vtilde} \right) \nonumber \\
  &= -\int d\rr^N\, \beta\frac{\partial \Vtilde}{\partial\lambda}e^{-\beta \Vtilde} \nonumber \\
  &= -\int d\rr^N\, \beta \Wtilde e^{-\beta \Vtilde}
\end{align}
And the non-ideal part of the free energy (i.e.~the excess free
energy) and its derivative are
\begin{align}
  \Fex &= -\kT\ln\left( \frac{Z_\xi}{V^N} \right)
\end{align}
\begin{align}
  \frac{\partial \Fex}{\partial\xi} &= -\kT\frac{V^N}{Z_\xi}\frac{\partial}{\partial\xi}\left( \frac{Z_\xi}{V^N} \right) \nonumber \\
  &= -\kT\left( \frac{1}{Z_\xi}\frac{\partial Z_\xi}{\partial\xi} \right) \nonumber \\
  &= \frac{1}{Z_\xi} \int d\rr^N\, \Wtilde e^{-\beta \Vtilde} \label{eqn:dfdlambda_Z}\\
  &= \left\langle \Wtilde \right\rangle_{\xi} \label{eqn:dfdlambda}
\end{align}
(where $-\kT$ and $-\beta$ have divided out to $1$ before equation~\ref{eqn:dfdlambda_Z}). Here, $V^N$ is
the total volume taken to the $N^{th}$ power, and $\left\langle \Wtilde
\right\rangle_{\xi}$ is the ensemble average of $\Wtilde$, with the
subscript indicating some value of $\xi$.
Note that equation~\ref{eqn:dfdlambda} is exact; now take a series expansion around $\xi=0$:
\begin{align}
  \left\langle \Wtilde \right\rangle_{\xi} &= \left\langle \Wtilde\right\rangle_{\xi = 0} + (\xi)\frac{\partial\left\langle \Wtilde \right\rangle_{\xi}}{\partial\xi}\bigg|_{\xi = 0} + \mathcal{O}(\xi^2) \label{eqn:Wn-expansion}
\end{align}
The derivative term is
\begin{widetext}
  \begin{align}
    \frac{\partial}{\partial\xi}\left\langle \Wtilde \right\rangle_{\xi} &= \frac{\partial}{\partial\xi}\left( \frac{1}{Z_\xi}\int d\rr^N\, \Wtilde e^{-\beta \Vtilde}\right) \nonumber \\
    &= \frac{1}{Z_\xi}\frac{\partial}{\partial\xi}\left( \int d\rr^N\, \Wtilde e^{-\beta \Vtilde} \right) - \left( \int d\rr^N\, \Wtilde e^{-\beta \Vtilde} \right)\frac{1}{Z_\xi^2}\frac{\partial}{\partial\xi}Z_\xi \nonumber \\
    &= -\beta\left[ \frac{1}{Z_\xi}\int d\rr^N\, \Wtilde^2e^{-\beta \Vtilde} - \left( \frac{1}{Z_\xi}\int d\rr^N\, \Wtilde e^{-\beta \Vtilde} \right)^2 \right] \nonumber \\
    &= -\beta\left[ \left\langle \Wtilde^2 \right\rangle_{\xi} - \left\langle \Wtilde \right\rangle_{\xi}^2 \right] \nonumber \\
    &\equiv -\beta\sigma_{\xi}^2
  \end{align}
\end{widetext}
where $\sigma_xi^2$ is the variance of the perturbative potential \Wtilde. Combining this result with equation~\ref{eqn:Wn-expansion} gives us
\begin{align}
  \left\langle \Wtilde \right\rangle_{\xi} &= \left\langle \Wtilde\right\rangle_{\xi = 0} - \beta\xi\sigma_{\xi=0}^2 + \mathcal{O}(\xi^2) \label{eqn:Wn-expansion-simplified}
\end{align}
and we can insert this into equation~\ref{eqn:dfdlambda}:
\begin{align}
  \frac{\partial \Fex}{\partial\xi} &= \left\langle \Wtilde\right\rangle_{\xi = 0} - \beta\xi\sigma_{\xi=0}^2 + \mathcal{O}(\xi^2)
\end{align}
Now integrate this expression over $\Fex\in[F_\text{ref},\Fex]$ and $\xi\in[0,1]$ to obtain
\begin{align}
  \int_{F_\text{ref}}^{\Fex}\, d\Fex &= \int_0^1 d\xi\, \left[ \left\langle \Wtilde\right\rangle_{\xi = 0} - \beta\lambda\sigma_{\xi=0}^2 + \mathcal{O}(\xi^2) \right] \nonumber \\
  \hookrightarrow \Fex - F_\text{ref} &= (1-0)\left\langle \Wtilde \right\rangle_{\xi=0} - \frac{1}{2}\beta(1 - 0)^2\sigma_{\xi=0}^2 + \mathcal{O}(\beta^2) \nonumber \\
  \hookrightarrow \Fex &= F_\text{ref} + \left\langle \Wtilde \right\rangle_{\xi=0} - \frac{1}{2}\beta\sigma_{\xi=0}^2 + \mathcal{O}(\beta^2) \label{eqn:F-inTermsW} \\
  &= F_\text{ref} + a_1(n) + \beta a_2(n) + \mathcal{O}(\beta^2) \label{eqn:hi-T-exp}
\end{align}
with
\begin{align}
  a_1(n) &= \frac{1}{Z_\xi}\int d\rr^N\, \Wtilde e^{-\beta\Vtilde} \\
  a_2(n) &= -\frac{\sigma_{\xi=0}^2}{2}
\end{align}
Equation~\ref{eqn:hi-T-exp} is the \emph{high-temperature expansion}, first derived by
Zwanzig\cite{Zwanzig54}. He further showed that the $n^{th}$ term could be written in
terms of the mean fluctuations $\left\langle \left[ \Wtilde - \left\langle
  \Wtilde\right\rangle_{\xi=0} \right]^\nu \right\rangle_{\xi=0}$
with $\nu \leq n$. This requires us to know the form of higher-order
correlation functions of the reference system, which can be quite
complicated. In the next section we will see simple approximations for
the $n=1$ and $n=2$ terms.

\subsection{Square well liquid free energy}\label{subsec:SW}

This work applies entirely to the homogeneous case---that is, there
are no walls containing our liquid, nor are there any sort of solutes
in the liquid. Since our fluid is of infinite extent, the free energy
is infinite. Therefore, in the calculations I work with free energy density:
\begin{align}
  f(T,n) &\equiv \frac{F(T,n)}{V}
\end{align}

As a baseline for RG, I use the square well liquid free energy:\cite{Hughes13}
\begin{align}
  f_\text{TPT}(T,n) &= \fid + \fhs + \fsw
\end{align}
$\fid$ is the ideal gas free energy, $\fhs$ is the free energy contribution due to hard sphere
repulsion, and $\fsw$ is the free energy contribution due to square well attraction. Each function is outlined below.

\subsubsection{Ideal gas}\label{sub2sec:ID}
The ideal gas term is given as
\begin{align}
  \fid &= n\kT\left(\log(n) - 1\right)
\end{align}

\subsubsection{Hard sphere repulsion}\label{sub2sec:HS}
Hard sphere repulsive forces are dealt with by the White Bear version
of Fundamental-Measure Theory,\cite{Roth02} which simplifies to the
Carnahan-Starling free energy\cite{Carnahan69} in the homogeneous limit.
\begin{align}
  \fhs &= \kT\left( \Phi_1 + \Phi_2 + \Phi_3 \right)
\end{align}
The $\Phi_j$ terms are given as
\begin{align}
  \Phi_1 &= -n_0\ln(1 - n_3) \\
  \Phi_2 &= \frac{n_1n_2}{1 - n_3} \\
  \Phi_3 &= n_2^3\left( \frac{n_3 + (1 - n_3)^2\ln(1 - n_3)}{36\pi n_3^2(1 - n_3)^2} \right)
\end{align}
The \emph{fundamental measures} are given by
\begin{align}
  n_0 &= \frac{n_2}{4\pi R^2} \nonumber \\
  &= n
\end{align}
\begin{align}
  n_1 &= \frac{n_2}{4\pi R} \nonumber \\
  &= nR
\end{align}
\begin{align}
  n_2 &= n4\pi R^2
\end{align}
\begin{align}
  n_3 &= n\frac{4}{3}\pi R^3 \nonumber \\
  &= \eta \label{eqn:packingFrac}
\end{align}
where $\eta$ is the packing fraction.

\subsubsection{Attraction}\label{sub2sec:disp}
The attractive free energy includes square well attraction between
hard spheres and is founded in Thermodynamic Perturbation Theory
(TPT).
\begin{align}
  \fsw &= n \left( a_1(n) + \frac{1}{\kT}a_2(n) \right)
\end{align}
The terms $a_1(n)$ and $a_2(n)$ come from TPT, outlined in section~\ref{subsec:TPT}.

The first term, $a_1$, is the zeroth-order interaction. The
second term, $a_2$, describes the effect of fluctuations resulting
from compression of the fluid due to the dispersion interaction
itself. It is approximated using the local compressibility
approximation (LCA), which assumes the energy fluctuation is simply
related to the compressibility of a hard-sphere reference
fluid.\cite{Barker76}

Gil-Villegas\cite{gil-villegas97} gives an approximation for $a_1$ as
\begin{align}
  a_1^\text{SW}(n) &= -4\eta\varepsilon(\lambda^3 - 1)\frac{1 - \left( \eta_\textit{eff}/2 \right)}{(1 - \eta_\textit{eff})^3}
\end{align}
where
\begin{align}
  \eta_\textit{eff} &= c_1\eta + c_2\eta^2 + c_3\eta^3
\end{align}
with
\begin{align}
  \left( \begin{array}{c}
    c_1 \\
    c_2 \\
    c_3
    \end{array} \right)
  &= \left( \begin{array}{ccc}
    2.25855 & -1.50349 & 0.249434 \\
    -0.669270 & 1.40049 & -0.827739 \\
    10.1576 & -15.0427 & 5.30827
    \end{array} \right)
  \left( \begin{array}{c}
    1 \\
    \lambdaSW \\
    \lambdaSW^2
    \end{array} \right)
\end{align}


%% In Gil-Villegas' paper,\cite{gil-villegas97} HS and Dispersion are wrapped up into the
%% \textit{monomer} contribution, expressed in terms of energy densities
%% as:
%% \begin{align}
%%   f_\text{mono} = f_\text{HS} - \frac{\alpha^\text{VDW}n}{kT}
%% \end{align}
%% with $\alpha^\text{VDW}$ given by
%% \begin{align}
%%   \alpha^\text{VDW} &= 2\pi\varepsilon\int_\sigma^\infty r^2\phi(r)\,dr \\
%%   \intertext{using reduced units of $x = r/\sigma$ (\textsc{N.B.}~this is unrelated to the RGT amplitude of fluctuations $x$ from section~\ref{subsec:RGT}), we have}
%%   &= 2\pi\sigma^3\varepsilon\int_1^\infty x^2\phi(x)\,dx \nonumber \\
%%   &= 3b^\text{VDW}\varepsilon\int_1^\infty x^2\phi(x)\,dx
%% \end{align}
%% where $b^\text{VDW}$ is the van der Waals size parameter. It
%% corresponds to the volume excluded by two spherical particles of
%% volume $b$: $b^\text{VDW} = 4b = 4\left(\pi\sigma^3/6\right)$.

%% Then, the high temperature expansion is an expansion of the monomer term:
%% \begin{align}
%%   f_\text{mono} &= f_\text{HS} + \beta a_1 + \beta^2 a_2 + \mathcal{O}(\beta^3)
%% \end{align}

%% The term $a_1$ is given by
%% \begin{align}
%%   a_1 &= -2\pi n \varepsilon\int_\sigma^\infty r^2\phi(r)g^\text{HS}(r)\,dr \\
%%   &= -3 n  b^\text{VDW}\varepsilon\int_1^\infty x^2\phi(x)g^\text{HS}(x)\,dx
%% \end{align}
%% If we assume random correlations between the particles' positions, for
%% all distances, we have $g^\text{HS}(r) = 1$. This yields \fixme{derive this---or at least explain a little bit}
%% \begin{align}
%%   a_1^\text{VDW} = - n \alpha^\text{VDW},
%% \end{align}
%% the van der Waals mean-field energy.

The second-order term is quite a bit more complicated. As discussed in
section~\ref{subsec:TPT}, one must have knowledge of higher-order
correlation functions to exactly determine $a_2$. This term describes the
response of the attractive energy due to the compression of the fluid
from the attractive well. Gil-Villegas \emph{et al.} base their
expression on an approximation by Barker and Henderson.\cite{Barker67}
Barker and Henderson's approximation, called the Local Compressibility
Approximation (LCA), considers fluctuations in the number of particles
in the potential well. In this way, the fluctuations in $a_2$ are
related to the pressure and compressibility of the liquid. Given
pressure $P^\text{HS}$ and isothermal compressibility $K^\text{HS} =
kT\left(\partial n /\partial P^\text{HS}\right)_T$, we have:
\begin{align}
  a_2 &\approx \frac{1}{2}\varepsilon n K^\text{HS}\frac{da_1}{dn}
  %% a_2 &\approx -\pi n \varepsilon^2kT\int_0^\infty r^2\left[\phi(r)\right]^2\frac{\partial n  g^\text{HS}}{\partial P^\text{HS}}(r)\,dr \nonumber \\
  %% &= -\pi n \varepsilon^2K^\text{HS}\frac{\partial}{\partial n }\left[\int_\sigma^\infty r^2\left[\phi(r)\right]^2 n  g^\text{HS}\,dr\right] \nonumber \\
  %% &= \frac{1}{2}\varepsilon K^\text{HS}\frac{\partial}{\partial n }\left[-3 n  b^\text{VDW}\varepsilon\int_1^\infty x^2\left[\phi(x)\right]^2 g^\text{HS}(x)\,dx \right]
\end{align}

We use the isothermal compressibility in terms of packing fraction
$\eta$ (see equation~\ref{eqn:packingFrac}), given by the Percus-Yevick expression:\cite{Barker76}
\begin{align}
  K^\text{HS} &= \frac{\left(1 - \eta\right)^4}{1 + 4\eta + 4\eta^2}
\end{align}

\subsection{Renormalization Group Theory}\label{subsec:RGT}
Renormalization group theory (RGT) is a recursive procedure that was
developed to deal with large-scale fluctuations of the density
near the critical point of a system. The formulation used here is
provided by Forte~\textit{et al}.\cite{Forte11}

We begin by dividing the total volume into cells of volume $V_i$. We
assume that within a cell the number density $n(\rr)$ varies only slightly, and that the
cells do not interact with one another. This approach is referred to as Wilson's
phase cell method.\cite{Ramana12}

Then consider a baseline free energy $F_0(T,n)$, which we will take
from TPT. We assume this baseline accounts for all fluctuations of
wavelength $ < \lambda_0$. We then double the size of our phase cell
and add a correction to the free energy $\delta F_1(T,n)$. This
correction accounts for fluctuations with a wavelength up to
$2\lambda_0$. The new free energy, $F_1(T,n)$, is
\begin{align}
  F_1(T,n) &= F_0(T,n) + \delta F_1(T,n)
\end{align}
Then double the length scale again; $\delta F_2(T,n)$ accounts for
fluctuations up to $4\lambda_0$ and
\begin{align}
  F_2(T,n) &= F_1(T,n) + \delta F_2(T,n)
\end{align}
As we continue this process, the $i^{th}$ correction accounts for
fluctuations of $2^i\lambda_0$, and
\begin{align}
  F_i(T,n) &= F_{i-1}(T,n) + \delta F_i(T,n)
\end{align}
In the end, the free energy is
\begin{align}
  F(T,n) = F_0(T,n) + \Fattr + \sum_{i=1}^\infty\delta F_i(T,n)
\end{align}
where $\Fattr$ is the contribution due to the longest-wavelength
density fluctuations. At this stage, I do not entirely understand why
Forte separates out $\Fattr$.

The $\delta F_i(T,n)$ term is a difference in free energies:
\begin{align}
  \delta F_i(T,n) &= -\kT\ln Z - \left( -\kT\ln Z^* \right) \nonumber \\
  &= -\kT\ln\left[ \frac{Z(T,n)}{Z^*(T,n)} \right]
\end{align}
where $Z$ is the partition function for the entire system, and $Z^*$
is the partition function of a reference system consisting only of the
term $\FbarD$. Forte gives these terms as
\begin{multline}
  Z(T,n,x) = \\ \int dx\, \exp\left[-\beta V\left( \FbarD + \UbarD \right)\right] \label{eqn:Z} \\
\end{multline}
\begin{multline}
  Z^*(T,n,x) = \int dx\, \exp\left[ -\beta V\FbarD \right] \label{eqn:Zstar}
\end{multline}
where $x$ is the amplitude of density fluctuations. $\FbarD$ and $\UbarD$ are formulated similar to averages:
\begin{align}
  \FbarD &= \frac{F_{i-1}(T,n+x) + F_{i-1}(T,n-x)}{2} - F_{i-1}(T,n)
\end{align}
\begin{multline}
  \UbarD = \frac{U(T,n+x,\lambdaRG) + U(T,n-x,\lambdaRG)}{2} \\
  - U(T,n,\lambdaRG)
\end{multline}
\FbarD represents the free energy change due to fluctuations if interactions between cells are neglected, and \UbarD is the contribution of the interactions between cells. Finally, $U(T,n,\lambdaRG)$ is given by Forte as
\begin{multline}
  U(T,n,\lambdaRG) = g_\sigma^\text{HS}(\neff)\alpha(\lambdaRG) \\
  - \left( \frac{4\pi^2}{2^{2i+1}L^2} \right)\alpha(\lambdaRG)\omega^2(\lambdaRG) \\
  + \left( \frac{4\pi^2}{2^{2i+1}L^2} \right)^2\alpha(\lambdaRG)\gamma(\lambdaRG) \label{eqn:URG} % URG! It sounds like it hurt! :P
\end{multline}
with
\begin{align}
  \alpha(\lambdaRG) &= \frac{2}{3}\pi\epsilon\sigma^3(\lambdaRG^3 - 1) \\
  \omega^2(\lambdaRG) &= \frac{1}{5}\sigma^2\frac{\lambdaRG^5 - 1}{\lambdaRG^3 - 1} \\
  \gamma(\lambdaRG) &= \frac{1}{70}\sigma^4\frac{\lambdaRG^7 - 1}{\lambdaRG^3 - 1}
\end{align}
The pair correlation function $g_\sigma(n)$ tells us how likely we
are to find two hard spheres touching, given some density. In
equation~\ref{eqn:URG} we are considering this quantity with an
effective density \neff.

%%%%%%%%%%%%%%%%%%%%%%%%%%%%%%%%%%%%%%%%%%%%%%%%%%%%%%%%%%%%
\section{Methods}\label{sec:methods}

\subsection{Coexistence Curve Algorithm}\label{subsec:coexis}
Coexistence curves are found by modifying the chemical potential $\mu$
in the grand free energy per volume $\phi(T,n)/V$. Grand free energy
density is defined as
\begin{align}
  \frac{\Phi}{V}(T,n) &= \phi(T,n) \nonumber \\
                 &= f(T,n) - n\mu \nonumber \\
                 &= f(T,n) - n\frac{df}{dn}(T,n)\bigg|_{\npart}\ .
\end{align}
$\npart$ is some particular density, which defines the chemical
potential; it is this value that we adjust to find coexistence curves.

\begin{figure}
  \centering
  \includegraphics[width=0.5\textwidth]{figs/SW-phi-lowT}
  \caption{Grand free energy per unit volume, square well, low temperature}
  \label{fig:SW-phi-lowT}
\end{figure}

The grand free energy density below the critical temperature has two distinct minima when
plotted versus density. (See fig.~\ref{fig:SW-phi-lowT} for an
example.) The fluid will be in liquid-vapor equilibrium when those
minima have the same grand free energy. I have designed an algorithm
that adjusts \npart\ (that is, adjusts $\mu$) until the system is in
liquid-vapor equilibrium.  The algorithm is as follows; start at some
temperature $T=T_{start}$, with an initial guess for $\npart$. This
initial guess can be read off of a plot, and optimized ``by hand''
with minimal effort.
\begin{enumerate}
  \item Calculate $\nvap$ and $\nliq$ by calving for the minima in $\phi(T_{start},n)$ over appropriate regions
  \item Calculate $\phi(T_{start},\nvap)$ and $\phi(T_{start},\nliq)$ \label{while-start}
  \item Calculate an estimated change in chemical potential using the slope \[\delta\mu = \frac{\phi(T_{start},\nvap) - \phi(T_{start},\nliq)}{\nliq - \nvap}\]
  \item Calculate $\phi^*(T_{start},n) = \phi(T_{start},n) + n\delta\mu$
  \item Calculate a new $n$ particular value, $\npart^*$, based on $\phi^*(T_{start},n)$ by finding the local maximum between $\nvap$ and $\nliq$
  \item Recalculate the minima in the grand free energy using $\npart^*$ to find an improved estimate for \nvap\ and \nliq
  \item Go back to step~\ref{while-start} and repeat until the grand free energy at liquid and vapor densities are sufficiently similar. I used a dimensionless computation tolerance of 10$^{-5}$.
  \item Increase the temperature to $T = T + \delta T$ and repeat using the final $\npart^*$ as our initial guess for the dividing point between the liquid and vapor free energy minima
\end{enumerate}

\begin{figure}
  \begin{center}
  \includegraphics[width=0.5\textwidth]{figs/coexistence_SW}
  \end{center}
  \caption{Liquid-vapor coexistence for a square well liquid.}
  \label{fig:coexistance_SW}
\end{figure}

Figure~\ref{fig:coexistance_SW} is a demonstration of this algorithm.

\subsection{RG partition functions}\label{subsec:fbar-ubar}
I use the midpoint method for evaluating the integrals $Z$ and $Z^*$
(equations~\ref{eqn:Z} and~\ref{eqn:Zstar}). This method was chosen
for ease of implementation and to avoid problematic endpoint
cases. The approximation is
\begin{align}
  \int_a^b dx\, f(x) &\approx dx\sum_{i=0}^{N-1} f(x_i) \\
  x_i &= a + dx(i+1/2)
\end{align}

%%%%%%%%%%%%%%%%%%%%%%%%%%%%%%%%%%%%%%%%%%%%%%%%%%%%%%%%%%%%
\section{Results and discussion}\label{sec:results}

\begin{figure}
  \begin{center}
  \includegraphics[width=0.5\textwidth]{figs/phi-RG-lowT}
  \end{center}
  \caption{Free energy density for RG at low temperature}
  \label{fig:phi-RG-lowT}
\end{figure}

\begin{figure}
  \begin{center}
  \includegraphics[width=0.5\textwidth]{figs/phi-RG-highT}
  \end{center}
  \caption{Free energy density for RG near critical temp}
  \label{fig:phi-RG-highT}
\end{figure}

\begin{figure}
  \begin{center}
  \includegraphics[width=0.5\textwidth]{figs/coexistence-RG-forte}
  \end{center}
  \caption{Liquid-vapor coexistence, compared with Forte's 2011 data, and with molcular dynamics simulations from White\cite{White00}}
  \label{fig:coexistence-RG-forte}
\end{figure}

Figures~\ref{fig:phi-RG-lowT} and~\ref{fig:phi-RG-highT} show the
grand free energy per unit volume at a low temperature and near the
critical point, respectively, for a value of $\lambdaSW = 1.5$. Results are not particularly clear at
this point, as the program has not been fully debugged. This is
evident in figure~\ref{fig:phi-RG-lowT}; one notes that the curve for
$i=1$ is neither complete nor
accurate. Figure~\ref{fig:coexistence-RG-forte} gives another
picture of the shortcomings of my RG implementation. We see that the
baseline ($i=0$; identical to SW calculations) is reasonable, with
expected shortcomings near the critical point. However, calculations
including even one iteration ($i=1$) are not included because they flat-out fail.


%%%%%%%%%%%%%%%%%%%%%%%%%%%%%%%%%%%%%%%%%%%%%%%%%%%%%%%%%%%%
\section{Conclusion}\label{sec:conclusion}

I have implemented a the use of renormalization group theory in
calculations for the square well liquid. The program is not entirely
debugged. I have also developed an algorithm that calculates
liquid-vapor coexistence curves, even close to the critical point.

We do not yet know if RGT is worth implementing in DFT. Since the program has
not been fully debugged, I am unable to say what the computational
cost would be to obtain an accurate liquid-vapor coexistence curve.

%%%%%%%%%%%%%%%%%%%%%%%%%%%%%%%%%%%%%%%%%%%%%%%%%%%%%%%%%%%%
\bibliographystyle{unsrt}
\bibliography{project} % Produces the bibliography via BibTeX.

\end{document}

