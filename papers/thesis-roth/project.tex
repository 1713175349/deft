\documentclass[letterpaper,twocolumn,amsmath,amssymb,prb]{revtex4-1}
\usepackage{graphicx}% Include figure files
\usepackage{dcolumn}% Align table columns on decimal point
\usepackage{bm}% bold math
\usepackage{color}

%%%%%%%%%%%%%%%%%%%%%%%%%%%%%%%%%%%%%%%%%%%%%%%%%%%%%%%%%%%%
%% Definitions

% 1/k_B/T
\newcommand{\kt}{k_BT}

% special values for n
\newcommand{\npart}{n_{particular}}
\newcommand{\nliq}{n_{liquid}}
\newcommand{\nvap}{n_{vapor}}

%%%%%%%%%%%%%%%%%%%%%%%%%%%%%%%%%%%%%%%%%%%%%%%%%%%%%%%%%%%%

\begin{document}
\title{Applying Renormalization Group Theory to the Square Well Liquid}

\author{Dan Roth}
\affiliation{Department of Physics, Oregon State University, Corvallis, OR
97331}

\author{David Roundy}
\affiliation{Department of Physics, Oregon State University, Corvallis, OR
97331}

%%%%%%%%%%%%%%%%%%%%%%%%%%%%%%%%%%%%%%%%%%%%%%%%%%%%%%%%%%%%
\begin{abstract}
{}

%\tableofcontents
\end{abstract}

\maketitle

%%%%%%%%%%%%%%%%%%%%%%%%%%%%%%%%%%%%%%%%%%%%%%%%%%%%%%%%%%%%
\section{Introduction}

%%%%%%%%%%%%%%%%%%%%%%%%%%%%%%%%%%%%%%%%%%%%%%%%%%%%%%%%%%%%
\section{Theory and Methods}

\subsection{Renormalization Group Theory}
Renormalization group (RG) theory is an iterative procedure that was
developed to deal with large-scale fluctuations of the order parameter
near the critical point of a system.

The free energy density for each iteration is given by
\begin{equation}
  f_i(T,n) = f_{i-1}(T,n) + \delta f_i(T,n)\ .
\end{equation}
The total free energy density is obtained when one takes the
iterations to infinity and adds in the attractive component:
\begin{equation}
  f(T,n) = \lim_{i \to \infty} f_i(T,n) + f_{att}(T,n)\ .
\end{equation}
The initial free energy density, $f_0$ depends on the liquid model of study. For this work, I use the square well model:
\begin{equation}
  f_0(T,n) = f_{id}(T,n) + f_{HS}(T,n) + a_2^{SW}(T,n)\ ,
\end{equation}
where $a_2^{SW}(T,n)$ is the second-order term in a high-temperature expansion around $1/\kt$.~\cite{Gil-Villegas97}

\subsection{Coexistence Curve Algorithm}
Coexistence curves are found by modifying the chemical potential $\mu$
in the grand free energy per volume $\Phi/V$. Grand free energy
density is defined as
\begin{align}
  \frac{\Phi}{V} &= \phi \nonumber \\
                 &= f(T,n) - n\mu \nonumber \\
                 &= f(T,n) - n\frac{d}{dn}f(T,\npart)\ .
\end{align}
It is the value $\npart$ that we adjust to find coexistence curves.

The grand free energy density generally has two distinct minima when
plotted versus density. The liquid will be in liquid-vapor equilibrium
when those minima have the same value for $\phi$. There is a local
maximum between those minima, and the value of the density at that
maximum is used for $\npart$. Find this value at a given temperature
then plot $\phi$ vs $n$ at slightly higher temperature. Find the new
value for $\npart$ such that the two minima have the same value for
$\phi$. (It is helpful to note that if $\npart$ is increased,
$\phi(\nliq)$ increases, and $\phi(\nliq)$ decreases if $\npart$ is
decresed.) The program I wrote (npart.py) uses the previous
temperature's $\npart$ as a ``first guess,'' maximizing
$\phi(T,\npart)$. From there, the program adjusts $\npart$
until $\phi(T,\nvap) = \phi(T,\nliq)$. When that condition is met, it
uses this new value of $\npart$ as the ``first guess'' at the next
higher temperature.

%%%%%%%%%%%%%%%%%%%%%%%%%%%%%%%%%%%%%%%%%%%%%%%%%%%%%%%%%%%%
\section{Results and discussion}

\includegraphics[width=\columnwidth]{figs/coexistance_SW}


%%%%%%%%%%%%%%%%%%%%%%%%%%%%%%%%%%%%%%%%%%%%%%%%%%%%%%%%%%%%

\section{Conclusion}

\bibliographystyle{unsrt}
\bibliography{project} % Produces the bibliography via BibTeX.

\end{document}

