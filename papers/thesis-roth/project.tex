\documentclass[letterpaper,twocolumn,amsmath,amssymb,prb]{revtex4-1}
\usepackage{graphicx}% Include figure files
\usepackage{dcolumn}% Align table columns on decimal point
\usepackage{bm}% bold math
\usepackage{color}

%%%%%%%%%%%%%%%%%%%%%%%%%%%%%%%%%%%%%%%%%%%%%%%%%%%%%%%%%%%%
%% Definitions

% 1/k_B/T
\newcommand{\kT}{k_BT}

% special values for n
\newcommand{\npart}{n_{particular}}
\newcommand{\nliq}{n_{liquid}}
\newcommand{\nvap}{n_{vapor}}

% Free energy terms
\newcommand{\fid}{f_\text{ID}(T,n)}
\newcommand{\fhs}{f_\text{HS}(T,n)}
\newcommand{\fdisp}{f_\text{disp}(T,n)}
\newcommand{\fattr}{f_\text{attr}(T,n)}

\newcommand{\fixme}[1]{\textcolor{red}{\textbf{[#1]}}}

%%%%%%%%%%%%%%%%%%%%%%%%%%%%%%%%%%%%%%%%%%%%%%%%%%%%%%%%%%%%

\begin{document}
\title{Applying Renormalization Group Theory to the Square Well Liquid}

\author{Dan Roth}
\affiliation{Department of Physics, Oregon State University, Corvallis, OR
97331}

\author{David Roundy}
\affiliation{Department of Physics, Oregon State University, Corvallis, OR
97331}

%%%%%%%%%%%%%%%%%%%%%%%%%%%%%%%%%%%%%%%%%%%%%%%%%%%%%%%%%%%%
\begin{abstract}

%\tableofcontents
\end{abstract}

\maketitle

%%%%%%%%%%%%%%%%%%%%%%%%%%%%%%%%%%%%%%%%%%%%%%%%%%%%%%%%%%%%
\section{Introduction}

%%%%%%%%%%%%%%%%%%%%%%%%%%%%%%%%%%%%%%%%%%%%%%%%%%%%%%%%%%%%
\section{Theory}\label{sec:theory}

This work applies entirely to the homogeneous case---that is, there are no walls containing our liquid, nor are there any sort of solutes in the liquid. This being the case, the free energy is infinite. Therefore, I work with free energy density:
\begin{align}
  f(T,n) &\equiv \frac{F(T,n)}{V}
\end{align}

\subsection{Renormalization Group Theory}\label{subsec:RGT}
Renormalization group (RG) theory is an iterative procedure that was
developed to deal with large-scale fluctuations of the order parameter
near the critical point of a system. The formulation used here is provided by Forte~\textit{et al}\cite{Forte11}.

\fixme{Discuss Wilson phase cell}

We begin with a baseline free enrgy $f_0(T,n)$. We assume this baseline accounts for all fluctuations of wavelength $\lambda < \lambda_0$. We then double the size of our phase cell and add a correction to the free energy $\delta f_1(T,n)$. This correction accounts for fluctuations with a wavelength $\lambda = 2\lambda_0$. The new free energy is $f_1(T,n) = f_0(T,n) + \delta f_1(T,n)$. Then double the length scale again; $\delta f_2(T,n)$ accounts for fluctionations with $\lambda = 4\lambda_0$ and $f_2(T,n) = f_1(T,n) + \delta f_2(T,n)$. As we continue this process, the $i^{th}$ correction accounts for fluctuations with $\lambda = 2^i\lambda_0$, and $f_i(T,n) = f_{i-1}(T,n) + \delta f_i(T,n)$. In the end, your free energy is
\begin{align}
  f(T,n) = f_0(T,n) + \sum_{i=1}^\infty\delta f_i(T,n) + \fattr
\end{align}

Where $\fattr$ is the contribution due to the longest-wavelenght density fluctuations. At this stage, I do not entirely understand why Forte splits up $\fattr$.

%% \fixme{The following comes from Forte~\cite{Forte11} and doesn't make a whole lot of sense to me}

%% The free energy for each iteration is given by
%% \begin{equation}
%%   f_i(T,n) = f_{i-1}(T,n) + \delta f_i(T,n)\ .
%% \end{equation}
%% The total free is obtained when one takes the
%% iterations to infinity and adds in the attractive component:
%% \begin{equation}
%%   f(T,n) = \lim_{i \to \infty} f_i(T,n) + \fattr\ .
%% \end{equation}
%% The initial free energy density, $f_0$ depends on the liquid model of study. For this work, I use the square well model:
%% \begin{equation}
%%   f_0(T,n) = \fid + \fhs + \frac{a_2^{SW}(T,n)}{\kT}\ ,
%% \end{equation}
%% where $a_2^{SW}(T,n)$ is the second-order term in a high-temperature expansion around $1/\kT$. (See ref~\cite{Gil-Villegas97}.)
%% \fixme{End of Forte confusion}

\subsection{Square well liquid free enrgy}\label{subsec:SW}
For the baseline, this work is based on the square well liquid free energy\cite{Hughes13}:
\begin{align}
  f_0(T,n) &= \fid + \fhs + \left( \fdisp - n a_1(n) \right) \ .
\end{align}
The long-range attractive term, $\fattr$ is given by
\begin{align}
  \fattr &= n a_1(n) \ .
\end{align}
$\fid$ is the ideal gas free energy, $\fhs$ is the hard sphere repulsion, and $\fdisp$ is the attraction. The functions $a_j(n)$ are the first two terms from a high-temperature perturbation expansion. Each function is outlined below.

\subsubsection{Ideal gas}\label{sub2sec:ID}
\begin{align}
  \fid &= n\kT\left(\log(n) - 1\right)
\end{align}

\subsubsection{Hard sphere repulsion}\label{sub2sec:HS}


\subsubsection{Attraction}\label{sub2sec:disp}
The attractive free energy includes van der Waals attraction between hard spheres and is founded in Thermodynamic Perturbation Theory (TPT).
\begin{align}
  \fdisp &= n \left( a_1(n) + \frac{1}{\kT}a_2(n) \right)
\end{align}
The $a_j(n)$ terms come from TPT, outlined in section~\ref{subsec:TPT}. I give their forms here. \fixme{Pull this from the Square well contribution in homogeneous case section in the SW paper}

\subsection{Thermodynamic perturbation theory}\label{subsec:TPT}
\fixme{Pull TPT section from SW paper}

%%%%%%%%%%%%%%%%%%%%%%%%%%%%%%%%%%%%%%%%%%%%%%%%%%%%%%%%%%%%
\section{Methods}\label{sec:methods}

\subsection{Coexistence Curve Algorithm}\label{subsec:coexis}
Coexistence curves are found by modifying the chemical potential $\mu$
in the grand free energy per volume $\phi(T,n)/V$. Grand free energy
density is defined as
\begin{align}
  \frac{\Phi(T,n)}{V} &= \phi(T,n) \nonumber \\
                 &= f(T,n) - n\mu \nonumber \\
                 &= f(T,n) - n\frac{df(T,n)}{dn}\bigg|_{\npart}\ .
\end{align}
It is the value $\npart$ that we adjust to find coexistence curves.

The grand free energy density generally has two distinct minima when
plotted versus density. (See fig.~\ref{fig:SW-phi-lowT} for an example.) The liquid will be in liquid-vapor equilibrium
when those minima have the same value for $\phi(T,n)$. There is a local
maximum between those minima, and the value of the density at that
maximum is used for $\npart$. Find this value at a given temperature
then plot $\phi(T,n)$ vs $n$ at slightly higher temperature. Find the new
value for $\npart$ such that the two minima have the same value for
$\phi(T,n)$. (It is helpful to note that if $\npart$ is increased,
$\phi(T,\nliq)$ increases, and $\phi(T,\nliq)$ decreases if $\npart$ is
decresed.) The program I wrote uses the previous
temperature's $\npart$ as a ``first guess,'' maximizing
$\phi(T,\npart)$. From there, the program adjusts $\npart$
until $\phi(T,\nvap) = \phi(T,\nliq)$. When that condition is met, it
uses this new value of $\npart$ as the ``first guess'' at $\npart$ for the next
higher temperature.

The algorithm is as follows; start at some temperature $T=T_{start}$, with an intial guess for $\npart$.
\begin{enumerate}
  \item Calculate $\nvap$ and $\nliq$ by calculating the minima in $\phi(T_{start},n)$ over appropriate regions
  \item Calculate $\phi(T_{start},\nvap)$ and $\phi(T_{start},\nliq)$ \label{while-start}
  \item Calculate \[\delta\mu = \frac{\phi(T_{start},\nvap) - \phi(T_{start},\nliq)}{\nliq - \nvap}\]
  \item Calculate $\phi^*(T_{start},n) = \phi(T_{start},n) + \delta\mu$
  \item Calculate a new $\npart$, $\npart^*$, based on $\phi^*(T_{start},n)$ by finding the local maximum between $\nvap$ and $\nliq$
  \item Calculate $\phi(T_{start},\npart^*)$
  \item Re-caluclate $\nvap$, $\nliq$, $\phi(T_{start},\nvap)$, and $\phi(T_{start},\nliq)$, using $\npart^*$
  \item Go back to step~\ref{while-start} and repeat until \[ \frac{\phi(T_{start},\nvap) - \phi(T_{start},\nliq)}{\phi(T_{start},\npart^*)} > tol  \] where $tol$ is some computational tolerance (I used $tol=10^{-5}$)
  \item Increase the temperature to $T = T + \delta T$ and repeat
\end{enumerate}

\begin{figure}
  \centering
  \includegraphics[width=\columnwidth]{figs/SW-phi-lowT}
  \caption{Grand free energy per unit volume, SW, low temp}
  \label{fig:SW-phi-lowT}
\end{figure}

%%%%%%%%%%%%%%%%%%%%%%%%%%%%%%%%%%%%%%%%%%%%%%%%%%%%%%%%%%%%
\section{Results and discussion}

\begin{figure}
  \begin{center}
  \includegraphics[width=\columnwidth]{figs/coexistance_SW}
  \end{center}
  \caption{Liquid-vapor coexistance for a square well liquid.}
  \label{fig:coexistance_SW}
\end{figure}

\begin{figure}
  \begin{center}
  \includegraphics[width=\columnwidth]{figs/SW-RG-compare-phi-lowT}
  \end{center}
  \caption{Free energy density for RG at low temperature}
  \label{fig:SW-RG-compare-lowT}
\end{figure}

\begin{figure}
  \begin{center}
  \includegraphics[width=\columnwidth]{figs/SW-RG-compare-phi-highT}
  \end{center}
  \caption{Free energy density for RG near critical temp}
  \label{fig:SW-RG-compare-highT}
\end{figure}


%%%%%%%%%%%%%%%%%%%%%%%%%%%%%%%%%%%%%%%%%%%%%%%%%%%%%%%%%%%%

\section{Conclusion}


\bibliographystyle{unsrt}
\bibliography{project} % Produces the bibliography via BibTeX.

\end{document}

