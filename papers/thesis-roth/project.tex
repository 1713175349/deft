\documentclass[letterpaper,twocolumn,amsmath,amssymb,prb]{revtex4-1}
\usepackage{graphicx}% Include figure files
\usepackage{dcolumn}% Align table columns on decimal point
\usepackage{bm}% bold math
\usepackage{color}

%%%%%%%%%%%%%%%%%%%%%%%%%%%%%%%%%%%%%%%%%%%%%%%%%%%%%%%%%%%%
%% Definitions

% 1/k_B/T
\newcommand{\kT}{\ensuremath{k_BT}}

% vector
\newcommand{\rr}{\ensuremath{\mathbf{r}}}

% special values for n
\newcommand{\npart}{\ensuremath{n_{particular}}}
\newcommand{\nliq}{\ensuremath{n_{liquid}}}
\newcommand{\nvap}{\ensuremath{n_{vapor}}}

% Free energy terms
\newcommand{\Fattr}{\ensuremath{F_\text{attr}(T,n)}}

\newcommand{\fid}{\ensuremath{f_\text{ID}(T,n)}}
\newcommand{\fhs}{\ensuremath{f_\text{HS}(T,n)}}
\newcommand{\fdisp}{\ensuremath{f_\text{disp}(T,n)}}
\newcommand{\fattr}{\ensuremath{f_\text{attr}(T,n)}}

% pair potential
\newcommand{\1}{\ensuremath{\textbf{r}_1}}
\newcommand{\2}{\ensuremath{\textbf{r}_2}}
\newcommand{\3}{\ensuremath{\textbf{r}_3}}
\newcommand{\4}{\ensuremath{\textbf{r}_4}}

% RG stuff
\newcommand{\FbarD}{\ensuremath{\bar{F}_D(T,n)}}
\newcommand{\UbarD}{\ensuremath{\bar{U}_D(T,n)}}

% fixme
\newcommand{\fixme}[1]{\textcolor{red}{\textbf{[#1]}}}
\newcommand{\davidsays}[1]{\textcolor{blue}{\textit{[#1]}}}

%%%%%%%%%%%%%%%%%%%%%%%%%%%%%%%%%%%%%%%%%%%%%%%%%%%%%%%%%%%%

\begin{document}
\title{Applying Renormalization Group Theory to the Square Well Liquid}

\author{Dan Roth}
\affiliation{Department of Physics, Oregon State University, Corvallis, OR
97331}

%%%%%%%%%%%%%%%%%%%%%%%%%%%%%%%%%%%%%%%%%%%%%%%%%%%%%%%%%%%%
\begin{abstract}

Near the critical point of a liquid, there are density fluctuations at
all length scales. In this regime, mean field theories break down. As
the critical point is approached, there becomes less and less of a
difference in free energy of the liquid and vapor states; this causes
difficulty in modeling liquid-vapor coexistence. This work develops a
method for calculating liquid-vapor coexistence, and applies
renormalization group theory to the square well liquid.

%\tableofcontents
\end{abstract}

\maketitle

%%%%%%%%%%%%%%%%%%%%%%%%%%%%%%%%%%%%%%%%%%%%%%%%%%%%%%%%%%%%
\section{Introduction}

Gasses have been accurately modeled since the 19$^{th}$
century\cite{Lederman92} and the kinetic theory of gasses. With the
advent of quantum mechanics in the early 20$^{th}$ century, we have
had a pretty good grasp on solids. A robust model of liquids continues
to be elusive. Liquids are not periodic structures like solids, but
neither are they disperse, as are gasses.

However, understanding liquids has both philosophical and economic
desireability. Liquid water is---as far as we know---essential for
developing and sustaining life. All celluar processes take place in an
aqueous environment. Understanding liquids is in many ways a step
towards understanding the development and existence of life. On the
economic side of the coin, we recognize liquids to be an important
part of a wide array of industrial applications.

One particularly troublesome regime is around the so-called ``critical
point'' of a liquid. At the appropriate density and temperature, the
distinction between liquid and vapor dissapears. Beyond this point,
you have a ``supercritical fluid.'' Supercritical fluids are used in
many industrial processes\cite{Perrut00}. For example (and of personal
interest), supercritical fluids are used in coffee decaffeination and
the extraction of hops resins for brewing beer.

The simplest model of a liquid is to consider each molecule to be a
hard sphere (i.e.~spheres cannot overlap) with a square-well potenial
between any pair of spheres. This is called the ``square well
liquid,'' and it is a widely used approximation to simple
liquids. This is one of the foundations of this work.

We use statistical mechanics to study liquids; this requires large
sample sizes, long simulation times, or some combination of the
two. Monte Carlo simulations are exact, if you allow them to run
``long enough.'' Our models are ultimately compared to these
simulations, but Monte Carlo simulations are otherwise outside the
scope of this work. The Roundy group uses classical Density-funtional
Theory (DFT) to calculate free energies and probability distributions
of bulk liquids. The end goal of this work is to investigate into
wheather or not implementing renormalization group theory is
worthwhile to the Roundy research group.

In section~\ref{sec:theory}, I present some of the main theories I
deal with: renormalization group theory, thermodynamic perturbation
theory and the square well liquid free energy. In
section~\ref{sec:methods} I discuss the the methods used. I cover the
calculation of liquid-vapor coexistence curves in some detail, as
caluclating coexistence near the critical point is tricky.

%%%%%%%%%%%%%%%%%%%%%%%%%%%%%%%%%%%%%%%%%%%%%%%%%%%%%%%%%%%%
\section{Theory}\label{sec:theory}


\subsection{Renormalization Group Theory}\label{subsec:RGT}
Renormalization group theory (RGT) is an iterative procedure that was
developed to deal with large-scale fluctuations of the order parameter
near the critical point of a system. The formulation used here is
provided by Forte~\textit{et al}.\cite{Forte11}

We start by dividing the total volume into cells of volume $V_i$. We
assume that within a cell $n(\rr)$ varies only slightly, and that the
cells do not interact with one another. This is refered to as Wilson's
phase cell method.\cite{Ramana12}

Then consider a baseline free enrgy $F_0(T,n)$. We assume this
baseline accounts for all fluctuations of wavelength $\lambda <
\lambda_0$. We then double the size of our phase cell and add a
correction to the free energy $\delta F_1(T,n)$. This correction
accounts for fluctuations with a wavelength $\lambda =
2\lambda_0$. The new free energy is
\begin{align}
  F_1(T,n) &= F_0(T,n) + \delta F_1(T,n)
\end{align}
Then double the length scale again; $\delta F_2(T,n)$ accounts for
fluctionations with $\lambda = 4\lambda_0$ and
\begin{align}
  F_2(T,n) &= F_1(T,n) + \delta F_2(T,n)
\end{align}
As we continue this process, the $i^{th}$ correction accounts for
fluctuations with $\lambda = 2^i\lambda_0$, and
\begin{align}
  F_i(T,n) &= F_{i-1}(T,n) + \delta F_i(T,n)
\end{align}
In the end, your free energy is
\begin{align}
  F(T,n) = F_0(T,n) + \sum_{i=1}^\infty\delta F_i(T,n) + \Fattr
\end{align}

Where $\Fattr$ is the contribution due to the longest-wavelength
density fluctuations. At this stage, I do not entirely understand why
Forte splits up $\Fattr$.

The $\delta F_i(T,n)$ term is a difference in free energies:
\begin{align}
  \delta F_i(T,n) &= -\kT\ln Z - \left( -\kT\ln Z^* \right) \\
  &= -\kT\ln\left[ \frac{Z_D(T,n)}{Z_D^*(T,n} \right]
\end{align}
Here, $Z$ is the partition function for the entire system, and $Z^*$ is the partition function of a reference system consisting only of the term $\FbarD$. Forte gives these terms as
\begin{align}
  Z_D(T,n) &= \int dx\, \exp\left[ -\frac{V_D}{\kT}\left( \FbarD + \bar{U}_D(T,n \right) \right] \\
  Z_D^*(T,n) &= \int dx\, \exp\left[ -\frac{V_D}{\kT}\FbarD \right]
\end{align}
where $x$ is the amplitude of density fluctuations. $\FbarD$ and $\UbarD$ are formulated similar to averages:
\begin{align}
  \FbarD &= \frac{F_{i-1}(T,n+x) + F_{i-1}(T,n-x)}{2} - F_{i-1}(T,n) \\
  \UbarD &= \frac{U(T,n+x) + U(T,n-x)}{2} - U(T,n)
\end{align}
Finally, $U(T,n)$ is given by Forte as
\begin{align}
  U(T,n) = &{} g^\text{HS}(n_{eff};\sigma)\alpha - \left( \frac{4\pi^2}{2^{2i+1}L^2} \right)\alpha\omega^2 \nonumber \\
  &{} + \left( \frac{4\pi^2}{2^{2i+1}L^2} \right)^2\alpha\gamma \label{eqn:URG} % URG! It sounds like it hurt! :P
\end{align}
with
\begin{align}
  \alpha &= \frac{2}{3}\pi\epsilon\sigma^3(\lambda^3 - 1) \\
  \omega^2 &= \frac{1}{5}\sigma^2\frac{\lambda^5 - 1}{\lambda^3 - 1} \\
  \gamma &= \frac{1}{70}\sigma^4\frac{\lambda^7 - 1}{\lambda^3 - 1}
\end{align}
The pair correlation function $g(n,r)$ tells us how likely we are to find another hard sphere some distance away, given some density. In equation~\ref{eqn:URG} we are considering this quantity at contact ($r = \sigma$) with some effective density.

\subsection{Thermodynamic perturbation theory}\label{subsec:TPT}

Thermodynamic perturbation theory (TPT) gives a good description of
simple liquids, as long as you are far from the critical point---which
is most of where we calculate. TPT is a rigorous power series
expansion in a small parameter; as long as the parameter is small, the
theory is valid. This discussion follows primarily from \textit{Theory
  of Simple Liquids}, by Hansen and McDonald\cite{Hansen06}.

Thermodynamic perturbation theory treats the intermolecular potential
as separable into two parts, one for short-range repulsion and another
for long-range attraction.

The pair potential $v(\1,\2)$ describes the interaction between two
hard spheres. For the square-well model, the pair potential is
\begin{align}
  v_\text{HS}(\1,\2) &=
    \begin{cases}
      \infty & r < \sigma \\
      0 & \sigma < r
    \end{cases}
\end{align}

We consider this as our referecne, and take a square well attraction as a perturbation:
\begin{align}
  v_\lambda(\1,\2) &= v_\text{HS}(\1,\2) + \lambda w^\text{SW}(\1,\2) \label{eqn:small-perturbation}
\end{align}
where
\begin{align}
  w^\text{SW}(\1,\2) &= - \varepsilon \text{ for } \sigma < r < \lambda\sigma
\end{align}

The total potential energy of our system and its derivatives with resepect to $\lambda$---which become important later---are
\begin{align}
  \widetilde{V} &= \sum_{i=1}^N\sum_{j>i}^N v_\lambda(\mathbf{r}_i,\mathbf{r}_j) \\ % \label{eq:VN} \\
  \hookrightarrow \frac{\partial \widetilde{V}}{\partial\lambda} &= \sum_{i=1}^N\sum_{j>i}^N w^\text{SW}(\mathbf{r}_i,\mathbf{r}_j) \nonumber \\
  &\equiv W_N \\
  \hookrightarrow \frac{\partial^2 \widetilde{V}}{\partial\lambda^2} &= 0
\end{align}

The \emph{configuration integral} (the potential energy component of the partition function) and its derivative with respect to $\lambda$ are
\begin{align}
  Z_\lambda &= \int d\rr^N\, e^{-\beta \widetilde{V}(\lambda)} \\
  \hookrightarrow \frac{\partial Z_\lambda}{\partial\lambda} &=  \int d\rr^N\, \frac{\partial}{\partial\lambda}e^{-\beta \widetilde{V}(\lambda)} \\
  &= \int d\rr^N\, -\beta\frac{\partial \widetilde{V}}{\partial\lambda}e^{-\beta \widetilde{V}(\lambda)} \\
  &= \int d\rr^N\, -\beta W_N e^{-\beta \widetilde{V}(\lambda)}
\end{align}
And the non-ideal part of the free energy (i.e.~the excess free energy) and its derivative are
\begin{align}
  F &= -\kT\ln\left( \frac{Z_\lambda}{V^N} \right) \\
  \hookrightarrow \frac{\partial F}{\partial\lambda} &= \frac{V^N}{Z_\lambda}\frac{\partial}{\partial\lambda}\left( \frac{Z_\lambda}{V^N} \right) \\
  &= -\kT\left( \frac{1}{Z_\lambda}\frac{\partial Z_\lambda}{\partial\lambda} \right) \\
  &= \frac{1}{Z_\lambda} \int d\rr^N\, W_N e^{-\beta \widetilde{V}(\lambda)}\\
  &= \left\langle W_N \right\rangle_\lambda \label{eqn:dfdlambda}
\end{align}
(where $-\kT$ and $-\beta$ have divided out to $1$). Here, $V^N$ is
the total volume taken to the $N^{th}$ power, and
$\left\langle W_N \right\rangle_\lambda$ is the ensemble
average of $W_N$, with the subscript indicating some given
value of $\lambda$.

\fixme{The following could probably go someplace else (is it even necessary?):
  We can simplify this even further by noting
  \begin{align}
    \left\langle W_N \right\rangle &= \left\langle \sum_i\sum_jw(\mathbf{r}_i,\mathbf{r}_j) \right\rangle \\
    &= \frac{1}{2}N^2\left\langle w(\1,\2) \right\rangle
    \intertext{which gives us}
    \frac{\partial F}{\partial\lambda} &= \frac{1}{2}N^2\left\langle w(\1,\2) \right\rangle_\lambda
\end{align}}

Note that equation~\ref{eqn:dfdlambda} is exact; now take a series expansion:
\begin{align}
  \left\langle W_N \right\rangle_\lambda &= \left\langle W_N\right\rangle_{\lambda = 0} + (\lambda)\frac{\partial}{\partial\lambda}\left\langle W_N \right\rangle_{\lambda}\bigg|_{\lambda = 0} + \mathcal{O}(\lambda^2) \label{eqn:Wn-expansion}
\end{align}
The derivative term is
\begin{widetext}
  \begin{align}
    \frac{\partial}{\partial\lambda}\left\langle W_N \right\rangle_{\lambda} &= \frac{\partial}{\partial\lambda}\left( \frac{1}{Z_\lambda}\int d\rr^N\, W_N e^{-\beta \widetilde{V}}\right) \\
    &= \frac{1}{Z_\lambda}\frac{\partial}{\partial\lambda}\left( \int d\rr^N\, W_N e^{-\beta \widetilde{V}} \right) - \left( \int d\rr^N\, W_N e^{-\beta \widetilde{V}} \right)\frac{1}{Z_\lambda^2}\frac{\partial}{\partial\lambda}Z_\lambda \\
    &= \frac{1}{Z_\lambda}\int d\rr^N\, \left( e^{-\beta \widetilde{V}}\frac{\partial}{\partial\lambda}W_N + W_N\frac{\partial}{\partial\lambda}e^{-\beta \widetilde{V}} \right) - \left( \int d\rr^N\, W_N e^{-\beta \widetilde{V}} \right)\frac{1}{Z_\lambda^2}\int d\rr^N\, \frac{\partial}{\partial\lambda}e^{-\beta \widetilde{V}} \\
  \end{align}
\end{widetext}

Noting that $\partial W_N/\partial\lambda = 0$, we have
\begin{widetext}
  \begin{align}
    \frac{\partial}{\partial\lambda}\left\langle W_N \right\rangle_{\lambda} &= \frac{1}{Z_\lambda}\int d\rr^N (-\beta) W_N^2 e^{-\beta \widetilde{V}} - \frac{1}{Z_\lambda^2}\left(\int d\rr^N\, W_N e^{-\beta \widetilde{V}} \right)\left( \int d\rr^N\, (-\beta)W_Ne^{-\beta \widetilde{V}} \right) \\
    &= -\beta\left[ \frac{1}{Z_\lambda}\int d\rr^N\, W_N^2e^{-\beta \widetilde{V}} - \left( \frac{1}{Z_\lambda}\int d\rr^N\, W_N e^{-\beta \widetilde{V}} \right)^2 \right] \\
    &= -\beta\left[ \left\langle W_N^2 \right\rangle_\lambda - \left\langle W_N \right\rangle_\lambda^2 \right] \\
    &\equiv -\beta\sigma_\lambda^2
  \end{align}
\end{widetext}

Combining this result with equation~\ref{eqn:Wn-expansion} gives us
\begin{align}
  \left\langle W_N \right\rangle_\lambda &= \left\langle W_N\right\rangle_{\lambda = 0} - \beta\lambda\sigma_{\lambda=0}^2 + \mathcal{O}(\lambda^2) \label{eqn:Wn-expansion-simplified}
\end{align}
and we can insert this into equation~\ref{eqn:dfdlambda}:
\begin{align}
  \frac{\partial F}{\partial\lambda} &= \left\langle W_N\right\rangle_{\lambda = 0} - \beta\lambda\sigma_{\lambda=0}^2 + \mathcal{O}(\lambda^2)
\end{align}

We can integrate this expression over $F\in[F_0,F]$ and $\lambda\in[0,1]$ to obtain
\begin{align}
  F - F_0 &= (1-0)\left\langle W_N \right\rangle_\lambda - \frac{1}{2}\beta(1 - 0)^2\sigma_{\lambda=0}^2 + \mathcal{O}(\beta^2) \nonumber \\
  \hookrightarrow F &= F_0 + \left\langle W_N \right\rangle_\lambda - \frac{1}{2}\beta\sigma_{\lambda=0}^2 + \mathcal{O}(\beta^2) \label{eqn:F-inTermsW}
\end{align}
Multiply by $\beta$ and we have the \emph{high-temperature expansion}, first derived by
Zwanzig:\cite{Zwanzig54}
\begin{align}
  \beta F &= \beta F_0 + \beta \left\langle W_N \right\rangle_{\lambda=0} - \beta^2\lambda\sigma_{\lambda=0}^2 + \mathcal{O}(\beta^3)
\end{align}

Zwanzig further showed that the $n^{th}$ term could be written in terms of the mean fluctuations $\left\langle \left[ W_N - \left\langle W_N\right\rangle_{\lambda=0} \right]^\nu \right\rangle_{\lambda=0}$ with $\nu \leq n$. This requires us to know the form of higher-order correlation functions of the reference system, which can be quite complicated. In the next section we will see a simple aproximation for the $n=2$ term.

%% Writing the statistical averages in terms of the pair density
%% $n_\lambda^{(2)}(\1,\2)$ (i.e. the probability per volume squared of having one sphere at \1
%%   and another at \2) gives us
%% \begin{align}
%%   \beta F &=\beta F_0 + \frac{1}{2}\beta\int_0^1 d\lambda\, \iint d\1 d\2\, n_\lambda^{(2)}(\1,\2)w(\1,\2) \label{eqn:Henderson5-2-12}
%% \end{align}
%% \davidsays{I think this discussion is pretty redundant with
%%   parts of the previous derivation in terms of $\widetilde{V}$.}  The pair
%% density itself can be expanded as
%% \begin{align}
%%   n_\lambda^{(2)}(\1,\2) &= n_0^{(2)}(\1,\2) + \lambda\frac{\partial n_\lambda^{(2)}(\1,\2)}{\partial\lambda}\bigg|_{\lambda = 0} + \mathcal{O}(\lambda^2)
%% \end{align}
%% Applying this to the 2nd term in equation~\ref{eqn:Henderson5-2-12}, we
%% have
%% \begin{align}
%%   \frac{\beta F_1}{N} &= \frac{\beta}{2N}\iint d\1 d\2\, n_0^{(2)}(\1,\2)w(\1,\2) \\
%%   &= \frac{\beta n}{2}\int d\mathbf{r}_{12}\, g_0(\1,\2)w(\1,\2)
%% \end{align}
%% The second-order $\lambda$ term is quite a bit more complicated:
%% \begin{widetext}
%%  \begin{align}
%%    \begin{split}
%%    \frac{\beta F_2}{N} =&{} -\frac{\beta^2}{2}\left[ \frac{n}{2}\int d\2\, g_0(\1,\2)w^2(\1,\2) \right. \\ &{} + n^2 \iint d\2 d\3\, g_0^{(3)}(\1,\2,\3)w(\1,\2)w(\1,\3) \\ &{} \left. + \frac{n^3}{4}\iiint d\2 d\3 d\4\, \left( g_0^{(4)}(\1,\2,\3,\4) - g_0^{(2)}(\1,\2)g_0^{(2)}(\3,\4) \right)w(\1,\2)2(\3,\4) \right] \\ &{} - \frac{1}{4}S_0(0)\left[ \frac{\partial}{\partial n}\left( n^2\int d\2\, g_0(\1,\2)w(\1,\2) \right)^2 \right]
%%    \end{split}
%%  \end{align}
%% \end{widetext}
%% This term requires use to know the form of higher-order correlation functions and the structure function $S_0(k)$ of the reference system. These things are very difficult. Hence, we use approximations, as outlined in section~\ref{sub2sec:disp}.

\subsection{Square well liquid free enrgy}\label{subsec:SW}
For the baseline, this work is based on the square well liquid free energy:\cite{Hughes13}
\begin{align}
  f_0(T,n) &= \fid + \fhs + \left( \fdisp - n a_1(n) \right) \ .
\end{align}
The long-range attractive term, $\fattr$ is given by
\begin{align}
  \fattr &= n a_1(n) \ .
\end{align}
$\fid$ is the ideal gas free energy, $\fhs$ is the hard sphere
repulsion, and $\fdisp$ is the attraction. The function $a_1(n)$ is
the first term from a high-temperature perturbation expansion,
described in further detail in sections~\ref{subsec:TPT}
and~\ref{sub2sec:disp}. Each function is outlined below.

\subsubsection{Ideal gas}\label{sub2sec:ID}
The ideal gas term is given as
\begin{align}
  \fid &= n\kT\left(\log(n) - 1\right)
\end{align}

\subsubsection{Hard sphere repulsion}\label{sub2sec:HS}
Hard sphere repulsive forces are dealt with by the White Bear version
of Fundamental-Mearuse Theory.\cite{Roth02} The hard sphere excess
free energy is given in the homogeneous case as
\begin{align}
  \fhs &= \kT\left( \Phi_1(n) + \Phi_2(n) + \Phi_3(n) \right)
\end{align}
The $\Phi_j(n)$ terms are given as
\begin{align}
  \Phi_1(n) &= -n_0(n)\ln(1 - n_3(n)) \\
  \Phi_2(n) &= \frac{n_1(n)n_2(n)}{1 - n_3(n)} \\
  \Phi_3(n) &= n_2^3(n)\left( \frac{n_3(n) + (1 - n_3(n))^2\ln(1 - n_3(n))}{36\pi n_3^2(n)(1 - n_3(n))^2} \right)
\end{align}
The \emph{fundamental measure densities} are
\begin{align}
  n_0(n) &= \frac{n_2}{4\pi R^2} \nonumber \\
  &= n \\
  n_1(n) &= \frac{n_2}{4\pi R} \nonumber \\
  &= nR \\
  n_2(n) &= n4\pi R^2 \\
  n_3(n) &= n\frac{4}{3}\pi R^3 \nonumber \\
  &= \eta
\end{align}
where $\eta$ is the packing fraction.

\subsubsection{Attraction}\label{sub2sec:disp}
The attractive free energy includes van der Waals attraction between
hard spheres and is founded in Thermodynamic Perturbation Theory
(TPT).
\begin{align}
  \fdisp &= n \left( a_1(n) + \frac{1}{\kT}a_2(n) \right)
\end{align}
The $a_j(n)$ terms come from TPT, outlined in section~\ref{subsec:TPT}.

The first term, $a_1$, is the mean-field dispersion interaction. The
second term, $a_2$, describes the effect of fluctuations resulting
from compression of the fluid due to the dispersion interaction
itself, and is approximated using the local compressibility
approximation (LCA), which assumes the energy fluctuation is simply
related to the compressibility of a hard-sphere reference
fluid.\cite{Barker76}

In Gil-Villegas' paper,\cite{gil-villegas97} HS and Dispersion are wrapped up into the
\textit{monomer} contribution, expressed in terms of energy densities
as:
\begin{align}
  f_\text{mono} = f_\text{HS} - \frac{\alpha^\text{VDW}n}{kT}
\end{align}
with $\alpha^\text{VDW}$ given by
\begin{align}
  \alpha^\text{VDW} &= 2\pi\epsilon\int_\sigma^\infty r^2\phi(r)\,dr \\
  \intertext{using reduced units of $x = r/\sigma$, we have}
  &= 2\pi\sigma^3\int_1^\infty x^2\phi(x)\,dx \\
  &= 3b^\text{VDW}\epsilon\int_1^\infty x^2\phi(x)\,dx
\end{align}
where $b^\text{VDW}$ is the van der Waals size parameter. It
corresponds to the volume excluded by two spherical particles of
volume $b$: $b^\text{VDW} = 4b = 4\left(\pi\sigma^3/6\right)$.

Then, the high-temp expansion is an expansion of the monomer term:
\begin{align}
  f_\text{mono} &= f_\text{HS} + \beta a_1 + \beta^2 a_2 + \cdots
\end{align}

The term $a_1$ is given by
\begin{align}
  a_1 &= -2\pi n \epsilon\int_\sigma^\infty r^2\phi(r)g^\text{HS}(r)\,dr \\
  &= -3 n  b^\text{VDW}\epsilon\int_1^\infty x^2\phi(x)g^\text{HS}(x)\,dx
\end{align}
If we assume random correlations between the particles' positions, for
all distances, we have $g^\text{HS}(r) = 1$. This yields
\begin{align}
  a_1^\text{VDW} = - n \alpha^\text{VDW},
\end{align}
the van der Waals mean-field energy.

The second-order term is quite a bit more complicated; one must have
knowledge of higher-order correlation functions to determine
$a_2$. This term describes the response of the attractive energy due
to the compression of the fluid from the attractive well. Gil-Villegas
\emph{et al.} base their expression on an approximation by Barker and
Henderson.\cite{Barker67} Barker and Henderson's approximation
considers fluctuations in the number of particles in the potential
well. In this way, the fluctuations in $a_2$ are related to the
pressure and compressibility of the liquid. Given pressure
$P^\text{HS}$ and isothermal compressibility $K^\text{HS} =
kT\left(\partial n /\partial P^\text{HS}\right)_T$, we have:
\begin{align}
  a_2 &= -\pi n \epsilon^2kT\int_0^\infty r^2\left[\phi(r)\right]^2\frac{\partial n  g^\text{HS}}{\partial P^\text{HS}}(r)\,dr \\
  &= -\pi n \epsilon^2K^\text{HS}\frac{\partial}{\partial n }\left[\int_\sigma^\infty r^2\left[\phi(r)\right]^2 n  g^\text{HS}\,dr\right] \\
  &= \frac{1}{2}\epsilon K^\text{HS}\frac{\partial}{\partial n }\left[-3 n  b^\text{VDW}\epsilon\int_1^\infty x^2\left[\phi(x)\right]^2 g^\text{HS}(x)\,dx \right]
\end{align}

We use the isothermal compressibility in terms of packing fraction
$\eta$, given by the Percus-Yevick exprssion:\cite{Barker76}
\begin{align}
  K^\text{HS} &= \frac{\left(1 - \eta\right)^4}{1 + 4\eta + 4\eta^2}
\end{align}
This is called the Local Compressibility Approximation (LCA).
%%%%%%%%%%%%%%%%%%%%%%%%%%%%%%%%%%%%%%%%%%%%%%%%%%%%%%%%%%%%
\section{Methods}\label{sec:methods}

This work applies entirely to the homogeneous case---that is, there
are no walls containing our liquid, nor are there any sort of solutes
in the liquid. Since our fluid is of infinite extent, the free energy
is infinite. Therefore, in the calculations, I work with free energy density:
\begin{align}
  f(T,n) &\equiv \frac{F}{V}(T,n)
\end{align}

\subsection{Coexistence Curve Algorithm}\label{subsec:coexis}
Coexistence curves are found by modifying the chemical potential $\mu$
in the grand free energy per volume $\phi(T,n)/V$. Grand free energy
density is defined as
\begin{align}
  \frac{\Phi}{V}(T,n) &= \phi(T,n) \nonumber \\
                 &= f(T,n) - n\mu \nonumber \\
                 &= f(T,n) - n\frac{df}{dn}(T,n)\bigg|_{\npart}\ .
\end{align}
It is the value $\npart$ that we adjust to find coexistence curves.

\begin{figure}
  \centering
%%  \includegraphics[width=\columnwidth]{figs/SW-phi-lowT}
  \caption{Grand free energy per unit volume, SW, low temp}
  \label{fig:SW-phi-lowT}
\end{figure}

The grand free energy density generally has two distinct minima when
plotted versus density. (See fig.~\ref{fig:SW-phi-lowT} for an
example.) The liquid will be in liquid-vapor equilibrium when those
minima have the same value for $\phi(T,n)$. There is a local maximum
between those minima, and the value of the density at that maximum is
used for $\npart$. Find this value at a given temperature then plot
$\phi(T,n)$ vs $n$ at slightly higher temperature. Find the new value
for $\npart$ such that the two minima have the same value for
$\phi(T,n)$. (It is helpful to note that if $\npart$ is increased,
$\phi(T,\nliq)$ increases, and $\phi(T,\nliq)$ decreases if $\npart$
is decresed.) The program I wrote uses the previous temperature's
$\npart$ as a ``first guess,'' maximizing $\phi(T,\npart)$. From
there, the program adjusts $\npart$ until $\phi(T,\nvap) =
\phi(T,\nliq)$. When that condition is met, it uses this new value of
$\npart$ as the ``first guess'' at $\npart$ for the next higher
temperature.

The algorithm is as follows; start at some temperature $T=T_{start}$,
with an intial guess for $\npart$.
\begin{enumerate}
  \item Calculate $\nvap$ and $\nliq$ by calculating the minima in $\phi(T_{start},n)$ over appropriate regions
  \item Calculate $\phi(T_{start},\nvap)$ and $\phi(T_{start},\nliq)$ \label{while-start}
  \item Calculate \[\delta\mu = \frac{\phi(T_{start},\nvap) - \phi(T_{start},\nliq)}{\nliq - \nvap}\]
  \item Calculate $\phi^*(T_{start},n) = \phi(T_{start},n) + \delta\mu$
  \item Calculate a new $\npart$, $\npart^*$, based on $\phi^*(T_{start},n)$ by finding the local maximum between $\nvap$ and $\nliq$
  \item Calculate $\phi(T_{start},\npart^*)$
  \item Re-caluclate $\nvap$, $\nliq$, $\phi(T_{start},\nvap)$, and $\phi(T_{start},\nliq)$, using $\npart^*$
  \item Go back to step~\ref{while-start} and repeat until \[ \frac{\phi(T_{start},\nvap) - \phi(T_{start},\nliq)}{\phi(T_{start},\npart^*)} > tol  \] where $tol$ is some computational tolerance (I used $tol=10^{-5}$)
  \item Increase the temperature to $T = T + \delta T$ and repeat
\end{enumerate}

\subsection{Free Energies}\label{subsec:free-energies}
The $a_j(n)$ terms require even more simplification when
``converting'' from the mathematical formulation to Python code, to
rid ourselves of troublesome integrals. Gil-Villegas gives us
\newcommand\eff{\textit{eff}}
\begin{align}
  a_1^\text{SW}(n) &= a_1^\text{VDW}(n)g^\text{HS}(1;\eta_{\eff}) \\
  &= -4\eta\epsilon(\lambda^3 - 1)\frac{1 - \left( \eta_{\eff}/2 \right)}{(1 - \eta_{\eff})^3}
\end{align}
and
\begin{align}
  \eta_{\eff} &= c_1\eta + c_2\eta^2 + c_3\eta^3
\end{align}
with
\begin{align}
  \left( \begin{array}{c}
    c_1 \\
    c_2 \\
    c_3
    \end{array} \right)
  &= \left( \begin{array}{ccc}
    2.25855 & -1.50349 & 0.249434 \\
    -0.669270 & 1.40049 & -0.827739 \\
    10.1576 & -15.0427 & 5.30827
    \end{array} \right)
  \left( \begin{array}{c}
    1 \\
    \lambda \\
    \lambda^2
    \end{array} \right)
\end{align}

\subsection{RG partition functions}\label{subsec:fbar-ubar}
I use the midpoint method for evaluating the integrals of $\FbarD$
and $\UbarD$.

%%%%%%%%%%%%%%%%%%%%%%%%%%%%%%%%%%%%%%%%%%%%%%%%%%%%%%%%%%%%
\section{Results and discussion}\label{sec:results}

For this work, I use $\lambda_\text{SW} = 1.5$.

\begin{figure}
  \begin{center}
%%  \includegraphics[width=\columnwidth]{figs/coexistance_SW}
  \end{center}
  \caption{Liquid-vapor coexistance for a square well liquid.}
  \label{fig:coexistance_SW}
\end{figure}

\begin{figure}
  \begin{center}
%%  \includegraphics[width=\columnwidth]{figs/phi-RG-lowT}
  \end{center}
  \caption{Free energy density for RG at low temperature}
  \label{fig:phi-RG-lowT}
\end{figure}

\begin{figure}
  \begin{center}
%%  \includegraphics[width=\columnwidth]{figs/phi-RG-highT}
  \end{center}
  \caption{Free energy density for RG near critical temp}
  \label{fig:phi-RG-highT}
\end{figure}

Figures~\ref{fig:phi-RG-lowT} and~\ref{fig:phi-RG-highT} show the
grand free energy per unit volume at a low temperature, and near the
critical point. It is clear that the program is not de-bugged.


%%%%%%%%%%%%%%%%%%%%%%%%%%%%%%%%%%%%%%%%%%%%%%%%%%%%%%%%%%%%
\section{Conclusion}\label{sec:conclusion}

I have implemented a the use of renormalization group theory in
calculations for the square well liquid. The program is not entirely
de-bugged. I have also developed an algorithm that calculates
liqiud-vapor coexistence curves, even close to the critical point.

I conclude that it would be worthwhile to further investigate implementing renormalization group theory in the Roundy DFT.


\bibliographystyle{unsrt}
\bibliography{project} % Produces the bibliography via BibTeX.

\end{document}

