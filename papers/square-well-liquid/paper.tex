\documentclass[letterpaper,twocolumn,amsmath,amssymb,pre,aps,10pt]{revtex4-1}
\usepackage{graphicx}% Include figure files
\usepackage{color}

\newcommand\xx{\mathbf{x}}
\newcommand\rr{\mathbf{x}}
\newcommand\fixme[1]{\textcolor{red}{\textbf{[#1]}}}

\begin{document}
\title{Classical density functional theory for the square-well liquid}

\author{Student One}
\author{Michael MI? Perlin}
\author{David Roundy}
\affiliation{Department of Physics, Oregon State University, Corvallis, OR 97331}

\begin{abstract}
  We have developed a functional.
\end{abstract}

\maketitle

The SAFT Helmholtz free energy is composed of five terms:
\begin{equation} \label{eq:SAFT-free-energy}
  F = F_\textit{id} + F_\textit{hs} + F_\textit{disp} +
  F_\textit{assoc} + F_\textit{chain},
\end{equation}
where the first three terms---ideal gas, hard-sphere repulsion and
dispersion---encompass the \emph{monomer} contribution
to the free energy, the fourth is the \emph{association} free energy,
describing hydrogen bonds, and the final term is the chain formation
energy for fluids that are chain polymers.  While a
number of formulations of SAFT have been published, we will focus on
SAFT-VR\cite{gil-villegas-1997-SAFT-VR}.

\newcommand\etadisp{\ensuremath{\eta_\textit{d}}}
\newcommand\epsilondisp{\ensuremath{\epsilon_\textit{d}}}
\newcommand\lambdadisp{\ensuremath{\lambda_\textit{d}}}
\newcommand\lscale{\ensuremath{s_d}}


\section{Theory}

\subsection{Thermodynamic perturbation theory}

\fixme{Add here an introduction to thermodyamic perturbation theory,
  explaining $a_1$ and $a_2$ and their ``exact'' formulas that we wish
  to approximate.}



\subsection{Square well contribution in homogeneous case}
The square well contribution to the free energy is based on
thermodynamic perturbation theory (sometimes known as the ``high
temperature expansion'').  We use a dispersion term based on the
SAFT-VR approach\cite{gil-villegas-1997-SAFT-VR}, which has two
parameters an interaction energy $\epsilondisp$ and a length scale
$\lambdadisp R$.

The SAFT-VR dispersion free energy has the form~\cite{gil-villegas-1997-SAFT-VR}
\begin{align}
  F_\text{disp}[n] &= \int \left(a_1(\xx) + \beta a_2(\xx)\right)n(\xx)d\xx,
\end{align}
where $a_1$ and $a_2$ are the first two terms in a high-temperature
perturbation expansion and $\beta=1/k_BT$.  The first term, $a_1$, is 
the mean-field dispersion interaction. The second term, $a_2$, describes the
effect of fluctuations resulting from compression of the fluid due
to the dispersion interaction itself, and is approximated
using the local compressibility approximation (LCA), which
assumes the energy fluctuation is simply related to the
compressibility of a hard-sphere reference fluid\cite{barker1976liquid}.

\fixme{Add here a complete exposition of this functional form,
  reproducing what Gil Villegas says.}

\subsection{A simple weighted density approach}

Hughes \emph{et al.} used a simple weighted density approach to
approximate the square-well contribution to the free energy, which we
will describe here~\cite{hughes2013classical}.

The form of $a_1$ and $a_2$ for SAFT-VR is given in
reference~\cite{gil-villegas-1997-SAFT-VR}, expressed in terms
of the packing fraction.  In order to apply this form to an
\emph{inhomogeneous} density distribution, we construct an effective local
packing fraction for dispersion $\etadisp$, given by a Gaussian
convolution of the density:
\begin{align}
%    ((4/3.0*M_PI)*Pow(3)(R)*GaussianConvolve(2*lambdainput*lscale*radius,
%                                             2*lambdaE*length_scalingE*Rexpr)
%  \etadisp(\xx) &= \color{red}
%  \int \frac{\Theta(2\lambdadisp\lscale R - \left|\xx-\xx'\right|)}{
%             \left(2\lambdadisp\lscale\right)^3} n(\xx') d\xx'.
  \etadisp(\xx) &= \frac{1}{6\sqrt{\pi} \lambdadisp^3\lscale^3}
  \int n(\xx')\exp\left({-\frac{|\xx-\xx'|^2}{2(2 \lambdadisp
      \lscale R)^2}}\right)d\xx'.
\end{align}
This effective packing fraction is used throughout the dispersion
functional, and represents a packing fraction averaged over the
effective range of the dispersive interaction.  Here we have
introduced an additional empirical parameter $\lscale$ which modifies
the length scale over which the dispersion interaction is correlated.

\subsection{Others}

\fixme{Check out what other groups have done, especially Jackson
  and/or Gross.  Also Arias.}

\subsection{Weighted density approach}

\cite{peng2008meanfield} used by
\cite{sundararaman2013efficient} and many others.

See also \cite{yu2009novel} which explicitly addresses the square well
liquid.

Also look up \cite{shen2013hybrid} which might follow Yu above.

\subsection{Using the pair distribution function}

\fixme{We have introduced an approximation to the pair distribution
  function...}

\section{Simulation}

\fixme{A short explanation of MC methods here?}

\begin{figure}
  \includegraphics[width=\columnwidth]{figs/periodic-ww13-ff30-N200-E.pdf}
  \caption{A simple DOS plot!}
\end{figure}

\begin{figure}
  \includegraphics[width=\columnwidth]{figs/periodic-ww30-ff30-N200-E.pdf}
  \caption{A simple DOS plot!}
\end{figure}

\section{Results}

\subsection{Homogeneous fluid}

\fixme{Here we present the comparison between various theories and in
  the homogeneous case.  This will mostly be our pair distribution
  function vs. Gil Villegas.}

\subsection{Hard wall}

\fixme{Here we present the comparison between various theories and
  simulation for a fluid at a hard wall.}

\subsection{Test particle}

\fixme{Here we present the comparison between various theories and
  simulation for a test-particle simulation (i.e. the radial
  distribution function).}

\section{Conclusion}

\fixme{We have an awesome theory.}

\bibliography{paper}% Produces the bibliography via BibTeX.

\end{document}
