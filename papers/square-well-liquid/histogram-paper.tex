\documentclass[letterpaper,twocolumn,amsmath,amssymb,pre,aps,10pt]{revtex4-1}
\usepackage{graphicx}% Include figure files
\usepackage{color}

\begin{document}
\title{Applying a clever histogram method to the square-well liquid}

\author{Michael MI? Perlin}
\author{David Roundy}
\affiliation{Department of Physics, Oregon State University, Corvallis, OR 97331}

\begin{abstract}
  We have applied the clever histogram method to the square-well liquid.
\end{abstract}

\maketitle

\begin{figure}
  \includegraphics[width=\columnwidth]{figs/periodic-ww13-ff30-N200-E.pdf}\
  \includegraphics[width=\columnwidth]{figs/periodic-ww30-ff30-N200-E.pdf}
  \caption{A histogram plot demonstrating the difficulty of using
    canonical Monte Carlo to simulate a system.  A simulation at a
    give temperature only provides information fo a small number of
    energy states.\label{fig:histograms}}
\end{figure}

\begin{figure}
  \includegraphics[width=\columnwidth]{figs/periodic-ww13-ff30-N200-dos.pdf}\
  \includegraphics[width=\columnwidth]{figs/periodic-ww30-ff30-N200-dos.pdf}
  \caption{A density of states plot demonstrating the difficulty of using
    canonical Monte Carlo to simulate a system.  A simulation at a
    give temperature only provides information fo a small number of
    energy states.\label{fig:histograms}}
\end{figure}

As shown in Fig.~\ref{fig:histograms}, there are several difficulties
encountered when running Monte Carlo simulations at a fixed
temperature.  On the top plot, which shows three simulations run at
$\lambda = 1.3$, it is apparent that each given canonical simulation
only provides statistical information for a small number of energy
states.  The bottom plot shows an even more serious issue:  when
running a simulation at a fixed temperature, it is possible to become
frozen in a state that is far from the ground state, as happened in
the $kT=0.1\epsilon$ simulation.

\end{document}
