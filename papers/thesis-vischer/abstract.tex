\thispagestyle{plain}
\begin{center}
    \Large
    \textbf{Free Energy Decomposition of the Critical Square-Well Liquid Using a Renormalization Group Method}
   
    \vspace{0.8cm}
    \large
    A Computational Project
    
    \vspace{0.8cm}
    \textbf{Brenden Vischer}
    
    \vspace{1.2cm}
    \textbf{Abstract}
\end{center}

Analytic models using renormalization methods to approximate the free energy of fluids have been developed, but there is currently no way to investigate the intermediate steps of the approximation. We present a computational method to determine the free energy contributions of each length scale by exploiting both the behavior of the critical fluid and the periodicity of Metropolis-Hastings Monte-Carlo simulations. We further detail a method by which to compute the ``absolute'' free energy, including both the ideal gas free energy and the excess free energy. We further present an application of this method to the well-research square-well liquid near the critical region. We find that the square-well liquid is well-characterized by a base cell length of $\sqrt2\sigma$, the side length of an FCC lattice, by establishing liquid-vapor coexistence between scaled temperatures $0.5 \leq T < .75$.   
\clearpage