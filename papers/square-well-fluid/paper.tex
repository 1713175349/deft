\documentclass[letterpaper,twocolumn,amsmath,amssymb,pre,aps,10pt]{revtex4-1}
\usepackage{graphicx}% Include figure files
\usepackage{color}

\newcommand\kk{\mathbf{k}}
\newcommand\xx{\mathbf{x}}
\newcommand\rr{\mathbf{r}}
\newcommand\dd{\mathrm{d}}
\newcommand\fixme[1]{\textcolor{red}{\textbf{[#1]}}}

\begin{document}
\title{Classical density functional theory for the square-well liquid}

\author{Student One}
\author{Michael A. Perlin}
\author{David Roundy}
\affiliation{Department of Physics, Oregon State University, Corvallis, OR 97331}

\begin{abstract}
  We have developed a functional.
\end{abstract}

\maketitle

The SAFT Helmholtz free energy is composed of five terms:
\begin{equation} \label{eq:SAFT-free-energy}
  F = F_\textit{id} + F_\textit{hs} + F_\textit{disp} +
  F_\textit{assoc} + F_\textit{chain},
\end{equation}
where the first three terms---ideal gas, hard-sphere repulsion and
dispersion---encompass the \emph{monomer} contribution
to the free energy, the fourth is the \emph{association} free energy,
describing hydrogen bonds, and the final term is the chain formation
energy for fluids that are chain polymers.  While a
number of formulations of SAFT have been published, we will focus on
SAFT-VR\cite{gil-villegas-1997-SAFT-VR}.

\newcommand\etadisp{\ensuremath{\eta_\textit{d}}}
\newcommand\epsilondisp{\ensuremath{\epsilon_\textit{d}}}
\newcommand\lambdadisp{\ensuremath{\lambda_\textit{d}}}
\newcommand\lscale{\ensuremath{s_d}}


\section{Theory}

\subsection{Thermodynamic perturbation theory}

\fixme{Add here an introduction to thermodyamic perturbation theory,
  explaining $a_1$ and $a_2$ and their ``exact'' formulas that we wish
  to approximate.}

\fixme{And introduction to TPT---why do we use it?}

\fixme{For now, the discussion follows entirely from the Simple Liquids book~\cite{simple-liquids}}


The pair potential $V(\rr)$ describes the interaction between two hard spheres. For the square-well model,
\begin{align}
  v_\text{SW}(\rr) &=
    \begin{cases}
      \infty & r < \sigma \\
      -\epsilon & \sigma < r < \lambda\sigma \\
      0 & \lambda\sigma < r
    \end{cases}
\end{align}

Now, consider this to be the reference, and add a small perturbation:
\begin{align}
  v_\lambda(\rr) &= v_\text{SW}(\rr) + \lambda w(\rr)
\end{align}

The total potential energy of our system is
\begin{align}
  V_N(\lambda) = \sum_\rr v_\lambda(\rr)
\end{align}



\subsection{Square well contribution in homogeneous case}
The square well contribution to the free energy is based on
thermodynamic perturbation theory (sometimes known as the ``high
temperature expansion'').  We use a dispersion term based on the
SAFT-VR approach\cite{gil-villegas-1997-SAFT-VR}, which has two
parameters an interaction energy $\epsilondisp$ and a length scale
$\lambdadisp R$.

The SAFT-VR dispersion free energy has the form~\cite{gil-villegas-1997-SAFT-VR}
\begin{align}
  F_\text{disp}[n] &= \int \left(a_1(\xx) + \beta a_2(\xx)\right)n(\xx)d\xx,
\end{align}
where $a_1$ and $a_2$ are the first two terms in a high-temperature
perturbation expansion and $\beta=1/k_BT$.  The first term, $a_1$, is
the mean-field dispersion interaction. The second term, $a_2$, describes the
effect of fluctuations resulting from compression of the fluid due
to the dispersion interaction itself, and is approximated
using the local compressibility approximation (LCA), which
assumes the energy fluctuation is simply related to the
compressibility of a hard-sphere reference fluid\cite{barker1976liquid}.

\fixme{The following hopes to be a complete exposition of this functional form,
  reproducing what Gil Villegas says.}

In Gil-Villegas' paper, HS and Dispersion are wrapped up into the
\textit{monomer} contribution, expressed in terms of energy densities
as:
\begin{align}
  f_\text{mono} = f_\text{HS} - \frac{\alpha^\text{VDW}n}{kT}
\end{align}
with $\alpha^\text{VDW}$ given by
\begin{align}
  \alpha^\text{VDW} &= 2\pi\epsilon\int_\sigma^\infty r^2\phi(r)\,dr \\
  \intertext{using reduced units of $x = r/\sigma$, we have}
  &= 2\pi\sigma^3\int_1^\infty x^2\phi(x)\,dx \\
  &= 3b^\text{VDW}\epsilon\int_1^\infty x^2\phi(x)\,dx
\end{align}
where $b^\text{VDW}$ is the van der Waals size parameter. It corresponds to the volume excluded by two spherical particles of volume $b$: $b^\text{VDW} = 4b = 4\left(\pi\sigma^3/6\right)$.

Then, the high-temp expansion is an expansion of the monomer term:
\begin{align}
  f_\text{mono} &= f_\text{HS} + \beta a_1 + \beta^2 a_2 + \cdots
\end{align}

The term $a_1$ is given by
\begin{align}
  a_1 &= -2\pi n \epsilon\int_\sigma^\infty r^2\phi(r)g^\text{HS}(r)\,dr \\
  &= -3 n  b^\text{VDW}\epsilon\int_1^\infty x^2\phi(x)g^\text{HS}(x)\,dx
\end{align}
If we assume random correlations between the particles' positions, for all distances, we have $g^\text{HS}(r) = 1$. This yields
\begin{align}
  a_1^\text{VDW} = - n \alpha^\text{VDW},
\end{align}
the van der Waals mean-field energy.

The second-order term is quite a bit more complicated; one must have
knowledge of higher-order correlation functions to determine
$a_2$. This term describes the response of the attractive energy due
to the compression of the fluid from the attractive well. Gil-Villegas
\emph{et al.} base their expression on an approximation by Barker and
Henderson\cite{barker1967-SW-perturbation}. Barker and Henderson's
approximation considers fluctuations in the number of particles in the
potential well. In this way, the fluctuations in $a_2$ are related to
the pressure and compressibility of the liquid. Given pressure
$P^\text{HS}$ and isothermal compressibility $K^\text{HS} =
kT\left(\partial n  /\partial P^\text{HS}\right)_T$, we have:
\begin{align}
  a_2 &= -\pi n \epsilon^2kT\int_0^\infty r^2\left[\phi(r)\right]^2\frac{\partial n  g^\text{HS}(r)}{\partial P^\text{HS}}\,dr \\
  &= -\pi n \epsilon^2K^\text{HS}\frac{\partial}{\partial n }\left[\int_\sigma^\infty r^2\left[\phi(r)\right]^2 n  g^\text{HS}\,dr\right] \\
  &= \frac{1}{2}\epsilon K^\text{HS}\frac{\partial}{\partial n }\left[-3 n  b^\text{VDW}\epsilon\int_1^\infty x^2\left[\phi(x)\right]^2 g^\text{HS}(x)\,dx \right]
\end{align}

We use the isothermal compressibility in terms of packing fraction $\eta$, given by the Percus-Yevick exprssion\cite{barker1976liquid}:
\begin{align}
  K^\text{HS} &= \frac{\left(1 - \eta\right)^4}{1 + 4\eta + 4\eta^2}
\end{align}

\subsection{A simple weighted density approach}

Hughes \emph{et al.} used a simple weighted density approach to
approximate the square-well contribution to the free energy, which we
will describe here~\cite{hughes2013classical}.

The form of $a_1$ and $a_2$ for SAFT-VR is given in
reference~\cite{gil-villegas-1997-SAFT-VR}, expressed in terms
of the packing fraction.  In order to apply this form to an
\emph{inhomogeneous} density distribution, we construct an effective local
packing fraction for dispersion $\etadisp$, given by a Gaussian
convolution of the density:
\begin{align}
%    ((4/3.0*M_PI)*Pow(3)(R)*GaussianConvolve(2*lambdainput*lscale*radius,
%                                             2*lambdaE*length_scalingE*Rexpr)
%  \etadisp(\xx) &= \color{red}
%  \int \frac{\Theta(2\lambdadisp\lscale R - \left|\xx-\xx'\right|)}{
%             \left(2\lambdadisp\lscale\right)^3} n(\xx') d\xx'.
  \etadisp(\xx) &= \frac{1}{6\sqrt{\pi} \lambdadisp^3\lscale^3}
  \int n(\xx')\exp\left({-\frac{|\xx-\xx'|^2}{2(2 \lambdadisp
      \lscale R)^2}}\right)d\xx'.
\end{align}
This effective packing fraction is used throughout the dispersion
functional, and represents a packing fraction averaged over the
effective range of the dispersive interaction.  Here we have
introduced an additional empirical parameter $\lscale$ which modifies
the length scale over which the dispersion interaction is correlated.

\subsection{Others}

\fixme{Check out what other groups have done, especially Jackson
  and/or Gross.  Also Arias.}

\subsection{Weighted density approach}

\cite{peng2008meanfield} used by
\cite{sundararaman2013efficient} and many others.

See also \cite{yu2009novel} which explicitly addresses the square well
liquid.

Also look up \cite{shen2013hybrid} which might follow Yu above.

\subsection{The contact value approximation for the hard-sphere pair distribution function}

We have recently introduced an efficient approximation for the pair
distribution function of the inhomogeneous hard-sphere fluid.  This
approximation takes the form:
\begin{align}
  g^{(2)}(\rr_1,\rr_2) = \frac{g_S(r_{12}, g_\sigma(\rr_1)) +
    g_S(r_{12}, g_\sigma(\rr_2))}{2} \label{eq:g2-our-mean}
\end{align}
where $g_S(r,g_\sigma)$ is a separable fit to the radial distribution
function of the homogeneus hard-sphere fluid, and $g_\sigma(\rr)$ is
an approximation for the pair distribution function of an
inhomogeneous hard-sphere fluid averaged over contact with a sphere
located at position $\rr$.

\subsubsection{Separable radial distribution function
  $g_S(r,g_\sigma)$}

The separable radial distribution function $g_S(r,g_\sigma)$ is given
by
\newcommand\kappaO{\kappa_0}
\newcommand\kappaI{\kappa_1}
\newcommand\kappaZ{\kappa_2}
\newcommand\kappazero{3.68}
\newcommand\kappaone{2.16}
\newcommand\kappatwo{2.79}

\begin{multline}\label{eq:gs}
  g_S(r,g_\sigma) = 1 + (g_\sigma-1) e^{-\kappaO \left(\frac{r}{\sigma}-1\right)} \\
  + (g_\sigma -1)(\kappaO - 2g_\sigma)  \left(\frac{r}{\sigma}-1\right)e^{-\kappaI  \left(\frac{r}{\sigma}-1\right)} \\
  + I(g_\sigma)  \left(\frac{r}{\sigma}-1\right)^2e^{-\kappaZ  \left(\frac{r}{\sigma}-1\right)}
\end{multline}
\begin{equation}
  I(g_\sigma) = \kappaZ^5 \frac{
    \frac{\chi-1}{24\eta} + (1-g_\sigma) \frac{(\kappaO-2
      g_\sigma)(\kappaI^2 + 4 \kappaI + 6)}{\kappaI^4}
    + (1-g_\sigma)\frac{\kappaO^2 + 2\kappaO + 2}{\kappaO}
  }{2 \kappaZ^2 + 12 \kappaZ + 24}
\end{equation}
\begin{equation}
  \chi = \frac{(1-\eta)^4}{\eta^4 - 4\eta^3 + 4\eta^2 + 4\eta + 1}
\end{equation}
Finally, we can convert between the packing fraction $\eta$ and the
radial distribution function at contact $g_\sigma$ using the
Carnahan-Starling approximation for $g_\sigma$:
\newcommand\nastyetacuberoot{\sqrt[3]{54 g_\sigma^2 -
    6\sqrt{81g_\sigma^4 - 6g_\sigma^3}}}
\begin{align}
  g_\sigma &= \frac{1-\tfrac{\eta}{2}}{(1-\eta)^3}\text{, and its inverse} \\
  \eta &= 1 - \frac{1}{\nastyetacuberoot} - \frac{\nastyetacuberoot}{6g_\sigma}.
\end{align}
This function includes three fitted parameters, $\kappaO =
\kappazero$, $\kappaI = \kappaone$,
and $\kappaZ = \kappatwo$.

\subsubsection{Fourier transform of radial distribution function}

\newcommand\kapovsig[1]{\frac{\kappa_{#1}}{\sigma}}

For or purposes, we want to Fourier transform $g_s(r,g_\sigma)u_{SW}(\rr)$ and
the square well potential since it is used in the free energy functional.
\begin{align}
  \tilde{g_s}(k, g_\sigma) =\int^\infty_{-\infty} g_s(r, g_\sigma)u_{SW}(\rr)e^{-i\kk \cdot \rr}\mathrm{d}^3\rr
\end{align}
Since the radial distribution function and potential are radially symmetric, we can
express the dot product differently so that the transform is
\begin{align}
  \tilde{g_s}(k, g_\sigma) &=\int^\infty_{-\infty} g_s(r,
  g_\sigma)u_{sw}(r)e^{-ikr \cos \theta}\mathrm{d}^3\rr \\
  &= 2 \pi \int_\sigma^{\lambda \sigma} \int_0^\pi g_s(r, g_\sigma) e^{-ikr \cos
    \theta} r^2 \sin \theta \, \mathrm{d}r \mathrm{d}\theta \\
  &= 2\pi\int_\sigma^{\lambda \sigma} \int_{1}^{-1} g_s(r, g_\sigma) e^{-ikr \cos
    \theta} r^2 \mathrm{d}r \mathrm{d}\cos \theta \\
  &= 2\pi\int_\sigma^{\lambda \sigma} g_s(r, g_\sigma) \frac{1}{-ikr}\bigg[e^{ikr} -
    e^{-ikr} \bigg]r^2 \mathrm{d}r \\
 &= \frac {4\pi}{k}\int_\sigma^{\lambda \sigma} r g_s(r,g_\sigma) \sin (kr) \mathrm{d}r.
\end{align}
Now I'll do the separate integrals for each part on the right hand
side of Eqn. \ref{eq:gs}. The first term is 1, so we are left with
a delta function.  We will ignore this term, and
move onto the next one whose integral is
\begin{align}
  \int_\sigma^{\lambda \sigma}r(g_\sigma -1) e^{-\kappa_0
    \left( \frac{r}{\sigma}-1 \right)} \sin(kr)\mathrm{d}r.
\end{align}
If we pull out constants and put this into Mathematica,
\begin{widetext}
  \begin{align}
  \int_\sigma^{\lambda
    \sigma}re^{-\frac{\kappa_0}{\sigma}r}\sin(kr)\mathrm{d}r
  &= \frac{1}{(k^2 + (\kapovsig{0})^2)^2} \bigg\{ e^{-\kappaO} \left[k
      \left(2\kapovsig{0} + \sigma (k^2 + (\kapovsig{0})^2) \right)
      \cos (\sigma k) + \left((\kapovsig{0})^2(1 + \kappaO) +
      k^2(\kappaO-1)\right) \sin (\sigma k) \right] \notag \\
  &- e^{-\lambda
    \kappaO} \left[ k
      \left(2\kapovsig{0} + \lambda \sigma (k^2 + (\kapovsig{0})^2) \right)
      \cos (\lambda \sigma k) + \left((\kapovsig{0})^2(1 + \lambda \kappaO) +
      k^2(\lambda\kappaO-1)\right) \sin (\lambda \sigma k)\right] \bigg\}
  \end{align}
\end{widetext}
\subsection{Polynomial expansion}

A different way to do this that avoids some of the Fourier transform
troubles is to formulate $g_S$ as a polynomial expansion.  We do this
up to fouth order so that
\begin{align}
  g_S(r;g_\sigma) &= g_\sigma + \sum_i \gamma_i(g_\sigma) \xi_i(r) \\
  &=
  g_\sigma + \sum_{i=1}^{4} \left(\sum_{j=1}^{4} (g_\sigma
  - 1)^j \kappa_{ji} \right)
  \left(\frac{r}{\sigma}-1\right)^i,
  \label{eq:fit-form}
\end{align}
which makes the transforms for $i=1,2,3,4$
\begin{align}
 \frac {4\pi}{k}\int r \xi_i(r) \Phi(r) \sin (kr) \dd r =  \frac
       {-4\pi\epsilon}{k}\int_\sigma^{\lambda \sigma} r \xi_i(r)\sin (kr) \dd r
\end{align}
For the ``zeroth'' term, we treat it like $\xi=1$, so that we're only
transforming the potential well.
%% For the other i's, we'll do a
%% u-substitution where $u= \frac{r}{\sigma}-1$, then the
%% integral becomes
%% \begin{align}
%%  \frac {4\pi}{k}\int_0^{\lambda -1} \sigma^2 (u+1)u^i \sin(k\sigma
%%        [u+1]) \mathrm{d}u.
%% \end{align}
%% Now we calculate the integrals.
\begin{widetext}
\begin{align}
  \int_\sigma^{\lambda \sigma} r \sin (kr) \mathrm{d}r =&
  \frac{1}{k^2}\left[ \sin (k\lambda\sigma) - k\lambda\sigma
    \cos (k\lambda\sigma) - \sin (k\sigma) + k\sigma \cos (k-sigma) \right]\\
  \int_\sigma^{\lambda \sigma} r
 \left( \frac{r}{\sigma}-1\right) \sin (kr) \mathrm{d}r =&
 \frac{1}{k^3 \sigma}\left[(2-\lambda k^2
 \sigma^2(\lambda-1))\cos(k\lambda\sigma)-k\sigma(\sin(k\sigma)
 +(1-2\lambda) \sin(k\lambda\sigma)) -\cos(k\sigma) \right] \\
  \int_\sigma^{\lambda \sigma} r
 \left( \frac{r}{\sigma}-1\right)^2 \sin (kr) \mathrm{d}r =&
 \frac{1}{k^4\sigma^2} \big[
 6\sin(k\sigma)+(k^2\sigma^2(1-4\lambda+3\lambda^2)
 -6)\sin(k\lambda\sigma) \notag \\
 &-2k\sigma\cos(k\sigma) - k\sigma(4 +
 \lambda(k^2\sigma^2(\lambda-1)^2 - 6))\cos(k\lambda\sigma)\big] \\
  \int_\sigma^{\lambda \sigma} r
 \left( \frac{r}{\sigma}-1\right)^3 \sin (kr) \mathrm{d}r =&
 \frac{1}{k^5\sigma^3} \big[k\sigma(6\sin(k\sigma) + (18-24\lambda
   +k^2\sigma^2(\lambda-1)^2(4\lambda-1))\sin(k\lambda\sigma)) \notag \\
   &+24\cos(k\sigma) + (6k^2\sigma^2(\lambda-1)(2\lambda-1)-\lambda
   k^4\sigma^4(\lambda-1)^3)\cos(k\lambda\sigma)\big] \\
  \int_\sigma^{\lambda \sigma} r
 \left( \frac{r}{\sigma}-1\right)^4 \sin (kr) \mathrm{d}r =&
 \frac{1}{k^6\sigma^4}
 \big[(k^2\sigma^2(\lambda-1)(36-60\lambda+k^2\sigma^2(\lambda - 1)^2(5\lambda-1))
   + 120)\sin(k\lambda\sigma) - 120\sin(k\sigma) \notag \\
   &+k\sigma(24\cos(k\sigma) -
   (24(5\lambda-4)-4k^2\sigma^2(\lambda-1)^2 (5\lambda-2) +
   \lambda k^4\sigma^4(\lambda-1)^4))\cos(k\lambda\sigma) \big]
\end{align}
\end{widetext}
\subsubsection{Averaged pair distribution  $g_\sigma(\rr)$}

The approximation for the averaged pair distribution $g_\sigma(\rr)$
is from Schulte \emph{et al.}~\cite{schulte2012using}.  We will here
summarize the key results needed to implement this approximation.

%%Stuff commented out below was me typing out this section before I
%%realised we had something like this.  I'm leaving the code here for now.

%% If we take into account
%% the $\beta$ term then $a_2$ becomes
%% \begin{align}\label{eq:A2-exact}
%%   A_2 &= \int \int \textrm{d$\rr_1$
%%     d$\rr_2$}n(\rr_1)n(\rr_2)g^{(2)}(\rr_1\,\rr_2)\Phi(r_{12})
%% \end{align}
%% We want to use our approximation for $g^{(2)}$
%% written as Eqn. \ref{eq:g2-our-mean}:
%% \begin{align}
%%   g^{(2)}(\rr_1,\rr_2) \approx 
%%   \frac{g_s \left(\rr_{12},g_{\sigma}(\rr_1)\right) + g_s
%%     \left(\rr_{12},g_{\sigma}(\rr_2)\right)}{2}.
%% \end{align}
%% If the look at the homogeneous case then
%% \begin{align}
%%   g_s \left(\rr_{12},g_{\sigma}(\rr_1)\right) = g_s
%%   \left(\rr_{21},g_{\sigma}(\rr_2)\right).
%% \end{align}
%% With which we can rewrite equation \ref{eq:A1-exact} to be  
%% \begin{align}
%%   A_2 &= \int \int \textrm{d$\rr_1$
%%     d$\rr_2$}n(\rr_1)n(\rr_2)g_s(\rr_{12}\,g_\sigma(\rr_1))\Phi(r_{12}).
%% \end{align}
%% We can write $g_s$ as a sum of functions
%% \begin{align}\
%%   g_s(\rr_{12}\,g_\sigma(\rr_1)) = \sum_i f_i(r_{12})G_i(g_\sigma)
%% \end{align}
%% where the ith function we assume is a product of two functions that explicitly
%% depend on $r_{12}$ and $g_\sigma(\rr_1)$ respectively.

\subsection{Local-compressibility approximation}

Barker and Henderson presented two approximations for the second order
term in the inverse temperature expansion, one of which they name the
local-compressiblitity
approximation. \cite{barker2004perturbationSW,barker2004perturbationII}
\fixme{below is some perturbation theory stuff}
We start with the definition for the Helmholtz free energy of the
reference system and add a correction to the energy due to a perturbation correction:
\begin{align}
  A &= A_{ref}+A_a \\
  &=-kT \ln \left( \sum_i e^{-\beta [E_{ref}(i) + \lambda E_a(i)]}
  \right).
\end{align}
Take the derivative at $\lambda = 0$:
\begin{align}
  \frac{\partial A}{\partial \lambda} \bigg{|}_{\lambda = 0} &= -kT
  \frac{\sum_i -\beta E_a(i) e^{-\beta [E_{ref}(i) + \lambda E_a(i)]}}{\sum_i e^{-\beta
      [E_{ref}(i) + \lambda E_a(i)]}} \\
  &= \frac{\sum_i E_a(i) e^{-\beta [E_{ref}(i) + \lambda E_a(i)]}}{\sum_i e^{-\beta
      [E_{ref}(i) + \lambda E_a(i)]}} \\
  &= \sum_i E_{a,i}{P_i}|_{\lambda = 0}.
\end{align}
Then we take the second derivative;
\begin{align}
  \frac{\partial^2 A}{\partial \lambda^2} \bigg{|}_{\lambda = 0} &=
  \sum_i E_{a,i} \frac{\partial P_i}{\partial
    \lambda}\bigg{|}_{\lambda = 0} \\
  &= \sum_i E_{a,i}\Bigg[ \frac{-\beta E_a(i)e^{-\beta [E_{ref}(i) +
        \lambda E_a(i)]}}{\sum_j e^{-\beta [E_{ref}(j) + \lambda E_a(j)]}}
  \notag \\
  &- \frac{e^{-\beta [E_{ref}(i) + \lambda
        E_a(i)]}\sum_k -\beta E_a(k)e^{-\beta [E_{ref}(k) + \lambda
        E_a(k)]}}{\left( \sum_j e^{-\beta [E_{ref}(j) + \lambda
        E_a(j)]} \right)^2} \Bigg] \\
  &= \sum_i E_a(i)\bigg[ -\beta E_a(i)P_i + \beta P_i \sum_j
  E_a(j)P_j \bigg] \\
  &= -\beta \sum_i E_a^2(i)P_i + \beta \sum_i P_i E_a(i)\sum_j E_a(j)
  P_j \\
  &= -\beta \sum_i E_a^2(i)P_i + \beta \bigg(\sum_i E_a(i)P_i\bigg)^2
\end{align}
Since we set $\lambda = 0$, the probabilities are over the reference state so these become
averages over those states.
\begin{align}
  \frac{\partial^2 A}{\partial \lambda^2} \bigg{|}_{\lambda = 0} &=
  -\beta \left( \left< E_a^2 \right>_0 - {\langle E_a \rangle}_0^2 \right)
\end{align}
%% Our derivation for this looks something like this:
%% \begin{align}
%%   A_2 &\approx -\left(\frac{\partial n}{\partial p}\right)
%%   \int\textrm{d$\rr'$} \notag \\
%%   &\times \frac{\delta}{\delta n(\rr')} \int \int
%%   \textrm{d$\rr_1$ d$\rr_2$} \left[n(\rr_1)
%%   n(\rr_2)g^{(2)}(\rr_1\,\rr_2)\Phi(r_{12})^2 \right]
%% \end{align}
%% \begin{align}
%%   A-A_{ref} &= -kT \ln \left(\frac{Z_a}{Z_{ref}}\right) 
%%   \\ &= -kT \ln \left( \frac{\sum_i e^{-\beta[E_{ref}(i) + \lambda
%%         E_a(i)]}} {\sum_i e^{-\beta E_{ref}(i)}} \right) \\
%% \end{align}
%% Since our perturbed Energy $E_a$ is caused by some small potential
%% $u_a$, we can expand $\exp(-\beta \lambda u_a)$ in a power series.
%% This makes
%% \begin{align}
%%   A-A_{ref} &= -kT \ln \bigg{\langle} 1 - \beta
%%   \lambda u_a + \frac{1}{2} \beta^2 \lambda^2 u_a^2 + \cdots \bigg{\rangle}_0
%% \end{align}
%% Here, the average is over the reference energies.  Now we can expand
%% the the natural log in a series
%% \begin{align}
%%   A-A_{ref} &= -kT \ln \bigg{\langle} 1 - \beta
%%   \lambda u_a + \frac{1}{2} \beta^2 \lambda^2 u_a^2 + \cdots \bigg{\rangle}_0
%% \end{align}



\subsection{Using the pair distribution function}

We begin with the definition of $A_1$,
\begin{align}
  A_1 = \frac{1}{2}\iint n^{(2)}(\rr_1,\rr_2) \Phi(|\rr_1-\rr_2|)
  \dd\rr_1 \dd\rr_2
\end{align}

But we must use the relation
\begin{align}
  n^{(2)}(\rr_1,\rr_2) = n(\rr_1)n(\rr_2)g^{(2)}(\rr_1,\rr_2)
\end{align}
and equation \ref{eq:g2-our-mean} so that
\begin{align}
  A_1 &=\frac{1}{4}\iint n(\rr_1)n(\rr_2)\Big[
    g_s(r_{12},g_\sigma(\rr_1)) \notag \\
    &\hspace{2cm} + g_s(r_{12},g_\sigma(\rr_2)) \Big] \Phi(r_{12})
  \dd\rr_1 \dd\rr_2.
\end{align}
$g_s$ is symmetric in the exchange of $\rr_1$ and $\rr_2$, and we'll
expand $g_s$ into its polynomial form
\begin{align}
  A_1 &= \frac{1}{2}\iint n(\rr_1)n(\rr_2)g_s(r_{12},g_\sigma(\rr_1))
  \Phi(r_{12}) \dd\rr_1 \dd\rr_2 \\
  &= \frac{1}{2}\iint n(\rr_1)n(\rr_2)\sum_i
  \gamma_i(g_\sigma(\rr_1))\xi_i(r_{12})\Phi(r_{12}) \dd\rr_1 \dd\rr_2 \\
  &= \frac{1}{2} \sum_i \iint \dd\rr_1 \dd\rr_2
  n(\rr_1)n(\rr_2)\gamma_i(g_\sigma(\rr_1)) \xi_i(r_{12})\Phi(r_{12}).
\end{align}
Then we can rearrange things a little better
\begin{align}
  A_1 &= \frac{1}{2} \sum_i \int \dd\rr_1 n(\rr_1)
  \gamma_i(g_\sigma(\rr_1)) \int \dd\rr_2 n(\rr_2)
  \xi_i(r_{12}) \Phi(r_{12})
\end{align}
\fixme{We have introduced an approximation to the pair distribution
  function above... here we discuss an implementation of thermodynamic
  perturbation theory for an inhomogeneous system using this pair
  distribution function.}


\section{Simulation}

\fixme{A short explanation of MC methods here?}

%% \begin{figure}
%%   \includegraphics[width=\columnwidth]{figs/periodic-ww130-ff30-N200-E.pdf}
%%   \caption{A simple DOS plot!}
%% \end{figure}

%% \begin{figure}
%%   \in cludegraphics[width=\columnwidth]{figs/periodic-ww300-ff30-N200-E.pdf}
%%   \caption{A simple DOS plot!}
%% \end{figure}

\section{Results}

\subsection{Homogeneous fluid}

\fixme{Here we present the comparison between various theories and in
  the homogeneous case.  This will mostly be our pair distribution
  function vs. Gil Villegas.}

\subsection{Hard wall}

\fixme{Here we present the comparison between various theories and
  simulation for a fluid at a hard wall.}

\subsection{Test particle}

\fixme{Here we present the comparison between various theories and
  simulation for a test-particle simulation (i.e. the radial
  distribution function).}

\section{Conclusion}

\fixme{We have an awesome theory. You should agree.}

\bibliography{paper}% Produces the bibliography via BibTeX.

\end{document}
