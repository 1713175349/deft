\documentclass[letterpaper,twocolumn,amsmath,amssymb,pre,aps,10pt]{revtex4-1}
\usepackage{graphicx} % Include figure files
\usepackage{color}
\usepackage{nicefrac} % Include for in-line fractions
\usepackage{harpoon}% <---

\newcommand{\red}[1]{{\bf \color{red} #1}}
\newcommand{\green}[1]{{\bf \color{green} #1}}
\newcommand{\blue}[1]{{\bf \color{blue} #1}}
\newcommand{\cyan}[1]{{\bf \color{cyan} #1}}

\newcommand{\davidsays}[1]{{\color{red} [\green{David:} \emph{#1}]}}
\newcommand{\jpsays}[1]{{\color{red} [\blue{Jordan:} \emph{#1}]}}
\newcommand{\tssays}[1]{{\color{red} [\cyan{Tanner:} \emph{#1}]}}

\newcommand{\Int}{\int\limits}

\begin{document}
\title{Two-dimensional Broad Histogram Monte Carlo}

\author{Jordan K. Pommerenck}
\author{Cade MIDDLE INITIAL? Trotter}
\author{David Roundy}
\affiliation{Department of Physics, Oregon State University,
  Corvallis, OR 97331}

\begin{abstract}
We present several histogram methods and compare the performance and
efficiency at treating the square-well fluid.
\end{abstract}

\maketitle

In component notation, the free energy can be broken down into its constituent components.
\begin{align}
\overrightharp{g} = \bigg(\frac{\partial S}{\partial U}\bigg)_N \hat{U} + \bigg(\frac{\partial S}{\partial S}\bigg)_U \hat{N}
\end{align}
\begin{align}
\overrightharp{g} = \frac{1}{T_{\min}} \hat{U} - \frac{\mu_{\max}}{T_{\min}}  \hat{N}
\end{align}
The gradient of the free energy can be written as follows (thinking of a $\Delta S$ near the minimum entropy $S_{\min}$:
\begin{align}
\Delta S = \left|\overrightharp{g} \cdot \overrightharp{r} \right| = \left|\frac{E_{\min}}{T_{\min}} \right| + \left|\frac{\mu_{\max}N_{\max}}{T_{\min}} \right|
\end{align}
\begin{align}
\langle\Delta S \rangle = \frac{1}{3}\bigg( \left|\frac{E_{\min}}{T_{\min}} \right| + \left|\frac{\mu_{\max}N_{\max}}{T_{\min}} \right| \bigg)
\end{align}
We can write the entropy as a parabolic function with free parameters $A$ and $B$ in terms of the energy and particle number of the
system.
\begin{align}
S = -A E^2 - B N^2
\end{align}
By taking to partial derivatives of the entropy, we can solve for the free parameters.
\begin{align}
\bigg(\frac{\partial S}{\partial E}\bigg)_N = \frac{1}{T_{\min}} = - 2AE
\end{align}
\begin{align}
\bigg(\frac{\partial S}{\partial N}\bigg)_E = - \frac{\mu_{\max}}{T_{\min}} = - 2BN
\end{align}
Therefore the parabolic equation can now be written in terms of known parameters.
\begin{align}
S = \bigg( \frac{1}{2 T_{\min} E}\bigg) E^2 - \bigg( \frac{\mu_{\max}}{2 T_{\min} N}\bigg) N^2 \\
S = \bigg( \frac{1}{2 T_{\min}}\bigg) E - \bigg( \frac{\mu_{\max}}{2 T_{\min}}\bigg) N
\end{align}
Now we integrate this functional form of entropy in terms of the internal energy $E$ and particle number $N$.
\begin{align}
\Int_{0}^{N_{\max}} \Int_{E_{\min}}^{0} \bigg( S - S_{\min}\bigg) \,dE\,dN
\end{align}
Now we insert our derived expressions into the integral and obtain a functional form for the update factor $\gamma$.
\begin{align}
\Int_{0}^{N_{\max}} \Int_{E_{\min}}^{0} \bigg( \bigg( \frac{1}{2 T_{\min}}\bigg) E - \bigg( \frac{\mu_{\max}}{2 T_{\min}}\bigg) N \\- \frac{1}{3}\bigg( \frac{E_{\min}}{T_{\min}}  + \frac{\mu_{\max}N_{\max}}{T_{\min}} \bigg) \,dE\,dN
\end{align}
We then find that the result of the integral is equal to the following:
\begin{align}
\gamma =\frac{1}{t} \frac{E_{\min}N_{\max}}{T_{\min}}\bigg(E_{\min} + 7 \mu_{\max}N_{\max} \bigg)
\end{align}

\bibliography{paper}% Produces the bibliography via BibTeX.

\end{document}
