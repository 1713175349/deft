\documentclass[letterpaper,twocolumn,amsmath,amssymb,pre]{revtex4-1}

\usepackage{color}

\begin{document}
\title{Harmonics with symmetries of platonic solids}

\author{David Roundy}
\affiliation{Department of Physics, Oregon State University}
%\pacs{61.20.Ne, 61.20.Gy, 61.20.Ja}
%%%%%%%%%%%%%%%%%%%%%%%%%%%%%%%%%%%%%%%%%%%%%%%%%%%%%%%%%%%%
\begin{abstract}
  This is just some notes to myself regarding harmonics.
\end{abstract}

\newcommand\rot{\ensuremath{\mathbf{\varpi}}}
\newcommand\rhat{\ensuremath{\mathbf{\hat{r}}}}
\newcommand\solidangle{\ensuremath{\Omega}}

\maketitle

I plan on adding here an introduction to symmetrized linear
combinations of spherical harmonics with the symmetries of platonic
solids.

\section{Cubic}

Von der Lage and Bethe in 1947 give the first few cubic harmonics in
Cartesian coordinates.  They refer to these as ``Kubic Harmonics.''
We will want to convert these into spherical harmonics, using $x/\rho
= \sin\theta\cos\phi$, etc.
\begin{align}
  f_0 &= 1 \\
  f_4 &= \frac{5\sqrt{3\cdot 7}}{4}\left(
  x^4 + y^4 + z^4 - \frac35 \rho^4
  \right) \\
  f_6 &= \frac{3\cdot 7 \cdot 11 \sqrt{2\cdot 13}}{8}\left(
  x^2y^2z^2 + \frac1{22} f_4 - \frac1{105}\rho^6
  \right)\\
  f_8 &= \frac{5\cdot 13  \sqrt{3\cdot 11 \cdot 17}}{16}\left(
  x^8 + y^8 - \frac{28}{f} f_6 \rho^2 - \frac{210}{143} f_4 \rho^4 - \frac16\rho^8
  \right)
\end{align}

\section{Icosohedral}

Zheng and Doerschuk give the icosohedral harmonics in terms of
$\theta$ and $\phi$.  They also give formulas for these in terms of
spherical harmonics, but these are hard to read.
\begin{align}
  I_{00}(\theta,\phi) &= 1 \\
  I_{60}(\theta,\phi) &= 2^3 \cdot 3^2 \cdot 5 \cdot 11
  P_{60}(\cos\theta) - P_{65}(\cos\theta)\cos(5\phi)
  \\
  P_{10,0} &= 2^8 \cdot 3^2 \cdot 5^2 \cdot 7 \cdot 13
  \cdot 19 P_{10,10}
\end{align}

\section{Notes}

The following are defined on page 5 of Zheng and Doerschuk:
\begin{align}
  \phi_k &\equiv \phi_0 + k \frac{2\pi}{5} \\
  \cos \gamma_k &\equiv \cos\beta\cos\theta_0 + \sin\beta
  \sin\theta_0\cos \phi_k \\
  \sin \alpha_k &\equiv - \frac{\sin\theta_0 \sin\phi_k}{\sin\gamma_k}
\end{align}

\end{document}
