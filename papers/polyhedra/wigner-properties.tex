\documentclass[letterpaper,twocolumn,amsmath,amssymb,pre]{revtex4-1}

\begin{document}
\title{Properties of the Wigner D Function}

\author{David Roundy}
\affiliation{Department of Physics, Oregon State University}
%\pacs{61.20.Ne, 61.20.Gy, 61.20.Ja}
%%%%%%%%%%%%%%%%%%%%%%%%%%%%%%%%%%%%%%%%%%%%%%%%%%%%%%%%%%%%
\begin{abstract}
  This is just some notes to myself regarding the Wigner $D$ function,
  which I hope to find useful in my research.
\end{abstract}

\newcommand\rot{\ensuremath{\mathbf{\varpi}}}
\newcommand\rhat{\ensuremath{\mathbf{\hat{r}}}}
\newcommand\solidangle{\ensuremath{\Omega}}

\maketitle

\subsection{Properties of the Wigner $D$ matrix}

There are various ways one can understand the Wigner $D$ function, but
I think the simplest is to state that it defines how spherical
harmonics behave under a rotation that I will call $\rot$, which we
can think of as being defined by a set of Euler angles.  I will
describe the rotation matrix as $R(\rot)$.  Without further ado, I
will state that the $D$ functions (or matrices) are defined by:
\begin{align}\label{eq:Ylm-rotation}
  Y_{lm}(R(\rot)\rhat) &= \sum_{|m'|<l} D^{l}_{mm'}(\rot) Y_{lm'}(\rhat)
\end{align}
which is to say that the $D$ matrix defines how spherical harmonics
transform under rotation.  $D$ functions also form an orthogonal basis
set in the space of rotations (or orientations).
\begin{align}
  \int D^{l}_{mn}(\rot)^{*}D^{l'}_{m'n'}(\rot) d\rot
  &= \frac{8\pi^2}{2l+1}\delta_{ll'}\delta_{mm'}\delta_{nn'}
\end{align}
Sadly, the $D$ matrices aren't normalized... but if they were, then
that normalization would have popped up in
Equation~\ref{eq:Ylm-rotation}, so you can't have everything.

The complex conjugate of a $D$ matrix just corresponds to an inverse
rotation:
\begin{align}
  D^{l}_{mm'}(\rot)^{*} &= (-1)^{m'-m}D^{l}_{-m-m'}(\rot)
\end{align}

\subsection{Properties of spherical harmonics}

Spherical harmonics are properly orthonormal
\begin{align}
  \int Y_{lm}(\rhat)^{*}Y_{l'm'}(\rhat) d\solidangle
  &= \delta_{ll'}\delta_{mm'}\delta_{nn'}
\end{align}

The complex conjugate of a $Y_{lm}$ corresponds to time reversal, or
an opposite rotation (up to a sign):
\begin{align}
  Y_{lm}^{*}(\rhat) &= (-1)^{m} Y_{l,-m}(\rhat) \\
  Y_{lm}(\rhat) &= (-1)^{m} Y_{lm}(-\rhat)
\end{align}

\end{document}
