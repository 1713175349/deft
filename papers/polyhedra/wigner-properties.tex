\documentclass[letterpaper,twocolumn,amsmath,amssymb,pre]{revtex4-1}

\usepackage{color}

\begin{document}
\title{Properties of the Wigner D Function}

\author{David Roundy}
\affiliation{Department of Physics, Oregon State University}
%\pacs{61.20.Ne, 61.20.Gy, 61.20.Ja}
%%%%%%%%%%%%%%%%%%%%%%%%%%%%%%%%%%%%%%%%%%%%%%%%%%%%%%%%%%%%
\begin{abstract}
  This is just some notes to myself regarding the Wigner $D$ function,
  which I hope to find useful in my research.
\end{abstract}

\newcommand\rot{\ensuremath{\mathbf{\varpi}}}
\newcommand\rhat{\ensuremath{\mathbf{\hat{r}}}}
\newcommand\solidangle{\ensuremath{\Omega}}

\maketitle

\subsection{Properties of the Wigner $D$ matrix}

There are various ways one can understand the Wigner $D$ function, but
I think the simplest is to state that it defines how spherical
harmonics behave under a rotation that I will call $\rot$, which we
can think of as being defined by a set of Euler angles.  I will
describe the rotation matrix as $R(\rot)$.  Without further ado, I
will state that the $D$ functions (or matrices) are defined by:
\begin{align}\label{eq:Ylm-rotation}
  Y_{lm}(R(\rot)\rhat) &= \sum_{|m'|<l} D^{l}_{mm'}(\rot) Y_{lm'}(\rhat)
\end{align}
which is to say that the $D$ matrix defines how spherical harmonics
transform under rotation.  $D$ functions also form an orthogonal basis
set in the space of rotations (or orientations).
\begin{align}
  \int D^{l}_{mn}(\rot)^{*}D^{l'}_{m'n'}(\rot) d\rot
  &= \frac{8\pi^2}{2l+1}\delta_{ll'}\delta_{mm'}\delta_{nn'}
\end{align}
Sadly, the $D$ matrices aren't normalized... but if they were, then
that normalization would have popped up in
Equation~\ref{eq:Ylm-rotation}, so you can't have everything.

The complex conjugate of a $D$ matrix just corresponds to an inverse
rotation:
\begin{align}
  D^{l}_{mm'}(\rot)^{*} &= (-1)^{m'-m}D^{l}_{-m-m'}(\rot)
\end{align}

We have a final relationship that allows us to convert products of $D$
matrices into a linear combination of other $D$ matrices:
\begin{align}
  D^{l}_{mk}(\rot)D^{l'}_{m'k'} &=
  \sum_{l'', m'', k''} \left<lml'm' | l''m''\right>
  \left<lkl'k' | l''k''\right> D^{l''}_{m''k''}(\rot)
\end{align}
where the coefficients are Clebsch-Gordon coefficients.  Relating to
this, we have a special case that seems to be sort of handy for
certain Clebsch-Gordon coefficients:
\begin{align}
  \left< lm l'm' | 00 \right> &= \delta_{ll'}\delta{m,-m'}\frac{(-1)^{l-m}}{\sqrt{2l+1}}
\end{align}

\subsection{Properties of spherical harmonics}

Spherical harmonics are properly orthonormal
\begin{align}
  \int Y_{lm}(\rhat)^{*}Y_{l'm'}(\rhat) d\solidangle
  &= \delta_{ll'}\delta_{mm'}\delta_{nn'}
\end{align}

The complex conjugate of a $Y_{lm}$ corresponds to time reversal, or
an opposite rotation (up to a sign):
\begin{align}
  Y_{lm}^{*}(\rhat) &= (-1)^{m} Y_{l,-m}(\rhat) \\
  Y_{lm}(\rhat) &= (-1)^{m} Y_{lm}(-\rhat)
\end{align}

\subsection{A simple integral}
Let's consider the following integral, where $n>1$:
\begin{align}
  X_{lmkn} &\equiv \int \left(\Delta^{l}_{mk}(\rot)\right)^{2n} d\rot
\end{align}
This is useful in the above.  Of course, odd powers vanish.  I expect
that this integral is:
\begin{align}
  X_{lmkn} &=  8\pi^2 \frac{\prod_{i=1}^{n} (2i-1) }{ \prod_{i=1}^{n} i}
\end{align}
based on the assumption that it is independent of the particular value
of $m$ and $k$.  If this is true, then we can simply evaluate
\begin{align}
  X_{ll0n} &= \int \left(\Delta^{l}_{l0}(\rot)\right)^{2n} d\rot \\
  &= \int \left( \sin(l\phi) \right)^{2n} \sin\theta d\theta d\phi
  d\chi \\
  &= 4\pi \int_0^{2\pi} \left( \sin(l\phi) \right)^{2n} d\phi \\
  &=  8\pi^2 \frac{\prod_{i=1}^{n} (2i-1) }{ \prod_{i=1}^{n} 2i}
\end{align}
So if this particular integral, which could be gotten from appropriate
sums of Clebsch-Gordon coefficients, is independent of $m$ and
$k$---which seems likely, somehow, since it shouldn't be affected by a
rotation---then we have a relatively nice closed-form solution for its
value.

\begin{widetext}
\subsection{An interesting integral}

Let's consider the following quantity, which is the simplest
nontrivial real and positive function of orientation that I can
imagine:
\begin{align}
  f(\rot) &= a + b\frac{D^{l}_{mk}(\rot) +
    (-1)^{m}D^{l}_{-m,-k}(\rot)}{\sqrt 2}
  \\
  &= a + b \Delta^{l}_{mk}(\rot)
\end{align}
where I've taken a constant plus a single $lmk$ value, but made it
real.  If $a > |b|$ by enough, then this will always be real, which
will turn out to be important for a density.  I have defined the
function $\Delta^{l}_{mk}(\rot)$, which is a real version of the
Wigner $D$ function, and has the interesting properties that
\begin{align}
  \int \Delta^{l}_{mk}(\rot) \Delta^{l'}_{m'k'}(\rot)d\rot
  &= \frac{8\pi^2}{2l+1}\delta_{ll'}\delta_{mm'}\delta_{nn'}
  \\
  \Delta^{l}_{mk}(\rot)^2 &= \frac{\Delta^{0}_{00}(\rot)}{2l+1} + \text{h.o.t.}
\end{align}
where the higher-order terms are terms with non-zero $l$ value.

We are interested in evaluating the integral:
\begin{align}
  I &= \int f(\rot)\ln(f(\rot)) d\rot
  \\
  &=
  \int f(\rot)\left(\ln a
  + \ln\left(1 + \frac{b}{a}\Delta^{l}_{mk}(\rot)\right)\right) d\rot
  \\
  &=
  \int f(\rot)\left(\ln a
  + \frac{b}{a}\Delta^{l}_{mk}(\rot)
  - \frac12 \frac{b^2}{a^2}(\Delta^{l}_{mk}(\rot))^2
  + \frac13 \frac{b^3}{a^3}(\Delta^{l}_{mk}(\rot))^3
  - \frac14 \frac{b^4}{a^4}(\Delta^{l}_{mk}(\rot))^4
  + \cdots\right) d\rot \\
  &=
  \int f(\rot)\left(\ln a
  - \sum_{n=1}^{\infty} \frac{(-1)^{n}}{n} \left(\frac{b}{a}\Delta^{l}_{mk}\right)^n
  \right) d\rot
  \\
  &=
  \int (a + b \Delta^{l}_{mk}(\rot))\left(\ln a
  - \sum_{n=1}^{\infty} \frac{(-1)^{n}}{n} \left(\frac{b}{a}\Delta^{l}_{mk}\right)^n
  \right) d\rot
  \\
  &=
  a\int \left(\ln a
  - \sum_{n=1}^{\infty} \frac{(-1)^{n}}{n} \left(\frac{b}{a}\Delta^{l}_{mk}\right)^n
  \right) d\rot
  -
  b\int \left(\sum_{n=1}^{\infty} \frac{(-1)^{n}}{n} \left(\frac{b}{a}\Delta^{l}_{mk}\right)^{n+1}
  \right) d\rot
  \\
  &=
  a\int \left(\ln a
  - \sum_{n=1}^{\infty} \frac{1}{2n} \left(\frac{b}{a}\Delta^{l}_{mk}\right)^{2n}
  \right) d\rot
  +
  b\int \left(\sum_{n=1}^{\infty} \frac{1}{2n-1} \left(\frac{b}{a}\Delta^{l}_{mk}\right)^{2n}
  \right) d\rot
  \\
  &=
  8\pi^2 a \ln a
  - 8\pi^2a\left(
  \sum_{n=1}^{\infty} \frac{1}{2n} \left(\frac{b}{a}\right)^{2n}
   \frac{\prod_{i=1}^{n} (2i-1) }{ \prod_{i=1}^{n} 2i}
  \right)
  +
  8\pi^2 b \left(
   \sum_{n=1}^{\infty} \frac{1}{2n-1} \left(\frac{b}{a}\right)^{2n}
   \frac{\prod_{i=1}^{n} (2i-1) }{ \prod_{i=1}^{n} 2i}
  \right)
  \\
  &=
  8\pi^2 a \ln a
  - 8\pi^2a\left(
  \sum_{n=1}^{\infty} \frac{1}{2n} \left(\frac{b}{a}\right)^{2n}
   \frac{\prod_{i=1}^{n} (2i-1) }{ \prod_{i=1}^{n} 2i}
  \right)
  +
  8\pi^2 b \left(
   \sum_{n=1}^{\infty} \left(\frac{b}{a}\right)^{2n}
   \frac{\prod_{i=1}^{n-1} (2i-1) }{ \prod_{i=1}^{n} 2i}
  \right)
  \\
  &=
  8\pi^2 a \left\{
  \ln a
  -
  \sum_{n=1}^{\infty} \frac{1}{2n} \left(\frac{b}{a}\right)^{2n}
   \frac{\prod_{i=1}^{n} (2i-1) }{ \prod_{i=1}^{n} 2i}
  +
   \sum_{n=1}^{\infty} \left(\frac{b}{a}\right)^{2n+1}
   \frac{\prod_{i=1}^{n-1} (2i-1) }{ \prod_{i=1}^{n} 2i}
  \right\}
\end{align}
\end{widetext}
The next idea would be to take these power series sums, and see if I
could turn them into an analytic form.  As it is, I don't recognize
them, but it's quite possible that they have a relatively simple
form.  If they don't, I'll come up with a reasonable approximation
that has the correct behavior in the limit as $b \ll a$, and also has
the correct divergent behavior as $b$ approaches $a$.

\end{document}
