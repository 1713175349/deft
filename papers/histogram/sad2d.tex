\documentclass[letterpaper,twocolumn,amsmath,amssymb,pre,aps,10pt]{revtex4-1}
\usepackage{graphicx} % Include figure files
\usepackage{color}
\usepackage{nicefrac} % Include for inline fractions

\usepackage{xargs}                      % Use more than one optional parameter in a new commands
\usepackage[pdftex,dvipsnames]{xcolor}  % Coloured text etc.
\usepackage[colorinlistoftodos,prependcaption,textsize=normalsize]{todonotes}
\usepackage{mdframed}

% define colors for comments
\definecolor{dark-gray}{gray}{0.10}
\definecolor{light-gray}{gray}{0.70}

\newcommand{\red}[1]{{\bf \color{red} #1}}
\newcommand{\green}[1]{{\bf \color{green} #1}}
\newcommand{\blue}[1]{{\bf \color{blue} #1}}
\newcommand{\cyan}[1]{{\bf \color{cyan} #1}}

\newcommand{\davidsays}[1]{{\color{red} [\green{David:} \emph{#1}]}}
\newcommand{\jpsays}[1]{{\color{red} [\blue{Jordan:} \emph{#1}]}}
\newcommandx{\jpcom}[2][1=inline]{\todo[linecolor=gray,backgroundcolor=light-gray,bordercolor=dark-gray,#1]{\textbf{Jordan says:} #2} }
\begin{document}

\title{Notes on 2D sad
}

\author{Jordan K. Pommerenck} \author{David Roundy}
\affiliation{Department of Physics, Oregon State University,
  Corvallis, OR 97331}

\begin{abstract}
  Notes for a possible 2D sad algorithm
\end{abstract}

\maketitle

\section{The basics}

\begin{align}
  dU &= TdS - pdV + \mu dN \\
  dE_{exc} &= TdS_{exc} - p_{exc}dV + \mu_{exc} dN
\end{align}
This tells us that
\begin{align}
  \frac{1}{T} &= \frac{S_{exc}(N,E+\Delta E) - S_{exc}(N,E)}{\Delta E}
\end{align}
and
\begin{align}
  \mu_{exc} &= \left(\frac{\partial E_{exc}}{\partial N}\right)_{S_{exc}}
  \\
  &= -\frac{
    \left(\frac{\partial E_{exc}}{\partial S_{exc}}\right)_{N}
  }{
    \left(\frac{\partial N}{\partial S_{exc}}\right)_{E_{exc}}
  }
  \\
  &= -\frac{
    \left(\frac{\partial S_{exc}}{\partial N}\right)_{E_{exc}}
  }{
    \left(\frac{\partial S_{exc}}{\partial E_{exc}}\right)_{N}
  }
  \\
  &= -T\left(\frac{\partial S_{exc}}{\partial N}\right)_{E_{exc}}
\end{align}
where I used the cyclic chain rule to make the derivatives feasible.

\section{Moving forward}

\begin{align}
  U &= ST - pV + \mu N \\
  E_{exc} &= S_{exc}T - p_{exc}V + \mu_{exc}N
\end{align}
We want a maximum total pressure $p_{\max}$, which makes for a maximum excess pressure of
$p_{\max} - p_{id} = p_{\max} - NkT/V$.  Thus at higher $N$ values, we have
\begin{align}
  E_{exc} &= S_{exc}T - \left(p_{\max}-\frac{NkT}{V}\right)V - NT\left(\frac{\partial S_{exc}}{\partial N}\right)_{E}
  \\
  &= S_{exc}T - p_{\max}V + NkT - NT\left(\frac{\partial S_{exc}}{\partial N}\right)_{E}
\end{align}
solving for the rate of change, we see
\begin{align}
  \left(\frac{\partial S_{exc}}{\partial N}\right)_{E} &= \frac{S_{exc}T - E_{exc} - p_{\max}V + NkT}{NT}
  \\
  &= \frac{S_{exc}}{N} - \frac{E_{exc}+p_{\max}V}{NT} + k
\end{align}
So at a finite value of $N>N_{hi}$ let's start with an ansatz:
\begin{align}
  S_{exc}(N,E) &= A e^{BN} + C \\
  \left(\frac{\partial S_{exc}}{\partial N}\right)_{E} &= AB e^{BN} \\
  &= BS_{exc}(N,E) - BC
\end{align}
thus we can see that our ansatz was nonsense.


\end{document}
