\documentclass[letterpaper,twocolumn,amsmath,amssymb,pre,aps,10pt]{revtex4-1}
\usepackage{graphicx} % Include figure files
\usepackage{color}
\usepackage{nicefrac} % Include for inline fractions

\newcommand{\red}[1]{{\bf \color{red} #1}}
\newcommand{\green}[1]{{\bf \color{green} #1}}
\newcommand{\blue}[1]{{\bf \color{blue} #1}}
\newcommand{\cyan}[1]{{\bf \color{cyan} #1}}

\newcommand{\davidsays}[1]{{\color{red} [\green{David:} \emph{#1}]}}
\newcommand{\jpsays}[1]{{\color{red} [\blue{Jordan:} \emph{#1}]}}
\newcommand{\tssays}[1]{{\color{red} [\cyan{Tanner:} \emph{#1}]}}

\begin{document}
\title{Comparing novel parameter independent broad-histogram methods
applied to the square-well liquid}

\author{Jordan K. Pommerenck} \author{Michael A. Perlin}
\author{Tanner T. Simpson} \author{David Roundy}
\affiliation{Department of Physics, Oregon State University,
  Corvallis, OR 97331}

\begin{abstract}
  We present several histogram methods and compare the performance and efficiency at treating the square-well fluid.
\end{abstract}

\maketitle

\section{Introduction}
Over the past several decades a number of advanced Monte-Carlo
simulation algorithms have been developed which calculate the
properties of a thermal system over a range of temperatures.  This
idea began with the original histogram method, which used a single
canonical Monte Carlo simulation to predict properties for nearby
temperatures~\cite{ferrenberg1988new}.  For large simulations this
approach is limited to a narrow temperature range because a single
canonical simulation explores only a small range of energies.  This
led to a variety of ``broad'' (or ``flat'') histogram
methods~\cite{penna1996broad, penna1998broad, swendsen1999transition,
  wang2001determining, wang2001efficient, trebst2004optimizing}, which
attempt to explore a wide range of energies.  These methods also
benefit in that they are unlikely to be trapped in a local energy
minimum.

Wang and Landau introduced an algorithm that uses an update
factor and a statistical histogram to compute the density of states of
a given system~\cite{wang2001determining, wang2001efficient}.  While
the method is incredibly powerful, it has several user-defined
parameters.  This can make its application to a variety of systems
something of an art-form since the user needs to determine the ideal
parameters for his particular system.  Also, detailed balance is
violated (although for only short periods of time), yet the method
converges in most situations.  This raises several questions such as
when does the method converges, what rate will the method converge to
the correct density of states, and how to appropriately decide what
parameters to choose so that the algorithm solves the problem in the
best way. Transition Matrix Monte Carlo~\cite{wang1999transition,
swendsen1999transition, fitzgerald2000monte} became an attractive
complimentary simulation algorithm to Wang-Landau in that, if detailed
balance is ensured, the system is guaranteed to converge (something
which is not guaranteed for the Wang-Landau Algorithm).  Unfortunately,
the algorithm can be considerably more difficult to implement when
compared to Wang-Landau: in particular, the infinite temperature
transition matrix~\cite{wang2002transition}.  Also, the time necessary
for the density of states to converge could be considerable
due to the algorithm spending too much time exploring low energy states.
~\tssays{(note to self) The above paragraph could be condensed a bit}

Shell et al.~\cite{shell2003improved, shell2004flat} originially
implemented Wang-Landau Transition Matrix Monte Carlo (WL-TMMC) in an
effort to quickly explore the energies of a given system using WL then
switch over to TMMC to guarantee convergence. Considerable effort has
been spent trying to determine the flatness criteria and cutoff for the
WL portion, but there remains no rigorous systematic way to determine
these parameters~\cite{rane2013monte}.  The algorithm is first run for
several trial simulations and/or `experience' is used to determine the
best parameters for a users particular system of
study~\cite{siderius2013use}.  Around the same time, the convergence
rate for Wang-Landau was being explored~\cite{zhou2005understanding}.
It was found that utilizing modification factors that decrease faster
than $1/t^2$ leads to nonconvergence~\cite{belardinelli2007fast}.  This
lead to the so-called $1/t$-WL algorithm which ensures that excess CPU
time is not wasted by continuing to perfrom calculations once the error
in the density of states becomes
saturated~\cite{belardinelli2008analysis}. An effort to reduce the
number of user defined parameters of WL and formulate a theory for why
the method converges despite detailed balance being violated although
infrequently was also being udertaken.  Liang, Liu, and Carrol began to
consider whether WL could be considered a special case of Monte Carlo
whose convergence could be mathematically
proven~\cite{liang2006theory, liang2007stochastic}. In 2007, Liang et
al.~\cite{liang2007stochastic} argued that WL can be considered a form
of Stochastic Approximation Monte Carlo (SAMC).  While SAMC can
guarantee convergence, the method still has a system specific
user-defined variable which makes applying this algorithm to arbitrary
systems difficult.

In our work, we have developed several novel algorithms that do not
require user-defined inputs and therefore should be easily applicable
to a given system.  These methods, each of which will be discussed in
detail in the methods section, can also be made to satisfy theoretical
requirements, guaranteeing convergence. These algorithms are Transition
Matrix Initialization (TMI), Dynamic Stochastic Approximation (SAD),
and Stochastic Approximation Transition Matrix Monte Carlo (SA-TMMC).
TMI works by using logical conditionals about changes in the logrithm
of the density of states to control the weights of the simulation.  SAD
operates effectively the same as standard SA; however, we claim that
the user defined parameter can be set independent of the system type or
size.  SA-TMMC works by performing SA for a system-independent amount
of time in order to explore the entire energy space.  The simulation
then switches over to TMMC for the remainder of the simulation.  All of
these methods have no parameter inputs that depend on the type of
system or size of system under study.

We consider the square-well fluid i.e. a system of particles whose
interactions are governed by a square-well
potential~\cite{singh2003surface, barker2004perturbationSW}.  The
square-well potential is an ideal test-bed as it is the simplest model
that ensures both attractive and repulsive forces are experienced by
interacting particles~\cite{barker1967-SW-perturbation, vega1992phase}.
The potential $U(\textbf{r})$ for such a system is given by
\begin{equation}
 U(\textbf{r})=\begin{cases} \infty &
 \lvert\textbf{r}\rvert< \sigma\\-\epsilon &
 \sigma<\lvert\textbf{r}\rvert<\lambda\sigma\\0 &
 \lvert\textbf{r}\rvert > \lambda\sigma\end{cases}
\end{equation}
where $\sigma$ is the hard-sphere diameter of the particle, $\lambda$ is the
reduced range of the potential well, and $\epsilon$ is its depth.

The organization of this paper is as follows: In Section II, we
describe in detail several broad-histogram methods used to calculate
the density of states for the liquid-vapor system.  Section III
contains the results and comparison of the methods applied to the
square-well fluid.  In Section IV, we discuss the performance of the
various methods applied to the model system.

\section{Methods}

Here we employ a variety of broad-histogram methods.  We outline the
general workings of algorithm we developed in detail and summarize
algorithms that can be found elsewhere.  The following algorithms are
introduced and applied to the square-well fluid: Transition Matrix
Initialization (TMI), Transition Optimized Ensemble (TOE), Transition
Matrix Monte-Carlo (TMMC), Wang-Landau (WL), Wang-Landau Transition
Matrix Monte-Carlo (WL-TMMC), Dynamic Stochastic Approximation
Monte-Carlo (SAD), and Stochastic-Approximation Transition Matrix
Monte-Carlo (SA-TMMC).

\subsection{TMI2 Algorithm}

The second version of transition matrix initialization algorithm works
by looking at the logarithm of the density of states.  We begin at
high energies ($\epsilon_{S_{\max}}$ the energy that maximizes the
entropy) and work our way
downward.

\paragraph{Reaching $T_{\min}$:}
If the change in the logarithm of the density of states is greater
than the reciprocal of the minimum temperature, i.e. if
\begin{align}
  \ln\frac{D(\epsilon_i)}{D(\epsilon_{i-1})} &>
  \frac{\epsilon_{i-1} - \epsilon_i}{k_BT_{\min}}
  \\
  \frac{D(\epsilon_i)}{D(\epsilon_{i-1})} &>
  e^{\beta_{\max}(\epsilon_{i-1} - \epsilon_i)}
\end{align}
then the algorithm has reached the minimum temperature of interest.
In this case, we set the log weights at all lower energies using a
Boltzmann ratio relative to this energy.
\paragraph{Confident at $\epsilon_i$:} We are confident in our density
of states at energy $\epsilon_i$ provided two conditions are true.
\begin{enumerate}
\item The density of states must be decreasing as the energy decreases
  from its next-higher energy.
\item The change in density of states (relative to the next higher
  energy) must be smaller than a pessimistic estimate of the counting
  uncertainty at the energy we are examining:
  \begin{align}
    \frac{D(\epsilon_i)}{D(\epsilon_{i-1})} < \frac1{\sqrt{N_i}}
  \end{align}
  where $N_i$ is the number of ``pessimistic
  samples'' at $\epsilon_i$, the lower of the two energies under
  consideration.
\end{enumerate}
If both of these are true, we believe we know the density of states at
this energy well, and the weights are set by
\begin{align}
  \ln w(\epsilon_i) = -\ln{D}(\epsilon_i)
\end{align}
\paragraph{Reached $T_\text{converged}$:}
If this is not the case, $\epsilon_i$ is the lowest energy that is
adequately converged.  We set the weights for energies lower than this
to a Boltzmann ratio with a temperature determined using the slope of
a secant line on the $\ln w$ versus $\epsilon$ graph.
\begin{align}
  \beta_\text{secant} = \bigg(\frac{\ln{D}(\epsilon_{S_{\max}}) - \ln{D}
  (\epsilon_\text{pivot})}{\epsilon_{S_{\max}}-\epsilon_\text{pivot}}\bigg)
\end{align}
This ``slope'' is a positive quantity.  The weights at energies below
$\epsilon_\text{pivot}$ are given by
\begin{align}
  \ln w(\epsilon) = \ln w(\epsilon_\text{pivot}) -
  \beta_\text{secant}(\epsilon - \epsilon_\text{pivot})
\end{align}
indicating that there are higher weights at lower energies, since
$\beta$ is positive.  We have one additional caveat: we examine the
lower energies for which we have samples (``pessimistic'' samples),
and if we find any energy at which the density of states is greater
than the inverse of the weight, we adjust $\beta$ to cause it to pass
through that point.  This ensures that the product of the weight with
the estimated density of states is always less than or equal one for
energies below $\epsilon_\text{converged}$.

\subsection{TMI3 Algorithm}
The TMI$_3$ algorithm is like the TMI$_2$ algorithm, with one
difference, which is that instead of a secant line being used to
estimate the converged temperature, we use a tangent line, estimated
using the lowest two ``converged'' energies.

\subsection{TOE3 Algorithm}
The TOE$_3$ method desribed by Trebst et.
al.~\cite{trebst2004optimizing} uses the metric of round trips defined
by a walker visiting the extreme energies for the density of states.
The rate for the walker is dependent of the diffusivity $D(\epsilon)$
at a given energy $E$.  The goal of the method is to maximize the
number of round trips that the Monte-Carlo simulation makes.  The
optimized density of states is given by the following:
\begin{align}
  n_{w}^{(opt)} = \frac{1}{\sqrt{D(E)}}
\end{align}
where unlike in the original paper detailing the method, we compute the
diffusivity directly from the transition matrix normalized with
canonical weights.  The optimal ensemble therefore relates the density
of states to the inverse of the square root of the diffusivity.
\begin{align}
  D(E) = \mid (T(\Delta\epsilon)^2 - (T(\Delta\epsilon))^2) \mid
\end{align}

\subsection{TMMC Algorithm}
The Transition Matrix Monte-Carlo (TMMC) method as introduced by Fitzgerald et al.
consists of three primary steps.  The first two steps are common to the Metropolis
algorithm while the third makes use of the actual transition
probabilities~\cite{fitzgerald2000monte}.

\subsubsection{define a transition probability}
Assuming that the system under study is in a state denoted old and a
proposed transition is made to a new state denoted new, the probability
is defined as
\begin{align}
  p_{old \rightarrow new} = p_{new \rightarrow old}
\end{align}
where we have allowed the probabilies to be equal for simplicity of presentation.

\subsubsection{define an acceptance probability}
The probability for accepting a transition from an old state to a new one can be
defined
\begin{align}
  p_{a} = \min\bigg[1,\frac{\alpha_{new\rightarrow old}}
  {\alpha_{old \rightarrow new}}\frac{\pi_{new}}{\pi_{old}}\bigg]
\end{align}
where $\alpha_{new\rightarrow old}$ is the probability of generating a
new system configuration the old.  For conventional Monte-Carlo moves
$\alpha_{old \rightarrow new} =\alpha_{new\rightarrow old}$ although
for advanced Monte-Carlo moves this would not be the
case~\cite{paluch2008comparing, siepmann1990method}.  The probabily of
observing the system in the old state can be found based on the
ensemble constraints placed on the system.  Fitzgerald et al.
considered the system of interest using the canonical ensemble however
extension to other ensembles is quite natural.
\begin{align}
  \pi_{old} = \frac{e^{-\beta E_{old}}}{Z}
\end{align}
The acceptance probability given for TMMC is idential to that introduced by
Swendsen and Wang~\cite{wang2002transition}.
\begin{align}
  p_{a} = \min\bigg[1,e^{-\frac{\delta\epsilon}{k_{B} T}}\bigg]
\end{align}
\subsubsection{define a bookkeeping step}
The third step in the TMMC algorithm consists of recording the data in a collection
matrix (analogous to the transition histogram matrix).
\begin{align}
  C_{N,\delta} = C_{N,\delta} + p_{a}\\
  C_{N,0} = C_{N,0} +(1 - p_{a})
\end{align} If $\delta=0$, that is, the energy state has not been
changed, then only $C_{N,0}$ is incremented by unity. Fitzgerald et
al.~\cite{fitzgerald2000monte} show that defining the third step this
way results in a uniform improvement over using histogram estimators to
determine the particle number probability distribution (PNPD).

\subsection{Wang-Landau Algorithm}

The Wang-Landau Algorithm developed by Wang and Landau proposes a
random walk in energy space in order to estimate the density of
states~\cite{wang2001efficient}. While methods such as Metropolis
Sampling and Swendsen-Wang cluster
flipping~\cite{swendsen1987nonuniversal} generate a narrow distribution
and require sampling at individual temperatures, the Wang-Landau
algorithm uses a flat histogram and performs a random walk in energy
space to determine the density of states~\cite{wang2001determining,
landau2014guide}. Wang-Landau's major tenant is that when counting the
histogram $H(\epsilon)$, the energy occurences should form a flat
distribution.  The criteria for flat sampling is given by
\begin{equation}
	\frac{\min_{\epsilon} H(\epsilon)}
	{\big\langle H(\epsilon)\big\rangle }
	> \eta
\end{equation} where $\eta$ is usually set to $0.80$ and determines how
many times each energy is sampled relative to the mean number of
visits.  Difficulties can occur with this flatness criteria due to the
fact that some energies on a given energy range might never be
sampled~\cite{haber2014transition}.  Over the past decade much
attention was directed at examining the rate of convergence and error
saturation~\cite{lee2006convergence, belardinelli2007wang}.  The
algorithm is briefly outlined below where $w$ represents the weights:
\begin{equation}
	\ln{w_{k+1}(\epsilon,N)}=\ln{w_{k}(\epsilon,N)}
	+\gamma
\end{equation}
\begin{equation}
	\ln{\gamma_{k+1}}=\frac{u}{2}\ln{\gamma_{k}}
\end{equation}
for $u$ typically greater than 1 and $\gamma$ is the modification factor for
the density of states.  We then accept a new configuration for the density of
states with a probability given by the following
\begin{equation}
	\mathcal{P}(\epsilon_i \rightarrow \varepsilon_j)
	= \min[1,e^{\ln{w(\epsilon_i)}-\ln{w(\epsilon_j)}}]
\end{equation}
Zhou and Bhatt explored the convergence of the Wang-Landau algorithm and
determined that choosing a large modification factor $\gamma_0 = e^{4}$ and
rapidly reducing the factor by 10 during each step resulted in reduced statistical
error~\cite{zhou2005understanding}.


\subsection{WL-TMMC Algorithm}

Wang-Landau Transition Matrix Monte-Carlo (WL-TMMC) method as proposed
by Shell and coworkers~\cite{shell2003improved,shell2004flat} takes
advantage of the ease of implementation and applicability to a variety
of systems, while making use of a transition matrix to adress WL
weaknesses such as a limited statistical accuracy plateau which is not
improved with additional Monte-Carlo steps.  This WL version is more
accurate for a given number of simulation steps.

\paragraph{Initializing Wang-Landau:} We begin the simulation by
defining a minimum important energy and a maximum entropy state. In
traditional Wang-Landau the WL factor begins at 1.0 and the factor is
modified by 2.0 until the simulation is cutoff typically around
$10^{-10}$.  Rane et. al. published an implemenation of
WL-TMMC~\cite{rane2013monte} that suggested using a WL factor between 1.0 and
$10^{-2}$ with a cutoff $<10^{-5}$. In our implementation of WL-TMMC,
we set the WL factor to be 1 and the cutoff at $10^{-10}$. A single
sweep is defined as each macrostate being sampled a number of times in
this case 10,000 (Siderius et. al. does 100). Wang-Landau sets a
flatness criteria~\cite{wang2001determining, wang2001efficient,
hatch2015computational, mahynski2017predicting} for the accumulated
histogram usually between 0.75 to 0.90.  For our simulation, it is
sufficient that each macrostate is visited at least once (each
histogram bin has one entry)~\cite{shell2003improved}.  The goal of
this WL initialization is to prefill the transition matrix for the next
portion of the simulation.
\paragraph{Initializing TMMC:} The purpose of prefilling the transition
matrix is apparent because if numerous zeros existed in the collection
matrix, the infinite temperature transition matrix would be
ill-defined.  Sampling over large ranges of density of states would
therefore take an unacceptable length of time to
complete~\cite{shell2003improved, shen2014elucidating}.  The transition
matrix in terms of the collection matrix is given as follows:
\begin{align}
\widetilde{T}_{\infty}(I\rightarrow J) = \frac{C(I,J)}
{\sum_{K} C(I,K)}
\end{align}
We now run the TMMC portion of the simulation.  Siderius does this for
20 additional sweeps (does about 8 sweeps for WL).  We do this until we
reach a specified temperature of interest.
We have not yet implemented a biasing function to control acceptance
rate of trial moves but we probably could/should? at some point.
\begin{align}
  P_{acc}(i\rightarrow j) = \min\bigg[1,\frac{\widetilde{T}_{\infty}(J\rightarrow I)}
  {\widetilde{T}_{\infty}(I\rightarrow J)}\bigg]
\end{align}
this is only used to control actual transitions not update the collection
matrices.




\subsection{SAD Algorithm}
The SAD (Dynamic Stochastic Approximation) is a version of the SAMC
Algorithm that attempts to dynamically choose the modification factor
rather than relying on system dependent parameters such as $t_0$ or
$\gamma_0$.  There is an immediate advantage of such an algorithm where
parameters are chosen independent of system size or type. SAMC can be
applied as a dynamic importance sampling algorithm that uses an update
factor to explore energy space~\cite{liang2007stochastic,
werlich2015stochastic, schneider2017convergence}.  The update factor is
defined in the original implementation~\cite{liang2007stochastic} in
terms of an arbitrary tunable parameter $t_0$.
\begin{align}
\gamma_{t}^{\text{SA}} = \frac{t_0}{\max(t_0,t)}
\end{align}
We argue that it is possible to set this parameter to a system
independent parameter such that $t_0 = N_E^2/t$ where $N_E$ is the
number of energies that have been found.  One of the energy found terms
can be thought of as a change in the entropy $\Delta S$.  The change in
the entropy can be thought of as a slope of the triangle made by
connecting the minimum important energy and the maximum entropy state
divided by the minimum temperature:
\begin{align}
\Delta S = \frac{E_{\text{max}}-E_{\text{min}}}{T_{\text{min}}}
\end{align}
This gives an update factor as follows:
\begin{align}
\gamma_{t}^{\text{SA}} = \frac{N_E\Delta S}{t} =
\frac{N_E}{t}\frac{E_{\text{max}}-E_{\text{min}}}{T_{\text{min}}}
\end{align}
The update factor has the same $1/t$ behavior as the original SAMC
algorithm; however now every time a new energy is discovered during the
course of the simulation, the update factor experiences a jump.

\subsection{SA-TMMC Algorithm} We introduce a new algorithm consisting
of the merger of the Stochastic Approximation (SA) method proposed by
Liang et al.~\cite{liang2007stochastic} and the TMMC Algorithm with
suitable adaptations.  While the Stochastic Approximation method has
recently been implemented in comparison with the Wang-Landau
Algorithm~\cite{werlich2015stochastic, schneider2017convergence}, it
seems quite natural to create a new Monte-Carlo method which we will
call SA-TMMC.  Like in the original implementation, we define an update
factor where $t_0$ is the number of Monte-Carlo moves and $t$ is the
current iteration. \begin{align} \gamma_{t}^{\text{SA}} =
\frac{t_0}{\max(t_0,t)} \end{align} One potential problem of merging SA
and TMMC is that SA provides no nice transition to the later method due
to the lack of a user-controlled cutoff.  Also for SA Monte-Carlo, the
choice of parameter $t_0$ proves difficult particularly for complex
systems.  Finally, we enforce a minimum temperature probability which
prevents the system from sampling states below the minimum temperature.
We discuss here how we resolve all of the potential issues we
encountered when integrating both SA and TMMC.

\subsubsection{Define an adaptive parameter $t_0$}
It has been shown that if $F_{t}^{\text{SA}}$ satisfies two conditions
the estimator is proven to converge to the real
DOS~\cite{liang2006theory, liang2007stochastic}.
\begin{align}
\sum_{t=1}^\infty \gamma_{t}^{\text{SA}} = \infty \quad\textrm{and}\quad
\sum_{t=1}^\infty (\gamma_{t}^{\text{SA}})^\zeta < \infty
\end{align}
where $\zeta \in \{1,2\}$.  For practical implementation with TMMC, we
find it necessary to reset $t_0$ everytime a new energy state is found.
This ensures that all energy states are sampled before TMMC move
probabilities begin to be favored.

\subsubsection{Define a statistical weight to control the transition between SA and TMMC}
Our algorithm can alternate between SA and TMMC on the fly using
weights to determine the probability that a move between adjacent
energy states is allowed.  We define the transition probability
$\text{SA}_{\text{Weight}}$ as
\begin{align}
\delta_{\text{TMMC}}^2 =\frac{1}{n_\downarrow}+\frac{1}{n_\uparrow}, \quad\quad
\delta_{\text{SA}}^2 = \frac{1}{N_E}
\end{align}
\begin{align}
\text{SA}_{\text{Weight}} =\frac{\delta_{\text{TMMC}}^2}{\delta_{\text{TMMC}}^2 + \delta_{\text{SA}}^2}
\end{align}
where $N_\text{found}$ is the number of energy states that have been
found i.e. visited, and $n_{\uparrow\downarrow}$ are the number of
upward or downward moves between a pair of energy states respectively.
The rational behind the transition probability is that the transition
weight can be related to the fractional uncertainty in upward and
downward moves assuming there is no system correlation (the moves are
independent).  Although this is not necessarily the case, we find that
this definition for the transition weight is satifactory. The
probability of a move is then the geometic average of either using the
SA or TMMC Algorithm.

\begin{align}
\mathcal{P}_\text{move} = \mathcal{P}_\text{move}^{\text{SA}_\text{Weight}} \mathcal{P}_\text{move}^{1-\text{SA}_\text{Weight}}
\end{align}

\subsubsection{Enforce a minimum temperature probability}
One potentially damaging aspect to the convergence of the Stochastic
Approximation method is the need to enforce a minimum temperature
probability.  We define this probability as
\begin{align}
P_{\text{min}} = e^{\beta_{\text{max}}\Delta\epsilon}
\end{align}
where $\beta_{\text{max}} = \nicefrac{1}{k_B T_\text{min}}$.  If the
transition probability is ever below the minimum temperature transition
probability, we set the former equal to the later.

\section{Results}

For the square-well fluid, our simulations treat systems with a
well-width of $\text{ww} = 1.30$ and a filling fraction of either
$\text{ff} = 0.22$ or $\text{ww} = 0.30$.  The simulation explores the
energy space of the system down to a reduced temperature of
$T_{\text{min}} = 0.2$.  All simulations lead to the minimum important
energy $E_{\text{min}}$ and maximum entropy state $S_{\text{max}}$
being calculated (with the exception of the WL-TMMC method where both
of these parameters are needed a priori).  In the sections below, we
examine systems with different particles sizes, system dimensions, and
sticky-wall.

\subsubsection{Sticky-wall configuration}
A sticky-wall is an attractive wall that impacts particle moves.  Only
one such wall is inserted into a simulation and the intent is to assist
in visualizing phase separation.  As the temperature of the system
decreases, an increase in particle aggregation near the sticky-wall
becomes evident. As the energy of the particles is decreased, the
particles are unlikely to be able to escape the attractive sticky-wall
potential.  By examining the particle number density vs the system
length (z-axis), phase transitions to a liquid-vapor system from a
vapor dominated system become evident.

\begin{figure}
  \includegraphics[width=\columnwidth]{figs/lv/ww1_30-ff0_22-30x4-entropy-error.pdf}
  \caption{The average entropy error for each method is shown with tmi3 serving as
  the `golden' standard.}\label{fig:avgS-30x4}
\end{figure}
\begin{figure}
  \includegraphics[width=\columnwidth]{figs/lv/ww1_30-ff0_22-60x8-entropy-error.pdf}
  \caption{The average entropy error for each method is shown with sad serving as
  the `golden' standard.}\label{fig:avgS-60x8}
\end{figure}
\begin{figure}
  \includegraphics[width=\columnwidth]{figs/lv/ww1_30-ff0_22-100x10-entropy-error}
  \caption{The average entropy error for each method is shown with
    WL-TMMC using a cutoff of 10$^{-10}$ serving as the
    standard.}\label{fig:avgS-100x10}
\end{figure}
\begin{figure}
  \includegraphics[width=\columnwidth]{figs/s000/periodic-ww1_30-ff0_20-N50-entropy-error.pdf}
  \caption{The average entropy error is shown with tmi3 acting as the
  'golden' standard.}
\end{figure}
\begin{figure}
\includegraphics[width=\columnwidth]{figs/s000/periodic-ww1_30-ff0_30-N500-entropy-error.pdf}
  \caption{The average entropy error is shown with tmi3 acting as the
  'golden' standard.}
\end{figure}

\subsubsection{A $30\times4\times4$ $(N = 25)$ system with a sticky
wall}
In order to determine how the various methods scale with size,
simulations are run for systems with different particle numbers.  For
the $N = 25$ system, a sticky-wall is added to ensure that we see the
liquid and vapor phase separation.

A system is configured with dimensions $x = y = 4\sigma$ and $z =
30\sigma$.  A filling fraction $\text{ff} = 0.22$ is chosen which
yields $N = 25$ particles in the rectangular system.  These particles
are randomly inserted into the system and random moves are proposed
(with a translation scale $\delta_0 = 0.05$). The boundaries in $x$ and
$y$ are periodic while the $z$-axis is not periodic.  A sticky-wall is
inserted along the $z$-axis at one end to promote phase separation.
~\jpsays{should we include a sticky-wall.py plot or something else or
are we not going to talk about phase separation in this paper?}

We use the entropy error vs iterations as a metric to compare
simulation runtimes.  Simulations that take more iterations to reach
the same entropy error as a different method take longer to converge.
This behavior is seen in Fig.~\ref{fig:avgS-30x4}.
~\jpsays{SAD appears to do pretty well as does WL-TMMC for a cutoff of $10^{-10}$
although this seems to depend on what is used as the `standard'.}

\subsubsection{A Tanner dimensions? $(N = 50)$ system with no sticky
wall}

A system is configured with dimensions $x = y = z = 10\sigma?$.  A
filling fraction $\text{ff} = 0.20$ is chosen which yields $N = 50$
particles in the cubical system.  These particles are randomly
inserted into the system and random moves are proposed (with a
translation scale $\delta_0 = 0.05$). The boundaries in $x$ and $y$ are
periodic while the $z$-axis is not periodic.  A sticky-wall is not inserted
along the $z$-axis.

\subsubsection{A $60\times8\times8$ $(N = 202)$ system with a sticky
wall}

A system is configured with dimensions $x = y = 8\sigma$ and $z =
60\sigma$.  A filling fraction $\text{ff} = 0.22$ is chosen which
yields $N = 202$ particles in the rectangular system.  These particles
are randomly inserted into the system and random moves are proposed
(with a translation scale $\delta_0 = 0.05$). The boundaries in $x$ and
$y$ are periodic while the $z$-axis is not periodic.  A sticky-wall is
inserted along the $z$-axis at one end to promote phase separation.

\subsubsection{A $100\times10\times10$ $(N = 525)$ system with a sticky
wall}

A system is configured with dimensions $x = y = 10\sigma$ and $z =
100\sigma$.  A filling fraction $\text{ff} = 0.22$ is chosen which
yields $N = 525$ particles in the rectangular system.  These particles
are randomly inserted into the system and random moves are proposed
(with a translation scale $\delta_0 = 0.05$). The boundaries in $x$ and
$y$ are periodic while the $z$-axis is not periodic.  A sticky-wall is
inserted along the $z$-axis at one end to promote phase separation.

\subsubsection{A Tanner dimensions? $(N = 500)$ system with no sticky
wall}

A system is configured with dimensions $x = y = z = 19\sigma?$.  A
filling fraction $\text{ff} = 0.30$ is chosen which yields $N = 500$
particles in a cubical system.  These particles are randomly inserted
into the system and random moves are proposed (with a translation scale
$\delta_0 = 0.05$). The boundaries in $x$ and $y$ are periodic while
the $z$-axis is not periodic.  A sticky-wall is not inserted along the
$z$-axis.

\section{Conclusions}

We introduced a new algorithm SAD (which is Stochastic Approximation
with the modification factor dynamically chosen) that effectively
samples the entire energy space for a variety of system sizes.  Also,
SAD proves effective in comparison with other Monte-Carlo methods at
converging to the correct density of states.  The ease of
implementation and the absence of user-defined parameters makes SAD an
ideal algorithm.

~\jpsays{We need about a paragraph here that says exactly how much
better SAD performed than other methods and or how easy it was to
implement (Likely best added once simulations run a little longer).}

We note that SAD should be tested on additional systems, and the
accuracy of the method should be examined.  Also, a rigorous
mathematical justification for our choice of the update factor is
necessary to ensure the most optimal convergence of the algorithm.

\section{Acknowledgement}

J.K. Pommerenck would like to gratefully acknowledge D.W. Siderius and
J.R. Errington for helpful communications regarding the implementation
of WL-TMMC.

% \jpsays{Everything below this is TMI1 and GC-TMMC}

%\subsection{GC-TMMC Algorithm}
%Grand Canonical Transition Matrix Monte-Carlo (GC-TMMC) allows the energy and
%particle number of the system to fluctuate while holding the chemical potential,
%volume, and temperature constant~\cite{chen2001aggregation, chen2002simulating,
%errington2003direct, fenwick2006accurate, fenwick2008direct, errington2005direct}.
%The algorithm boasts a lower statistical noise than conventional GCMC due to its
%incorporation of a transition matrix to estimate ensemble averages~\cite{siderius2013use,
%paluch2008comparing, grzelak2008computation}.  The steps for the algorithm are
%outlined below:

%\subsubsection{Define particle insertion/deletion}
%In the system of consideration, the number of particles $\mathcal{N}$ can change
%by an amount $\delta$.  It is assumed that in the system either no particles can be
%inserted or deleted, a single insertion takes place, or a single deletion takes place.
%This yields a condition $\delta={-1,0,+1}$ for particle insertion.

%\subsubsection{Define an acceptance probability}
%Now it is necessary to define criteria for determining whether a particle insertion/deletion
%is allowed.  This criteria is called the acceptance probability and is denoted as
%\begin{align}
  %p_{a} = \min\bigg[1,\frac{\alpha_{new\rightarrow old}}
  %{\alpha_{old \rightarrow new}}\frac{\pi_{new}}{\pi_{old}}\bigg]
%\end{align}
%where $\alpha_{new\rightarrow old}$ is the probability of generating an old system
%configuration from the new.  For conventional Monte-Carlo moves $\alpha_{old \rightarrow new}
%=\alpha_{new\rightarrow old}$ although for advanced Monte-Carlo moves this would not
%be the case~\cite{siepmann1990method}.  The probabilies of actually observing the
%system in the old state is given by
%\begin{align}
  %\pi_{old} = \frac{1}{\Xi}\frac{V^{N_{old}}}{\Lambda^{3N_{old}}N_{old}!}
  %e^{-\beta(\epsilon_{old}-\mu N_{old})}
%\end{align}
%where $\Xi$ is the grand canonical partition function and $\Lambda$ is the De Broglie
%wavelength.

%\subsubsection{Define a Collection Matrix}
%In order to calculate an overall transition probability $\mathcal{P}_{N,\delta}$, we
%must first define how the system procession is to be recorded.  A Collection Matrix
%can be defined, such that for any one move, two elements are updated.
%\begin{align}
  %{C}_{N,\delta} = {C}_{N,\delta} + p_{a}\\
  %{C}_{N,0} = {C}_{N,0} +(1 - p_{a})
%\end{align}
%If $\delta=0$, that is, the number of particles has not been changed, then only
%$\mathcal{C}_{N,0}$ is incremented by unity.  The overal transition probability can
%now be defined as
%\begin{align}
  %\mathcal{P}_{N,\delta} = \frac{{C}_{N,\delta}}
  %{\sum_{j=-1}^{+1} {C}_{N,j}}
%\end{align}

%\subsubsection{Define the Particle Number Probability Distribution}
%The Particle Number Probability Distribution (PNPD) $\Pi(N;\mu,V,T)$ can be found
%by enforcing detailed balance in the system.
%\begin{align}
  %\ln(\Pi_{N+1}) = \ln(\Pi_{N}) + \ln\bigg(\frac{\mathcal{P}_{N\rightarrow N+1}}
  %{\mathcal{P}_{N+1\rightarrow N}}\bigg)
%\end{align}
%This equation will hold as long as N-space can be sampled sufficiently.  In an
%implemented algorithm $N$, $N_{min}$, and $N_{max}$ are specified for macrostate domain
%sampling.

%\subsubsection{Define a biasing function}
%The primary difference between TMMC as defined by Fitzgerald et al. and GC-TMMC as
%pioneered by JR Errington et al. is the use of a biasing function to control the acceptance
%rate of trial moves~\cite{siderius2013use, fitzgerald2000monte}.  This type of
%multi-canonical sampling augments the acceptance probability in the following way
%\begin{align}
  %p_{\eta,old\rightarrow new} = \min\bigg[1,\frac{e^{\eta(N_{new})}}
  %{e^{\eta(N_{old})}}\frac{\alpha_{new\rightarrow old}}
  %{\alpha_{old \rightarrow new}}\frac{\pi_{new}}{\pi_{old}}\bigg]
%\end{align}

%Although $p_{\eta}$ is used to control the actual transitions for trial configurations
%in GC-TMMC, $p_{a}$ is still used to update the Collection Matrices.  Ideally the biasing
%function would be given by
%\begin{align}
  %\eta(N) = -\ln\Pi(N;\mu,V,T)
%\end{align}
%Since $\Pi$ is not usually known a priori, the biasing function is usually set to some
%arbitrary constant for initialization.  The goal of this biasing function is to achieve
%uniform sampling (a common feature shared by the flat histogram Monte-Carlo family).

%\subsubsection{Histogram reweighting for the PNPD}
%Once the PNPD has beeen collected for a given chemical potential $\mu_0$, histogram
%reweighting is used to determine thermodynamic properties at other chemic potentials.
%Since the canonical partition function $Z(N,V,T)$ does not depend on the chemical
%potential, we can write
%\begin{align}
%\begin{split}
  %\Pi(N;\mu,V,T) = \frac{e^{\beta\mu N}Z(N,V,T)}{\Xi(\mu,V,T)}\\
  %\ln\Pi(N;\mu,V,T) = \ln\Pi(N;\mu_0,V,T) \\
  %+ \beta N(\mu-\mu_0) + C
%\end{split}
%\end{align}
%where $C$ is a normalization constant independent of the number of particles.


\bibliography{paper}% Produces the bibliography via BibTeX.

\end{document}
