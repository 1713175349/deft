\documentclass[letterpaper,twocolumn,amsmath,amssymb,pre,aps,10pt]{revtex4-1}
\usepackage{graphicx}% Include figure files
\usepackage{color}

\newcommand{\red}[1]{{\bf \color{red} #1}}
\newcommand{\green}[1]{{\bf \color{green} #1}}
\newcommand{\blue}[1]{{\bf \color{blue} #1}}
\newcommand{\cyan}[1]{{\bf \color{cyan} #1}}

\newcommand{\davidsays}[1]{{\color{red} [\green{David:} \emph{#1}]}}
\newcommand{\mpsays}[1]{{\color{red} [\blue{Michael:} \emph{#1}]}}

\begin{document}
\title{Applying the optimized ensemble histogram method to the
  square-well liquid}

\author{Michael A. Perlin} \author{David Roundy}
\affiliation{Department of Physics, Oregon State University,
  Corvallis, OR 97331}

\begin{abstract}
  We have applied the clever histogram method to the square-well
  liquid.
\end{abstract}

\maketitle

\begin{figure}
  \includegraphics[width=\columnwidth]{figs/periodic-ww130-ff30-N20-E}
  \caption{A histogram plot demonstrating the difficulty of using
    canonical Monte Carlo to simulate a system.  A simulation at a
    give temperature only provides information fo a small number of
    energy states.\label{fig:histograms}}
\end{figure}

\begin{figure}
  \includegraphics[width=\columnwidth]{figs/periodic-ww130-ff30-N20-weights}
  \caption{The weight functions from different methods.}
\end{figure}

\begin{figure}
  \includegraphics[width=\columnwidth]{figs/periodic-ww130-ff30-N20-dos}
  \caption{A density of states plot demonstrating the difficulty of
    using canonical Monte Carlo to simulate a system.  A simulation at
    a give temperature only provides information fo a small number of
    energy states.\label{fig:dos}}
\end{figure}

\begin{figure}
  \includegraphics[width=\columnwidth]{figs/periodic-ww130-ff30-N20-sample-rate}
  \caption{Inverse sampling rates from different methods.}
\end{figure}

% \begin{figure}
%   \includegraphics[width=\columnwidth]{figs/periodic-ww130-ff30-scaling}
%   \caption{Scaling.\label{fig:scaling}}
% \end{figure}

\begin{figure}
  \includegraphics[width=\columnwidth]{figs/periodic-ww130-ff30-N20-u}
  \caption{Specific internal energy.\label{fig:u}}
\end{figure}

\begin{figure}
  \includegraphics[width=\columnwidth]{figs/periodic-ww130-ff30-N20-hc}
  \caption{Specific heat capacity.\label{fig:hc}}
\end{figure}

\begin{figure}
  \includegraphics[width=\columnwidth]{figs/periodic-ww130-ff30-N20-S}
  \caption{Specific entropy.\label{fig:S}}
\end{figure}

\begin{figure}
  \includegraphics[width=\columnwidth]{figs/periodic-ww130-ff30-N20-u_err}
  \caption{Specific internal energy.\label{fig:u}}
\end{figure}

\begin{figure}
  \includegraphics[width=\columnwidth]{figs/periodic-ww130-ff30-N20-hc_err}
  \caption{Specific heat capacity.\label{fig:hc}}
\end{figure}

\begin{figure}
  \includegraphics[width=\columnwidth]{figs/periodic-ww130-ff30-N20-s_err}
  \caption{Error in specific heat entropy.\label{fig:Serr}}
\end{figure}

%% \begin{figure}
%%   \includegraphics[width=\columnwidth]{figs/periodic-ww130-ff30-N20-rdf}
%%   \caption{Radial distribution function.\label{fig:rdf}}
%% \end{figure}

As shown in Fig.~\ref{fig:histograms}, there are several difficulties
encountered when running Monte Carlo simulations at a fixed
temperature.  On the top plot, which shows three simulations run at
$\lambda = 1.3$, it is apparent that each given canonical simulation
only provides statistical information for a small number of energy
states.  The bottom plot shows an even more serious issue: when
running a simulation at a fixed temperature, it is possible to become
frozen in a state that is far from the ground state, as happened in
the $kT=0.1\epsilon$ simulation.

% \begin{table*}
% \input{figs/scaling-table-ww130-ff30}
% \caption{Scaling data}
% \end{table*}

\section{New ideas for algorithms}

\subsection{Finding maximum probability state}

Technically maximum histogram or maximum entropy.

Loop:

run iterations until we see $N$ energy changes, counting the number of
returns to the initial energy.  If there are sufficient returns, then
we are at the maximum, where sufficient returns will be determined by
a random walk (weighted?) where we compute the expected number of
returns if the walk is not weighted.  We may want to correct to get
all the way to the top.

If we are not at the top yet, we try again.  Eventually we will be.

\subsection{Finding variance of the histogram}

I assume we first ran the max-entropy/histogram algorithm, which gives
us a maximum value of $E_0$.  We then zero the histogram.

We run for a short while, and then compute the mean and standard
deviation of the energy from the histogram: $\bar E$ and $\sigma$.
While doing this, we track the number of returns to the initial state
$n$.  Now the uncertainty in the mean is given by
\begin{equation}
  \Delta \bar E \approx \frac{\sigma}{\sqrt{n}}
\end{equation}
if $\Delta \bar E \ll (E_0 - \bar E)$ then we can conclude that we are
certain where the maximum is, and how wide it is.

\begin{figure*}
  \includegraphics[width=0.33\textwidth]{figs/periodic-ww130-ff30-N20-tmmc-golden-transitions}\hfill%
\includegraphics[width=0.33\textwidth]{figs/periodic-ww130-ff30-N20-tmmc-transitions}\hfill%
\includegraphics[width=0.33\textwidth]{figs/periodic-ww130-ff30-N20-tmmc-golden-tmmc-compare-transitions}
  \caption{A plot showing the probability of energy transitions (with
    no weighting) from given energy states.\label{fig:transitions}}
\end{figure*}

\subsection{Tracking transitions}

We can track the number of attempted transitions from each energy
eestate to each other energy state.  This statistic has the advantage of
being approximately independent of the weighting function used
(although it is highly dependent on the step size).  Thus we could
continue to refine this statistic, even as we are updating the weight
array.  Here are two papers that seem highly relevant to this
method~\cite{wang1999transition, wang2002transition}, but I'm not sure
if they're identical.

The ratio of density of states at two energies should be proportional
to the ratio of upgoing and downgoing transition rates between those
two energies.  Thus these transitions give us information about
$\Delta D$ (where $D$ is the density of states).

Figure~\ref{fig:transitions} below shows the transition matrix for a
20-sphere system, computed during initialization using the Wang-Landau
algorithm.

How can we use this?

\subsubsection{Generating $w$ from the transition matrix}

We should be able to generate a set of weighting functions directly
from the transition matrix.  We could do this either by optimizing for
a flat histogram, or by optimizing the round-trip rate.  Both would be
interesting to implement.

One natural approach would use linear algebra.  If the normalized
transition matrix is $T$, then the density of states is given by
\begin{equation}
  D = TD
\end{equation}
By solving this equation for $D$, we could find a good approximation
for $w$ to obtain a flat histogram.

I expect that we could use similar math to the optimized ensemble
folks to optimize the round-trip rate.

\subsubsection{Initializing using the transition matrix directly}

An alternative initialization algorithm involves using the transition
matrix directly during initialization.  This process was proposed by
Swendsen \emph{et al.} as a modification for their initialization
procedure~\cite{swendsen1999transition}.  In this case, the transition
matrix is continually updated, thus modifying the relative weighting
of different energies as the simulation proceeds.  This strictly
speaking no longer satisfies detailed balance, but as the simulation
proceeds this violation will quickly become increasingly small.

This raises the question of how to use the transition matrix to
directly determine transition probabilities, since the self-consistent
eigenvalue solution is unsuitable.  The key is to recognize that we do
not require weights for each energy but only \emph{differences} of
weights for on pair of energies at a time.  This is a much smaller
problem.

Swendsen \emph{et al.} find for a flat distribution, the acceptance
rate for a given transition is equal to the ratio of that element in
the transition matrix to the element corresponding to the reverse
transition~\cite{swendsen1999transition}.  Naturally, for a different
distribution, we can scale the acceptance ratios according to the
ratio of occupations desired.  We modify this algorithm to always
accept transitions to a lower energy, thus employing a broad histogram
method that does not much sample energies above the maximum-entropy
energy.

Swendsen only uses this approach after a slightly hokey two-stage
initialization.  It seems to work pretty well, and asymptotically it
does have detailed balance.

\begin{figure}
  \includegraphics[width=\columnwidth]{figs/periodic-ww130-ff30-u_errors}
  \caption{$u$ error scaling.}
  \label{fig:scaling-u_err}
\end{figure}
\begin{figure}
  \includegraphics[width=\columnwidth]{figs/periodic-ww130-ff30-cv_errors}
  \caption{$c_V$ error scaling.}
  \label{fig:scaling-cv_err}
\end{figure}
\begin{figure}
  \includegraphics[width=\columnwidth]{figs/periodic-ww130-ff30-s_errors}
  \caption{$s$ error scaling.}
  \label{fig:scaling-s_err}
\end{figure}

% \subsection{Convergence tests}

% \begin{figure}
%   \includegraphics[width=\columnwidth]{figs/periodic-ww130-ff30-N20-dos-conv-T10}
%   \caption{Convergence}
%   \label{fig:conv-dos}
% \end{figure}

% \begin{figure}
%   \includegraphics[width=\columnwidth]{figs/periodic-ww130-ff30-N20-u-conv-T10}
%   \caption{Convergence}
%   \label{fig:conv-u}
% \end{figure}

% \begin{figure}
%   \includegraphics[width=\columnwidth]{figs/periodic-ww130-ff30-N20-hc-conv-T10}
%   \caption{Convergence}
%   \label{fig:conv-hc}
% \end{figure}
% \begin{figure}
%   \includegraphics[width=\columnwidth]{figs/periodic-ww130-ff30-N20-S-conv-T10}
%   \caption{Convergence}
%   \label{fig:conv-S}
% \end{figure}

\bibliography{paper}% Produces the bibliography via BibTeX.

\end{document}
