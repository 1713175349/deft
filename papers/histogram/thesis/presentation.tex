\documentclass[xcolor=dvipsnames]{beamer}
\usepackage{beamerthemelined}
\usepackage{pstricks}

\setbeamertemplate{navigation symbols}{}
\setbeamertemplate{caption}[numbered]

\usepackage{caption}
\captionsetup{textfont={small,bf,color=BlueViolet}}
\captionsetup{labelformat=empty}

%%% physics and math
\usepackage[boldvectors,braket]{physymb}
\newcommand{\bk}{\Braket} % shorthand for braket notation
\usepackage{amsmath}
\newcommand{\p}[1]{\left(#1\right)} % parenthesis
\renewcommand{\sp}[1]{\left[#1\right]} % square parenthesis
\renewcommand{\set}[1]{\left\{#1\right\}} % curly parenthesis
\newcommand{\f}[2]{\dfrac{#1}{#2}}
\renewcommand{\d}{\partial}

\let\olditem\item
\renewcommand{\item}{\setlength{\itemsep}{6pt}\olditem}
\usepackage{graphicx}

\title[Optimizing MC Simulation of the SW Fluid]
{Optimizing Monte Carlo Simulation of the Square-Well Fluid}
\author[M. A. Perlin]{Michael A. Perlin}
\institute{\bf Department of Physics, Oregon State University}
\date{02 June 2015}

\begin{document}

\begin{frame}
  \maketitle
\end{frame}

\begin{frame}
  \frametitle{Phase transitions}
  \begin{figure}
    \centering
    \includegraphics[width=0.5\textwidth]
    <1>{figs/phase-diagram.pdf}
    \includegraphics[width=0.5\textwidth]
    <2>{figs/phase-diagram-critical-point.pdf}
    \caption{A generic phase diagram}
  \end{figure}
\end{frame}

\begin{frame}
  \frametitle{Square-well fluid}
  \begin{itemize}
  \item Simplest system with a liquid-vapor phase transition
  \item Spheres with pair potential
  \begin{align*}
    v_{sw}\p{r}=\left\{
      \begin{array}{ll}
        \infty & r<\sigma \\
        -\epsilon & \sigma<r<\lambda\sigma \\
        0 & r>\sigma
      \end{array}
    \right.
  \end{align*}
  \end{itemize}
\end{frame}

\begin{frame}
  \frametitle{Square-well fluid}
  \begin{figure}
    \centering
    \includegraphics[width=0.5\textwidth]<1>{figs/square-well.pdf}
    \includegraphics[width=0.5\textwidth]<2>{figs/square-well-v1.pdf}
    \includegraphics[width=0.5\textwidth]<3>{figs/square-well-v2.pdf}
    \includegraphics[width=0.5\textwidth]<4>{figs/square-well-v3.pdf}
    \caption{Pair potential of the square-well fluid}
  \end{figure}
\end{frame}

\begin{frame}
  \frametitle{Monte Carlo fluid simulation}
  \begin{itemize}
  \item Model systems and the real world; theory testing
  \item Dynamic vs equilibrium simulation
  \item Metropolis Monte Carlo: random walk in configuration space
  \item Histogram $H\p{E}$ of observed energies is proportional to the
    density of states $D\p{E}$
  \end{itemize}
\end{frame}

\begin{frame}
  \frametitle{Thermodynamic properties}
  \begin{align*}
    Z\p{T}=\sum_se^{-E_s/kT}=\sum_ED\p{E}e^{-E/kT}
  \end{align*}
  \begin{align*}
    \bk{X}_T=\sum_EX\p{E}P\p{E}=\f1{Z\p{T}}\sum_EX\p{E}D\p{E}e^{-E/kT}
  \end{align*}
\end{frame}

\begin{frame}
  \frametitle{Thermodynamic properties}
  \begin{align*}
    U\p{T}=\bk{E}_T
  \end{align*}
  \begin{align*}
    C_V\p{T}=\p{\f{\d U}{\d T}}_V
    =\bk{\p{\f{E}{kT}}^2}_T-\bk{\f{E}{kT}}_T^2
  \end{align*}
\end{frame}

\begin{frame}
  \frametitle{Density of states}
  \begin{figure}
    \centering
    \includegraphics[width=0.5\textwidth]
    <1>{figs/dos-poster-example.pdf}
    \includegraphics[width=0.5\textwidth]
    <2>{figs/dos-thesis-example.pdf}
    \caption{Sample density of states for a particular square-well
      fluid}
  \end{figure}
\end{frame}

\begin{frame}
  \frametitle{Histogram methods}
  \begin{itemize}
  \item {\it Biased} walk in configuration space
  \item To each energy $E$, assign a weight $w\p{E}$
  \item Accept Monte Carlo moves with probability
    \begin{align*}
      P\p{s_i\to s_f}=\min\set{\f{w\p{E_f}}{w\p{E_i}},1}
    \end{align*}
  \item $w\p{E}$ is proportional to the bias on states with energy $E$
  \item Histogram method: an algorithm for finding $w\p{E}$
  \end{itemize}
\end{frame}

\end{document}