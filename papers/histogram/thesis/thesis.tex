\documentclass[11pt]{article}

%%% standard header
\usepackage[margin=1in]{geometry} % one inch margins
\usepackage{fancyhdr} % easier header and footer management
\pagestyle{fancyplain} % page formatting style
\usepackage{hyperref} % for linking references
\newcommand{\psl}{6pt} % store spacing as command for consistency
\setlength{\parindent}{0cm} % don't indent new paragraphs...
\parskip \psl % ... place a space between paragraphs instead
\usepackage[inline]{enumitem} % include for \setlist{}, use below
\frenchspacing % add a single space after a period
\usepackage{lastpage} % for referencing last page
\cfoot{\thepage~of \pageref{LastPage}} % "x of y" page labeling

%%% physics
\usepackage[boldvectors,braket]{physymb} % physics package
\newcommand{\bk}{\Braket} % shorthand for braket notation

%%% math
\usepackage{amsmath} % math package
\renewcommand{\t}{\text} % for text in math environment
\newcommand{\f}[2]{\dfrac{#1}{#2}} % shorthand for fractions
\newcommand{\p}[1]{\left(#1\right)} % parenthesis
\renewcommand{\sp}[1]{\left[#1\right]} % square parenthesis
\renewcommand{\set}[1]{\left\{#1\right\}} % curly parenthesis

%%% figures
\usepackage{graphicx,float,subcaption} % floats, etc.
\usepackage{footnote} % for footnotes in floats
\usepackage[font=small,labelfont=bf]{caption} % caption text options

%%% bibliography, table of contents
\usepackage[sort&compress,numbers]{natbib} % bibliography options
\bibliographystyle{apsrev4-1}
\usepackage[english]{babel} % allow editing bibliography title
\addto\captionsenglish{ % set bibliography title
  \renewcommand{\contentsname}{Table of Contents}
  \renewcommand\refname{\bf References}}
\usepackage{tocloft}
\renewcommand{\cftsecleader}{\cftdotfill{\cftdotsep}}

%%% algorithm float
\floatstyle{plaintop}
\newfloat{algorithm}{thb}{lop}
\floatname{algorithm}{Algorithm}
\newenvironment{alg}
{\hrulefill\begin{enumerate}}
{\end{enumerate}\hrulefill}

\renewcommand{\title}{Optimizing Monte Carlo simulation of the
  square-well fluid} \renewcommand{\author}{Michael A. Perlin}
\newcommand{\institution}{Oregon State University}
\newcommand{\department}{Department of Physics}
\newcommand{\supervisor}{Dr. David Roundy}
\renewcommand{\date}{08 May 2015}

%%% notes, etc.
\usepackage{color}
\newcommand{\red}[1]{{\bf \color{red} #1}}
\newcommand{\fixme}[1]{[\red{fixme:} \emph{#1}]}

\fancypagestyle{abstract}{
  \rhead{}
  \lhead{}
}

\begin{document}

%%%%%%%%%%%%%%%%%%%%%%%%%%%%%%%%%%%%%%%%%%%%%%%%%%%%%%%%%%%%%%%%%%%%%%
\begin{titlepage}

  \newcommand{\HRule}{\rule{\linewidth}{0.5mm}}

  \center

  \textsc{\LARGE \institution}\\[1.5cm]
  \textsc{\Large \department}\\[0.5cm]

  \HRule \\[0.4cm]
  { \huge \bfseries \title}\\[0.4cm]
  \HRule \\[1.5cm]

  \begin{minipage}{0.4\textwidth}
    \begin{flushleft} \large
      \emph{Author:}\\
      \author
    \end{flushleft}
  \end{minipage}
  ~
  \begin{minipage}{0.4\textwidth}
    \begin{flushright} \large
      \emph{Supervisor:} \\
      \supervisor
    \end{flushright}
  \end{minipage}\\[4cm]

  {\large\date}

  \vfill

\end{titlepage}

\thispagestyle{empty}

\newpage

\tableofcontents

\thispagestyle{empty}

\newpage

\listof{algorithm}{\Large List of Algorithms}

\listof{figure}{\Large List of Figures}

\thispagestyle{empty}

\newpage

\setcounter{page}{1}


%%%%%%%%%%%%%%%%%%%%%%%%%%%%%%%%%%%%%%%%%%%%%%%%%%%%%%%%%%%%%%%%%%%%%%
\section*{Abstract}
\label{sec:abstract}

\thispagestyle{abstract}

We identify and develop efficient numerical methods for determining
thermodynamic properties of the square-well fluid in order to test
square-well density functional theories. The liquid-vapor phase
interface is an interesting regime for testing density functional
theories, but unbiased Metropolis Monte Carlo simulations are
incapable of sampling the low energy fluid states which dominate the
partition function at low (i.e. liquid-state) temperatures. Previous
works have developed several generic Monte Carlo histogram methods for
collecting statistics on low energy system states, but little or no
literature exists on these methods' systematic comparison, as well as
their application to the square-well fluid. The square-well fluid in
particular introduces application challenges not manifest in
traditional models for testing and benchmarking such numerical
techniques (e.g. the Ising model). We propose to implement these
methods in Monte Carlo simulations of the square-well fluid, determine
appropriate performance metrics for their comparison, and identify
those methods which are most efficient and effective at sampling
otherwise improbable system states.

\fixme{Rewrite after the rest of the thesis is completed}


\newpage


%%%%%%%%%%%%%%%%%%%%%%%%%%%%%%%%%%%%%%%%%%%%%%%%%%%%%%%%%%%%%%%%%%%%%%
\section{Introduction}
\label{sec:intro}

\fixme{give more intro about phase transitions and the critical
  point?}

Understanding the behavior of fluids near the critical point, or the
point at which the distinction between liquid and gas ceases to become
well-defined, is a major challenge in the field of classical density
functional theory. Studying the critical point of a fluid via
simulations requires exploring the boundary between its liquid and
gaseous phases. The square-well fluid is the simplest system with a
liquid-vapor phase transition, and is therefore of interest for
studying the critical point of fluids. For a generic square-well fluid
to be observed in a liquid state, it must have a localized
distribution of the spheres which make up the fluid. Such a
distribution implies a low energy fluid state, as all interactions
between spheres are attractive. Conversely, a low energy liquid state
of a square-well fluid otherwise observable in both liquid and gaseous
phases implies a localized distribution of spheres. The search for
liquid states of the square-well fluid is thus equivalent to the
search for low energy states.

Monte Carlo simulations are a standard means for studying the
equilibrium thermodynamic properties of a system. In the most
straightforward implementation of Monte Carlo fluid simulations,
however, highly localized distributions of fluid elements are
extremely unlikely to occur, making such simulations impractical for
studying the critical point of the square-well fluid. Several generic
computational methods, called histogram methods, exist for dealing
with such a problem; namely, that interesting regions of a system's
state space cannot be sampled via standard Monte Carlo methods in a
reasonable amount of time. There is no common consensus, however, on
the relative efficacy of these methods. Furthermore, some methods,
such as the Wang-Landau method\cite{wang_landau}, require tuning
several free parameters that lack both ``obvious'' values and
systematic means for assigning them. This work will thus attempt to
quantify and compare the effectiveness of a few such methods as
applied to the square-well fluid.


%%%%%%%%%%%%%%%%%%%%%%%%%%%%%%%%%%%%%%%%%%%%%%%%%%%%%%%%%%%%%%%%%%%%%%
\section{Background}
\label{sec:background}

\subsection{The square-well fluid}
\label{sec:sw_fluid}

\begin{figure}[tb]
  \centering
  \includegraphics[width=0.47\textwidth]{figs/square-well.pdf}
  \caption[The square-well pair potential]{The square-well pair
    potential can be used to model short range forces to first
    order. The infinite potential at $r<\sigma$ simply enforces the
    condition that spheres cannot overlap. At $r>\lambda\sigma$, the
    potential is null, and makes no contribution to the fluid's
    internal energy.}
  \label{fig:pair_potential}
\end{figure}

The square-well (SW) fluid is a simple model used in classical density
functional theories to capture low order effects of short-range
attractive forces, such as the van der Waals force. The fluid is
composed of spheres with diameter $\sigma$ which have a pair potential
\begin{align}
  v_{sw}\p{\vec r}=\left\{
    \begin{array}{ll}
      \infty & \abs{\vec r}<\sigma \\
      -\epsilon & \sigma<\abs{\vec r}<\lambda\sigma \\
      0 & \abs{\vec r}>\lambda\sigma
    \end{array}
  \right., \label{eq:pair_potential}
\end{align}
also shown graphically in Figure \ref{fig:pair_potential}, where the
parameters $\lambda$ and $\epsilon$ are referred to as the well width
and depth, respectively. The first ($\abs{\vec r}<\sigma$) part of
this potential forbids spheres from overlapping, whereas the second
($\sigma<\abs{\vec r}<\lambda\sigma$) associates an energy $-\epsilon$
with each pair of spheres whose centers are within distance of
$\lambda\sigma$ of each other (where typically, $\lambda\in(1,3]$).
The net potential energy of the square-well fluid is thus
\begin{align}
  E=\sum_{i<j}v_{sw}\p{\vec r_i-\vec r_j},
  \label{eq:internal_energy}
\end{align}
where $\vec r_i$ is the position of the $i$-th sphere. As the
potential energy $E$ is the primary form of energy concerning us in
this paper, we will refer to $E$ as simply the ``energy'' of the
fluid. An important feature of the square-well fluid is that its
energy is always an integer multiple of the well depth. A homogeneous
square-well fluid is thus uniquely identified by its well width
$\lambda$ and filling fraction $\eta$ (i.e. the proportion of space
filled by spheres; a dimensionless density), as all other properties
can be normalized to the natural energy scale $\epsilon$ and length
scale $\sigma$. In practice, our simulation codes use dimensionless
energies $E/\epsilon$, temperatures $kT/\epsilon$, and distances
$r/\sigma$.

\subsection{Monte Carlo fluid simulations}
\label{sec:mc_sim}

While model systems are powerful tools for understanding complex
physical systems, they do not themselves exist in the real
world. Consequently, direct experimental tests of theories for model
system (e.g. square-well density functional theories) are not
possible. For this reason, model system theories are commonly tested
against Monte Carlo simulations. Proper implementation of Monte Carlo
methods to study completely characterized systems ensures that
statistical results from simulations converge on the exact properties
of the simulated system in the infinite simulation time
limit. Furthermore, uncertainties in quantities computed via Monte
Carlo simulations are typically well-defined, monotonically decreasing
functions of simulation time, allowing one to run simulations to the
desired level of accuracy.

\subsubsection{Implementation}
\label{sec:mc_implementation}

\begin{algorithm}[tb]
  \caption{Metropolis Monte Carlo fluid simulation}
  \label{alg:metropolis}
  \begin{alg}

  \item Construct an initial ``typical'' (i.e. non-ordered) fluid
    configuration. \fixme{explain how}

  \item Randomly attempt to change the position of one sphere (in
    general, a single fluid ``atom'' or ``molecule'') to another
    location, rejecting the change if it results in a forbidden fluid
    configuration (e.g. two or more spheres overlap) and accepting the
    change otherwise. Attempting to move a single sphere is referred
    to as a move. \label{alg:metropolis_move}

  \item Repeat step \ref{alg:metropolis_move} for every other sphere
    in the fluid. Attempting to move every sphere once is referred to
    as an iteration of the simulation.
    \label{alg:metropolis_iteration}

  \item Repeat step \ref{alg:metropolis_iteration} indefinitely, or
    until data of sufficient quality has been generated, periodically
    collecting and dumping statistics on fluid states (e.g. energy,
    pair distribution histograms, etc.) to data files.

  \end{alg}
\end{algorithm}

Algorithm \ref{alg:metropolis} provides a sketch of unbiased,
Metropolis Monte Carlo fluid simulations. Such an algorithm is
``unbiased'' in the sense that it collects statistics (i.e. data) on
all valid system configurations with equal probability. Statistics
whose collection time scales as $\mathcal O\p{1}$ with system size,
meaning that increasing the number of spheres $N$ in the simulation
does not affect collection time, can be collected after every move
(defined in the algorithm). Statistics whose collection time scales as
$\mathcal O\p{N}$, meaning that doubling $N$ doubles collection time,
can be collected after each iteration. In general, collection with
$\mathcal O\sp{\chi\p{N}}$ time scaling should not occur more often
than once every $\chi\p{N}$ moves, where $\chi\p{N}$ may be an
arbitrary function of $N$ void of constant prefactors, e.g. $N\log N$,
or $2^N$. These collection rules ensure that scaling up simulations
does not cause them to asymptotically spend all computation time only
collecting statistics, or only simulating the fluid. Collected
statistics are used to find thermodynamic properties of the simulated
fluid.

In this work, we are concerned with simulating the \emph{homogeneous}
square-well fluid. To avoid edge effects resulting from fluid behavior
near a wall, we employ periodic boundary conditions. The use of a
finite cell with periodic boundary conditions suppresses all density
fluctuations on scales larger than the dimensions of the simulated
fluid cell, thereby introducing a source of error. Addressing and
sequestering this error, however, is outside the scope of this work,
and involves considering the limit of numerical results as the number
of spheres $N\to\infty$ (keeping the filling fraction $\eta$
constant).

\subsubsection{Computing observables}
\label{sec:computing_observables}

The sort of Monte Carlo simulations described above are only capable
of collecting statistics on functions of system microstates, and
finding correlations between these functions. For example, in
simulation one might periodically compute both the energy $E\p{s}$ and
some system property $X\p{s}$, both of which are determined by the
microstate $s$, in order to find the mean value of $X$ at any given
energy $E$, i.e. $\bk{X}_E$. In the real world, however, information
about a system's microstates, and by extension information about
functions of microstates (e.g. the energy $E$), is inaccessible. One
therefore cannot measure $\bk{X}_E$ directly. Instead, one typically
measures the dependence of thermodynamic properties on macroscopic
state variables such as temperature, i.e. $\bk{X}_T$.

To find $\bk{X}_T$, we first need to understand the concept of a
density of states. Given an arbitrary system property $Y\p{s}$
(e.g. energy) determined by the microstate $s$, one can define a
density of states $D\p{Y}$ such that
\begin{enumerate*}[label=\roman*)]
\item for arbitrary $Y_0$, the value of $D\p{Y_0}$ is proportional to
  the number of microstates $s$ for which $Y\p{s}=Y_0$, and
\item $\sum_YD\p{Y}=1$.
\end{enumerate*}
The units of $D\p{Y}$ are inverse to the units of $Y$. The temperature
dependence of $\bk{X}_T$ can be expressed in terms of the expectation
value $\bk{X}_E$ and the density of states $D\p{E}$ by
\begin{align}
  \bk{X}_T=\f1{Z\p{T}}\sum_E\bk{X}_ED\p{E}e^{-E/kT},
  \label{eq:XT_norm}
\end{align}
where the partition function $Z\p{T}$ is simply a normalization
factor, given by
\begin{align}
  Z\p{T}=\sum_ED\p{E}e^{-E/kT}.
\end{align}
To reduce redundant computations and numerical error in
implementation, we will generally use a partition function with an
unnormalized density of states $\tilde D\p{E}$, i.e.
\begin{align}
  \tilde Z\p{T}=\sum_E\tilde D\p{E}e^{-E/kT},
\end{align}
in terms of which
\begin{align}
  \bk{X}_T=\f1{\tilde Z\p{T}}\sum_E\bk{X}_E\tilde D\p{E}e^{-E/kT}.
  \label{eq:XT}
\end{align}

\subsection{Histogram methods}
\label{sec:histogram_methods}

Due to the fact that Metropolis Monte Carlo simulations sample all of
state space randomly and without preference, a histogram $H\p{X}$ of
observations of some system property $X$ is directly proportional to
the density of states $D\p{X}$; that is, the number of observations
$H\p{X_0}$ of the system with $X\p{s}=X_0$ is proportional to the
total number of states $s$ for which $X\p{s}=X_0$. It is sometimes the
case, however, that the density of states in some region $R$ in the
range of possible $X$ is so low that it is practically impossible (via
Metropolis Monte Carlo) to sufficiently sample $R$, that is, to
accumulate a statistically significant histogram $H\p{X\in R}$, in any
reasonable amount of time. Given that the entropy $S\p{X}$ can be
expressed in terms of the number of microstates $\Omega\p{X}$ as
$S\p{X}=k\ln\Omega\p{X}\propto\ln D\p{X}$, we may refer to regions $R$
low state densities $D\p{X\in R}$ as low entropy states.

\subsubsection{Biased sampling}
\label{sec:biased_sampling}

Histogram methods provide a means to address Metropolis Monte Carlo's
inability to sufficiently sample low entropy states by introducing a
bias into the otherwise random sampling of state space. A weighting
function $w$ is introduced, whose domain is the value of some property
$X\p{s}$ which depends on the system state $s$. An additional
condition is then added to step \ref{alg:metropolis_move} of Algorithm
\ref{alg:metropolis} in order to accept an attempted move: the weights
$w\sp{X\p{s_i}}$ and $w\sp{X\p{s_f}}$ of the initial (pre-move) state
$s_i$ and final (post-move) state $s_f$ are used to determine the
probability $P_m\p{s_i\to s_f}$ of accepting an otherwise valid move
via
\begin{align}
  P_m\p{s_i\to s_f}=\min\set{\f{w\sp{X\p{s_f}}}{w\sp{X\p{s_i}}},1}.
  \label{eq:move_prob}
\end{align}
This formula means that when $w\sp{X\p{s_f}}>w\sp{X\p{s_i}}$, the move
$s_i\to s_f$ is accepted; when $w\sp{X\p{s_f}}<w\sp{X\p{s_i}}$, the
ratio of these weights determines the probability of accepting the
move. Due to the fact that only ratios of weights determine
$P_m\p{s_i\to s_f}$, the weights $w\sp{X\p{s}}$ are scale-invariant,
meaning that their effect on simulations is unchanged by scale factors
that are constant with respect to $X\p{s}$.

\begin{algorithm}[tb]
  \caption{Biased Monte Carlo fluid simulation}
  \label{alg:biased_MC}
  \begin{alg}

  \item Construct an appropriate weight function $w\sp{X\p{s}}$, whose
    argument is a system property $X\p{s}$ determined by the
    microstate $s$.

  \item Construct an initial typical fluid configuration.

  \item Randomly attempt to change the position of one sphere,
    rejecting the change if
    \begin{enumerate*}[label=\roman*)]
    \item it results in a forbidden fluid configuration, or
    \item a newly chosen random number on the interval $\sp{0,1}$ is
      larger than the probability determined by (\ref{eq:move_prob})
      for the initial and final states $s_i$ and $s_f$, respectively.
    \end{enumerate*}
    \label{alg:biased_mc_move}

  \item Repeat step \ref{alg:biased_mc_move} the simulation produces
    statistics of sufficient quality.

  \end{alg}
\end{algorithm}

Employing biased Monte Carlo simulations, sketched out in Algorithm
\ref{alg:biased_MC}, allows one to construct weight functions that
favor some region of state space over others, as transitions to states
with higher weights are always accepted, whereas transitions to states
with lower weights may be rejected, artificially preventing the
simulated system from leaving interesting regions of state
space. Crucially, the bias introduced by weights can be reversed when
computing system properties from sampling statistics, as the weight of
a particular state is directly proportional to the probability bias of
that state; that is, a state with a weight of 2 will be sampled twice
as often as it would have been with a weight of 1. If we wish to
convert a histogram $H\p{X}$ of observations in a biased Monte Carlo
simulation into a numerical unnormalized density of states $\tilde
D\p{X}$ in $X$, we therefore divide the histogram by the corresponding
weights $w\p{X}$, i.e.
\begin{align}
  \tilde D\p{X}=\f{H\p{X}}{w\p{X}}.
  \label{eq:dos}
\end{align}
The normalized density of states $D\p{X}$ in $X$ is then
\begin{align}
  D\p{X}=\f{\tilde D\p{X}}{\sum_X\tilde D\p{X}}
  =\f{H\p{X}/w\p{X}}{\sum_XH\p{X}/w\p{X}}.
  \label{eq:dos_norm}
\end{align}
In general, the domain and shape of the weight function will depend on
the desired yields (e.g.  density of states, heat capacity) of a
simulation. A histogram method is simply an algorithm or procedure for
determining a weight function appropriate for a particular simulation.

In this paper, we will consider weights $w\p{E}$ which depend only on
the energy $E$ of the square-well fluid. Due to the fact that the
square-well fluid can only have discrete energies $E=-n\epsilon$,
where $n$ is a non-negative integer and $\epsilon$ is the well depth,
in simulation we store the weight function as an array of values. We
may therefore use the terms ``weight function,'' and ``weight array,''
and ``weights'' interchangeably.

\fixme{mention that simulating with flat weights $w\p{E}=w_0$ is the
  same as running a Metropolis MC simulation.}

\subsubsection{Canonical weights}
\label{sec:canonical_weights}

The most common weight array $w\p{E}$ used by physicists for what is
called a canonical Monte Carlo simulation involves choosing a
particular temperature $T_0$, and using weights proportional to the
Boltzmann factor at that temperature, i.e.
\begin{align}
  w\p{E}=e^{-E/kT_0},
\end{align}
where there is no reason to normalize $w\p{E}$ due to the fact that
only ratios of weights, as per (\ref{eq:move_prob}), are ever used in
simulation. The fact that only ratios of weights are used in
simulation makes $w\p{E}$ scale-invariant.

The partition function at $T=T_0$ for simulations with canonical
weights is
\begin{align}
  \tilde Z\p{T_0}=\sum_E\tilde D\p{E}e^{-E/kT_0}
  =\sum_E\f{H\p{E}}{w\p{E}}~e^{-E/kT_0}
  =\sum_E\f{H\p{E}}{e^{-E/kT_0}}~e^{-E/kT_0}=\sum_EH\p{E},
  \label{Z_canonical}
\end{align}
and the value of a thermodynamic property $X\p{T_0}$
\begin{align}
  X\p{T_0}=\f1{\tilde Z\p{T_0}}\sum_E\bk{X}_E\tilde D\p{E}e^{-E/kT_0}
  =\f{\sum_E\bk{X}_EH\p{E}}{\sum_EH\p{E}}.
  \label{eq:X_canonical}
\end{align}
The simplifications in in (\ref{Z_canonical}) and
(\ref{eq:X_canonical}), which have no explicit dependence on $T_0$,
occur because canonical Monte Carlo simulations sample energies in
proportion to the distribution (over energy) of microstates at a
temperature of $T_0$.  Canonical Monte Carlo simulations can therefore
fail to sufficiently sample energies which are important
(i.e. energies with a non-negligible state probability density) at
different temperatures. As a consequence, such simulations should not
be used to determine properties $X\p{T\ne T_0}$, and are thus referred
to as ``fixed temperature'' simulations.

Though canonical Monte Carlo is simple to implement, its inability to
investigate a system at more than one temperature at a time is a
disadvantage for determining the temperature dependence of system
properties. In order to find the behavior of $\bk{X}_T$, one must run
many simulations at discrete temperature intervals; each such
simulation will yield one sample (e.g. one data point) of $\bk{X}_T$.

Sampling low energies, however, or understanding system behavior at
low temperatures, is even more problematic with canonical
weights. Using low temperature canonical weights will indeed force a
simulated system down to low energies, but will also likely freeze the
system into a local minimum of its energy landscape. Freezing into a
state means that a simulation will sample only a small portion of the
energy landscape, even though there may (and generally will) be many
other states with the same energy.

Due to these problems, we will not use canonical weights alone to
study the square-well fluid. We will, however, use canonical weights
for part of all weight arrays, as discussed in Section
\ref{sec:min_energy}.

\subsubsection{Broad energy sampling}
\label{sec:broad_energy_sampling}

Given the expression for $\bk{X}_T$ in (\ref{eq:XT}), sufficient
accumulation of statistics on $\bk{X}_E$ at all available energies in
principle allows one to determine $\bk{X}_T$ for any temperature
$T$. In practice, the density of states can fall off so quickly with
energy that some range of allowable energies is practically
inaccessible via Monte Carlo simulations, biased or otherwise. In such
a case, computing $\bk{X}_T$ to a reasonable degree of accuracy
requires sufficiently sampling the energies at which $\tilde
D\p{E}e^{-E/kT}$ dominates the sum in (\ref{eq:XT}). Section
\ref{sec:energy_range} provides more discussion on this subject,
namely on identifying the energies which are important for computing
system properties at a given temperature.

We will employ histogram methods in order to sample as broad of an
energy range as possible in each simulation. Broad energy sampling
will allow us to determine, for various square-well fluids,
\begin{enumerate*}[label=\roman*)]
\item the density of states $D\p{E}$, and
\item the temperature dependence $\bk{X}_T$ of various thermodynamic
  properties $X$, particularly at temperatures near the liquid-gas
  phase boundary.
\end{enumerate*}


%%%%%%%%%%%%%%%%%%%%%%%%%%%%%%%%%%%%%%%%%%%%%%%%%%%%%%%%%%%%%%%%%%%%%%
\section{Methods}
\label{sec:methods}

\fixme{Give an overview of the methods section}

\fixme{Make a note on storing the logarithm of weights and
  periodically rescaling them (resetting the zero on $\ln w$) to
  reduce numerical error.}

\subsection{The flat histogram (multi-canonical) method}
\label{sec:flat_histogram}

The flat histogram method, also called the multi-canonical method,
assumes complete knowledge of the density of states $D\p{E}$ of the
system in question, and solves (\ref{eq:dos}) for a weight array
$w\p{E}$ which should yield a flat energy histogram $H\p E=H_0$ to get
\begin{align}
  w\p E=\f1{D\p E},
  \label{eq:flat_weights}
\end{align}
where we may neglect the constant scale factor $H_0$ and do not worry
about normalization factor distinguishing $D\p{E}$ from $\tilde
D\p{E}$ due to the scale-invariance of $w\p{E}$.

For all but the most trivial or well-studied systems, however, the
density of states $D\p{E}$ is not known prior to simulation, and is in
fact one of the system properties which a simulation is intended to
determine. Nonetheless, the this method has pedagogical value by
enabling discussion of an ``optimal'' weight array for generating a
flat energy histogram in simulations. All algorithms to construct
$w\p{E}$ with the aim of producing a flat energy histogram should
converge on the same weight array, given by (\ref{eq:flat_weights}).

\begin{algorithm}[b]
  \caption{A naive flat histogram method}
  \label{alg:flat_histogram}
  \begin{alg}

  \item Construct an initial typical fluid configuration.

  \item Simulate (without weights) for some time $\tau$, which may be
    real time, computer time, or a number of simulation iterations,
    collecting a histogram $H\p{E}$ of energy observations after every
    move.

  \item Set weights $w\p{E}=1/H\p{E}$. At energies with $H\p{E}=0$,
    set $w\p{E}=1$.

  \end{alg}
\end{algorithm}

Algorithm \ref{alg:flat_histogram} provides a naive way to implement
the flat histogram method without knowing a priori the density of
states $D\p{E}$. This algorithm involves first running a simulation
without weights (or with constant weights $w\p{E}=w_0$) while
collecting a histogram $H\p{E}$ of the observed energies $E$. After
some time (loosely defined), the histogram $H\p{E}$ may be taken as an
approximation of the density of states $\tilde D\p{E}$ (see Section
\ref{sec:histogram_methods}), and used via (\ref{eq:flat_weights}) to
generate a weight array for an actual simulation. Algorithm
\ref{alg:flat_histogram} has one free parameter, which determines the
amount of time for which to collect an energy histogram $H\p{E}$.

The problem with this implementation of the flat histogram method is
that histogram methods are generally necessary when some range of
energies $R$ is inaccessible via Metropolis Monte Carlo
simulations. In such a case, the energy histogram $H\p{E\in R}$ after
simulating for any reasonable amount of time will be statistically
insignificant or null, invalidating the approximation
$H\p{E}\approx\tilde D\p{E}$. This method will therefore only
``flatten'' the histogram at energies $E$ which have been sufficiently
sampled for $H\p{E}$ to be statistically significant. The end goal in
implementing histogram methods, however, is precisely to sample those
energies that cannot be sufficiently sampled without employing clever
schemes to bias Monte Carlo simulations. The flat histogram method
therefore does not directly help us study the square-well fluid. This
method does, however, motivate the method in Section
\ref{sec:simple_flat}, and provides some of the theory behind the
methods in Sections \ref{sec:tmmc} and \ref{sec:oetmmc}.

\subsection{The simple flat method}
\label{sec:simple_flat}

The ``simple flat'' method, developed by ourselves, extends the
implementation of the flat histogram method discussed in Section
\ref{sec:flat_histogram} to an algorithm intended to be the simplest
viable broad energy histogram method. While we do not expect this
method to outperform other, more sophisticated methods, the simple
flat method should provide a lower bar and a standard of comparison
for the performance of other methods. Any nontrivial histogram method
should, at a minimum, consistently outperform the simple flat method
in order to be considered for any applications.

The idea behind the simple flat method is such: after simulating for
some time, the energy histogram $H\p{E}$ will have a peak, or some
other uneven distribution of observations, in a region of energies $R$
for which $H\p{E\in R}$ statistically significant. One can thus use
the existing histogram to construct weights which should flatten
$H\p{R}$. Subsequent simulations should then spend less time at
energies with the highest state densities, and more time at energies
with relatively low state densities. The energy histogram from such
simulations should therefore be statistically significant in a larger
range of energies, improving the available estimate for a density of
states with which to compute flat histogram weights. One could then
use the improved estimate to construct new weights, and repeat this
simulation and updating process until simulations satisfy some end
condition (in fact, most histogram methods will require end
conditions, which is discussed in Section \ref{sec:end_conditions}).

\begin{algorithm}[!t]
  \caption{The simple flat method}
  \label{alg:simple_flat}
  \begin{alg}

  \item Construct an initial typical fluid configuration.

  \item Set $k=0$, choose an initial number of iterations $n$ for
    which to simulate, and initialize weights $w\p{E}=1$ for all $E$.

  \item Simulate for $n$ iterations (with weights), collecting a
    histogram $H\p{E}$ of energy observations after every move.
    \label{alg:simple_flat_sim}

  \item If some predetermined initialization end condition has
    \emph{not} been met, then
    \begin{enumerate}
    \item increment $k\leftarrow k+1$
    \item set $n_k=un_{k-1}$ with a predetermined factor $u\ge1$,
      \label{alg:simple_flat_f}
    \item update weights as $w\p{E}\leftarrow w\p{E}/H\p{E}$ for all
      $E$ at which $H\p{E}\ne0$,
      \label{alg:simple_flat_weight_update}
    \item return to step \ref{alg:simple_flat_sim}.
    \end{enumerate}

  \end{alg}
\end{algorithm}

Algorithm \ref{alg:simple_flat} provides an implementation of the
simple flat method. This algorithm has two free parameters: the
initial number of iterations $n_0$ for which to simulate and the
factor $u$. Due to the exponential growth of $n_k=u^kn_0$, this
algorithm should not be sensitive to the value of $n_0$.

The most interesting part of Algorithm \ref{alg:simple_flat} is step
\ref{alg:simple_flat_weight_update}, which updates the weights as
$w\p{E}\leftarrow w\p{E}/H\p{E}$. The reasoning behind this step is
such: say that for two energies $E_0$ and $E_b$ we have that
$H\p{E_0}=\bk{H}_E$ and $H\p{E_b}=b\bk{H}_E$, where $\bk{H}_E$ is the
mean histogram value over all energies. In this case, we wish to
decrease the current simulation bias on $E_b$ by a factor of $b$ while
keeping the bias on $E_0$ the same. As the weights $w\p{E}$ are
proportional to the simulation bias on each respective energy $E$,
dividing the weights $w\p{E}$ by the current histogram $H\p{E}$
achieves the desired changes to the simulation biases. The factor
$\bk{H}_E$ in this weight update has no effect on the function of the
weights, and is in practice immediately scaled out of the weight
array.

In our implementation of the simple flat method, we used $u=2$ and
$n_0=L\exp\p{\epsilon/kT_{\t{min}}}$, where $L$ is the number of
energy levels of the given square-well fluid and $T_{\t{min}}$ is a
minimum temperature of interest for the current simulation. The factor
of $L$ appears because initialization time should scale with the range
of energies of the simulated system. If we have twice the number of
energies, we should explore energy space for twice as long. To
estimate $L$, we multiplied the maximum number of spheres $M$ which
fit within a radius of $\lambda\sigma$ of a single sphere on a
face-centered cubic lattice (i.e. the maximum number of spheres with
which a given sphere could interact) by the total number of spheres
$N$, and divided the result by two so as to not double-count
interactions between spheres (so $L=MN/2$). The exponential
$\exp\p{\epsilon/kT_{\t{min}}}$, a Boltzmann factor for two adjacent
energies, appears because lower minimum temperatures of interest
warrant sampling lower energies, which in turn requires simulating for
longer (in step \ref{alg:simple_flat_sim} of Algorithm
\ref{alg:simple_flat}) because these energies have lower state
densities. The introduction of a minimum temperature may seem ad-hoc,
but all simulations in this paper require a choice of minimum
temperature anyways (discussed in Section \ref{sec:energy_range}). In
this paper, we will always use $kT_{\t{min}}=0.2\epsilon$.

\subsection{The Wang-Landau (WL) method}
\label{sec:wang_landau}

\fixme{find more sources to cite for some of the claims made in this
  section}

The Wang-Landau method is the first published histogram method we
discuss\cite{wang_landau, wang_landau_mod, wang_landau_analysis}. This
method has been used to study a wide variety of systems\fixme{which
  systems? cite}, and is the standard method\fixme{cite} to achieve
broad energy sampling via Monte Carlo simulations. \fixme{more intro
  on WL}

Unlike the simple flat method, the Wang-Landau method modifies the
weight array on the fly. After every move during initialization
Wang-Landau decreases the weight $w\p{E}$ on the current energy $E$ by
some factor of $f$, so as to decrease weights on the most commonly
observed energies relative to those of the least commonly observed
energies. Eventually, such a simulation with constant modification of
the weights $w\p{E}$ should yield a flat histogram
$H\p{E}\approx\bk{H}_E$ for all $E$, as the modifications to $w\p{E}$
increasingly pushes the simulated system away from the most sampled
energies, and toward the least sampled energies. When the energy
histogram is sufficiently flat, $H\p{E}$ is reset (i.e. set to
$H\p{E}=0$ for all $E$), the factor $f$ is decreased (geometrically
approaching unity), and the entire process is repeated until $f-1$
falls below some cutoff $c\ll 1$.

Resetting and repeating this initialization process is necessary to
perform fine tuning and refinement of the weights. A flat histogram
after a simulation in which $w\p{E}$ has changed does not guarantee
that simulating with the final weights will likewise result in a flat
histogram. Eventually, when $f-1<c\ll 1$, i.e. $f\approx 1$, the
initialization process no longer makes any appreciable modifications
to the weights $w\p{E}$, so that upon achieving a flat histogram one
can be confident that a regular simulation which fixes the current
weights will yield similar results.

\begin{algorithm}[!b]
  \caption{Wang-Landau initialization of weights}
  \label{alg:wang_landau}
  \begin{alg}

  \item Construct an initial typical fluid configuration

  \item Set $k=0$, choose some factor $f_0>1$, and initialize
    $w\p{E}=1$ for all $E$.

  \item Simulate for $n$ iterations. After each move, increment the
    histogram $H\p{E}$ of energy observations and update
    $w\p{E}\leftarrow w\p{E}/f_k$, where $E$ is the current energy of
    the system.
    \label{alg:wang_landau_sim}

  \item If the histogram $H\p{E}$ is not sufficiently flat, i.e. if it
    fails a ``flatness'' condition $C\sp{H}$, return to step
    \ref{alg:wang_landau_sim}, otherwise
    \begin{enumerate}
    \item reset $H\p{E}=0$ for all $E$,
    \item increment $k\leftarrow k+1$,
    \item set $f_k=\sqrt[u]{f_{k-1}}$ for some predetermined $u>1$,
      \label{alg:wang_landau_update}
    \item if $f_k\ge c$ for some predetermined $c$, return to step
      \ref{alg:wang_landau_sim}.
    \end{enumerate}

  \end{alg}
\end{algorithm}

Algorithm \ref{alg:wang_landau} spells out the Wang-Landau
initialization process, which has five free parameters: the number of
iterations $n$ to run in step \ref{alg:wang_landau_sim}, the initial
factor $f_0$, the update factor $u$, the flatness condition $C\sp{H}$,
and the cutoff $c$. The first of these parameters, $n$, controls how
often to check the flatness condition $C\sp{H}$ and run updates, if
necessary. The value of $n$ is not too important, so long as it is
reasonable for the system at hand. Too small value of $n$ will waste
computation time determining $C\sp{H}$, while too large of a value
will waste time simulating when $C\sp{H}$ has already been
satisfied. In general, $n$ should be proportional to the computation
time of $C\sp{H}$, which scales with energy range of the simulated
system. We therefore use $n=L$, where $L$ is, as described in Section
\ref{sec:simple_flat}, the number of energy levels for the given
square-well fluid. Other works typically neglect providing any
heuristic or formula for $n$, reporting a system-independent
$n=10^3$\cite{wang_landau_mod} or $10^5$\cite{wang_landau}.
\fixme{find more citations for this last claim}

The next parameter, $f_0$, controls the initial amount by which to
adjust the weight of the current energy after each move. As mentioned
in Algorithm \ref{alg:wang_landau}, one should always have $f_0>1$,
but the appropriate value of $f_0$ will generally depend on the system
at hand. Too large a value of $f_0$ will magnify stochastic error in
initialization, while too small a value of $f_0$ will cause
simulations to take an exceedingly long time to converge. Lacking any
heuristics for assigning $f_0$, literature typically suggests $f_0=e$
\cite{wang_landau}. After some experimentation with the free
parameters in the Wang-Landau method, we decided on $f_0=e^{1/8}$. In
practice, one usually works with $\ln f$, making the commonly
suggested value $\ln f_0=1$ and our own $\ln f_0=1/8$.

The factor $u$ appearing in step \ref{alg:wang_landau_update} of
Algorithm \ref{alg:wang_landau} controls the amount by which to adjust
$f$ when the histogram $H\p{E}$ is sufficiently flat. In principle,
one need not take a root of $f$ at each reset: the general idea behind
Algorithm \ref{alg:wang_landau} requires only $f_k<f_{k-1}$ to
converge (assuming that $C\sp{H}$ can be satisfied) and
$\lim_{k\to\infty}f_k=1$ to terminate. Taking the $u$-th root of $f$
at each increment of $k$ is a natural and convenient iterative means
to satisfy these requirements. Previous works have reported using
values such as $u=2$ \cite{wang_landau, wang_landau_mod} and $u=10$
\cite{wang_landau_analysis}; we use the former, conservatively
favoring more initialization iterations by decreasing $f$ by a smaller
amount on each reset.

One might guess why the flatness condition $C\sp{H}$ for updating $f$
and restarting the Wang-Landau initialization process should not be
made too lax: doing so would prompt the initialization process to
\begin{enumerate*}[label=\roman*)]
\item reset and decrease $f$ too early, i.e. when the weight array
  could still use some heavy handed adjustment, and
\item quit before one has any reason to think that simulations might
  yield a flat histogram.
\end{enumerate*}
Less obvious, however, is the fact that the flatness condition
$C\sp{H}$ should not be made too stringent: if the histogram is
completely flat, say $H\p{E}=H_0$ for all $E$, then the weights
$w\p{E}$ have all been modified by the same factor of $f^{H_0}$.
Remembering the scale-invariance of $w\p{E}$, this result would mean
that the weights have not actually changed at all!

The success of Wang-Landau thus relies on some degree of leniency in
the flatness condition, which makes tuning $C\sp{H}$ crucial to the
efficient and effective implementation of this method. Some previous
works\cite{wang_landau} suggest that $C\sp{H}$ should be made as
stringent as possible, cautioning only that some simulations might not
satisfy too strict of a flatness condition in any reasonable amount of
time. These works fail to recognize the reliance of Wang-Landau on the
{\it failure} of $C\sp{H}$ to enforce a perfectly flat histogram. A
commonly suggested flatness condition is that the minimum histogram
value be at least some fixed proportion of the mean histogram value,
i.e. $C\sp{H}:\min\sp{H\p{E}}\ge x\bk{H}_E$, and several papers report
using $x=0.95$. We used the same condition with $x=0.25$.

The final free parameter in the Wang-Landau method is the cutoff $c$,
which determines how small $f$ can get before the algorithm stops
modifying the weight array. Too large of a value for $c$ will make
Wang-Landau quit prematurely, whereas too small a value will cause the
algorithm to waste time simulating after the weights $w\p{E}$ have
essentially converged. One might imagine using an alternate end
condition for Wang-Landau, which involves comparing the current
weights $w_k\p{E}$ to those at the end of the previous cycle,
$w_{k-1}\p{E}$, to check whether the initialization process is still
making appreciable modifications to the weights, but such a check
would require both a metric and a free parameter to define what an
appreciable change is. For simplicity, we stuck with the standard end
condition given in Algorithm \ref{alg:wang_landau}. As with the other
free parameters in Wang-Landau, we are not aware of any heuristics for
determining an appropriate value of $c$, but literature cites typical
values of $\ln c=10^{-8}$; we used $\ln c=10^{-6}$.

\subsection{The transition matrix Monte Carlo method}
\label{sec:tmmc}

\fixme{add citations}

Unlike the previous histogram methods, the transition matrix Monte
Carlo (TMMC) method does not use weights, and merely compute the
density of states $D\p{E}$ of a system\cite{tmmc}. The density of
states can in turn be used to determine flat histogram weights via
(\ref{eq:flat_weights}). TMMC introduces a new object: the energy
transition matrix, $T$, whose components $T_{ij}$ are the
probabilities that a system will transition {\it to} a state with
energy $E_i$ {\it from} a given state with energy $E_j$,
i.e. $T_{ij}=P\p{E_j\to E_i}$.

The transition matrix is trivially square and positive-definite, but
it is not symmetric. Given that Metropolis Monte Carlo samples all of
state-space randomly, meaning all possible system states are equally
likely to occur during simulation, a transition $E_i\to E_f$ from an
energy with a low state density to one with a high state density,
$D\p{E_i}<D\p{E_j}$, is more probable than the inverse transition
$E_j\to E_i$ simply due to the fact that there are more states with
energy $E_i$ than those with energy $E_j$. Thus $D\p{E_i}<D\p{E_j}$
implies $P\p{E_i\to E_j}>P\p{E_j\to E_i}$, which means
$T_{ji}>T_{ij}$. The asymmetry of the transition matrix is precisely
the origin of the second law of thermodynamics: systems are
exceedingly likely to evolve towards macroscopic states with higher
accessible microstate densities.

Knowing the transition matrix of a system determines, among other
properties, the system's density of states. Consider an ensemble of
Metropolis Monte Carlo simulations with a distribution of energies at
a time $t$ given by $\hat D^{t}\p{E}$, such that the probability of
any given simulation to have an energy $E_i$ at a time $t$ is $\hat
D_i^{t}=\hat D^{t}\p{E_i}$. The probability $\hat D_i^{t}$ can be
expressed in terms of the distribution $\hat D^{t-1}\p{E}$ one time
step (i.e. one Metropolis Monte Carlo move) prior by
\begin{align}
  \hat D_i^{t}=\sum_jP\p{E_i\to E_j}\hat D_j^{t-1} =\sum_j T_{ij}\hat
  D_j^{t-1}.
  \label{eq:transition_evolution}
\end{align}
At equilibrium, the probability distributions $\hat D^{t}=\hat
D^{t-1}$. Furthermore, the distribution of states in an equilibrium
ensemble of Metropolis Monte Carlo simulations is precisely the
density of states $D\p{E}$\fixme{explain why?}, which means
$D_i=T_{ij}D_j$. Simultaneously expressing this condition for all
energies,
\begin{align}
  D=TD. \label{eq:dos_eigen}
\end{align}
Finding the density of states of a system can thus be reduced to
finding the eigenvector of the transition matrix which has a
corresponding unit eigenvalue\fixme{why is this eigenvector
  unique?}. Once one has determined the density of states of a system,
one can initialize weights via the flat histogram method,
i.e. (\ref{eq:flat_weights}).

Computationally determining the transition matrix of a system is
straightforward: one need only simulate the system and collect a
histogram $\tilde T_{ij}=\tilde T\p{E_i,E_j}$ of the energy
transitions after each Metropolis Monte Carlo move, i.e. after step
\ref{alg:metropolis_move} of Algorithm \ref{alg:metropolis}. A
histogram of transitions $\tilde T\p{E_i,\bar E_j}_{}$ from a fixed
energy $\bar E_j$ to energies $E_i$ is then proportional to the
probability distribution of transitions from $\bar E_j$, i.e. the
transition matrix $T\p{E_i,\bar E_j}$. Crucially, this proportionality
does not depend on the history of a simulation, or on how the
simulation got to be in the energy $\bar E_j$, so long as samples of
the energy $\bar E_j$ are themselves void of systemic
bias. Furthermore, one does not actually need to transition from $E_j$
to $E_i$ in order to collect statistics on $T\p{E_i,E_j}$: after
deciding whether to accept a move based on whether it results in a
valid system state, one can increment $\tilde T\p{E_i,E_j}$ before
deciding to reject the move for some other reason, e.g. because
$w\p{E_j}\gg w\p{E_i}$, or even because one wishes to collect more
statistics on $T\p{E_i,\bar E_j}$.

The proportionality $\tilde T_{ij}\propto T_{ij}$ can be made explicit
by enforcing that the probabilities of all transitions from an energy
$E_j$ sum to unity, i.e.
\begin{align}
  \sum_iP\p{E_j\to E_i}=\sum_iT_{ij}=1,
  \label{eq:transition_norm_condition}
\end{align}
which implies
\begin{align}
  T_{ij}=\f{\tilde T_{ij}}{\sum_i\tilde T_{ij}}.
  \label{eq:transition_normalization}
\end{align}
To compute $T$ from $\tilde T$, one must therefore normalize each row
of $\tilde T$ independently.

If a system has $L$ energy levels, the transition histogram $\tilde
T\p{E_i,E_j}$ is an $L\times L$ matrix. Given that $L$ is proportional
to the system size $N$, the memory footprint of $\tilde T\p{E_i,E_j}$
grows as $N^2$. For most systems, however, there will be a maximum
energy $M$ independent of system size by which a single move can
change the energy of the system. The transition histogram for such
systems is therefore band-diagonal, which one can exploit to save
memory. In particular, it is natural to store the transition histogram
in the form $\tilde T_d\p{E,\Delta E}$, which gives the probability
that a system will transition from the energy $E$ to the energy
$E+\Delta E$ in a single move. One can convert between $\tilde
T_d\p{E,\Delta E}$ and $\tilde T\p{E_i,E_j}$ using the relation
\begin{align}
  \tilde T\p{E_i,E_j}=\tilde T_d\p{E_j,E_i-E_i},
\end{align}
which means, from (\ref{eq:transition_normalization}),
\begin{align}
  T_{ij}=\f{\tilde T_d\p{E_j,E_i-E_j}}{\sum_i\tilde T_d\p{E_j,E_i-E_j}}
  =\f{\tilde T_d\p{E_j,E_i-E_j}}
  {\sum_{\Delta E}\tilde T_d\p{E_j,\Delta E}}.
  \label{eq:transition_conversion}
\end{align}
Expressed in this form, the memory footprint of the transition
histogram grows linearly with system size $N$ and the maximal energy
difference $M$.

While one could imagine many initialization routines to determine the
transition matrix $T$, we use Algorithm \ref{alg:tmmc}, which is a
modified version of a routine provided in \cite{tmmc_prl}\fixme{is
  this citation correct?}. The primary motivation behind this
algorithm is such: as lower energies are always more difficult to
sample than higher energies, transitions to lower energies should
always be accepted, while transitions to higher energies should pass a
probabilistic acceptance test. To understand the probability in step
\ref{alg:tmmc_prob} of Algorithm \ref{alg:tmmc} of accepting a move
$s_i\to s_f$ for which $E_f>E_i$, first consider the case of $c=0$ and
$T_{\t{min}}\to0$, in which case the probability becomes
\begin{align}
  P_m\p{s_i\to s_f}=\f{\p{\tilde T_d\p{E_i,\Delta E}}\big/
    \p{\sum_{\Delta E'}\tilde T_d\p{E_i,\Delta E'}}}
  {\p{\tilde T_d\p{E_f,-\Delta E}}\big/
    \p{\sum_{\Delta E'}\tilde T_d\p{E_f,\Delta E'}}}
  =\f{T\p{E_i,\Delta E}}{T\p{E_f,-\Delta E}}
  =\f{P\p{E_i\to E_f}}{P\p{E_f\to E_i}},
  \label{eq:tmmc_prob_simplified}
\end{align}
where $\Delta E=E_f-E_i$ and $P\p{E_m\to E_n}$ is the unbiased
probability of a transition from an energy $E_m$ to an energy
$E_n$. The result in (\ref{eq:tmmc_prob_simplified}) is the
probability of accepting moves $s_i\to s_f$ required for simulation
via Algorithm \ref{alg:tmmc} to yield a flat energy
histogram\cite{tmmc}. Unlike the previous histogram methods, we desire
a flat histogram during initialization via Algorithm \ref{alg:tmmc}
not because we are constructing weights $w\p{E}$, but because we wish
to sample the transition histogram $\tilde T_d\p{E,\Delta E}$ equally
at all energies $E$. \fixme{briefly explain
  (\ref{eq:tmmc_prob_simplified}) by considering a two-level system,
  and extending to larger systems}.

Aside from a minimum temperature $T_{\t{min}}$ and an end condition,
Algorithm \ref{alg:tmmc} contains one free parameter, $c$, used in
step \ref{alg:tmmc_prob} to modify (\ref{eq:tmmc_prob_simplified}).
This parameter makes the rejection of transitions more conservative
when there are too few statistics in the histogram $\tilde T_d$ to be
confident in the current estimate of the transition matrix. In order
to make the effects of $c$ negligible as $\tilde T_d$ collects more
statistics throughout initialization, $c$ should be of order unity,
though its optimal value is generally system-dependent. In our
implementation of Algorithm \ref{alg:tmmc}, we use $c=16$.

The last interesting part of the probability in step
\ref{alg:tmmc_prob} of Algorithm \ref{alg:tmmc} is the limit on
$P_m\p{s_i\to s_f}$ to a minimum value of $\exp\p{-\Delta
  E/kT_{\t{min}}}$. This limit caps the bias on energies to that
introduced by canonical (fixed-temperature) weights at a temperature
of $T_{\t{min}}$. As a result, Algorithm \ref{alg:tmmc} will not waste
time oversampling energies which are unimportant for determining
system properties at temperatures $T\ge T_{\t{min}}$.

\begin{algorithm}[tb]
  \caption{Transition matrix Monte Carlo initialization}
  \label{alg:tmmc}
  \begin{alg}

  \item Construct an initial typical fluid configuration.

  \item Initialize a transition histogram $\tilde T_d\p{E,\Delta E}=0$
    for all $E$ and $\Delta E\in\sp{-M,M}$, where $M$ is the possible
    maximum energy transition of a single move.

  \item Randomly attempt to change the position of one sphere,
    tentatively accepting the transition from state $s_i$ to state
    $s_f$ if $s_f$ is not a forbidden fluid configuration.
    \label{alg:tmmc_move}

  \item Let $\Delta E=E_f-E_i$ and increment $\tilde
    T_d\p{E_i,\Delta{E}}$, where $E_k$ is the energy of state $s_k$.

  \item If $\Delta E<0$, accept the move. Otherwise
    \begin{enumerate}
    \item compute the normalization factor
      \begin{align*}
        n_k=\sum_{\Delta E'}\tilde T_d\p{E_k,\Delta E'}
      \end{align*}
      for $k\in\set{i,f}$, and

    \item accept the move $s_i\to s_f$ with probability
      \begin{align*}
        P_m\p{s_i\to s_f}
        =\max\set{\f{\p{\tilde T_d\p{E_i,\Delta E}+c}/\p{n_i+c}}
          {\p{\tilde T_d\p{E_f,-\Delta E}+c}/\p{n_f+c}},
          \exp\p{-\f{\Delta E}{kT_{\t{min}}}}}
      \end{align*}
      for a predetermined number $c$ and temperature $T_{\t{min}}$.
      \label{alg:tmmc_prob}
    \end{enumerate}

  \item If some predetermined initialization end condition has
    \emph{not} been met, return to step \ref{alg:tmmc_move}.

  \end{alg}
\end{algorithm}

\subsection{The optimized ensemble method}
\label{sec:optimized_ensemble}

All histogram methods introduced in this paper thus far have focused
on determining a weight array $w\p{E}$ for a flat energy histogram
$H\p{E}$, in order to sample and collect statistics on all energies
equally. This motivation considers the quantity of statistics, but not
their quality. Biased simulations collect statistics on low energy
states by first getting into these states, and then rejecting moves
which move the system into higher energy states. Statistics on low
energy states will therefore generally be highly correlated, as they
are based on many samples of only a few low energy regions of a
system's energy landscape. The optimized ensemble
(OE)\cite{optimized_ensemble} attempts to address the autocorrelation
between low energy samples by finding weights to maximize the rate at
which a simulation makes round trips between low and high energy
states. Maximizing the round trip rate, in turn, maximizes the rate at
which a simulation makes independent, uncorrelated samples of low
energies.

The optimized ensemble works by considering an ensemble of
simulations, each of which defines the position of a ``walker'' in
energy space. Given that we want walkers to move back and forth
between two extrema $E_+$ and $E_-$ of some specified energy range, we
label each walker by which extrema it has visited most recently. We
can then identify ``down-going'' (``up-going'') walkers by those which
more recently visited $E_+$ ($E_-$), which means that to make a round
trip they first needed to get to $E_-$ ($E_+$). Denoting the total
walker and down-going density at an energy $E$ respectively by
$n\p{E}$ and $n_+\p{E}$, the ratio $f\p{E}=n_+\p{E}/n\p{E}$ gives the
proportion of walkers at $E$ which are down-going. We also denote the
walker diffusivity at an energy $E$ by $\alpha\p{E}$.

At equilibrium with a flat histogram, meaning $n$ is constant, the
down-going walker current $j\p{E}$ is defined in terms of the
diffusivity $\alpha\p{E}$ by
\begin{align}
  j=-\alpha\f{dn_+}{dE}.
  \label{eq:walker_current_definition}
\end{align}
This identity can be taken as the definition of the walker diffusivity
$\alpha$, which is the independent of the label we have assigned any
given walker. Substituting $n_+=fn$,
\begin{align}
  j=-\alpha\p{n\f{df}{dE}+f\f{dn}{dE}}=-\alpha n\f{df}{dE},
  \label{eq:walker_current}
\end{align}
where a constant $n$ means $dn/dE=0$. We can rearrange
(\ref{eq:walker_current}) and integrate over the energy range as
\begin{align}
  \int_{f=f\p{E_-}}^{f\p{E_+}}\f{df}{j}=-\int_{E=E_-}^{E_+}\f{dE}{\alpha n}.
  \label{eq:walker_current_integral_setup}
\end{align}
At equilibrium, the number of down-going walkers $n_+\p{E}$ at any
given energy $E$ is constant with time, which means that the walker
flux $j\p{E}$ must be constant with energy. We can therefore evaluate
one of the integrals in (\ref{eq:walker_current_integral_setup}) to
get
\begin{align}
  \f1{\abs j}=\int_{E=E_-}^{E_+}\f{dE}{\alpha n}.
  \label{eq:walker_current_integral}
\end{align}

The optimized ensemble method minimizes the integral in
(\ref{eq:walker_current_integral}), thereby maximizing the down-going
walker current $\abs{j}$, by varying the walker density $n$ with the
constraint that $n$ must remain normalized. With the additional
assumption that $\alpha\p{E}$ does not strongly depend on the weights
$w\p{E}$, the authors of \cite{optimized_ensemble} find that the
optimal walker density $n$ is
\begin{align}
  n_{\t{opt}}\propto\f1{\sqrt{\alpha}},
  \label{eq:optimal_walker_density}
\end{align}
where the proportionality is determined by enforcing that $n$ is
normalized.

Equating the set of observations in a single simulation with a single
observation of an ensemble of simulations, the walker density $n\p{E}$
is proportional to the energy histogram $H\p{E}=\tilde D\p{E}w\p{E}$,
which means
\begin{align}
  w_{\t{opt}}=\f1{D\sqrt{\alpha}},
  \label{eq:optimized_weights}
\end{align}
where we neglect the distinction between $\tilde D$ and $D$, which has
no effect on the weights $w$.

The algorithm provided in \cite{optimized_ensemble} to determine and
construct weights given by \ref{eq:optimized_weights} involves
simulating with and iteratively modifying flat histogram weights. This
algorithm is not meant to construct weights from scratch, but rather
to optimize them. We therefore will not be comparing the optimized
ensemble to other methods. We do, however, credit this method for
motivating the hybrid OE-TMMC method in Section \ref{sec:oetmmc}.

\subsection{The hybrid OE-TMMC method}
\label{sec:oetmmc}

By the central limit theorem, the probability distribution $P\p{E\to
  E+\Delta E}$ of Monte Carlo moves which changes the energy of a
system by $\Delta E$ is
\begin{align}
  P\p{E\to E+\Delta E}=\f1{\sqrt{2\pi\sigma^2}}
  \exp\p{-\f{\Delta E^2}{2\sigma^2}}
\end{align}
for some variance
\begin{align}
  \sigma^2=\sigma_E\p{\Delta E}^2=\bk{\Delta E^2}_E-\bk{\Delta E}_E
\end{align}
which generally depends on the energy $E$.

\vspace{3cm}

\begin{align}
  \bk{\Delta E}_E=\f{\sum_{\Delta E}\Delta E~\tilde T\p{E,\Delta E}
    w\p{E+\Delta E}/w\p{E}}{\sum_{\Delta E'}\tilde T\p{E,\Delta E'}
    w\p{E+\Delta E'}/w\p{E}}
\end{align}

\begin{align}
  \bk{\Delta E^2}_E=\f{\sum_{\Delta E}\Delta E^2~\tilde T\p{E,\Delta E}
    w\p{E+\Delta E}/w\p{E}}{\sum_{\Delta E'}\tilde T\p{E,\Delta E'}
    w\p{E+\Delta E'}/w\p{E}}
\end{align}

\begin{align}
  \alpha=\bk{\Delta E^2}_E-\bk{\Delta E}_E^2=\sigma\p{E}^2
\end{align}



\fixme{Explain, provide algorithm. This method is our own synthesis of
  other work.}

\begin{algorithm}[H]
  \caption{Modifying transition weights via the optimized ensemble}
  \label{alg:oetmmc}
\end{algorithm}


\subsection{Identifying the energy range of interest}
\label{sec:energy_range}

Histogram methods typically rely on knowledge of the minimum and
maximum energies of the system in question \fixme{how?}, which are
unknown for the square-well fluid. Worse still, some square-well fluid
energies have such an incredibly low density of states that one cannot
reasonably expect to ever observe them via Monte Carlo, biased or
otherwise. As such, we need to identify an energy range of interest,
bounded by minimum and maximum ``important'' energies. \fixme{Rewrite,
  expand}

\subsubsection{State of maximal entropy}
\label{sec:max_entropy}

\fixme{The energy at which the density of states is maximal, i.e. the
  state of maximal entropy, sets an upper bound for the energy range
  in which appropriate weights are not obvious. Explain why.}

\subsubsection{Minimum important energy}
\label{sec:min_energy}

\fixme{Determining the minimum important energy is considerably
  trickier. Punchline: we have to choose a minimum temperature of
  interest, $T_{\t{min}}$, and the minimum important energy is the
  energy at which the slope of the density of states is
  $\exp\p{kT_{\t{min}}/\epsilon}/\epsilon$. This condition for the
  minimum important energy is directly related to canonical weights at
  a temperature of $T_{\t{min}}$.}

\fixme{Mention that in all initializations which simulate for some
  time with unchanging weights (i.e. simple flat), the weight array
  should be set to canonical weights below the minimum important
  energy, as soon as it is identified.}

\fixme{Mention that weights should be set to canonical weights below
  the minimum important energy after all initializations.}

\subsubsection{Computing the density of states}
\label{sec:dos}

\fixme{This section might just become a part of the section on the
  minimum important energy. (\ref{eq:dos}) gives us an explicit
  formula for the density of states if we are confident in our energy
  histogram. We can also compute the density of states from the
  transition matrix, which is sometimes more appropriate to do.}

\subsection{End conditions}
\label{sec:end_conditions}

% \fixme{Talk about how we want method-independent end conditions, if
% possible. Introduce the concept of statistically independent energy
% samples, and explain two measures (one optimistic, and one
% pessimistic) for determining the number of statistically independent
% times we have samples a given energy.}

% \fixme{Talk about two end conditions: a fixed number of times which
% we require to independently sample the minimum important energy, and
% an enforced fractional error in the number of times we have
% independently sampled any energy (at a given minimum temperature).}


%%%%%%%%%%%%%%%%%%%%%%%%%%%%%%%%%%%%%%%%%%%%%%%%%%%%%%%%%%%%%%%%%%%%%%
\section{Results and Discussion}
\label{sec:results}

\subsection{Implementation and overview}
\label{sec:implementation}

\fixme{Talk about the relative difficulty of implementing each method,
  as well as the relative difficulty of optimizing or tuning each
  method. Free parameters are generally bad, and especially so if they
  come without ``obvious'' or prescribed values. Justify our
  modifications to Wang-Landau by showing that we did not make it
  worse. Include mention of how memory requirements for transition
  matrix Monte Carlo scale poorly with system size, as well as energy
  resolution for systems with continuous energies.}

\subsection{Simulation results and errors}
\label{sec:results_and_errors}

\fixme{Talk about the energy histogram, weights, and density of states
  yielded by the methods. These should all be fairly similar. Talk
  about why this is so.}

\begin{figure}[H]
  \centering
  \caption[Energy histograms]{Energy histograms}
  \label{fig:histograms}
\end{figure}

\begin{figure}[H]
  \centering
  \caption[Energy weights]{Energy weights}
  \label{fig:weights}
\end{figure}

\begin{figure}[H]
  \centering
  \caption[Normalized densities of states]{Normalized densities of
    states}
  \label{fig:density_of_states}
\end{figure}

\fixme{Talk about initialization times, and how they scale with system
  size}

\begin{figure}[H]
  \centering
  \caption[Initialization iterations vs. system size]{Initialization
    iterations vs. system size}
  \label{fig:scaling}
\end{figure}

\fixme{Discuss energy initialization times, sampling rates, and errors
  in computed thermodynamic properties. Explain that errors are given
  relative to a simulation which ran much longer, and talk about how
  we ran that simulation (which method did we use?). Show that all
  methods converge on the same answer as they simulate for longer.}

\begin{figure}[H]
  \centering
  \caption[Optimistic energy sampling rates]{Optimistic energy
    sampling rates}
  \label{fig:opt_sample_rate}
\end{figure}

\begin{figure}[H]
  \centering
  \caption[Pessimistic energy sampling rates]{Pessimistic energy
    sampling rates}
  \label{fig:pes_sample_rate}
\end{figure}

\begin{figure}[H]
  \centering
  \caption[Initialization iterations vs. data quality]{Initialization
    iterations vs. data quality}
  \label{fig:quality}
\end{figure}

\begin{figure}[H]
  \centering
  \caption[Specific internal energy]{Specific internal energy}
  \label{fig:internal_energy}
\end{figure}

\begin{figure}[H]
  \centering
  \caption[Specific heat capacity]{Specific heat capacity}
  \label{fig:heat_capacity}
\end{figure}


%%%%%%%%%%%%%%%%%%%%%%%%%%%%%%%%%%%%%%%%%%%%%%%%%%%%%%%%%%%%%%%%%%%%%%
\section{Conclusions}
\label{sec:conclusions}

\fixme{What have we learned? What recommendations can we make? When is
  it appropriate to use which histogram methods?}


%%%%%%%%%%%%%%%%%%%%%%%%%%%%%%%%%%%%%%%%%%%%%%%%%%%%%%%%%%%%%%%%%%%%%%
\nocite{*} \bibliography{thesis}

\end{document}
