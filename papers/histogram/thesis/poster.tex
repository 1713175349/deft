\documentclass[paperwidth=48in,paperheight=36in,
fontscale=0.27,margin=0.75in]{baposter}

\definecolor{border_color}{RGB}{40,40,40}
\definecolor{header_color_left}{RGB}{186,215,230}
\definecolor{header_color_right}{RGB}{80,80,80}
\definecolor{header_font_color}{RGB}{0,0,0}
\definecolor{box_color}{RGB}{186,215,230}

\frenchspacing

%%% enumerating
\usepackage{enumitem} % include for \setlist{}, use below
\setlist{nolistsep} % more compact spacing between environments
\setlist[itemize]{leftmargin=*} % nice margins for itemize ...

%%% math and physics
\usepackage{amsmath} % math packages
\renewcommand{\t}{\text} % for text in math environment
\usepackage[boldvectors,braket]{physymb} % physics packages
\renewcommand{\v}{\vec} % bold vectors
\newcommand{\p}[1]{\left(#1\right)} % bold vectors

%%% figures
\usepackage{graphicx,grffile,float,subcaption} % floats, etc.
\usepackage{multirow} % for multirow entries in tables

%% misc.
\usepackage[font=normalsize,labelfont=bf]{caption} % caption text

\renewcommand{\title}{Optimizing Monte Carlo Simulation of the
  Square-Well Fluid}
\renewcommand{\author}{\LARGE Michael A. Perlin, David Roundy}
\newcommand{\location}{Oregon State University, Department of Physics}

\begin{document}

\begin{poster}{
% options
    %% colors and appearance
    borderColor=border_color, %
    headerColorOne=header_color_left, %
    headerColorTwo=header_color_right, %
    headerFontColor=header_font_color, %
    boxColorOne=box_color, %
    textborder=rectangle, %
    headerborder=open, %
    background=none, %
    %% headers
    headerfont=\Large\sf\bf, %
    headerheight=0.11\textheight, %
    headershape=roundedright, %
    %% other
    columns=3 %
  }%
  {% top left logo
  }%
  {% title
    \LARGE\title
  }%
  {% authors
    \Large\author \\[1mm]
    \location
  }%
  {% top right logo
    % \includegraphics[width=2.5cm]{figs/osu.pdf}
    % \hspace{5mm}
  }%

  \headerbox{Motivation}
  {name=intro,column=0,row=0}{%

    The critical point of a substance is the point in phase space at
    which the distinction between liquid and gas phases becomes
    ill-defined. Understanding the behavior of fluids near the
    critical point is a major challenge in the field of classical
    density functional theory.

    % fixme: figure out how to properly do paragraph spacing
    \vspace{6pt}

    The square-well fluid is the simplest system with a liquid-vapor
    phase transition, and is therefore of interest for studying
    critical point behavior of fluids. We wish to study the
    square-well fluid via numerical methods so as to test theories of
    fluid behavior.

    \begin{figure}[H]
      \centering
      \includegraphics[width=0.5\columnwidth]{figs/phase-diagram.pdf}
      \caption{Generic phase diagram.}
      \label{fig:phase_diagram}
    \end{figure}

  }

  \headerbox{The Square-Well Fluid}
  {name=sw_fluid,column=0,row=0,below=intro,above=bottom}{%

    % The square-well fluid is a model system used to capture low order
    % effects of short-range attractive forces, such as the van Der
    % Waals force. The fluid is composed of spheres with diameter
    % $\sigma$ and the pair potential shown graphically in Figure
    % \ref{fig:pair_potential}.

    % \vspace{6pt}

    % The parameters $\lambda$ and $\epsilon$ are called the well width
    % and well depth, respectively. The first ($r<\sigma$) part of the
    % pair potential forbids spheres from overlapping, whereas the
    % second ($\sigma<r<\lambda\sigma$) associates an energy $-\epsilon$
    % with each pair of spheres whose centers are within distance of
    % $\lambda\sigma$ of each other.

    \begin{itemize}
    \item Model system used to capture low order effects of
      short-range attractive forces, e.g. the van Der Waals force
    \item Composed of spheres with diameter $\sigma$ and the pair
      potential given in Figure \ref{fig:pair_potential}
    \item $\lambda$ is called the well width, and $\epsilon$ the well
      depth
    % \item Natural distance scale $\sigma$ and energy scale $\epsilon$
    \item First part of potential, $v_{sw}\p{r<\sigma}=\infty$,
      forbids spheres from overlapping
    \item Second part of potential,
      $v_{sw}\p{r:\sigma<r<\lambda\sigma}=-\epsilon$, associates an
      energy $-\epsilon$ with each pair of spheres whose centers are
      within distance of $\lambda\sigma$ of each other
    \item Spheres whose centers are more than $\lambda\sigma$ apart do
      not interact
    \end{itemize}

    \begin{figure}[H]
      \centering
      \includegraphics[width=0.45\columnwidth]{figs/square-well.pdf}
      \caption{Square-well pair potential.}
      \label{fig:pair_potential}
    \end{figure}

  }

  \headerbox{Monte Carlo Fluid Simulation}
  {name=mc_sims,column=1,row=0}{%

    % Model systems such as the square-well fluid are powerful tools for
    % understanding complex physical systems, but they do not themselves
    % exist in the real world. Consequently, direct experimental tests
    % of theories for model system are not possible. For this reason,
    % model system theories are commonly tested against Monte Carlo
    % simulations.

    % \vspace{6pt}

    % Standard Monte Carlo simulations of a fluid sample the fluid's
    % configuration space by randomly moving fluid elements
    % (i.e. spheres), and collecting statistics on observed fluid
    % properties (e.g. internal energy, density, etc.). Studying the
    % square-well fluid's critical point, however, requires sampling
    % states with highly localized distributions of spheres, which are
    % extremely unlikely to be found in a random sampling of all
    % possible fluid configurations. In the case of the square-well
    % fluid, the search for localized distributions of spheres is
    % equivalent to the search for low energy states.


    \begin{itemize}
    \item Model systems are powerful tools for understanding complex
      physical systems, but they do not exist in the real world
    \item Theories tested against Monte Carlo simulations
    \item Monte Carlo fluid simulations: moving spheres randomly to
      collect statistics on properties (e.g. potential energy, density,
      etc.) of valid fluid configurations
    \item Studying critical point requires sampling states with highly
      localized distribution of spheres, which is unlikely to occur by
      randomly sampling of all possible fluid configurations (second
      law of thermodynamics)
    \item For the square-well fluid, localized distribution of spheres
      is synonymous with a low energy state
    \end{itemize}

    \begin{figure}[H]
      \centering
      \includegraphics[width=0.5\columnwidth]{figs/dos-example.pdf}
      \caption{Sample density of states of a particular square-well
        fluid ($\lambda=1.3$, $\eta=0.3$). The logarithmic span of the
        density of states is proportional to the number of moves one
        must simulate via standard Monte Carlo in order to observe the
        energies with the lowest state densities.}
      \label{fig:dos}
    \end{figure}

  }

  \headerbox{Histogram Methods}
  {name=histogram_methods,column=1,row=0,below=mc_sims,above=bottom} {%

    % Histogram methods are a means for biasing Monte Carlo simulations
    % in favor of some regions of state space, i.e. toward low
    % energies. The idea, as applied to the problem at hand, is simple:
    % accept all Monte Carlo moves into fluid states with a lower
    % energy, and reject, with some probability, moves into fluid states
    % with a higher energy.

    % \vspace{6pt}

    % There are various published histogram methods, but little to no
    % literature exists on these methods' systematic comparison, or on
    % their application to the square-well fluid. We implement several
    % methods, including Wang-Landau, transition matrix Monte Carlo, and
    % a modified version of the so-called optimized ensemble. We compare
    % low energy sampling rates for all methods, as well as relative
    % errors in thermodynamic properties computed via each method.

    \begin{itemize}
    \item Means for biasing Monte Carlo simulations in favor of lower
      energy states
    \item Accept all moves in simulation which lower the system's
      energy, probabilistically reject moves to higher energy
    \item Details of rejection process define a histogram method
    \item There are several published histogram methods, but little to
      no literature on their systematic comparison, or their
      application to the square-well fluid
    \item We implement several methods: our own ``simple flat''
      method, Wang-Landau, transition matrix Monte Carlo (TMMC), and a
      hybrid of TMMC with a so-called ``optimized ensemble'' method
    \item Comparison metrics: low energy sampling rates, and relative
      errors in thermodynamic properties computed using results from
      simulating via each method
    \end{itemize}

  }

  \headerbox{Results}
  {name=results,column=2}{%

    \begin{figure}[H]
      \centering
      \includegraphics[width=0.6\columnwidth]{figs/sample-rates.pdf}
      \caption{Iterations per statistically independent energy
        sample.}
      \label{fig:sample_rates}
    \end{figure}
    \begin{figure}[H]
      \centering
      \includegraphics[width=0.6\columnwidth]{figs/u-comps.pdf}
      \caption{Maximum errors in a set of simulations, relative to a
        method which controls for system size and simulation time.}
      \label{fig:errors}
    \end{figure}

  }

  \headerbox{Conclusions}
  {name=conclusions,column=2,below=results}{%



  }

  \headerbox{Acknowledgments}
  {name=thanks,column=2,below=conclusions,above=bottom}{%



  }

\end{poster}

\end{document}