\documentclass[letterpaper,twocolumn,amsmath,amssymb,pre,aps,10pt]{revtex4-1}
\usepackage{graphicx}% Include figure files
\usepackage{dcolumn}% Align table columns on decimal point
\usepackage{color}

\newcommand{\fixme}[1]{{\bf\color{red}{[#1]}}}
\newcommand{\rr}{\mathbf{r}}

\begin{document}
\title{Renormalization}

\author{David Roundy}
\affiliation{Department of Physics, Oregon State University, Corvallis, OR 97331}

\begin{abstract}
  Here is an interesting way to look at GRG.
\end{abstract}

\maketitle


Let's talk about how we would solve for the free energy precisely.
\begin{align}
  F(N, V, T) &= -kT\ln Z \\
  Z(N, V, T) &= \sum_{i}^{\text{microstates}} e^{-\beta E_i}
\end{align}
This is our exact free energy, which requires that we specify an
explicit volume $V$.  We typically assume that $N \gg 1$ and $V$ is a
macroscopic volume.  An interesting option is to consider the
possibility where $N$ is of arbitrary size, and $V$ likewise may be
small.  In this case, we absolutely need to ask ourselves what the
boundary conditions will be.  One option is to choose periodic
boundary conditions, and a cubic cell with side length $L$, so that $V
= L^3$.  This enables easy comparison with Monte Carlo simulations,
and we can obtain the thermodynamic limit by taking the limit as
$N\rightarrow \infty$ with the density held fixed.  The appeal of this
approach is that it enables us to define a hierarchy of free energies
$f_i$ that have lengths $L_i=2^i L_0$ which provide a physical meanint
to the intermediate free energies obtained in a GRG computation.

We define the $i^{th}$ free energy:
\begin{align}
  f_i(n) &\equiv -\frac{kT}{2^{3i}L_0^3}\ln \sum_{i}^{L_i} e^{-\beta E_i}
\end{align}
where the summation is over microstates with periodicity $L_i$.

We now ask how $f_i$ relates to $f_{i-1}$.  Mathematically, the
difference between these free energies is
\begin{align}
  f_i(n) - f_{i-1}(n) &\equiv \delta f_i \\
  &= -\frac{kT}{2^{3i}L_0^3}\ln \sum_{i}^{L_i} e^{-\beta E_i}
  +\frac{kT}{2^{3i}L_0^3}\ln \sum_{i}^{L_i} e^{-\beta E_i}
\end{align}

\end{document}
