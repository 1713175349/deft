\documentclass[letterpaper,twocolumn,amsmath,amssymb,pre,aps,10pt]{revtex4-1}
\usepackage{graphicx}% Include figure files
\usepackage{dcolumn}% Align table columns on decimal point
\usepackage{color}

\newcommand{\fixme}[1]{{\bf\color{red}{[#1]}}}
\newcommand{\rr}{\mathbf{r}}

\begin{document}
\title{Renormalization}

\author{Daniel ?. Roth}
\author{Michael ?. Perlin}
\author{David Roundy}
\affiliation{Department of Physics, Oregon State University, Corvallis, OR 97331}

\begin{abstract}
  We present an analysis of the convergence properties of
  renormalization group theory  when applied to the square-well
  liquid.
\end{abstract}

\maketitle

Long-wavelength fluctuations lead to significant changes in fluid
behavior in two scenarios: near the critical point, and at
liquid-vapor interfaces.  The failing of perturbation theory near the
critical point is well-understood, and is commonly treated in the
homogeneous case using either renormalization group
theory~\cite{white2000global, white2001global, del2002vapour,
  kiselev2002computer, reiner2002hierarchical, mi2004renormalization,
  mi2004improved, fu2006study, giacometti2009liquid, jiuxun2005simple,
  forte2011application, el2008integral, ramana2012generalized}, or
crossover theory~\cite{kiselev1999crossover, kiselev2000crossover,
  kiselev2001crossover, kiselev2000simplified, hu2003crossover,
  hu2003back, mccabe2004crossover, llovell2006global}.  The
liquid-vapor interface is beset by long-wavelength fluctuations in the
form of capillary waves~\cite{buff1965interfacial,
  weeks1989consistency}, which are handled---when treated at all---as
an analytic correction to the surface tension computed using density
functional theory~\cite{wadewitz2000application, winkelmann2001liquid,
  gross2009density}.  These corrections to the surface tension involve
an ad hoc cutoff wavelength, and do not correct the prediction of the
interfacial density profile.  The effect of capillary waves on surface
tension is largest near the critical point.  Together, these effects
suggest that a density functional theory directly incorporating
long-wavelength fluctuations would be a major gain for the study of
liquid interfaces.

I plan to develop such a functional using renormalization group theory
(RGT).  Implementation of RGT in an inhomogeneous system will be
computationally challenging, because it is an inherently recursive
theory.  This results in long-range fluctuations (and correlations)
near the critical point, which will be reflected in a long-range
density functional.  It will be more computationally
intensive than existing density functionals, but at the same time will
enable far greater predictive power, over a wide range of
temperatures, pressures, and density distributions.

There has been one study of the effect of critical behavior on surface
tension using a renormalization approach, which used the Density
Gradient Theory (DGT) rather than DFT due to its easier integration
with renormalization group theory~\cite{fu2006study}.  The use of DGT,
which is not self-consistent, puts a major limitation the power of the
method, particularly when applied to diverse interfaces.


We will study a hard-sphere fluid with a square well attractive
potential.  This is a widely used model fluid within the SAFT
community~\cite{mi2004renormalization, mi2004improved, fu2006study,
  forte2011application} as well as in the study of the critical
behavior of fluids~\cite{white2000global, white2001global,
  del2002vapour, kiselev2002computer, reiner2002hierarchical,
  mi2004renormalization, mi2004improved, fu2006study,
  giacometti2009liquid, jiuxun2005simple, forte2011application,
  el2008integral, ramana2012generalized}. This model is particularly
suitable for testing over a wide range of temperatures and pressures
due to its ease of computation.  Because the set of energy levels is
discrete, this system is suitable for the use of histogram-based Monte
Carlo methods~\cite{ferrenberg1988histogram, lee1993new, de1998broad}.
These approaches accelerate the computation of thermodynamic
properties over a range of temperatures, and are particularly helpful
near phase transitions and critical points.  There has been a recent
Monte Carlo study of phase behavior of the the square-well
fluid~\cite{liu2005direct}, which did not take advantage of the
histogram family of Monte Carlo methods, although it did use the
aggregation volume bias sampling method to accelerate sampling of
bonded and non-bonded configurations~\cite{chen2000novel}.

We will begin by studying the simple square-well liquid already
introduced in a previous paragraph.  This liquid is a standard test
case for theories of critical behavior, although its critical behavior
at the liquid-vapor interface has not yet been examined.  We will
closely follow the approach recently developed by Forte \emph{et al.}
who constructed a SAFT-VR equation of state using
RGT~\cite{forte2011application}.  Their approach to RGT is expressed
in terms of the pair distribution function
$g^{(2)}_{HS}(\rr_1,\rr_2)$, which will enable us to leverage our
earlier work (from the previous paragraph) to construct a
renormalization group classical density functional theory for the
inhomogeneous fluid.  Once we have developed a RGT density functional,
we will test it against Monte Carlo simulations of the surface
tension, as well as a fluid near the critical point at a hard wall.
%
Future work will include applying this approach to associating fluids.


\section{Methods}

\subsection{Renormalization Group Theory}

\fixme{Here we will explain the RGT that we use, and how it works
  out.}

\subsection{Monte Carlo}

\fixme{Here we will explain our Monte Carlo simulations.}

\section{Results}

\section{Conclusion}

\bibliography{paper}% Produces the bibliography via BibTeX.

\end{document}
