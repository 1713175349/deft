\documentclass{beamer}
\usepackage{graphicx}
\usepackage{color}
\usepackage{verbatim}
\usetheme{Frankfurt}

\title{Energy and Entropy and More}

\author{David Roundy}
\institute{Oregon State University}
\date{}

\setbeamertemplate{navigation symbols}{}

\begin{document}

\begin{frame}
  \titlepage
\end{frame}

%\section*{Outline}
% \begin{frame}
%   \frametitle{Outline}
%   \tableofcontents
% \end{frame}

\section{Energy and Entropy}
\subsection*{}

\begin{frame}
  \frametitle{Energy and Entropy paradigm}
  \begin{block}{The course}
    \begin{itemize}
    \item Originally developed by Allen Wassermann.
    \item I cotaught Energy and Entropy in 2009.
    \item I taught it on my own in 2010.
    \end{itemize}
  \end{block}
  \vfill
  \begin{block}{Course challenges}
    \begin{itemize}
    \item Introduce students to thermodynamics for the first time.
    \item Tricky thermodynamic differentials.
    \item Information-theory approach to the statistical approach.
    \end{itemize}
  \end{block}
\end{frame}

\newcommand\myderiv[3]{%
  \ensuremath{\left(\frac{\partial #1}{\partial #2}\right)_{#3}}}

\begin{frame}
  \frametitle{Energy and Entropy paradigm}
  \begin{block}{Thermodynamic derivatives $\longrightarrow$ experiments}
    \emph{``Name the experiment!''} For each of the derivatives I
    write on the board, describe the experiment that you would perform
    in order to measure it.
    \begin{align}
      \myderiv{p}{V}{T} \quad
      \myderiv{p}{V}{S} \quad
      \myderiv{p}{S}{V}
    \end{align}
    \begin{itemize}
    \item Maxwell relations help for confusing cases
    \item We have labs so students get to measure thermodynamic
      quantities themselves
    \end{itemize}
  \end{block}
\end{frame}

\begin{frame}
  \frametitle{Energy and Entropy paradigm}
  \begin{block}{Rubber band experiment and measuring entropy}
    Measuring tension versus temperature and length gives $\Delta U$,
    $\Delta F$ and $\Delta S$ for an isothermal stretch.

    \begin{columns}
      \begin{column}{2.4in}
        \begin{center}
          %\includegraphics[height=2in]{figs/hot-rubber}
        \end{center}
      \end{column}
      \begin{column}{2in}
        \begin{center}
          %\includegraphics[height=2in]{figs/ice-rubber}
        \end{center}
      \end{column}
    \end{columns}
  \end{block}
\end{frame}

\newcommand\fairness{\ensuremath{\mathcal{F}}}

\begin{frame}
  \frametitle{Energy and Entropy paradigm}
  \begin{block}{Information theory approach}
    \begin{itemize}
    \item Maximizing the (constrained) fairness gives the Boltzmann
      factor.
      \[
      \fairness = -k_B\sum_i^\text{all states} P_i \ln P_i
      \]
    \item Entropy is thus $\fairness_\text{max}$ rather than
      \[
      S = k_B \ln \Omega
      \]
    \item The Second Law is obvious, given this derivation of
      probabilities.
    \item Is it better?
    \end{itemize}
  \end{block}
\end{frame}


\section{Joining the Paradigms team}
\subsection*{}


\begin{frame}
  \frametitle{Teaching a new Paradigm for the first time}
  \begin{block}{}
    \begin{itemize}
    \item I taught \emph{Symmetries and Idealizations} in Fall 2009
    \item This is the \emph{first} paradigm
      \begin{itemize}
      \item Electrostatic potential
      \item Physics integrals
      \item Taylor expansions
      \item Working in groups
      \end{itemize}
    \end{itemize}
  \end{block}
\end{frame}


\begin{frame}[fragile]
  \frametitle{Teaching a new Paradigm for the first time}
  \begin{block}{The Paradigms Portfolios wiki}
    \begin{itemize}
    \item \verb!http://physics.oregonstate.edu/portfolioswiki!
    \item Homework
    \item Descriptions of activities and discussions
    \item Time estimates---shockingly accurate!
    \end{itemize}
  \end{block}
\end{frame}


\begin{frame}
  \frametitle{Teaching a new Paradigm for the first time}
  \begin{block}{Class every day, and how that changes things}
    \begin{itemize}
    \item It's easy to pick things up the next day: \\this changes how
      you think about ``finishing'' an activity
    \item Students who get behind are very behind
    \item Homework \emph{needs} to be very well synchronized
    \end{itemize}
  \end{block}
\end{frame}

\end{document}
