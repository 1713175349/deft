\documentclass{beamer}
\usepackage{graphicx}
\usepackage{color}
\usepackage{verbatim}
\usepackage{amsmath}
\usepackage{bm}
\usetheme{Frankfurt}

\title{A Classical Density-Functional Theory for Describing Water Interfaces}
\author{Jessica Hughes}
\date{October 11, 2011}

\setbeamertemplate{navigation symbols}{}

\begin{document}

\begin{frame}
  \titlepage
\end{frame}

\section{Theory}
\subsection*{}

\begin{frame}{Why Water is Cool}
\end{frame}

\begin{frame}{Density Functional Theory}
\end{frame}

\begin{frame}{Statistical Associating Fluid Theory}
\end{frame}

\begin{frame}[fragile]{The Functional Terms}
The Helmholtz free energy functional is
based on SAFT, and is composed of:
\begin{equation}
  F[n] = F_\text{id}[n] + F_\text{hs}[n] + F_\text{assoc}[n] + F_\text{disp}
\end{equation}
where $F_\text{id}$ is the ideal gas free energy, $F_\text{hs}$ is
a hard-sphere free energy, $F_\text{assoc}$ is the free energy of
association, and $F_\text{disp}$ is the dispersion energy. 
\end{frame}

\begin{frame}[fragile]{Ideal Gas Free Energy}
The ideal gas functional is given by
\begin{equation}
  F_\text{id}[n] = k_B T \int n(\textbf{r})\left( \ln{n(\textbf{r})} - 1\right) d\textbf{r}
\end{equation}
where $n(\textbf{x})$ is the density of water molecules.  
\end{frame}

\begin{frame}[fragile]{Hard-Sphere Free Energy}
Uses a Fundamental\nobreakdash-Measure Theory (FMT) functional,
which combines scaled-particle theory and the Percus-Yevick theory,
and an attractive interaction.  

The FMT functional treats
each water molecule as a hard sphere with a radius approximately that of
a water molecule $(R~=~3~\AA)$.
\end{frame}
 
\begin{frame}[fragile]{Dispersion Energy}
The dispersion free energy is based on the SAFT-VR
approach, which has two free
parameters, an interaction energy $\epsilon_\text{disp}$ and a
length scale $\lambda_\text{disp}$. It has the form
\begin{align}
  F_\text{disp} &= 
\end{align}
where $a_1$ and $a_2$ are the first two terms in a high-temperature
perturbation expansion.  $a_1$ is the mean-field dispersion
interaction and $a_2$ describes the
effect of fluctuations resulting from the compression of the fluid due
to the dispersion interaction itself, and is commonly approximated
using the local compressibility approximation (LCA).
\end{frame}

\begin{frame}[fragile]{Free Energy of Association}
The free energy of association describes the effects of hydrogen bonds. These are modeled as
four association sites on the surface of the
hard sphere. There is an attractive energy
$\epsilon_\text{assoc}$ when two molecules are arranged such that
the hydrogen of one interacts with the lone pair of the other.  The
volume of this interaction is $\kappa_\text{assoc}$ and the free energy is given by
\begin{align}
  F_\text{assoc}[n] &= 4 k_BT \int n_0(\textbf{r})
  \left(\ln X(\textbf{r}) - \frac{X(\textbf{r})}{2} + \frac12\right) d\textbf{r}
\end{align}
where the value of $4$ comes from the four association sites per
molecule, and the functional $X$ is the fraction of association sites
\emph{not} hydrogen-bonded. 
\end{frame}

\begin{frame}[fragile]{Other Thermodynamic Functions}
\begin{itemize}
 \item Grand free energy
 \item Pressure
 \item Internal energy
 \item Entropy
 \item Enthalpy
\end{itemize}
\end{frame}

\begin{frame}[fragile]{Pressure Calculation}
The pressure is obtained by taking a partial
derivative with respect to volume at fixed temperature.  Since
\emph{volume} isn't a parameter in density-functional theory, we
consider the \emph{free energy per unit volume}.
\begin{align}
  F[n] &= f[n]V \\
  p[n] &= -\left(\frac{\partial F[n]}{\partial V}\right)_{T} \\
  &= -\left(\frac{\partial f[n]V}{\partial V}\right)_{T} \\
  &= -V\left(\frac{\partial f[n]}{\partial V}\right)_{T}
   - f[n]\left(\frac{\partial V}{\partial V}\right)_{T} \\
  &= n \left(\frac{\partial f[n]}{\partial n}\right)_{T} - f[n]
\end{align}
\end{frame}

\section{Testing the Functional}
\subsection*{}

\begin{frame}[fragile]{Pressure - Water with no Constraints}
\begin{figure}
\begin{center}
\includegraphics[width=\columnwidth]{figs/pressure-with-isotherms}
\end{center}
\end{figure} 
\end{frame}

\begin{frame}[fragile]{Surface Tension}
\begin{figure}
\begin{center}
\includegraphics[width=\columnwidth]{figs/surace-tension}
\end{center}
\end{figure} 
\end{frame}

\begin{frame}[fragile]{Vapor Pressure}
\begin{figure}
\begin{center}
\includegraphics[width=\columnwidth]{figs/equation-of-state}
\end{center}
\end{figure}
\end{frame}


\begin{frame}[fragile]{Entropy, Internal Energy, Enthalpy}
[Add equations, calculations?]
\end{frame}

\begin{frame}[fragile]{Temperature Coexistence Curve}
\begin{figure}
\begin{center}
\includegraphics[width=\columnwidth]{figs/temperature-versus-density}
\end{center}
\end{figure}
\end{frame}

\begin{frame}[fragile]{Determining Densities - Common Tangent Rule}
\begin{figure}
\begin{center}
\includegraphics[width=\columnwidth]{figs/finding-vapor-pressure}
%\includegraphics[width=\columnwidth]{figs/near-critical-point}
\end{center}
\end{figure}
\end{frame}

\section{Results}
\subsection*{}

\begin{frame}[fragile]{Liquid-Vapor Interface}
\begin{figure}
\begin{center}
\includegraphics[width=\columnwidth]{figs/surface-298}
\end{center}
\end{figure} 
\end{frame}

\begin{frame}[fragile]{Water Constrained to a Slit (1D)}
Initial constraint: density equal to saturated liquid density inside cavity, 1000 times vapor density outside
\begin{figure}
\begin{center}
\includegraphics[width=\columnwidth]{figs/density-1D}
\end{center}
\end{figure} 
\end{frame}

\begin{frame}[fragile]{Water Constrained to a Slit (1D)}
\begin{figure}
\begin{center}
\includegraphics[width=\columnwidth]{figs/energy-1D}
\end{center}
\end{figure} 
\end{frame}

\begin{frame}[fragile]{Water Constrained to a Slit (1D)}
\begin{figure}
\begin{center}
\includegraphics[width=\columnwidth]{figs/xassoc-1D}
\end{center}
\end{figure} 
\end{frame}

\begin{frame}[fragile]{Water Constrained to a Slit (1D)}
\begin{figure}
\begin{center}
\includegraphics[width=\columnwidth]{figs/xassoc-1D-060}
\end{center}
\end{figure} 
\end{frame}

\begin{frame}[fragile]{Hydrophobic Rods in Water}
\begin{figure}
\begin{center}
\includegraphics[width=\columnwidth]{figs/density-single-rod}
\end{center}
\end{figure}
\end{frame}

\begin{frame}[fragile]{Hydrophobic Rods in Water}
\begin{figure}
\begin{center}
\includegraphics[width=\columnwidth]{figs/energy-vs-diameter}
\end{center}
\end{figure} 
\end{frame}

\begin{frame}[fragile]{Hydrophobic Rods in Water}
\begin{figure}
\begin{center}
\includegraphics[width=\columnwidth]{figs/xassoc-single-rod}
\end{center}
\end{figure} 
\end{frame}

\begin{frame}[fragile]{Hydrophobic Rods in Water}
Resolution and Convergence Tests
\begin{figure}
\begin{center}
\includegraphics[width=\columnwidth]{figs/density-vs-radius}
\end{center}
\end{figure} 
\end{frame}

\begin{frame}[fragile]{Hydrophobic Rods in Water}
\begin{figure}
\begin{center}
\includegraphics[width=\columnwidth]{figs/density-rods-in-water}
\end{center}
\end{figure} 
\end{frame}

\begin{frame}[fragile]{Hydrophobic Rods in Water}
\begin{figure}
\begin{center}
\includegraphics[width=\columnwidth]{figs/energy-rods}
\end{center}
\end{figure} 
\end{frame}

\begin{frame}[fragile]{Hydrophobic Rods in Water}
\begin{figure}
\begin{center}
\includegraphics[width=\columnwidth]{figs/rods-energy-vs-distance}
\end{center}
\end{figure} 
\end{frame}

\begin{frame}[fragile]{Hydrophobic Rods in Water}
The critical transition
\end{frame}

\end{document}
