% TODO:

% Write ``Liquid-vapor interface'' section A.

% Look up question-mark references (citations).

\documentclass[letterpaper,twocolumn,amsmath,amssymb,prb]{revtex4-1}
\usepackage{graphicx}% Include figure files
\usepackage{dcolumn}% Align table columns on decimal point
\usepackage{bm}% bold math
\usepackage{color}

%%%%%%%%%%%%%%%%%%%%%%%%%%%%%%%%%%%%%%%%%%%%%%%%%%%%%%%%%%%%
%% Definitions
\def\re{\text{Re}}
\def\im{\text{Im}}
\def\ket#1{\vert #1 \rangle}
\def\cU{{\cal{U}}}
\def\cD{{\cal{D}}}
\def\re{\text{Re}}
\def\im{\text{Im}}
\newcommand{\red}[1]{{\bf \color{red} #1}}
\newcommand{\blue}[1]{{\bf \color{blue} #1}}
\newcommand{\green}[1]{{\bf \color{green} #1}}
\newcommand{\rr}{\textbf{r}}
\newcommand{\xx}{\textbf{x}}
\newcommand{\refnote}{\red{[ref]}}

\newcommand{\fixme}[1]{\red{[#1]}}

% needsworklater is used to annotate bits that need work, but that we
% can postpone for a while.
\newcommand{\needsworklater}[1]{\emph{[#1]}}
% needsworknow is intended to prioritize stuff that needs fixing.
\newcommand{\needsworknow}[1]{\textcolor{red}{[\emph{#1}]}}

%%%%%%%%%%%%%%%%%%%%%%%%%%%%%%%%%%%%%%%%%%%%%%%%%%%%%%%%%%%%

\begin{document}
\title{A Classical Density-Functional Theory for Describing Water Interfaces}

\author{Dennis L. Jackson}
\affiliation{Department of Physics, Oregon State University, Corvallis, OR 97331}

\author{David Roundy}
\affiliation{Department of Physics, Oregon State University, Corvallis, OR 97331}

%%%%%%%%%%%%%%%%%%%%%%%%%%%%%%%%%%%%%%%%%%%%%%%%%%%%%%%%%%%%
\begin{abstract}
\needsworklater{ We develop a classical density functional theory for
  water by combining the fundamental-measure theory functional--which
  describes hard sphere fluids--with attractive interactions.  Using
  this approximation, we are able to describe various properties of
  water near liquid-vapor and liquid-solid interfaces.  We plan to use
  this functional as the foundation for a efficient method for
  predicting the interactions of chemicals in aqueous solution on an
  atomic level.}
\end{abstract}

\maketitle

%%%%%%%%%%%%%%%%%%%%%%%%%%%%%%%%%%%%%%%%%%%%%%%%%%%%%%%%%%%%
\section{Introduction}

\subsection{Water---the universal solvent}

A large fraction of interesting chemistry---including all of molecular
biology---takes place in aqueous solution.  However, while quantum
chemistry now enables us to calculate the ground state energies of
large molecules in vacuum, prediction of the free energy of even the
smallest molecules in the presence of a solvent poses a continuing
challenge due to the complex structure of a liquid and the
computational cost of \emph{ab initio} molecular
dynamics~\cite{car1985, grossman2004}.  The current state-of-the art
in \emph{ab initio} molecular dynamics is limited to around 64 water
molecules per unit cell, which places the minimum solute concentration
at 0.84 molar and allows contact between the second solvation shell of
neighboring solutes~\cite{izvekov2005, choe2007}.  On top of this,
\emph{ab initio} approaches using classical molecular dynamics have
\textcolor{red}{vague problems}\cite{weber2010ab-initio-water}.  A
more efficient approach is needed in order to study nanoscale and
hydrophobic solutes.

Water stands out among liquids for its extraordinary properties.  Water is
the only natural substance that is found in solid, liquid and gaseous
states at temperatures and pressures normally found on Earth.  Water has a
density maximum at 4$^\circ$C, and expands upon freezing.  The heat
capacity of water is the highest of any liquid, and its latent heat of
fusion and vaporization are also the highest.  Water's thermal conductivity
and surface tensions are the highest of any non-metallic liquid.  Water has
an unusually high dielectric constant.
Much of the unusual behavior of water is explained by the ability of each
water molecule to form \emph{four} strong, directional hydrogen bonds---two
acceptors and two donors.  This directional bonding dictates the open
structure of ice, leading to thermal expansion upon freezing.

Due to its large dielectric constant, small molecular size and strong
hydrogen-bonding network, water as ``the universal solvent'' are as
interesting as its anomalous thermodynamic properties.

\begin{figure}
\includegraphics[width=6cm, angle=270]{figs/correlation}
\caption{ Experimental atom-atom radial distribution functions for water at
room temperature and pressure from reference~\cite{Soper2000}.  }
\label{correlation}
\end{figure}

\subsection{Classical density-functional theory}

Numerous approaches have been developed to approximate the effect of water
as a solvent in atomistic calculations.  Each of these approaches gives
adequate description of some aspect of interactions with water, but none of
them is adequate for describing \emph{all} these interactions with an
accuracy close to that attained by \emph{ab initio} calculations.  The
theory of Lum, Chandler and Weeks (LCW)~\cite{LCW}, for instance, can
accurately describe the free energy cost of creating a cavity by placing a
solute in water, but does not lend itself to extensions treating the strong
interaction of water with hydrophilic solutes.  Treatment of water as a
continuum dielectric with a cavity surrounding each solute can give
accurate predictions of the energy of solvation of ions~\cite{latimer1939,
rashin1985, zhan1998, hsu1999, hildebrandt2004, hildebrandt2007}, but
provides no information about the size of this cavity.  In a physically
correct approach, the size of the cavity will naturally arise from a
balance between the free energy required to create the cavity, the
attraction between the water and the solute, and the steric repulsion which
opens up the cavity in the first place.

A very promising approach for an efficient continuum description of water
is that of classical density-functional theory (DFT), which is an approach
for evaluating the free energy and thermally averaged density of fluids in
an arbitrary external potential~\cite{ebner1976}.  Classical DFT is a
natural framework for creating a more flexible theory of hydrophobicity
that can describe interaction of water with arbitrary external
potentials---including potentials describing a \emph{strong} interaction
with a solute.

A number of exact properties are easily achieved in the density-functional
framework, such as the relationship between the pressure on a hard wall and
the contact density---which ensures a correct solvation free energy for
small hard solutes.  Much of the research on classical density-functional
theory has focussed on the hard-sphere fluid~\cite{curtin1985,
rosenfeld1989, rosenfeld1993, rosenfeld1997, tarazona1997, tarazona2000}.
This has led to a large number of very sophisticated functionals, such as
the fundamental-measure theory (FMT) functional~\cite{rosenfeld1989,
rosenfeld1993, rosenfeld1997, tarazona1997, tarazona2000}.  This functional
\needsworklater{has many nice properties}.

Several classical density functionals have already been developed for
water~\cite{ding1987, Yang1992, Jaqaman2004}, capturing some of the
qualitative behavior of water.  However, many of these are less
sophisticated than those describing the hard-sphere fluid, and a
quantitatively accurate functional describing water---which reproduces both
the macroscopic surface tension and its decrease as curvature
increases---has not yet been published.  For instance, Jaqaman \emph{et al}
developed a classical density-functional for water including orientational
effects, but their functional resulted in a surface tension twice the
experimental value, with the result that it can really only be used to make
very qualitative predictions~\cite{Jaqaman2004}.

\needsworknow{Add discussion and citations for more recent papers}

One of the most interesting functionals for water is that of Ding \emph{et
al}, which was developed in order to explain the freezing of
water~\cite{ding1987}.  This functional uses the approach of Chandler
\emph{et al}, which uses separate atomic densities for the constituent
atoms of a molecular fluid~\cite{chandler1986a, chandler1986b}.  Although
this is a compelling approach, the actual functional developed cannot
describe the vapor state or liquid-vapor interface, both of which are
crucial to correct predictions of hydrophobic behavior at large length
scales.  In addition, the functional of Ding \emph{et al} also fails to
rigorously enforce intramolecular correlations, which casts doubt on its
suitability for describing interactions with ionic solutes.

\needsworklater{
Density-Functional Theory is most commonly used for calculating the
densities of electrons in quantum mechanical systems.  We take the
classical limit of this theory to construct a functional that will predict
the density of water molecules in a classical system.  The model used here
starts with a treatment of each individual water molecule as a purely
classical hard sphere.  The only constraint in this model is
the minimum distance between any two molecules is twice the hard sphere
radius.}

\needsworklater{ I still need to rewrite the entire introduction... }

%%%%%%%%%%%%%%%%%%%%%%%%%%%%%%%%%%%%%%%%%%%%%%%%%%%%%%%%%%%%
\section{Theory and Methods}
We introduce a new classical density functional for water that
\needsworklater{predicts the vapor density, liquid density, bulk
  modulus, and bulk surface tension, as well as allows for
  liquid-vapor coexistence}.  The Helmholtz free energy functional is
based on SAFT, and is composed of the usual terms:
\begin{equation}
  F[n] = F_\textit{id}[n] + F_\textit{hs}[n] + F_\textit{assoc}[n] + F_\textit{disp},
\end{equation}
where $F_\textit{id}$ is the ideal gas free energy, $F_\textit{hs}$ is
a hard-sphere free energy, $F_\textit{assoc}$ is the free energy of
association, and $F_\textit{disp}$ is the dispersion energy.  In
addition, a chemical potential term is used, so we actually work in
the grand canonical ensemble.  In the following sections, we will
introduce the terms of this functional.

\subsection{Ideal gas functional}
The first term is the ideal gas free energy functional,
which incorporates effects of kinetic energy.  The ideal gas
functional is given by
\begin{equation}\label{idealgas}
  F_{id}[n] = k_B T \int n(\xx)\left( \ln{n(\xx)} - 1\right) d\xx
\end{equation}
where n(\xx) is the density of water molecules.  The ideal gas free
energy functional has the property that the contact density at a hard
surface $n_c$ is proportional to the pressure on that surface,
according to the equation
\begin{equation}
  p = n_c k_BT \:,
\end{equation}
which also leads to the property that the free energy of solvation of
a hard solute is proportional to its volume in the limit of small
solutes.  These properties are retained by the total functional,
provided the remaining terms are \emph{purely
  nonlocal}~\cite{ashcroft?}.

\subsection{Hard-sphere repulsion}
The repulsive intermolecular interactions are modeled using a
Fundamental\nobreakdash-Measure Theory~\cite{rosenfeld1997}(FMT) functional,
which combines scaled-particle theory and the Percus-Yevick theory,
and an attractive interaction.  The FMT functional treats
each water molecule as a hard sphere with a radius approximately that of
a water molecule $(R~=~2.3~bohr)$.\\


We model the repulsive interactions using the White Bear version of
the Fundamental-Measure Theory~(FMT) functional published in
reference~\cite{roth2002whitebear}.  \needsworklater{The FMT
  functional describes the excess free energy of a hard-sphere
  fluid. Fundamental-Measure Theory has several advantages over other
  approaches, one of the most important being the simplicity of
  calculations.  This particular FMT also reliably predicts bulk
  properties of fluids in one, two and three dimensions, as well as
  exactly determines the excess free energy in the zero dimensional
  limit.}
\begin{equation}
F_\textit{hs}[n] = k_B T \int \left(\Phi_1(\xx) + \Phi_2(\xx) + \Phi_3(\xx)\right) d\xx \; ,
\end{equation}
with integrands
\begin{align}
\Phi_1 &= -n_0 \ln\left( 1 - n_3\right)\\
\Phi_2 &= \frac{n_1 n_2 - \mathbf{n}_{V1} \cdot\mathbf{n}_{V2}}{1-n_3} \\
\Phi_3 &= (n_2^3 - 3 n_2 \mathbf{n}_{V2} \cdot \mathbf{n}_{V2}) \frac{
  n_3 + (1-n_3)^2 \ln(1-n_3)
}{
  36\pi n_3^2\left( 1 - n_3 \right)^2
} ,
\end{align}
using the weighted densities
\begin{align}
  n_3(\xx) &= \int n(\xx') \Theta(\left|\xx - \xx'\right| - R) d\xx' \\
  n_2(\xx) &= \int n(\xx') \delta(\left|\xx - \xx'\right| - R) d\xx'
  \\
  \mathbf{n}_{V2} &= \mathbf{\nabla} n_3 \\
  n_1 &= \frac{n_2}{4\pi R}\\
  \mathbf{n}_{V1} &= \frac{\mathbf{n}_{V2}}{4\pi R}\\
  n_0 &= \frac{n_2}{4\pi R^2}
\end{align}


\begin{figure}
\begin{center}
\includegraphics[width=\columnwidth]{figs/equation-of-state}
\end{center}
\caption{The theoretical versus experimental phase diagram, vapor
  pressure versus temperature.  }
\label{fig:equation-of-state}
\end{figure}

\subsection{Association free energy}
To the purely repulsive hard-sphere free energy, we add two attractive
energy terms.  The first is the free energy of \emph{association},
which describes the effects of hydrogen bonds.  These are modeled as
four attractive patches (``association sites'') on the surface of the
hard sphere.  These four sites represent the two hydrogens and two
electron lone pairs.  There is an attractive energy
$\epsilon_\textit{assoc}$ when two molecules are arranged such that
the hydrogen of one interacts with the lone pair of the other.  The
volume of this interaction is $\kappa_\textit{assoc}$.  Thus the
association term in the free energy has two free parameters that must
be fit based on experimental data.

The association free energy has the form
\begin{align}
  F_\text{assoc}[n] &= 4 k_BT \int n_0(\xx)
  \left(\ln X(\xx) - \frac{X(\xx)}{2} + \frac12\right) d\xx
\end{align}
where the value of $4$ comes from the four association sites per
molecule, and the functional $X$ is the fraction of assotiation sites
\emph{not} hydrogen-bonded.  The fraction $X$ is determined by the
quadratic equation
\begin{align}
  X(\xx) &= \frac{\sqrt{1 + 8n_0(\xx)\zeta(\xx)\Delta(\xx)} - 1}
  {4 n_0(\xx)\zeta(\xx)\Delta(\xx)}
\end{align}
where $\zeta$ comes from
reference\cite{yu2002fmt-dft-inhomogeneous-associating,
  fu2005vapor-liquid-dft} and describes the inhomogeneity of the fluid
density:
\begin{align}
  \zeta(\xx) &= 1 - \frac{\mathbf{n_{2V}}\cdot\mathbf{n_{2V}}}{n_2^2}
\end{align}
The function $\Delta$ is given... below?

The following contact density for the hard-sphere fluid is in the code
called \texttt{gHS}, and comes
from\cite{yu2002fmt-dft-inhomogeneous-associating,
  fu2005vapor-liquid-dft}.  Actually,
reference~\cite{fu2005vapor-liquid-dft} has what appear to be a number
of mistakes in this formula, so
reference~\cite{yu2002fmt-dft-inhomogeneous-associating} is more
useful.  In particular, the former has an incorrect formula for
$\zeta_2$ (wrong dimensions) and the formula for $\zeta_3$ disagrees
with the latter, which it cites.  Its formula for $\zeta_3$ also
doesn't match how they use it in ordinary FMT.
\begin{align}
  \Delta &= 4\pi \kappa_\textit{assoc} g^\textit{HS}(\sigma)e^{-\beta
    \epsilon_\textit{assoc}} \\
  g^\textit{HS}(\sigma) &= \frac1{1-n_3}\left(1+\frac14\frac{\zeta n_2}{1-n_3}
  \left(1 + \frac{n_2}{14 (1-n_3)}\right)\right) \\
  &= \frac1{1 - \zeta_3}
  + \frac32 \frac{\zeta \zeta_2}{\left(1-\zeta_3\right)^2}
  + \frac12 \frac{\zeta \zeta_2^2}{\left(1-\zeta_3\right)^3} \\
  \zeta_2 &= \frac{n_2 R}{3} \\
  \zeta_3 &= n_3
\end{align}
\textcolor{red}{Here is paper where I got the association functional:}\cite{clark2006developing}

Instead, I use
\begin{align}
  \Delta(\xx) &= \kappa_\textit{assoc}
  e^{-\beta\epsilon_\textit{assoc}} g^\textit{SW}_\sigma(\xx) \\
  g^\textit{SW}_\sigma(\xx) &= g^\textit{HS}_\sigma(\xx) +
  \frac{1}{4}\beta\left(\frac{\partial a_1(\xx)}{\partial n_3(\xx)} -
  \frac{\lambda}{3 n3(\xx)}\frac{\partial a_1(\xx)}{\partial \lambda}\right)
\end{align}
% This is equation 77 of gil-villegas-1997-SAFT-VR
which I got from reference\cite{gil-villegas-1997-SAFT-VR}.

\begin{figure}
\begin{center}
\includegraphics[width=\columnwidth]{figs/saturated-liquid-density}
\end{center}
\caption{The theoretical versus experimental saturated liquid density
  versus temperature.  }
\label{fig:saturated-liquid-density}
\end{figure}

\subsection{Dispersion free energy}
The final term in the free energy is the dispersion term.  We use a
dispersion term based on the SAFT-VR
approach\cite{gil-villegas-1997-SAFT-VR}, which has two free
parameters, an interaction energy $\epsilon_\textit{disp}$ and a
length scale $\lambda_\textit{disp}$.

The dispersion free energy has the form~\cite{gil-villegas-1997-SAFT-VR}
\begin{align}
  F_\text{disp} &= A_1 + \beta A_2
\end{align}
where $A_1$ and $A_2$ are the first two terms in a high-temperature
perturbation expansion.  $A_1$ is the mean-field dispersion
interaction, which for a homogeneous fluid is given by
\begin{align}
  A_1 &= \frac12 n_b \int \varphi(\left|\xx\right|)
  g_{HS}(\left|\xx\right|) d\xx
\end{align}
The second dispersion term in the free energy $A_2$ describes the
effect of fluctuations resulting from the compression of the fluid due
to the dispersion interaction itself, and is commonly approximated
using the local compressibility approximation (LCA).

We use a free square-well function for the dispersion $\varphi$, which
has been demonstrated \textcolor{red}{(CITE)} to give a reasonable fit
to the equation of state.  Thus our model interaction has the form:
\begin{equation}
  \varphi(r) = \Theta(r-2 \lambda_\textit{disp} R)
\end{equation}
where $\Theta$ is the Heaviside step function.

\textcolor{red}{TODO:  Add the actual functions that we use and the
  actual form.  The above is really just an introduction.}


\begin{figure}
\begin{center}
\includegraphics[width=\columnwidth]{figs/temperature-versus-density}
\end{center}
\caption{The theoretical versus experimental saturated liquid density
  versus temperature.  }
\label{fig:saturated-liquid-density}
\end{figure}

\begin{figure}
\begin{center}
\includegraphics[width=\columnwidth]{figs/entropy}
\end{center}
\caption{The free energy, internal energy, and temperature*entropy versus density at room temperature.  }
\label{fig:energy-room-temp}
\end{figure}

\begin{figure}
\begin{center}
\includegraphics[width=\columnwidth]{figs/entropy-at-690K}
\end{center}
\caption{The free energy, internal energy, and temperature*entropy versus density near the critical temperature.  }
\label{fig:energy-near-critical-temp}
\end{figure}

\subsection{Other thermodynamic functions}

From the Helmholtz free energy functional, we may obtain any other
thermodynamic functions we should choose.  The grand free energy
$\Phi$, which we ordinarily use in our computations, is obtained by
simply adding a chemical potential term:
\begin{equation}
  \Phi[n] = F[n] + \mu \int n(\xx) d\xx
\end{equation}

The pressure would naturally be obtained by taking a partial
derivative with respect to volume at fixed temperature.  Since
\emph{volume} isn't a parameter in density-functional theory, we
consider the \emph{free energy per unit volume}.
\begin{align}
  F[n] &= f[n]V \\
  p[n] &= -\left(\frac{\partial F[n]}{\partial V}\right)_{T} \\
  &= -\left(\frac{\partial f[n]V}{\partial V}\right)_{T} \\
  &= -V\left(\frac{\partial f[n]}{\partial V}\right)_{T}
   - f[n]\left(\frac{\partial V}{\partial V}\right)_{T} \\
  &= n \left(\frac{\partial f[n]}{\partial n}\right)_{T} - f[n]
\end{align}

\begin{figure}
\begin{center}
\includegraphics[width=\columnwidth]{figs/pressure-with-isotherms}
\end{center}
\caption{The pressure versus density for various temperatures, including experimental pressure data from NIST.  }
\label{fig:pressure-with-isotherms}
\end{figure}

The entropy is naturally obtained by taking a partial derivative with
respect to temperature:
\begin{align}
  S[n] &= \left(\frac{\partial F[n]}{\partial T}\right)_{V}
\end{align}
This derivative is trivial for the ideal gas and hard sphere free
energies, which are simply proportional to the temperature.  For the
association and dispersion free energies, it is a little more work,
but still not hard.

Once we have the entropy, finding the internal energy is a simple
matter of subtraction:
\begin{align}
  U[n] &= F[n] + TS[n]
\end{align}
and the enthalpy is not much more work:
\begin{align}
  H[n] &= F[n] + TS[n] - pV
\end{align}


\begin{figure}
\begin{center}
\includegraphics[width=\columnwidth]{figs/surface-tension}
\end{center}
\caption{The theoretical versus experimental surface tension
  versus temperature.  }
\label{fig:surface-tension}
\end{figure}

\begin{figure}
\begin{center}
\includegraphics[width=\columnwidth]{figs/surface-298}
\end{center}
\caption{Liquid-vapor interface profile at room temperature.  }
\label{fig:liquid-vapor-profile}
\end{figure}

\subsection{Crossover formalism for handling critical behavior}

I am currently working on implementing the crossover approach of
Kiselev and Ely which is designed to make the critical behavior come
out right\cite{kiselev2006new}.  This method works by defining a
\emph{classical} free energy $f_0$---which where classical really is
closer to \emph{mean field} in meaning, and indicates a theory that
doesn't include long-range fluctuations.  This free energy is then
used to construct a corrected free energy by feeding portions of it a
``fake'' density and temperature that are constructed so as to
reconstruct the correct universal scaling laws near the critical
point.

\subsection{Determining the empirical parameters}\label{sec:empirical}

In order to ensure that the empirical parameters are correct, we test
the vapor pressure (see Figure~\ref{fig:equation-of-state}) and
saturated liquid density (see
Figure~\ref{fig:saturated-liquid-density}) against experimental
values.  To find these densities, we use the ``common tangent'' rule,
as illustrated in Figure~\ref{fig:homogeneous}.

\begin{figure}
\begin{center}
\includegraphics[width=\columnwidth]{figs/finding-vapor-pressure}\\
\includegraphics[width=\columnwidth]{figs/near-critical-point}
\end{center}
\caption{By plotting the free energy vs. density it is possible to
  determine a saturated liquid or vapor state by the common tangent
  method.  }
\label{fig:homogeneous}
\end{figure}

\begin{table}
\begin{tabular}{ccc}
  Fitting Parameters& \hspace{2em} & Experimental Values\\
\begin{tabular}{cc}
  $R$ & 2.3 $a_o$ \\
  $A$ & -0.0100909 Hartree \\
  $\gamma$ & 5\\
  $V$ & 515.792 $a_o^3$  \\
  $b$ & 0.75 $a_o$ \\
  $\mu$ & -0.014063 Hartree 
\end{tabular}
&& 
\begin{tabular}{cc}
 $n_l$ & 4.939 $\times 10^{-3} a_o^{-3}$\\
 $n_v$ & 1.141 $\times 10^{-7} a_o^{-3}$\\
 $\sigma$ & ??? \\
 $B$ & $7.435\times 10^{-5}$ Hartree$/a_o^3$\\
 $T$ & 298.15 K \\
\end{tabular}
\end{tabular}

\caption{Empirical parameters of optimized functional and experimental
  values matched, where $a_o$ is the Bohr radius, and $\sigma$ is the
  surface tension.  This table is not yet complete, but it should
  be soon...  We'll also need to think about the best units to use
  here.  Atom units might be best, but it also might be better to
  consistently use eV and angstrom or something.  I'm not sure about
  putting $k_BT$ data in the table.  Maybe it belongs in the caption?}
\label{tab:parameters}
\end{table}

%%%%%%%%%%%%%%%%%%%%%%%%%%%%%%%%%%%%%%%%%%%%%%%%%%%%%%%%%%%%
\section{Results and discussion}
\subsection{Liquid-vapor interface}

We compute the density profile and surface tension by relaxing a slab
of water in a periodic unit cell.  This slab has two surfaces, so the
surface tension is just half the free energy of the cell, divided by
its cross-sectional area.  As mentioned in
section~\ref{sec:empirical}, the $b$ parameter is determined by
setting this surface tension to match the experimental value.  The
associated density profile is plotted in
Figure~\ref{fig:cartesianDensity}, which displays two peaks in
density.

The fluid density at a liquid-solid interface commonly displays
oscillatory behavior, but the presence of density oscillations at
liquid-vapor interfaces is more debated\cite{penfold2001structure}.
In fact, the \emph{meaning} of the density profile at a liquid-vapor
interface is clouded by the existence of long-wavelength perturbations
known as \emph{capillary waves}.  These waves are not accounted for in
density-functional theory (although the \emph{exact}
density-functionaly theory \emph{would} include them), with the result
that the density profile will be smoothed out by CW effects.

\fixme{It'd be worth another search for water intrinsic profile
  papers... but the review I cite is pretty thorough.}

\begin{figure}
\begin{center}
\includegraphics[width=\columnwidth]{figs/density-1D}
\end{center}
\caption{ Density profile for a slab of water with molecules constrained to move in one dimension. }
\label{fig:density-1D}
\end{figure}

\begin{figure}
\begin{center}
\includegraphics[width=\columnwidth]{figs/energy-1D}
\end{center}
\caption{ Energy density profile for a slab of water with molecules constrained to move in one dimension. }
\label{fig:energy-1D}
\end{figure}

\begin{figure}
\begin{center}
\includegraphics[width=\columnwidth]{figs/entropy-1D}
\end{center}
\caption{ Entropy for a slab of water with molecules constrained to move in one 
dimension. \textcolor{red}{FIX UNITS}}
\label{fig:entropy-1D}
\end{figure}

\begin{figure}
\begin{center}
\includegraphics[width=\columnwidth]{figs/xassoc-1D}
\end{center}
\caption{ Xassociation value (fraction of association sites
\emph{not} hydrogen-bonded) for a slab of water with molecules constrained to move in one 
dimension. }
\label{fig:xassoc-1D}
\end{figure}

\subsection{Test-particle approach}

More interesting applications are possible when we move into three
dimensions using spherical coordinates.  The first (but not simplest)
thing to try is looking at the response of liquid water to a single
fixed water molecule.  This is referred to as the \emph{test particle}
method\cite{FIXME}, and is as simple as constraining the density to
include a delta function at the origin---meaning that a single
molecule is fixed in location.  The free energy is then minimized
under this constraint, giving a density profile and an associated free
energy.  The density profile is just the self-correlation function
multiplied by the bulk liquid density.  \fixme{We should be
  distinguishing between the properties of the \emph{exact functional}
  and those of an approximate functional...}  Thus, a test-particle
calculation provides a stringent test of the correctness of any
approximate functional.

In order to perform the test-particle test, we introduce a sharp
gaussian density spike at the origin and relax the density under a
constraint that this spike remain.  Figure~\ref{fig:test-particle}
shows the resulting density profile, compared with the experimental
$g_{\textrm{OO}}$.  Note that this data was \emph{not} used in the
fitting of the empirical parameters of the functional.  The agreement
between experiment and theory is both qualitatively and quantitatively
poor.  The integral of the difference between this density profile
(excluding the delta function) and the bulk liquid density should be
0.06~molecules \fixme{look up and show derivation... or cut this},
but we find instead to be $-0.15$~molecules.  In addition, the associated
free energy should be positive: there is an entropic cost in fixing
one molecule in position, but for our functional the free energy of
this density configuration is in fact negative, with a value of
-3.5~eV, meaning that the homogeneous fluid is unstable.  This is a
serious flaw that must be addressed in future work.

\subsection{Hydration energy of hard-sphere solutes}

A more widely studied system that is simpler in many ways than the
test-particle problem is that of a \emph{hard sphere solute} dissolved
in water.  This is a common model for hydrophobic interactions.  The
effect of this solute is merely to create a cavity within which the
water may not penetrate.  This is the simples possible solute, and
will serve as a test case, which if passed could lead to the ability
to insert different chemicals into aqueous solution, instead of a
vapor cavity, in order to observe biologically and chemically
important aqueous solvation interactions.

The most obvious output of such a calculation is the free energy of
solvation.  In Figure~\ref{fig:surfaceTension} we plot the ratio of the
energy of the cavity system to the surface area of the cavity.  This
is an \emph{effective surface tension}, which approaches the bulk
surface tension as the cavity radius increases.  On the same plot, we
include the result from an SPC/E molecular dynamics
simulation\cite{huang2001shs}, showing very good agreement.  At very
small solute radius the effective surface tension is linear, meaning
that the free energy is proportional to the volume---this limiting
case is an exact result, which is exactly reproduced by any functional
that reduces to the ideal gas functional at small distances.

\begin{figure}
\begin{center}
%\includegraphics[width=\columnwidth]{radial-plots/surface-tension}
\end{center}
\caption{Free energy per surface area as a function of cavity
  radius. SPC/E results from~\cite{huang2001shs}}
\label{fig:surfaceTension}
\end{figure}

This exact result for small radius solutes relates to a general property of
fluids interacting with hard surfaces, which is that the pressure on a hard
surface is determined by the contact density of the fluid at the surface.
Thus, a functional that correctly predicts the free energy as a function of
radius will also give the proper contact density.
Figure~\ref{fig:cavities} shows the density profile for several hard sphere
radii, plotted together with with the results of the same SPC/E molecular
dynamics simulation shown in
Figure~\ref{fig:surfaceTension}\cite{huang2001shs}.  As expected---since
the free energies agree so well---the contact densities show good
agreement.  For small spheres the entire curve shows reasonable agreement,
with the agreement becoming worse for larger radii.

\begin{figure}[b]
\begin{center}
%\includegraphics[width=\columnwidth]{radial-plots/cavity-density}
\end{center}
\caption{Density profiles for multiple sizes of hard sphere solute.
  Results from a simulation of SPC/E water~\cite{huang2001shs} are
  shown as a dotted lines.}
\label{fig:cavities}
\end{figure}

In order to understand the worsening agreement in the density profile
for larger solutes in Figure~\ref{fig:cavities}, it is instructive to
consider the limit as the radius becomes very large.  In this limit,
the surface becomes essentially flat, and the liquid should develop a
\emph{vapor layer}, so that as the effective surface tension
approaches the bulk surface tension, the density profile should
approach the profile of the liquid-vapor interface as plotted in
Figure~\ref{fig:cartesianDensity}.  We demonstrate this effect in
Figure~\ref{fig:LargeR_RadialDensity}, which shows superimposed the
density profiles for several large solutes with that of the
liquid-vapor interface.  It is clear that as the radius is increased,
the profile is approaching that of the liquid-vapor interface.  Thus
we can understand the deviations from the molecular dynamics
simulations in Figure~\ref{fig:cavities} at larger radii as a direct
result of the oscillations observed in the planar liquid-vapor
interface.

\fixme{Not sure if we want to keep this paragraph or work it into the
  discussion somewhere?} For \emph{very} small cavities \fixme{not
  shown... should we add one?}, the contact density is equal to the
bulk liquid density, as the solute is too small to really perturb the
water (see the blue curve in Fig.~\ref{fig:cavities}).  As the cavity
size increases, the contact density (and pressure) first rise, and
then fall again, as seen in Figure~\ref{fig:cavities}.  This is due to
short range attractive interactions that are able to pass through the
cavities of small radii.  For larger radii, these interactions are
less prominent and there is less attraction to molecules across the
cavity.  Thus, the contact density is decreased.

\subsection{Two Hydrophobic Rods}

\begin{figure}
\begin{center}
\includegraphics[width=\columnwidth]{figs/density-rods-in-water}
\end{center}
\caption{ Density profile for two hydrophobic rods.}
\label{fig:density-rods}
\end{figure}

%%%%%%%%%%%%%%%%%%%%%%%%%%%%%%%%%%%%%%%%%%%%%%%%%%%%%%%%%%%%
\section{Conclusion}
By adding an attractive term to the fundamental-measure theory
functional, we developed a classical density functional for water
which is able reproduce expected results for hydrophobic
interactions.  We hope to extend this functional to allow for
hydrophilic interactions by coupling with electrostatic fields.

This method is computationally efficient, which will be useful when
introducing new functionals that include intramolecular and electromagnetic
interactions, that are not yet included in this model.

\bibliography{paper}% Produces the bibliography via BibTeX.

\end{document}

