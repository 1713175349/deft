

\documentclass[letterpaper,twocolumn,amsmath,amssymb,prb]{revtex4-1}
\usepackage{graphicx}% Include figure files
\usepackage{dcolumn}% Align table columns on decimal point
\usepackage{bm}% bold math
\usepackage{color}

%%%%%%%%%%%%%%%%%%%%%%%%%%%%%%%%%%%%%%%%%%%%%%%%%%%%%%%%%%%%
%% Definitions
\def\re{\text{Re}}
\def\im{\text{Im}}
\def\ket#1{\vert #1 \rangle}
\def\cU{{\cal{U}}}
\def\cD{{\cal{D}}}
\def\re{\text{Re}}
\def\im{\text{Im}}
\newcommand{\red}[1]{{\bf \color{red} #1}}
\newcommand{\blue}[1]{{\bf \color{blue} #1}}
\newcommand{\green}[1]{{\bf \color{green} #1}}
\newcommand{\rr}{\textbf{r}}
\newcommand{\xx}{\textbf{x}}
\newcommand{\refnote}{\red{[ref]}}

\newcommand{\fixme}[1]{\red{[#1]}}

% needsworklater is used to annotate bits that need work, but that we
% can postpone for a while.
\newcommand{\needsworklater}[1]{\emph{[#1]}}
% needsworknow is intended to prioritize stuff that needs fixing.
\newcommand{\needsworknow}[1]{\textcolor{red}{[\emph{#1}]}}

%%%%%%%%%%%%%%%%%%%%%%%%%%%%%%%%%%%%%%%%%%%%%%%%%%%%%%%%%%%%

\begin{document}
\title{A Classical Density-Functional Theory for Describing Water Interfaces}

\author{Jessica Hughes}
\affiliation{Department of Physics, Oregon State University, Corvallis, OR 97331}

\author{David Roundy}
\affiliation{Department of Physics, Oregon State University, Corvallis, OR 97331}

%%%%%%%%%%%%%%%%%%%%%%%%%%%%%%%%%%%%%%%%%%%%%%%%%%%%%%%%%%%%
\begin{abstract}
\needsworklater{ We develop a classical density functional theory for
  water by combining the fundamental-measure theory functional--which
  describes hard sphere fluids--with attractive interactions based on the 
  Statistical Associating Fluid Theory (SAFT).  Using
  this approximation, we are able to describe various properties of
  water near liquid-vapor and liquid-solid interfaces including water confined 
  to a slit and two hydrophobic rods immersed in water.  This functional will be the 
  foundation for a efficient method for future work
  predicting the interactions of chemicals in aqueous solution on an
  atomic level.}

\tableofcontents
\end{abstract}

\maketitle

%%%%%%%%%%%%%%%%%%%%%%%%%%%%%%%%%%%%%%%%%%%%%%%%%%%%%%%%%%%%
\section{Introduction}

\subsection{Water---the universal solvent}

\subsection{Classical density-functional theory}

%%%%%%%%%%%%%%%%%%%%%%%%%%%%%%%%%%%%%%%%%%%%%%%%%%%%%%%%%%%%
\section{Theory and Methods}
We introduce a new classical density functional for water that
\needsworklater{predicts the vapor density, liquid density, bulk
  modulus, and bulk surface tension, as well as allows for
  liquid-vapor coexistence}.  The Helmholtz free energy functional is
based on SAFT, and is composed of the usual terms:
\begin{equation}
  F[n] = F_\textit{id}[n] + F_\textit{hs}[n] + F_\textit{assoc}[n] + F_\textit{disp},
\end{equation}
where $F_\textit{id}$ is the ideal gas free energy, $F_\textit{hs}$ is
a hard-sphere free energy, $F_\textit{assoc}$ is the free energy of
association, and $F_\textit{disp}$ is the dispersion energy.  In
addition, a chemical potential term is used, so we actually work in
the grand canonical ensemble.  In the following sections, we will
introduce the terms of this functional.

\subsection{Ideal gas functional}
The first term is the ideal gas free energy functional,
which incorporates effects of kinetic energy.  The ideal gas
functional is given by
\begin{equation}\label{idealgas}
  F_{id}[n] = k_B T \int n(\xx)\left( \ln{n(\xx)} - 1\right) d\xx
\end{equation}
where n(\xx) is the density of water molecules.  The ideal gas free
energy functional has the property that the contact density at a hard
surface $n_c$ is proportional to the pressure on that surface,
according to the equation
\begin{equation}
  p = n_c k_BT \:,
\end{equation}
which also leads to the property that the free energy of solvation of
a hard solute is proportional to its volume in the limit of small
solutes.  These properties are retained by the total functional,
provided the remaining terms are \emph{purely
  nonlocal}~\cite{ashcroft?}.

\subsection{Hard-sphere repulsion}

We model the repulsive interactions using the White Bear version of
the Fundamental-Measure Theory~(FMT) functional for the hard-sphere
fluid~\cite{roth2002whitebear}.  In the homogeneous limit, the White
Bear functional reduces to the Mansoori-Carnahan-Starling-Leland
equation of state.  FMT functionals are expressed as the integral of
the \emph{fundamental measures} of a fluid, which provide local
measures of quantities such as the packing fraction, density of
spheres touching a given point and mean curvature.  The excess free
energy is written as:
\begin{equation}
F_{hs}[n] = k_B T \int (\Phi_1(\xx) + \Phi_2(\xx) + \Phi_3(\xx)) d\xx \; ,
\end{equation}
with integrands
\begin{align}
\Phi_1 &= -n_0 \ln\left( 1 - n_3\right)\\
\Phi_2 &= \frac{n_1 n_2 - \mathbf{n}_{V1} \cdot\mathbf{n}_{V2}}{1-n_3} \\
\Phi_3 &= (n_2^3 - 3n_2 \mathbf{n}_{V2} \cdot \mathbf{n}_{V2})
  \frac{
    n_3 + (1-n_3)^2\ln(1-n_3)
  }{
    36\pi n_3^2\left( 1 - n_3 \right)^2
  } ,
\end{align}
where the fundamental measure densities are given by:
\begin{align}
  n_3(\xx) &= \int n(\xx') \Theta(\left|\xx - \xx'\right| - R) d\xx' \\
  n_2(\xx) &= \int n(\xx') \delta(\left|\xx - \xx'\right| - R) d\xx'
  \\
  \mathbf{n}_{V2} &= \mathbf{\nabla} n_3 \\
  n_1 &= \frac{n_2}{4\pi R}\\
  \mathbf{n}_{V1} &= \frac{\mathbf{n}_{V2}}{4\pi R}\\
  n_0 &= \frac{n_2}{4\pi R^2}
\end{align}
For our water functional, we use the hard-sphere radius of
3.03420~\AA, which was found to be optimal by Clark
\emph{et al}.\cite{clark2006developing}

\begin{figure}
\begin{center}
\includegraphics[width=\columnwidth]{figs/equation-of-state}
\end{center}
\caption{The theoretical versus experimental phase diagram, vapor
  pressure versus temperature.  }
\label{fig:equation-of-state}
\end{figure}

\subsection{Association free energy}
To the purely repulsive hard-sphere free energy, we add two attractive
energy terms.  The first is the free energy of \emph{association},
which describes the effects of hydrogen bonds.  These are modeled as
four attractive patches (``association sites'') on the surface of the
hard sphere.  These four sites represent the two hydrogens and two
electron lone pairs.  There is an attractive energy
$\epsilon_\textit{assoc}$ when two molecules are arranged such that
the hydrogen of one interacts with the lone pair of the other.  The
volume of this interaction is $\kappa_\textit{assoc}$.  Thus the
association term in the free energy has two free parameters that must
be fit based on experimental data.

The association free energy has the form
\begin{align}
  F_\text{assoc}[n] &= 4 k_BT \int n_0(\xx)
  \left(\ln X(\xx) - \frac{X(\xx)}{2} + \frac12\right) d\xx
\end{align}
where the value of $4$ comes from the four association sites per
molecule, and the functional $X$ is the fraction of assotiation sites
\emph{not} hydrogen-bonded.  The fraction $X$ is determined by the
quadratic equation
\begin{align}
  X(\xx) &= \frac{\sqrt{1 + 8n_0(\xx)\zeta(\xx)\Delta(\xx)} - 1}
  {4 n_0(\xx)\zeta(\xx)\Delta(\xx)}
\end{align}
where $\zeta$ comes from
reference\cite{yu2002fmt-dft-inhomogeneous-associating,
  fu2005vapor-liquid-dft} and describes the inhomogeneity of the fluid
density:
\begin{align}
  \zeta(\xx) &= 1 - \frac{\mathbf{n_{2V}}\cdot\mathbf{n_{2V}}}{n_2^2}
\end{align}
The function $\Delta$ is given... below?

The following contact density for the hard-sphere fluid is in the code
called \texttt{gHS}, and comes
from\cite{yu2002fmt-dft-inhomogeneous-associating,
  fu2005vapor-liquid-dft}, but I don't use it.
\begin{align}
  \Delta &= 4\pi \kappa_\textit{assoc} g^\textit{HS}(\sigma)e^{-\beta
    \epsilon_\textit{assoc}} \\
  g^\textit{HS}(\sigma) &= \frac1{1-n_3}\left(1+\frac14\frac{\zeta n_2}{1-n_3}
  \left(1 + \frac{n_2}{14 (1-n_3)}\right)\right)
\end{align}
\textcolor{red}{Here is paper where I got the association functional:}

Instead, I use
\begin{align}
  \Delta(\xx) &= \kappa_\textit{assoc}
  e^{-\beta\epsilon_\textit{assoc}} g^\textit{SW}_\sigma(\xx) \\
  g^\textit{SW}_\sigma(\xx) &= g^\textit{HS}_\sigma(\xx) +
  \frac{1}{4}\beta\left(\frac{\partial a_1(\xx)}{\partial n_3(\xx)} -
  \frac{\lambda}{3 n3(\xx)}\frac{\partial a_1(\xx)}{\partial \lambda}\right)
\end{align}
% This is equation 77 of gil-villegas-1997-SAFT-VR
which I got from reference\cite{gil-villegas-1997-SAFT-VR}.

\begin{figure}
\begin{center}
\includegraphics[width=\columnwidth]{figs/saturated-liquid-density}
\end{center}
\caption{The theoretical versus experimental saturated liquid density
  versus temperature.  }
\label{fig:saturated-liquid-density}
\end{figure}

\subsection{Dispersion free energy}
The final term in the free energy is the dispersion term.  We use a
dispersion term based on the SAFT-VR
approach\cite{gil-villegas-1997-SAFT-VR}, which has two free
parameters, an interaction energy $\epsilon_\textit{disp}$ and a
length scale $\lambda_\textit{disp}$.

The dispersion free energy has the form~\cite{gil-villegas-1997-SAFT-VR}
\begin{align}
  F_\text{disp} &= A_1 + \beta A_2
\end{align}
where $A_1$ and $A_2$ are the first two terms in a high-temperature
perturbation expansion.  $A_1$ is the mean-field dispersion
interaction, which for a homogeneous fluid is given by
\begin{align}
  A_1 &= \frac12 n_b \int \varphi(\left|\xx\right|)
  g_{HS}(\left|\xx\right|) d\xx
\end{align}
The second dispersion term in the free energy $A_2$ describes the
effect of fluctuations resulting from the compression of the fluid due
to the dispersion interaction itself, and is commonly approximated
using the local compressibility approximation (LCA).

We use a free square-well function for the dispersion $\varphi$, which
has been demonstrated \textcolor{red}{(CITE)} to give a reasonable fit
to the equation of state.  Thus our model interaction has the form:
\begin{equation}
  \varphi(r) = \Theta(r-2 \lambda_\textit{disp} R)
\end{equation}
where $\Theta$ is the Heaviside step function.

\textcolor{red}{TODO:  Add the actual functions that we use and the
  actual form.  The above is really just an introduction.}


\begin{figure}
\begin{center}
\includegraphics[width=\columnwidth]{figs/temperature-versus-density}
\end{center}
\caption{The theoretical versus experimental saturated liquid density
  versus temperature.  }
\label{fig:saturated-liquid-density}
\end{figure}

\begin{figure}
\begin{center}
\includegraphics[width=\columnwidth]{figs/entropy}
\end{center}
\caption{The free energy, internal energy, and temperature*entropy versus density at room temperature.  }
\label{fig:energy-room-temp}
\end{figure}

\begin{figure}
\begin{center}
\includegraphics[width=\columnwidth]{figs/entropy-at-690K}
\end{center}
\caption{The free energy, internal energy, and temperature*entropy versus density near the critical temperature.  }
\label{fig:energy-near-critical-temp}
\end{figure}

\subsection{Other thermodynamic functions}

From the Helmholtz free energy functional, we may obtain any other
thermodynamic functions we should choose.  The grand free energy
$\Phi$, which we ordinarily use in our computations, is obtained by
simply adding a chemical potential term:
\begin{equation}
  \Phi[n] = F[n] + \mu \int n(\xx) d\xx
\end{equation}

The pressure would naturally be obtained by taking a partial
derivative with respect to volume at fixed temperature.  Since
\emph{volume} isn't a parameter in density-functional theory, we
consider the \emph{free energy per unit volume}.
\begin{align}
  F[n] &= f[n]V \\
  p[n] &= -\left(\frac{\partial F[n]}{\partial V}\right)_{T} \\
  &= -\left(\frac{\partial f[n]V}{\partial V}\right)_{T} \\
  &= -V\left(\frac{\partial f[n]}{\partial V}\right)_{T}
   - f[n]\left(\frac{\partial V}{\partial V}\right)_{T} \\
  &= n \left(\frac{\partial f[n]}{\partial n}\right)_{T} - f[n]
\end{align}

\begin{figure}
\begin{center}
\includegraphics[width=\columnwidth]{figs/pressure-with-isotherms}
\end{center}
\caption{The pressure versus density for various temperatures, including experimental pressure data from NIST.  }
\label{fig:pressure-with-isotherms}
\end{figure}

The entropy is naturally obtained by taking a partial derivative with
respect to temperature:
\begin{align}
  S[n] &= \left(\frac{\partial F[n]}{\partial T}\right)_{V}
\end{align}
This derivative is trivial for the ideal gas and hard sphere free
energies, which are simply proportional to the temperature.  For the
association and dispersion free energies, it is a little more work,
but still not hard.

Once we have the entropy, finding the internal energy is a simple
matter of subtraction:
\begin{align}
  U[n] &= F[n] + TS[n]
\end{align}
and the enthalpy is not much more work:
\begin{align}
  H[n] &= F[n] + TS[n] - pV
\end{align}


\begin{figure}
\begin{center}
\includegraphics[width=\columnwidth]{figs/surface-tension}
\end{center}
\caption{The theoretical versus experimental surface tension
  versus temperature.  }
\label{fig:surface-tension}
\end{figure}

\begin{figure}
\begin{center}
\includegraphics[width=\columnwidth]{figs/surface-298}
\end{center}
\caption{Liquid-vapor interface profile at room temperature.  }
\label{fig:liquid-vapor-profile}
\end{figure}

\subsection{Determining the empirical parameters (?)}\label{sec:empirical}

In order to ensure that the empirical parameters are correct, we test
the vapor pressure (see Figure~\ref{fig:equation-of-state}) and
saturated liquid density (see
Figure~\ref{fig:saturated-liquid-density}) against experimental
values.  To find these densities, we use the ``common tangent'' rule,
as illustrated in Figure~\ref{fig:homogeneous}.

\begin{figure}
\begin{center}
\includegraphics[width=\columnwidth]{figs/finding-vapor-pressure}\\
\includegraphics[width=\columnwidth]{figs/near-critical-point}
\end{center}
\caption{By plotting the free energy vs. density it is possible to
  determine a saturated liquid or vapor state by the common tangent
  method.  }
\label{fig:homogeneous}
\end{figure}

\begin{table}
\begin{tabular}{ccc}
  Fitting Parameters& \hspace{2em} & Experimental Values\\
\begin{tabular}{cc}
  $R$ & 2.3 $a_o$ \\
  $A$ & -0.0100909 Hartree \\
  $\gamma$ & 5\\
  $V$ & 515.792 $a_o^3$  \\
  $b$ & 0.75 $a_o$ \\
  $\mu$ & -0.014063 Hartree 
\end{tabular}
&& 
\begin{tabular}{cc}
 $n_l$ & 4.939 $\times 10^{-3} a_o^{-3}$\\
 $n_v$ & 1.141 $\times 10^{-7} a_o^{-3}$\\
 $\sigma$ & ??? \\
 $B$ & $7.435\times 10^{-5}$ Hartree$/a_o^3$\\
 $T$ & 298.15 K \\
\end{tabular}
\end{tabular}

\caption{Empirical parameters of optimized functional and experimental
  values matched, where $a_o$ is the Bohr radius, and $\sigma$ is the
  surface tension.  This table is not yet complete, but it should
  be soon...  We'll also need to think about the best units to use
  here.  Atom units might be best, but it also might be better to
  consistently use eV and angstrom or something.  I'm not sure about
  putting $k_BT$ data in the table.  Maybe it belongs in the caption?}
\label{tab:parameters}
\end{table}

%%%%%%%%%%%%%%%%%%%%%%%%%%%%%%%%%%%%%%%%%%%%%%%%%%%%%%%%%%%%
\section{Results and discussion}
\subsection{Liquid-vapor interface}

\subsection{Water confined to a slit}

One of the most computationally simple ways to study confined water is by
constraining it to a slit where the water molecules can only move in one
dimension. We apply a constraining potential of the form

\begin{align}
V = n_l*constraint + 100*n_v (FIXME)
\end{align}

such that the density is 100 times the vapor density outside the slit 
and approximately equal to the saturated liquid density inside. The slits
range in size from 5 bohrs to 150 bohrs (FIX - redo in nm?). We 
minimize the energy and calculate density, energy, and $X$ on a lattice
with a spacing of 0.2 bohrs between data points.

The density profile for a slit of size 60 bohrs is shown in Figure 
\ref{fig:density-1D}. Near the interface, there are oscillations around the 
liquid density which decrease near the center of the slit. We expect the
center of the slit to act approximately like bulk water, and the absence 
of oscillations supports this. For larger slits (~120 bohrs or more) the
majority of the water molecules in the slit act like the bulk. We also 
determine the smallest slit that can contain liquid water, approximately 
30 bohrs. For smaller slits, the initial liquid water inside will vaporize.

\begin{figure}
\begin{center}
\includegraphics[width=\columnwidth]{figs/density-1D}
\end{center}
\caption{ Density profile for a slab of water with molecules constrained 
to move in one dimension. }
\label{fig:density-1D}
\end{figure}

\begin{figure}
\begin{center}
\includegraphics[width=\columnwidth]{figs/energy-1D}
\end{center}
\caption{ Energy density profile for a slab of water with molecules 
constrained to move in one dimension. }
\label{fig:energy-1D}
\end{figure}

The value of $X$ behaves as expected for water constrained to a slit. We
show in Figure \ref{fig:xassoc-1D} the plot of $X$ for a slit of 60 bohrs.
Outside the slit, there is no hydrogen bonding, and inside $X$ is 
approximately equal to
that of saturated liquid water.

\begin{figure}
\begin{center}
\includegraphics[width=\columnwidth]{figs/xassoc-1D}
\end{center}
\caption{ Xassociation value (fraction of association sites
\emph{not} hydrogen-bonded) for a slab of water with molecules constrained 
to move in one dimension. }
\label{fig:xassoc-1D}
\end{figure}

The fluid density at a liquid-solid interface commonly displays
oscillatory behavior, but the presence of density oscillations at
liquid-vapor interfaces is more debated\cite{penfold2001structure}.
In fact, the \emph{meaning} of the density profile at a liquid-vapor
interface is clouded by the existence of long-wavelength perturbations
known as \emph{capillary waves}.  These waves are not accounted for in
density-functional theory (although the \emph{exact}
density-functionaly theory \emph{would} include them), with the result
that the density profile will be smoothed out by CW effects.

\textcolor{red}{I found a nice paper on the hard-slit
  problem... actually a couple of papers that address
  it~\cite{forsman1996computer, lum1999hydrophobicity}}

\subsection{Hydrophobic rods}

\subsubsection{One rod - diameter effects}

We move into two dimensions by studying a single hydrophobic rod
immersed in water. 

\subsubsection{Resolution and convergence tests}

(Plot with low (0.4 bohrs), regular (0.2), and high (0.05) resolution 
for single rod.)

\subsubsection{Two rods - hydrophobic interaction}

Here we will want to cite~\cite{walther2004hydrodynamic}, which gives
simulations for two interacting nanotubes in water.  It's not
precisely the same, but if we can't find something more similar, this
would be worth citing.

\begin{figure}
\begin{center}
\includegraphics[width=\columnwidth]{figs/density-rods-in-water}
\end{center}
\caption{ Density profile for two hydrophobic rods.}
\label{fig:density-rods}
\end{figure}

\subsection{Hard Sphere}

%%%%%%%%%%%%%%%%%%%%%%%%%%%%%%%%%%%%%%%%%%%%%%%%%%%%%%%%%%%%
\section{Conclusion}

\bibliography{paper}% Produces the bibliography via BibTeX.

\end{document}

